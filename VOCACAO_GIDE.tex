\chapter{A vocação de Gide\footnote[*]{``Gides Berufung'', in \versal{GS VII}-1, p. 257--269. Tradução de Pedro
  Hussak e Carla Milani Damião. Programa de rádio transmitido em 31 de outubro de 1929. Seleção dos textos de Gide
  em traduções para o português, citados por Benjamin, de Carla M. Damião. [\versal{N.~O.}]}}

Talvez esteja na hora de revisar, de uma vez por todas, o procedimento que nós cultivamos ao observar por ocasião de aniversários de pessoas vivas e de concebê"-lo de
forma um pouco mais razoável. Tempo de ativar algumas reflexões
substanciais em vez daquelas ``celebrações'' em que o crítico apenas
dá importância a si mesmo. Essas datas não têm mais sentido, caso se limitem a verificar a infância do homenageado, pensando na sua origem, colecionando os documentos de seus primeiros anos --- suas primeiras brincadeiras e primeiros escritos ---, apreendendo"-o naquilo que ele tem de encantador, desconhecido e significativo. Em
contrapartida, no que se refere a alguns dentre aqueles homenageados --- os maiores ---,
seria também o caso de pensar na exposição de sua influência, ou seja, seus intuitos ou
efeitos educacionais. Entretanto, o que nos possibilitou planejar celebrar o 60º aniversário de André Gide em novembro foi um raro feliz acaso: o fato de que assim como os traços de sua
obra já estão anunciados nos pensamentos e experiências da mais remota
juventude, a fidelidade às inspirações e destinos de sua
juventude continua a exprimir"-se na obra mais madura do poeta.
Tal é o caso de Gide, para quem o
fundamento do mundo intelectual concentra"-se,
em uma miniatura infinitamente precisa, na figura de sua vocação. Como essa
vocação aflora na sua infância remota, como ela não cessa de
acompanhá"-lo na juventude e na vida adulta: como ela nem sempre foi um grande e
patético imperativo, mas frequentemente uma perigosa voz vagante do
espírito da montanha vindo de maciços montanhosos
labirínticos que o chamou; como seus escritos em toda sua perfeição nunca foram para
ele uma propriedade, mas sim um fardo que ele descarregou para aumentar
sua constante disposição, é isso que quero mostrar, retirando os seguintes trechos das
suas mais variadas obras, de modo que, ao final,
eles permaneçam na memória dos senhores como um texto unitário. De
resto, os senhores perceberão nas passagens seguintes uma eminente
sobriedade, que eu quase diria uma cautela. Gide desenvolve em seus escritos
propriamente apenas uma coisa: a linha. E tanto a linha visível,
quanto a linha tátil. O que é raro em sua arte é que ela atinge
a mais alta concreção sensível sem abandonar a plenitude do que é
sensivelmente agradável e prazeroso. Daí a nobreza de seus escritos, pela
qual ele dificilmente pode ser comparado com um autor vivo. A
linguagem dos fragmentos que se seguem corresponde às linhas clássicas
da paisagem na qual o falante Menalque\footnote{Personagem do
  livro de Gide, \emph{O imoralista} (\emph{L'immoraliste}) de 1902. \versal{[N.~T.]}} é
pensado. Em um jardim sobre uma colina próxima à Florença, em frente à
Fiesole, ele fala aos seus amigos. O drama espiritual que desenrola
diante deles, veremos posteriormente desenvolver"-se em uma paisagem
palestina.

\begin{quote}
Vivia na perpétua espera, deliciosa, de um futuro qualquer. Ensinei a
mim mesmo, como perguntas ante respostas à espera, que a sede de
gozá"-la, nascida em face de cada fonte me revelava uma sede, e que, no
deserto sem água, onde a sede é insaciável, ainda preferisse o fervor de
minha febre sob a exaltação do sol. Havia, à noite, oásis maravilhosos,
mais frescos ainda por terem sido desejados durante o dia todo. Sôbre a
vastidão arenosa, acabrunhada ao sol e como um imenso sono --- mas a tal
ponto era grande o calor, e na própria vibração do ar --- senti ainda a
palpitação da vida, que não poderia adormecer, tremer de delíquio no
horizonte, inchar de amor a meus pés.

Cada dia, de hora em hora, nada mais buscava senão a penetração cada vez
mais simples da natureza. Possuía o dom precioso de não ser por demais
entravado por mim mesmo. {[}\ldots{}{]} Minha alma era o albergue aberto na
encruzilhada; o que quisesse entrar, entrava. Fiz"-me dútil,
amigàvelmente disponível por todos os meus sentidos, de ouvido atento a
ponto de não ter mais \emph{um} só pensamento pessoal, captador de tôda
emoção de passagem, e com reação tão mínima que nada mais considera mal,
de preferência a protestar contra o que quer que fosse. Demais, observei
muito breve em quão pouco ódio ao feio se alicerçava meu amor ao belo.
{[}\ldots{}{]}

Houve um tempo em que minha alegria se tornou tão grande que a quis
comunicar, ensinar a alguém o que dentro de mim a fazia viver.

Ao cair da tarde, contemplava os lares, dispersos durante o dia, se
reconstituírem em aldeias desconhecidas. O pai voltava, cansado do
trabalho; as crianças retornavam da escola. A porta da casa
entreabria"-se um instante numa acolhida da luz, de calor e risos, e
fechava"-se depois para a noite. Nada mais das coisas vagabundas, do
vento tremendo fora podia entrar. --- Família, eu vos odeio! Lares
fechados; portas cerradas; posses ciumentas da felicidade. --- Invisível
à noite, debrucei"-me por vêzes a uma janela, e fiquei durante muito
tempo a olhar os hábitos da casa. O pai ali estava, perto da lâmpada; a
mãe costurava; o lugar do avô permanecia vago; um menino perto do pai
estudava; e meu coração encheu"-se de desejo de levá"-lo comigo pela
estrada.

Vi"-o no dia seguinte ao sair da escola; falei"-lhe no outro dia; quatro
dias depois êle abandonou tudo para me seguir. Abri"-lhe os olhos diante
do esplendor da planície; êle compreendeu que se oferecia a êle. Ensinei
depois sua alma a tornar"-se mais vagabunda, alegre enfim --- e depois a
desprender"-se até de mim, a conhecer sua solidão.\footnote{Walter Benjamin cita o trecho de Gide em tradução própria (Cf. \versal{GS VII}-2, p. 625). Nem sempre assim o faz, recorrendo a traduções já
  existentes em alemão. Nesses trechos, utilizamos traduções brasileiras
  das obras de Gide. Os trechos citados nesta passagem são da obra
  \emph{Os frutos da terra}, traduzida por Sérgio Milliet, São Paulo,
  Difusão Europeia do Livro, 1961, páginas 49--50. No original não há a
  indicação de ruptura do texto, o que na presente tradução foi indicada
  pela utilização de colchetes. Mantivemos a acentuação em português da
  época da tradução de Milliet. \versal{[N.~T.]}}
  \end{quote}

Aqui fazemos uma pausa. A partir deste ponto --- a obra \emph{Norritures
terrestres}, da qual este trecho foi extraído, foi publicada em 1897 ---
nasceu, dez anos mais tarde, o escrito mais famoso de Gide: \emph{A
volta do filho pródigo}. Dele leio a última das cinco seções que compõem
o livro. São os diálogos com o pai reconciliado, com o inflexível irmão
mais velho e com a mãe misericordiosa. Lemos aqui a última
conversa com o irmão mais novo, na tradução de Rilke, publicada na
editora \emph{Insel}. Tivemos, aliás, a sorte de possuir também ou de
esperar os outros escritos de Gide em traduções felizes e de rara
confiabilidade para o alemão. A obra completa, que foi empreendida pela
\emph{Deutschen Verlagsanstalt}, da qual foram extraídos todos trechos
posteriores, está nas distintas mãos de Ferdinand Hardekopf.\footnote{As traduções utilizadas por Benjamin de Ferdinand Hardekopf são:
  \emph{Tagebuch der Falschmünzer}, Berlim, Leipzig, 1929, p. 90--93; e
  \emph{André Gide, Stirb und Werde}, Berlim, Leipzig, 1930, p. 28--30. \versal{[N.~T.]}}

\begin{quote}
Ao lado de seu quarto há outro, também amplo e de paredes nuas. O
pródigo, uma candeia à mão, por ele entra até o leito onde repousa o
irmão mais novo, rosto voltado em direção à parede. Começa a falar em
voz baixa para, se é que ele dorme, não lhe perturbar o sono.

--- Meu irmão, quero falar"-lhe.

--- Pois fale, quem o impede?

--- Pensei que você estava dormindo.

--- Não é preciso dormir para sonhar.

--- Ah! Estava sonhando; com quê?

--- Não importa! Se eu próprio já não compreendo meus sonhos, não há de
ser você agora quem me vai explicá"-los.

--- Serão tão sutis assim? Mas, me contando, talvez pudesse tentar.

--- Você, por acaso, poderia escolher seus sonhos? Olha que os meus são
como querem ser, mais livres do que eu\ldots{} Que foi que veio fazer aqui?
Por que perturbou meu sono?

--- Você não estava dormindo, e vim falar"-lhe com carinho.

--- Que tem a dizer?

--- Nada, se o toma nesse tom.

--- Então, adeus.

O pródigo sai em direção à porta, mas apóia no chão a candeia que alumia
debilmente o quarto, e, voltando, senta"-se à beira da cama e afaga
longamente, no escuro, a fronte do irmão, que está voltada contra ele.

O irmão rebelde ergueu"-se de repente.

--- Diga: foi o irmão quem o mandou?

--- Não, menino; não foi ele, foi nossa mãe.

--- Ah! Você não teria vindo por si mesmo.

--- No entanto, eu venho como amigo.

Semi"-erguido no leito, o menino encara fixamente o pródigo.

--- Como um dos meus poderia ser meu amigo?

--- Você se engana quanto a nosso irmão\ldots{}

--- Não me fale dele! Odeio"-o!\ldots{} Meu coração inteiro se impacienta
contra ele. Por causa dele foi que lhe respondi asperamente.

--- Como assim?

--- Você não compreenderia.

--- Fale, mesmo assim\ldots{}

O pródigo aperta"-o contra o peito e o irmão adolescente deixa abrir seu
coração:

--- Na noite em que você voltou, não consegui dormir. Fiquei o tempo todo
pensando: Tinha outro irmão, e não sabia\ldots{} Foi por isso que meu
coração bateu mais forte, quando o vi avançar pelo pátio da casa, todo
coberto de glória.

--- Santo Deus! Eu vinha então coberto de andrajos.

--- Sim, eu o vi, e o achei glorioso. E vi também o que fez o pai: pôs em
seu dedo um anel, um anel que nem mesmo nosso irmão tem igual. Não quis
perguntar nada a ninguém a seu respeito: sabia apenas que você vinha de
muito longe, e seu olhar, à mesa\ldots{}

--- Então você estava no festim?

--- Oh! Bem sei que não me reparou; durante o tempo todo do banquete
olhava para longe, sem ver nada. Que fosse, na segunda noite, falar com
o pai, ainda compreendo, mas que na terceira\ldots{}

--- Termine.

--- Ah! Ao menos uma palavra de carinho bem me poderia ter dito!

--- Você me esperava então?

--- E como! Acha que eu odiaria assim nosso irmão se você não tivesse ido
falar com ele e demorasse tanto aquela noite? Que poderiam dizer? Se
você se parece comigo, bem sabe que nada tem em comum com ele.

--- Cometi graves faltas contra nosso irmão.

--- Será possível?

--- Pelo menos para nosso pai e nossa mãe. Você sabe que eu fugi de casa.

--- Bem sei. Isto foi há muito tempo, não?

--- Mais ou menos quando tinha sua idade.

--- Ah!\ldots{} É isso que você chama de erros?

--- Sim, este foi meu erro, meu pecado.

--- Quando você partiu, achou que procedia mal?

--- Não; sentia"-me como na obrigação de partir.

--- Que aconteceu depois, para que sua verdade de então se transformasse
em erro?

--- Sofri muito.

--- E é isto que o faz dizer: estava errado?

--- Não, não é bem isso: foi isto que me fez refletir.

--- Então não havia refletido antes?

--- Havia, mas a debilidade de minha razão se deixava impor por meus
desejos.

--- Como mais tarde pelo sofrimento. De sorte que, hoje então, você volta\ldots{} vencido?

--- Não é bem assim: resignado.

--- Ou seja, renunciou a ser como queria.

--- Que meu orgulho me persuadiu a ser.

O menino permanece um instante em silêncio, depois de súbito soluça e
exclama:

--- Irmão! Eu sou igual a você quando partiu. Oh! Diga"-me: só encontrou
decepções pelo caminho? Tudo o que pressinto existir lá fora, diferente
daqui, não passa de miragem? Tudo o que sinto em mim de novo não é mais
que fantasia? Fale: que havia de desesperador em seu caminho? Oh! Que
foi que o fez regressar?

--- Perdi a liberdade que buscava; cativo, fui obrigado a servir.

--- Aqui eu me sinto cativo.

--- Sim, mas tive que servir a maus senhores; aqui, pelo menos, servimos
nossos pais.

--- Ah! Servir por servir, não se tem pelo menos a liberdade de escolher a
servidão?

--- Eu achava que sim. Tão longe quanto puderam ir meus pés, como Saul a
buscar jumentas, andei a perseguir o meu desejo; mas, onde esperava um
reino, só encontrei miséria. Contudo\ldots{}

--- Não se enganou de caminho?

--- Fui caminhando sempre em frente.

--- Tem certeza? Todavia, há outros reinos, ainda, e terras sem rei, a
serem descobertas.

--- Quem lhe disse?

--- Eu sei. Pressinto. Parece até que já as conquistei.

--- Orgulhoso!

--- Ah! Ah! Foi isso o que nosso irmão lhe disse. Por que me vem agora
repeti"-lo? Por que não conservou esse orgulho? Decerto não teria
regressado.

--- E assim não o teria conhecido.

--- Teria, sim; lá, onde iria a seu encontro, decerto me reconheceria como
irmão; mesmo agora, parece"-me que é para encontrá"-lo que eu sigo.

--- Segue?

--- Não percebeu? Não me encoraja igualmente a partir?

--- Quisera poupar"-lhe o retorno, dissuadindo"-o da partida.

--- Não, não, não me diga isto; não é isto que me quer dizer. Foi como um
conquistador que você também partiu.

--- E foi isso que me fez sentir ainda mais a servidão.

--- Então, para que submeter"-se? Já estava assim tão fatigado?

--- Não, ainda não; mas tive dúvidas.

--- Que quer dizer?

--- Duvidava de tudo, de mim mesmo; quis parar, fixar"-me enfim em qualquer
parte; o conforto que esse patrão me prometia acabou tentando"-me\ldots{}
sim, sinto"-o perfeitamente agora: fracassei.

O pródigo inclina a cabeça e oculta os olhos com a mão.

--- Mas, e a princípio?

--- Caminhei por muito tempo pela imensa terra inóspita.

--- O deserto?

--- Nem sempre era deserto.

--- E que buscava?

--- Eu próprio não sei bem.

--- Levante"-se da cama. Olhe o que está ali na mesa"-de"-cabeceira, junto a
esse livro em frangalhos.

--- Uma romã partida.

--- Foi o tratador de porcos que a trouxe numa tarde, depois de passar
três dias fora.

--- Sim, é uma romã silvestre.

--- Bem sei; é de uma acidez quase insuportável; sinto, no entanto, que a
morderia, se estivesse com bastante sede.

--- Ah! Agora eu lhe posso dizer: foi essa sede que eu buscava no deserto.

--- Uma sede que só este fruto amargo consegue aplacar\ldots{}

--- Não: mas nos faz amar essa sede.

--- Sabe onde colhê"-lo?

--- Num pequeno pomar abandonado, aonde se chega quase ao anoitecer. Já
nenhum muro o separa do deserto. Ali corria um regato; alguns frutos,
quase maduros, pendiam das ramagens.

--- Que frutos?

--- Os mesmos de nosso pomar, porém silvestres. Fizera calor o dia
inteiro.

--- Ouça; sabe por que o esperava esta noite? É que partirei esta noite.
Hoje, de madrugada, ao clarear\ldots{} Estou disposto a tudo e já tenho as
sandálias calçadas.

--- Como! Pensa em fazer o que eu não consegui?\ldots{}

--- Você me abriu o caminho, e me sustentarei de pensar em você.

--- E eu em admirá"-lo; mas trate de esquecer"-me, em vez disso. Que vai
levar daqui?

--- Bem sabe que, sendo o último, não tenho direito à partilha. Vou sem
levar nada.

--- É melhor.

--- Que está olhando pela janela?

--- O horto onde repousam nossos mortos.

--- Irmão\ldots{} (e o menino, erguendo"-se do leito, passa o braço em torno do
pescoço do pródigo, num gesto que se faz tão doce quanto sua voz) ---
Venha comigo.

--- Deixe"-me! Deixe"-me! Ficarei para consolar nossa mãe. Sem mim, você
será mais corajoso. A hora está chegando. O céu empalidece. Parta sem
ruído. Vamos! Abrace"-me, meu caro irmão, você leva todas as minhas
esperanças. Tenha força: esqueça"-nos: esqueça"-me. Que você possa nunca
mais voltar\ldots{} Saia sem ruído. Eu seguro a candeia\ldots{}

--- Ah! Dê"-me a mão até a porta.

--- Cuidado com os degraus do patamar\ldots{}\footnote{A tradução
  utilizada por Benjamin da obra \emph{A volta do filho pródigo} é de
  Rainer Maria Rilke, Leipzig, 1914, p. 30--38. Utilizamos a tradução
  brasileira de Ivo Barroso. Rio de Janeiro, Editora Nova Fronteira,
  1984, p. 165--172. \versal{[N.~T.]}}
  \end{quote}

Se por ora retomamos o trecho interrompido há pouco, podemos entender
completamente quais são os desejos com os quais o Gide de quarenta anos
de idade acompanha esse irmão mais novo em seu caminho. É como se ele
falasse, em seu nome, desta peregrinação na qual se realizou o que o
irmão perdido procurara em vão.

\begin{quote}
Sòzinho, saboreei a alegria violenta do orgulho. Gostava de
levantar"-me antes da alvorada; chamava o sol por cima das choças; o
canto da cotovia era minha fantasia e o orvalho minha loção da aurora.
Comprazia"-me com exageradas frugalidades, tão pouco comendo que minha
cabeça se fazia leve e tôda sensação se me tornava uma espécie de
embriaguez. Muitos vinhos bebi depois, mas nenhum me dava, bem o sei,
êsse aturdimento do jejum, essa vacilação da planície na madrugada,
antes que, em chegando o sol, eu dormisse no fundo de um monte de feno.

O pão que levava comigo, guardava"-o por vêzes até o semi desfalecimento;
parecia"-me então sentir menos estranhamente a natureza e que ela me
penetrava melhor; era um afluxo de fora; com todos os meus sentidos
abertos acolhia"-lhe a presença; tudo em mim a isso o convidava.

Minha alma enchia"-se enfim de lirismo, que minha solidão exasperava e
que me afadigava à tarde. Sustentava"-me por orgulho, mas lamentava então
a ausência de Hilaire que no ano anterior me desviava de tudo o que meu
humor tinha de demasiado selvagem.

Com êle falava, ao chegar à tarde. Êle próprio era poeta, compreendia
tôdas as harmonias. Cada efeito natural se nos tornava como uma
linguagem aberta em que se lhe podia ler a causa; aprendíamos a conhecer
os insetos pelo vôo, os pássaros pelo canto, e a beleza das mulheres
pelas marcas dos pés na areia. {[}\ldots{}{]} Aspirando tudo com delícia, em
vão procurávamos cansar nossos desejos; cada um de nossos pensamentos
era um fervor {[}\ldots{}{]}.\footnote{A tradução de \emph{Os frutos da
  terra} é de Benjamin. Utilizamos a tradução de Sérgio Milliet,
  p. 50--51. \versal{[N.~T.]}}
  \end{quote}

Paramos aqui, porque aqui adicionamos a notável frase de Gide que diz:
``A melancolia não é senão um fervor que descaiu''.\footnote{Segundo a tradução de Sérgio Milliet de \emph{Os frutos da terra},
  p. 18. \versal{[N.~T.]}} Essa frase evoca a lembrança da mais surpreendente personagem
que, no momento da peripécia, avança sobre o palco do drama existencial de
Gide. Trata"-se de Satã que subitamente aparece em frente dele com a voz do anjo da
vocação. Certamente não um Satã como o tentador da carne, mas
como príncipe da tristeza, como o belo demônio que olha profundamente na
alma, segredando as três grandes promessas enganosas: a ilimitada liberdade,
a ilimitada profundeza, a ilimitada espiritualidade.
Na existência de Gide, ele carregou os traços de Oscar Wilde. Reiteradamente,
da sua bela contribuição em \emph{In memoriam Oscar Wilde}, depois em
\emph{Pretextes} e finalmente ainda na autobiografia \emph{Se o grão não
morre}, Gide procurou apanhar este instante decisivo da sua vida: a
aparição de Oscar Wilde. Sem que o seu nome apareça, é sem dúvida também
Wilde o parceiro do diálogo a seguir, que nós retomamos do \emph{Diário
dos moedeiros falsos}.

\begin{quote}
--- Mas agora que estamos sozinhos, diga"-me, por favor, de onde vem essa
estranha necessidade de acreditar que há perigo ou pecado em tudo aquilo
que você vai empreender?

--- Pouco importa; o importante é que isso não me retenha.

--- Durante muito tempo achei que era apenas um resto de sua educação
puritana; mas agora comecei a achar que é preciso ver nisso um não sei o
quê de byronismo\ldots{} Oh! Não proteste: sei que você tem horror ao
romantismo: pelo menos você o diz; mas você tem amor pelo drama\ldots{}

--- Tenho amor pela vida. Se busco o perigo, é com a confiança, a certeza
de que irei triunfar. Quanto ao pecado\ldots{} o que me atrai nele\ldots{} oh!
Não creia que é esse refinamento que fazia a italiana dizer sobre o
sorvete que degustava: `\emph{Peccato che non sai um peccato}'. Não,
talvez seja antes o desprezo, a raiva, o horror de tudo aquilo que eu
chamava de virtude em minha juventude; é também que\ldots{} como dizer"-lhe\ldots{} não faz muito tempo que compreendi\ldots{} é que tenho o diabo no meu
jogo.

--- Nunca pude entender, confesso"-lhe, o interesse que havia em acreditar
no pecado, no inferno ou em diabruras.

--- Permita; permita; mas eu também não, não acredito nele, no diabo;
somente, e aí está o que me dilacera, enquanto não se pode servir a Deus
senão acreditando n'Ele, o diabo não tem necessidade que se acredite
nele para servi"-lo. Ao contrário, nunca o servimos tão bem quanto
ignorando"-o. Ele tem sempre interesse em não se deixar conhecer; e é
isso, já lhe disse, que me dilacera: é pensar que, quanto menos acredito
nele, mais eu o reforço.

Dilacera"-me, compreenda"-me bem, pensar que é precisamente isso que ele
deseja: que não se creia nele. Ele sabe bem como fazer, vá, para
insinuar"-se em nossos corações, onde só pode entrar se de início não for
percebido.

Refleti muito sobre isso, garanto"-lhe. Evidentemente, e apesar de tudo
que acabei de dizer, em perfeita sinceridade, não acredito no demônio.
Tomo tudo que diz respeito a isso como uma simplificação pueril e
explicação aparente de certos problemas psicológicos --- aos quais
repugna minha mente dar outras soluções senão as perfeitamente naturais,
científicas, racionais. Mas, de novo, o próprio diabo não falaria de
outro modo; ele está exultante; sabe como atrás dessas explicações
racionais, que o relegam ao rol das hipóteses gratuitas. Satã ou a
hipótese gratuita; isso deve ser seu pseudônimo preferido. Pois bem,
apesar de tudo o que lhe digo a respeito, apesar de tudo o que penso e
que não lhe digo, uma coisa é certa: a partir do instante em que admito
sua existência, --- e isso me acontece apesar de tudo, ainda que fosse por
um instante, às vezes --- desde esse instante, parece"-me que tudo fica
claro, que compreendo tudo; parece"-me que, de repente, descubro a
explicação de minha vida, de todo o inexplicável, de todo o
incompreensível, de toda a sombra de minha vida. Gostaria de escrever um
dia uma\ldots{} oh! Não sei como dizer --- isso se apresenta à minha mente
sob uma forma de diálogo, mas haveria algo mais\ldots{} enfim, isso talvez
se chamasse: ``Conversa com o diabo'' --- e você sabe como começaria?
Encontrei a primeira frase; a primeira a pôr em sua boca, entenda; mas
para encontrar essa frase é preciso já conhecê"-lo muito bem\ldots{} eu o
faria dizer no início: --- \emph{Por que me temerias? Sabes bem que eu não
existo}. Sim, creio que é isso. Isso resume tudo: é dessa crença na não
existência do diabo que\ldots{} Mas fale um pouco: preciso que me
interrompam.

--- Não sei o que lhe dizer. Você está me falando de coisas nas quais
percebo nunca ter pensado. Mas não posso esquecer que muitas mentes, que
considero como grandes, acreditam na existência do diabo, em seu papel
--- e até atribuindo"-lhe a melhor parte. Você sabe o que dizia Goethe?
Que o poder de um homem e sua força de predestinação eram reconhecíveis
por aquilo que carregassem em si de demoníaco.

--- Sim, já me falaram dessa frase; você deveria reencontrá"-la para
mim.\footnote{André Gide, ``Identificação do demônio'', in
  Apêndice ao \emph{Diário dos moedeiros falsos}. Tradução de Mário
  Laranjeira. São Paulo, Estação Liberdade, 2009, p. 135--139. \versal{[N.~T.]}}
\end{quote}

Wilde amava definir"-se como um grego póstumo. O distinto comentador de
Gide, Du Bos, disse: ``Gide é um grego póstumo em um sentido muito
diferente daquele de Wilde. O helenismo de Gide nasce na estufa de Argel
(em Argel ocorreram os encontros decisivos com Wilde), seu helenismo é
produto de uma cultura extremamente intensa''. Entretanto, não se pode
desconhecer um duplo efeito: na atividade artística e no gênio pedagógico
do humano. Há artistas, de quem a todo instante esquecemos, com cujas obras temos que lidar com a arte. E não é necessário que entre em jogo
aqui nenhuma ilusão. Podemos ler \emph{Os demônios} de Dostoiévski e ter
a consciência de que nos aprofundamos em um romance. Ainda assim, não
nos ocorre que ele o teria escrito como artista. Trata"-se disso:
Dostoiévski escreveu o livro, e para nós ele é arte. No caso de Gide, ao
contrário, não há nenhuma linha da qual não tenhamos o sentimento
imperativo de que ele a tenha escrito como artista. Daí, se podemos dizer
assim, surge o particular charme grego. Pois aquela aura incolor, sem luz e sem
calor no seu jogo de formas mais inexprimível e que resplandece em torno da
obra de arte é de tipo grego, como é grega também, como dissemos, a atitude
fundamentalmente pedagógica de seu espírito. Como sempre, para ele, toda
propriedade interna e externa era boa apenas para desfazer"-se dela,
assim ele inculca tal atitude também aos jovens, mesmo que aquilo do que
fogem seja o próprio Gide. Isso não significa que os melhores deles não o
tenham cortejado. Quando os surrealistas fundaram sua primeira revista
\emph{Littérature}, chamaram apenas Gide dentre todos os expoentes
da geração mais velha. E ele não pôde dar"-lhes nada de melhor do que sua
rigorosa e descompromissada modelagem da sua experiência juvenil. Ele fez isso
no seu último grande romance \emph{Os moedeiros falsos}. O
capítulo sobre ``Bernardo e o anjo'', em nosso contexto, é o que melhor
representaria essa obra. Nós queremos, entretanto, como conclusão,
retornar, como o próprio Gide o fez há alguns anos, à sua juventude.
Certamente, o grande poeta André Gide mostrou"-se como o grande homem,
muito mais tarde, ao menos ele tornou"-se perceptível a si mesmo muito
tarde.\footnote{Benjamin provavelmente refere"-se aqui à escrita autobiográfica de Gide como revelação de si e revelação de sua homossexualidade ao grande público. [\versal{N.~T.}]} Em lugar nenhum, ele aparece em sua grande obra autobiográfica
como um garoto prodígio. Mas é magnífico como, em muitas passagens,
e mais notadamente nas que se seguem, ele soube fixar os chamamentos
decisivos que ressoam em toda infância na decisão de segui"-los e
apenas por isso permanecem na memória.

\begin{quote}
Eu já estava deitado, mas um rumor singular, um frêmito de alto a
baixo, pela casa, unidos a ondas harmoniosas, me impediam o sono. Sem
dúvida, durante o dia, eu observara os preparativos. Sem dúvida me
teriam dito que haveria um baile nessa noite. Mas um baile, sabia eu o
que era isso? Não lhe tinha dado importância e deitara"-me como nas
outras noites. Mas agora havia aquele rumor\ldots{} Presto atenção; procuro
surpreender algum ruído mais distinto, compreender o que se passa. Apuro
o ouvido. Por fim, não podendo mais, levanto"-me, saio do quarto, avanço
às apalpadelas pelo corredor escuro e, descalço, alcanço a escada cheia
de luz. Meu quarto é no terceiro andar. As ondas de som sobem do
primeiro; é preciso ir ver; e à medida que, de degrau em degrau, me
aproximo, distingo ruídos de vozes, frufru de fazendas, cochichos e
risos. Nada apresenta o ar costumeiro; parece"-me que de repente vou ser
iniciado numa outra vida, misteriosa, diferentemente real, mais
brilhante e mais patética, e que somente começa quando as crianças
pequenas estão deitadas. Os corredores do segundo andar, todos cheios de
noite, estão desertos; a festa é embaixo. Continuo a descer? Vão ver"-me.
Vão castigar"-me por não estar dormindo, por ter visto. Enfio a cabeça
entre os gradis da rampa. Precisamente quando chegam os convidados, um
militar uniformizado, uma senhora cheia de fitas, sedas, com um leque na
mão, o criado, meu amigo Victor, que a princípio não reconheço por causa
dos calções e das meias brancas, posta"-se diante da porta aberta do
primeiro salão para anunciá"-los. De súbito alguém salta para mim: é
Marie, minha ama que, como eu, procurava ver, escondida um pouco mais
abaixo no primeiro ângulo da escada. Ela me toma nos braços; de início
penso que vai levar"-me de volta para o meu quarto; mas não, ao
contrário, ela quer descer"-me para o lugar onde estava, de onde o olhara
apanha uma nesga da festa. Agora ouço perfeitamente bem a música. Ao som
dos instrumentos que não posso ver, cavalheiros giram senhoras ataviadas
e todas são mais belas que comumente. A música cessa; os dançarinos
param; e o ruído das vozes substitui o dos instrumentos. Minha ama vai
reconduzir"-me ao quarto; mas nesse momento uma das belas senhoras, que
estava em pé, apoiada junto à porta se abanando, enxerga"-me. Ela corre
para mim, beija"-me e ri porque eu não a reconhecia. É evidentemente
aquela amiga de mamãe, que vi hoje de manhã; mas assim mesmo não estou
certo de que seja mesmo ela, realmente ela. E quando me reencontro na
minha cama, tenho as ideias embaralhadas e penso, confusamente, antes de
mergulhar no sono: existe a realidade e existem sonhos; e depois existe
\emph{uma segunda realidade}.\footnote{Benjamin cita a tradução
  de Ferdinand Hardekopf. Utilizamos a seguinte publicação e tradução
  como referência: André Gide. \emph{Se o grão não morre}. Tradução de
  Hamilcar de Garcia. Rio de Janeiro, Editora Nova Fronteira, 1982, p.
  20--22. \versal{[N.~T.]}}
  \end{quote}

Há uma palavra de Sainte"-Beuve que é como uma profecia metafórica a
André Gide. Ele falou certa vez da diferença entre a \emph{intelligence glaive} e a
\emph{intelligence miroir}, a inteligência da espada e a inteligência do espelho.
Gide mostra ambas em sua unidade perfeita. O Eu é a sua espada e seu
escudo é tão brilhante que sobre ele aparece o mundo inteiro, como no escudo de Aquiles.
