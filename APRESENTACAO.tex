%Editora HEDRA%

%Coleção Walter Benjamin%

%Direção: Amon Pinho e Francisco Pinheiro Machado%

%Volume 2%

%\textbf{Walter Benjamin e a Paris do século XX: o lugar do crítico
%literário}%

%Organização, tradução e notas: Carla Milani Damião e Pedro Hussak van
%Velthen Ramos%

%Apresentação: Carla Milani Damião%

%Revisão da tradução: Francisco Pinheiro Machado%

%Índice%

%\begin{enumerate}
%\def\labelenumi{\Roman{enumi}.}
%\item
%  \textbf{O ambiente crítico-literário de Paris}
%\end{enumerate}%

%\begin{enumerate}
%\def\labelenumi{\arabic{enumi}.}
%\item
%  Três franceses (1927)
%\item
%  O comerciante no poeta (1926)
%\item
%  Talentos parisienses (1930)
%\item
%  Diário parisiense (1929-1930)
%\end{enumerate}%

%\begin{enumerate}
%\def\labelenumi{\Roman{enumi}.}
%\setcounter{enumi}{1}
%\item
%  \textbf{Escritos sobre Paul Valéry}
%\end{enumerate}%

%\begin{enumerate}
%\def\labelenumi{\arabic{enumi}.}
%\setcounter{enumi}{4}
%\item
%  Paul Valéry na \emph{École Normale} (1926)
%\item
%  Paul Valéry. Em seu sexagésimo aniversário (1931)
%\item
%  Anotações ao programa de rádio em homenagem a Paul Valéry
%\end{enumerate}%

%\begin{enumerate}
%\def\labelenumi{\Roman{enumi}.}
%\setcounter{enumi}{2}
%\item
%  \textbf{Escritos sobre Proust}
%\end{enumerate}%

%\begin{enumerate}
%\def\labelenumi{\arabic{enumi}.}
%\setcounter{enumi}{7}
%\item
%  Imagem de Proust (1929-1934)
%\item
%  Anotações ao ensaio ``Imagem de Proust''
%\end{enumerate}%

%\begin{quote}
%\textbf{IV -- Escritos sobre André Gide}
%\end{quote}%

%\begin{enumerate}
%\def\labelenumi{\arabic{enumi}.}
%\setcounter{enumi}{9}
%\item
%  André Gide e a Alemanha. Conversação com o poeta (1928)
%\item
%  Conversação com André Gide (1928)
%\item
%  Vocação de Gide (1929)
%\item
%  Édipo ou o mito racional (1932)
%\end{enumerate}%

%\begin{quote}
%\textbf{V -- Cartas parisienses}
%\end{quote}%

%\begin{enumerate}
%\def\labelenumi{\arabic{enumi}.}
%\setcounter{enumi}{13}
%\item
%  Carta parisiense I. André Gide e seu novo adversário (1936)
%\item
%  Carta parisiense II. Pintura e fotografia (1936)
%\end{enumerate}

\chapter{Apresentação}

O presente volume da Coleção Walter Benjamin tem por objetivo chamar a
atenção do leitor para alguns trabalhos escritos em uma década: de 1926
a 1936. São textos de particular importância para nosso pensador que
presumia ser, segundo diz Gershom Scholem, o único crítico legítimo de
literatura alemã.\footnote{O contexto da carta de Gershom Scholem a
  Benjamin (Walter Benjamin, \emph{Gesammelte Briefe} [daqui em diante: \versal{GB}], vol. \versal{I}. Organização de Gershorn
  Scholem e Theodor W. Adorno. Frankfurt a. \versal{M}.: Suhrkamp, 1978, 1ª edição, Jerusalém, 20 de fevereiro de 1930, p.~511) alude à ambição de
  Benjamin de se tornar o único legítimo crítico de literatura alemã
  (``speziell von der Warte Deiner präsumptiven Stellung als einziger
      echter Kritiker der deutschen Literatur aus keine Notwendigkeit eines
      Weges zum Hebräischen abzusehen ist''). Note"-se que, no comentário de
  Scholem há uma queixa ao detectar o distanciamento com o aprendizado
  do hebraico.} Sabemos que o sentido de crítica em Benjamin não está
isolado no empenho de sua pena crítica literária, e sim permeado, de
partida, pela reflexão filosófica empreendida em sua tese de doutorado,
\emph{O conceito de crítica de arte no romantismo alemão}. Importante
notar que o caráter legítimo de crítica que ambicionava não era comum a
seus contemporâneos,\footnote{Cf. Heinrich Kaulen, ``Rehabilitierung der
  Polemik Walter Benjamin als Literaturkritiker und Rezensent''.
  \emph{https://literaturkritik.de/id/14357} (consulta realizada em
  12/06/2019).} pouco ou nada ortodoxo em comparação com a crítica que
era realizada então. É, particularmente, nesse sentido, que o convívio
com críticos literários franceses transforma a noção aprofundada de
crítica presente em sua tese de doutorado, ao participar ativamente em
um ambiente literário e político no qual Benjamin insere"-se antes mesmo
de seu exílio.

Os textos reunidos nesse volume nos dão notícias, portanto, do trânsito
entre Alemanha e França percorrido por Benjamin em vários sentidos:
literário"-crítico, filosófico, artístico, político e biográfico. O
título faz menção ao ``lugar'' de Benjamin, como crítico literário
alemão e refugiado político, no debate literário francês no período
entreguerras. ``O lugar social do escritor francês na atualidade'',
título\footnote{Segundo a tradução de João Barrento, Editora Autêntica,
  2017. p. 79-10.} de um dos ensaios mais importantes de Benjamin no
período considerado, ao lado do ensaio ``O autor como
produtor''\footnote{Título de uma suposta ``Conferência pronunciada no
  Instituto para o estudo do fascismo em Paris, 27 de abril de 1934''.
  De acordo com os editores Rolf Tiedemann e Hermann Schweppenhäuser,
  Benjamin dá a entender a Adorno que a conferência ainda iria
  acontecer, em carta enviada no dia 28 de abril daquele ano (\versal{GS II}-3, p.~1460-1463), porém os editores afirmam com
  precisão que a conferência não foi efetivamente realizada e que não
  havia apresentação de conferências no mencionado Instituto para
  estudos do fascismo.} e das ``Cartas parisienses \versal{I} e \versal{II}''\footnote{Cf.
  C. Kambas, verbete de \emph{Walter Benjamins Handbuch}, p. 420-436.}
traduzidas nesse volume, são claros exemplos do lugar político e social
que o crítico assumia à época, quando exige dos intelectuais
posicionamento em face do tempo em que viviam. A retórica utilizada por
Benjamin remete a Brecht, sem dúvida, com quem dialogava no período, mas
refletia igualmente o modo de expressão do contexto literário francês.
Ele participa, em junho de 1935, do \emph{Congrès International des
écrivains pour la défense de la culture}, que tem por presidente e
personagem central André Gide, condutor do debate ao redor do qual
redemoinham polêmicas. Nesse lugar, e momento, ele é lançado no ``olho do
furacão''.\footnote{Cf. T. Conner. \emph{André Gide. Rebellion and
  Ambivalence}. New York: Palgrave, 2000.}

No período que antecede o exílio de Benjamin, há o ambiente de
intercâmbio cultural, literário e crítico na Alemanha que já estabelece
o diálogo com os escritores franceses, de forma a tematizar a relação
existente, seja na divulgação de obras, seja no trabalho de tradução
dessas. Da entrevista com Gide realizada em 1929, vertida em dois textos
diferentes,\footnote{Da entrevista com Gide realizada por Benjamin em
  1928 em Berlim resultaram dois artigos aqui traduzidos --- ``André Gide
  und Deutschland. Gespräch mit dem Dichter'' e ``Gespräch mit André
  Gide'' --- publicados respectivamente no \emph{Deutsche Allgemeine
  Zeitung}, 29/1/1928 e na revista \emph{Die literarische Welt},
  17/2/1928. Os temas principais dessa entrevista relacionam"-se: ao
  intercâmbio cultural e político entre França e Alemanha contra o
  conceito nacionalista de cultura; à visita de Gide a Berlim; ao
  interesse de Gide pela filosofia alemã; às traduções de Gide para o
  inglês e para o alemão; a respeito de Proust.} ao ``Diário de %[Paulo] Esse Diário de Paris é outra coisa que o Diário parisiense? Não ficou muito claro para mim...
Paris'', do final de 1929 a junho de 1930, o interesse é marcado pela
preocupação com a recepção dos escritores, com as traduções e com uma
ideia de trânsito e de intercâmbio cultural. Ao mesmo tempo em que lida
com mediações, há o tema premente de abertura de fronteiras culturais
entre os dois países. Embora André Gide, com quem se encontra
pessoalmente, seja muito citado, Marcel Proust torna"-se uma presença
recorrente por meio de testemunhos dos que com ele conviveram, sendo
constantemente evocado. Por isso, o ensaio sobre Proust não é a única
referência a esse escritor cuja fama póstuma estava, nesse momento,
sendo construída pela crítica literária. Na nova tradução do ensaio
``Imagem de Proust'' que aqui oferecemos, incluímos material adicional
com as anotações de Benjamin a esse texto, nas quais se nota a presença
do escritor francês nas conversas e nas lembranças. Esse material possui
o mérito de indicar quantos outros ensaios latentes ali se encontram.
Como Proust, Paul Valèry sempre emerge nos escritos de Benjamin, seja
por meio de epígrafes, de citações ou nesses poucos textos escritos ora
traduzidos.

O interesse demonstrado por Gide, em comparação, parece ser mais
extenso, tendo em vista um número maior de escritos, mas pode"-se dizer
que a importância de Gide como escritor deve ser medida ao lado de
Proust e Valèry. No início da resenha ``Três franceses'', Benjamin, ao
escrever sobre o livro \emph{Les Documentaires}, de Paul Souday, dispõe
Gide ao lado de Proust e Valèry, a fim de compor ``o triângulo
equilátero da nova literatura francesa''. Ao traçar a posição do
intelectual na França de forma negativa, ele utiliza o provérbio ``para
toda regra há sempre uma exceção'', citando Proust e Gide como tais
exceções. Os dois, a seu ver, teriam ``modificado'' decisivamente a
técnica do romance. Coube à posteridade eleger Proust como o mais
reconhecido entre os três. Gide, um tanto esquecido na
contemporaneidade, ressurgiu associado às questões de gênero e a
questões coloniais, dois temas que não passaram despercebidos a
Benjamin. Em relação à questão de gênero, a seu ver, o público francês
não possuía interesse algum no debate sobre questões sexuais ou pela
polêmica que envolvia Oscar Wilde, de quem Gide era muito próximo. Por
outro lado, antevendo uma crítica de muitos autores atuais ao
\emph{Corydon}, Benjamin considera a tentativa de Gide de estabelecer a
homossexualidade como um ``puro fenômeno natural'' como inferior ao
caráter sociológico que essa assume em Proust.

O papel de crítico e interlocutor entre as culturas de Benjamin é
evidente neste período, tanto em relação à entrevista com Gide, às suas
próprias traduções de Proust e de Baudelaire, quanto à sua leitura
atenta não só de escritores, mas de críticos literários franceses. O
primeiro texto dessa coletânea traz o comentário de Benjamin sobre o
livro do crítico Paul Souday, que reúne três volumes sob o título
\emph{Les Documentaires}, sendo o primeiro dedicado a Marcel Proust; o
segundo, a André Gide; e o terceiro, a Paul Valèry, publicados em Paris,
pela editora Simon Kra, em 1927. Além de Souday, destacam"-se, neste
contexto, particularmente dois críticos literários com os quais Benjamin
estabelece um contato mais próximo: Léon Pierre"-Quint e Ramón Fernandez,\footnote{Algumas das observações seguintes remetem e podem repetir
  informações presentes nos anexos do livro de minha autoria,
  intitulado: \emph{Sobre o declínio da ``sinceridade''. Filosofia e
  Autobiografia de Jean"-Jacques Rousseau a Walter Benjamin.} São Paulo:
  Editora Loyola, 2006.} dois críticos que se ocuparam também em
escrever sobre Proust e Gide ao público francês, num momento em que a
fama póstuma, como já ressaltamos, ainda não os distinguia.
Particularmente, é com Pierre"-Quint que Benjamin dialoga também sobre os
insurgentes surrealistas, sobre os quais, aliás, não publicamos nenhum
texto específico nessa coletânea, mas que se encontram citados em
passagens como descendentes da estirpe gideana.

Pierre"-Quint é citado por Benjamin como o primeiro intérprete de Proust,
ao ressaltar sua percepção do humor na obra de Proust. No ``Diário
parisiense'' há três referências a Pierre"-Quint, sendo a primeira delas
(6/1/1930) uma citação sobre Léon"-Paul Fargue. Benjamin considera"-o como
``o grande poeta lírico da França'' e uma espécie de testemunha viva, de
quem pôde ouvir pessoalmente histórias de sua amizade de mais de vinte
anos com Proust. A segunda referência, de 11 de janeiro de 1930, está
diretamente associada a Gide. Pierre"-Quint fala sobre seu plano de
escrever um livro sobre Gide, publicado três anos mais tarde. Essa obra
em particular revela ligações com algumas observações de Benjamin sobre
Gide. Em 15/1/1933, Benjamin escreve uma carta a Pierre"-Quint,
agradecendo"-lhe o envio do livro sobre Gide, a dedicatória e
prometendo"-lhe entrar em contato com a editora \emph{Deutsche Verlag
Anstalt}, de Stuttgart, para publicação da obra. Não só faria a
indicação, como também se ocuparia da tradução do livro, cuja intenção
seria a de facilitar a negociação com a editora, responsável pela
publicação das obras de Gide na Alemanha. Concorda com Pierre"-Quint
sobre a importância de Gide na Alemanha,\footnote{Cf. A respeito da
  importância de Gide na Alemanha, a extensa bibliografia coletada e
  publicada por George Pistorius em \emph{André Gide und Deutschland}.
  Heidelberg: Carl Winter, Universitätverlag, 1990.} e, portanto, sobre
a pertinência da publicação do livro de Pierre"-Quint na versão alemã.
Benjamin, no entanto, não chegou de traduzir a obra. A terceira %chegou de traduzir??
referência no ``Diário parisiense'', de 11/2/1930, revela Pierre"-Quint
como o diretor da editora \emph{Simon Kra} (mais tarde conhecida como
\emph{Éditions du Sagittaire}), responsável pela publicação do ``Segundo
Manifesto Surrealista''. Pierre"-Quint entrega a Benjamin um exemplar do
Manifesto e o registro dessa conversa ressalta ainda a força e a
peculiaridade do movimento surrealista na França, apesar das
discordâncias com as ideias e direção do movimento por Breton. No
comentário, Benjamin evidencia a predileção pela produção literária
francesa contra a alemã. A associação de Gide com o Surrealismo, que
Benjamin estabelece em ``Vocação de Gide'', tornando Gide uma espécie de
``tio'' dos surrealistas, ocorre por intermédio da personagem
Lafcadio, da obra \emph{Os subterrâneos do Vaticano}, que
inaugura o gesto do ``ato gratuito''. Notemos que associação parecida é
feita por Pierre"-Quint, no anexo de seu livro sobre Gide, intitulado
``André Gide, ou l'Oncle Dada'',\footnote{L. Pierre"-Quint, \emph{André
  Gide. Sa vie -- son oeuvre}. Paris: Librarie Stock, 1952.} no qual
cita trechos de Breton e Aragon a respeito de Gide, e trechos da obra
\emph{Os subterrâneos do Vaticano}, com enfoque na personagem
Lafcadio e um diálogo de Breton com Gide, que atestaria a afinidade
entre eles.

Paris aos poucos vai se transformando, de um lugar cheio de referências
literárias vivas ou lembradas, em um ambiente de combate. É neste
contexto que Ramón Fernandez torna"-se uma presença em textos com tintas
políticas mais fortes. Ramón Fernandez é, como Pierre"-Quint, autor de
estudos sobre Proust e Gide.\footnote{L. Pierre"-Quint, \emph{Marcel %Nota de rodapé confusa, repetindo a citação ao livro de Pierre-Quint
  Proust. Sa vie, son oeuvre}. Paris, Éditions du Sagitaire;
  \emph{Comment travaillait Marcel Proust}, Éditions des Cahiers Libres.
  Épuisé, 1928. \emph{André Gide. Sa vie, son oeuvre}. Paris: Éditions
  Stock, 1933. R. Fernandez, \emph{Gide}. Paris: Corrêa, 1931.
  \emph{Proust}. Paris: Éditions de la Nouvelle Revue Critique, 1942.
  Réédition: \emph{Proust ou la généalogie du roman moderne}. Grasset,
  1979.} O estudo por ele realizado sobre Proust é conhecido por
Benjamin e, como dissemos, Fernandez é citado em seu ensaio ``Imagem de
Proust''. Benjamin diz que, ``com razão'', Fernandez distinguiu em
Proust um ``tema da eternidade'' (\emph{thème de l'éternité}) de um
``tema do tempo'' (\emph{thème du temps}), acrescentando não se tratar,
contudo, de uma eternidade platônica ou utópica. A associação mais
direta entre os dois dá"-se em torno da questão política na qual Gide é
novamente muito importante. Conhecido como ``homem de esquerda'' e amigo
de Gide, Fernandez foi escolhido para coordenar a mesa de debate com
escritores cujo objetivo era arguir Gide em sua ``conversão'' ao
comunismo. Esse debate ocorreu na associação católica ``Union pour la
Vérité'' em 1935, e resultou na publicação da ``Carta parisiense \versal{I} --
André Gide e seu novo adversário''.

A imagem de Fernandez, segundo Chryssoula Kambas,\footnote{C. Kambas.
  \emph{Walter Benjamin im Exil. Zum Verhältnis von Literaturpolitik und
  Ästhetik}. Tübingen: Max Niemeyer Verlag, 1983, p.~21-2.} de homem de
esquerda é reforçada. Seu ``lugar'' é de um estrangeiro que escreve em
francês. Mas é como colaborador de longa data da \emph{Nouvelle Revue
Française} que Kambas ressalta seu engajamento, citando a ``Carta aberta
a André Gide'', por ele escrita, que teria causado grande mal"-estar aos
assinantes ``burgueses'' da revista. Henri Peyre\footnote{Henry Peyre.
  \emph{Literature and Sincerity}. New Haven/London: Yale University
  Press, 1963, p.~249-250.} considera Fernandez, ao lado de Thibaudet, Du
Bos e Jacques Rivière, como os quatro verdadeiramente significativos
críticos franceses no período entre a primeira e segunda guerras
mundiais. Kambas ressalta a luta contra fascismo como principal ideia do
texto de Fernandez, a ``Carta aberta a André Gide'', e considera"-a a
maior proximidade de Benjamin com esse autor. Muitos dos conceitos de
Fernandez seriam, segundo a autora, repetidos por Benjamin em seu texto
``O autor como produtor'', ensaio que é justamente iniciado com uma
epígrafe do crítico.\footnote{``Il s'agit de gagner les intellectuels à
  la classe ouvrière, en leur faisant prendre conscience de l'identité
  de leurs démarches spirituelles et de leurs conditions de producteur''
  (Trata"-se de ganhar os intelectuais para a classe trabalhadora,
  conscientizando"-os sobre a identidade de seus caminhos espirituais e
  suas condições como produtores).} Outro autor interessado no debate
político que envolvia Fernandez e Gide, Michael Lucey,\footnote{M.
  Lucey. \emph{Gides Bent. Sexuality, Politics, Writings}. Oxford:
  Oxford University Press, 1995.} limita a militância de Fernandez, ao
dizer com relação à atuação deste no debate por ele dirigido na ``Union
pour la Vérité'', que: ``\ldots{} Ramon Fernandez, um crítico da Nouvelle
Revue Française, amigo de Gide e (por um breve período incluindo esta
noite em particular) um defensor da União Soviética''. Para este
intérprete, a simpatia de Fernandez com ``Gide parece consistir em seu
esforço para suprimir sua homossexualidade, uma supressão que só faz com
que a sexualidade e o desconforto que isso provoca sejam mais
evidentes''.\footnote{No texto original: ```friendliness' to Gide seems
  to consist in his effort to suppress Gide's homosexuality, a
  suppression that only makes that sexuality and the discomfort it
  provokes more evident'', p.~196.}

Cabe aqui observar que o ``lugar do crítico'' no contexto francês do
período não é exatamente amigável. Fernandez, em comparação com
Benjamin, é reconhecido na cena literária e escreve em francês.
Benjamin, apesar do conhecimento da língua francesa, de ser tradutor de
Baudelaire e Proust, escreve raramente em francês. Seu ensaio ``A obra
de arte na era de sua reprodutibilidade técnica'' foi traduzido por
Pierre Klossowski em 1936. O primeiro ensaio que Benjamin escreve e
publica em francês é sobre Johann Jakob Bachofen iniciado em 1934, já na
condição de exilado da Alemanha nazista. Segundo comenta a tradutora
Elisabetta Villari:\footnote{Elisabetta Vilari. ``Introduzione'', p.~13.
  In: \emph{Walter Benjamin. Il viaggiatore solitário e il flâneur.
  Saggio su Bachofen.} Genova, Il nuovo melangolo, 1998. Tradução nossa.}

%[Paulo] Não faz sentido manter a mesma referência na nota de rodapé e como citação direta na quote, né? Escolher uma.

\begin{quote}
O momento é difícil tanto por causa de problemas econômicos quanto
porque, a partir de 1934, os ``tempos já parecem escurecer'': as
condições dos emigrantes alemães em Paris começam a piorar. Por ora,
demasiadamente numerosos na França, começam a ser aceitos: sujeitos a
controles cada vez mais escrupulosos, obrigados pela polícia a
apresentar todos os anos um \emph{curriculum} elaborado em francês, com
uma descrição detalhada dos deslocamentos de residências, atividades
realizadas e uma série de referências. Estes documentos testemunham as
contínuas mudanças de domicílio e as dificuldades materiais que Benjamin
se vê forçado a enfrentar em seu estilo (\versal{VILLARI}, 1998, p.~13).
\end{quote}

Nessas circunstâncias, seu porto seguro é a Biblioteca Nacional, que lhe
permite acesso ao acervo e mantém sua atividade intelectual de pesquisa.
O ensaio sobre Bachofen ``se constitui como um vasto panorama e se torna
o pretexto para traçar uma história real da cultura alemã através da
análise desta figura singular, ainda desconhecida na França,
interpretada nesta perspectiva como um ponto nodal do pensamento
alemão''.\footnote{Idem, p.~13-14. Segundo Vilari, ``o título original,
  relatado no J.J. Bachofen: um mestre da `Allemagne inconnue', utiliza
  precisamente o termo favorito do círculo de Stefan George, fortemente
  influenciado por Bachofen, como explica Benjamin em seu ensaio''.}
Nesse sentido, é possível perceber, na contramão, o lugar de um mediador
crítico que, ao apresentar um autor alemão como Bachofen, na língua do
país que o acolhe, reconstitui a discussão empreendida em sua tese de
doutorado, sob as bases da crítica ao círculo de Stefan Georg, à
tradição que elege o ``símbolo'' como central para a formatação
imagética da cultura alemã, e a apropriação banalizada dessa tradição
pelo nazismo.\footnote{Em particular pelo filósofo nazista Alfred
  Bäumler, que publicou \emph{Bachofen und Nietzsche} em 1929,
  \emph{Nietzsche der Philosoph und Politiker} (1931), \emph{Aesthetik}
  (1934), \emph{Studien zur deutschen Geistesgeschichte} (1937).}

A politização, portanto, é o tema que lateja nesses textos, reunidos às
críticas previamente efetivadas, seja na crítica literária composta com
elementos da tradição da cultura alemã, seja no contexto vivo e ativo
que se impõe nesses anos de exílio. No bojo desse exercício crítico, o
tema da homossexualidade é igualmente politizado de maneira, a bem
dizer, inusitada, o que não escapa à observação de Benjamin. Os ataques
dirigidos a Gide por Henri Massis mostram de maneira implícita e, ao
mesmo tempo evidente, o imbricamento entre sexualidade e a adesão de
Gide ao partido comunista ou, em geral, a relação entre sexualidade e
política em Gide. Benjamin, na ``Carta parisiense \versal{I} - André Gide e seu
novo adversário'', traça uma pequena história das desavenças que
conduziram ao debate na ``Union pour la Vérité'' e se refere ao papel
central dessa relação. Segundo afirma, após a publicação de
\emph{Corydon}, em 1920, no qual Gide defende a pederastia como um
fenômeno natural, causando uma tempestuosa reação de seus
contemporâneos, tornou"-se um hábito para ele ir contra a maioria. É o
que novamente ele teria feito ao publicar, em 1931, o primeiro volume de
seu \emph{Diário}, no qual descreve seu ``caminho para o comunismo'', o
que teria, novamente, causado uma espessa polêmica. Benjamin, no
entanto, apesar de conhecer o \emph{Diário de volta da \versal{U.R.S.S}}, não
chegou a comentar esse novo redemoinho que torna Gide distanciado e
malquisto pelos comunistas franceses.

François Mauriac publicou três artigos contrários na revista ``Echo de
Paris''. Os ataques constantes fazem com que Gide se disponha ao debate
público. Benjamin nomeia o debate na ``Union pour la Vérité'' como o
auge desse processo. Ele não menciona Fernandez ou o rol de escritores
convidados, mas refere"-se principalmente a Thierry Maulnier. Seu texto é
uma defesa explícita de Gide --- ao mesmo tempo que defende seu
engajamento político, formula uma acusação ao que chama de
``posicionamento fascista'' de Maulnier. A publicação desse artigo em
1936 (artigo que compõe com ``O autor como produtor'' e o ensaio sobre a
obra de arte, o tema da arte associada à luta contra o fascismo, a
``politização da arte'' contra a ``estetização da política'') sofreu,
como já ressaltado, uma defasagem de tempo com relação à mudança de
posicionamento político de Gide. Ou seja, Gide já havia rompido com o
partido comunista quando de volta de sua visita à \versal{U.R.S.S.}, passa a
discordar do encaminhamento do comunismo via estalinismo e publica duas
obras: \emph{Retour de l'\versal{U.R.S.S.}} de 1936 e \emph{Retouches à mon
Retour de l'\versal{U.R.S.S.}} em 1937, ambos publicados pela editora Gallimard.

Fernandez não é citado por Benjamin em cartas desse período. Não parece,
portanto, haver uma proximidade pessoal tão clara quanto a que Benjamin
manteve com Pierre"-Quint. Poderíamos até afirmar que sua relação com
Pierre"-Quint era de ordem mais especulativa e investigativa, ao passo
que Fernandez traria o lado circunstancial do debate político e
literário desse período de exílio na França. O tema da ``luta contra o
fascismo'' e o tom engajado dos textos do período atestariam a
circunstancialidade de algumas ideias desenvolvidas por Benjamin nesse
período. Nesse sentido, a importância de Fernandez é também de fundo
conceitual, bem como comenta Kambas a respeito do papel do intelectual
na luta de classes pensado por Fernandez (no texto já citado ``Carta
aberta a André Gide''), ideia acolhida por Benjamin e que inspiraria seu
texto ``O autor como produtor''. Por outro lado, a amizade entre
Benjamin e Pierre"-Quint não se limitaria ao interesse puramente
conceitual, já que Pierre"-Quint, como atestam as cartas, teria procurado
inserir Benjamin no debate literário francês e auxiliá"-lo na publicação
de alguns artigos. Reciprocidade demonstrada por Benjamin ao sugerir a
publicação do livro de Pierre"-Quint sobre Gide na Alemanha. O que é
necessário sublinhar no interesse de Benjamin por Gide é,
principalmente, a importância desse escritor no debate político do
período, ocorrendo de diversas maneiras, seja como presidente do
Congresso do Escritores em 1935,\footnote{Cf. Thomas Conner.
  Introduction. In: \versal{CONNER}, T. (Org.). \emph{André Gide's Politics.
  Rebellion and Ambivalence}. New York: Palgrave, 2000. p.~1-11.} seja
como ativo oponente ao nacionalismo de Barrès, além da relação de
proximidade e distanciamento em relação ao comunismo ortodoxo. Gide foi
inegavelmente --- além de polemizador --- um escritor muito ativo no
contexto político e, mesmo após sua morte, continuou a receber elogios
ao lado de críticas mordazes. O artigo de Sartre é um exemplo notável
nesse sentido. O título --- ``Gide vivo'' --- marca a oposição ao jornal
comunista \emph{L'Humanité}, o qual, por ocasião da morte de Gide em
1951, publica o seguinte comentário: ``C'est un cadavre qui vient de
mourir''.\footnote{J.P. Sartre, ``Gide vivo''. In: \emph{Situações, \versal{IV}}.
  \emph{Situations, \versal{IV}}. Paris, \versal{N.R.F.}/Gallimard, 1964, p.~75-79.}

Essas críticas transitam do pessoal à obra, às vezes, a própria obra
serve como fundamentação para um ataque pessoal. O que se pode afirmar
sobre esse debate ``apaixonado'' em torno do escritor é que Benjamin,
sem dúvida, é um de seus defensores. A análise dos artigos que marcam e
demarcam esse interesse parecem manter o tom polêmico que circunda a
figura do escritor. Seria Benjamin um jovem admirador da obra e figura
de Gide? Ou um já experiente intelectual --- embora bem mais novo do que
Gide --- que procurava apenas manter uma relação de intercâmbio político
no terreno literário entre Alemanha e França?\footnote{Essa entrevista
  foi motivo dos dois artigos que primeiro comentaram a relação
  Benjamin"-Gide. O primeiro artigo publicado é de um autor francês,
  Claude Foucart, ``André Gide dialogue avec la nouvelle génération
  allemande: la recontre avec Walter Benjamin en 1928'', \versal{BAAG}, vol. \versal{VII},
  n. 44, octobre, 1979. Foucart é especialista em Gide e na relação
  deste com a cultura alemã e seus expoentes. Possui em torno de vinte
  artigos publicados no \versal{BAAG} a respeito das relações Gide"-Alemanha e o
  livro \emph{André Gide et l'Allemagne. Recherche de la
  complementarite} (1889-1932). Bonn, Romantischer Verlag, 1997. O
  segundo artigo é da autora alemã Chryssoula Kambas, especialista em
  Benjamin, com particular atenção aos ensaios deste, escritos no
  período de seu exílio na França, bem no tocante às relações pessoais e
  políticas travadas por Benjamin durante o mesmo período (Cf. ```Indem
  wir von uns scheiden, erblicken wir uns selbst'. André Gide, Walter
  Benjamin und der deutsch"-französische Dialog''. In: L. Jäger/T.
  Regehly (org.), \emph{Was nie geschrieben wurde, lesen}. Bielefeld:
  Aisthesis Verlag, 1992, p.~132-156; \emph{Walter Benjamin im Exil. Zum
  Verhältnis von Literaturpolitik und Ästhetik}. Tübingen: Max Niemeyer
  Verlag, 1983; verbete de \emph{Walter Benjamins Handbuch}, p.~420-436.
  Ambos artigos tratam do tema do intercâmbio cultural e político
  franco"-germânico. Citamos ainda um terceiro intérprete, Michael Lucey,
  \emph{Gide's Bent. Sexuality, Politics, Writing.} Oxford, Oxford
  University Press, 1995.} São questões levantadas por poucos
intérpretes que pesquisaram o período e o interesse de Benjamin em se
tornar esse crítico exemplar, um mediador entre culturas. A mediação
passa por diversos meios, seja pela escrita de resenhas sobre livros, a
conversação com Gide, seja pela escrita do ensaio sobre Proust, cujo
valor conceitual é de fundamental importância para seu pensamento, ou
ainda, na escrita de cartas, do diário e de programas de rádio. Essa
proliferação de gêneros de escrita é quase sempre sujeita a uma
especulação sobre os gêneros e a transgressão de sua escrita ou
composição. Se Proust o motiva a dizer que sua obra é inclassificável, e
que ``toda grande obra inaugura seu próprio gênero'', vemos em sua
atividade crítica, e em seus experimentos midiáticos, uma perene audácia
ao transformar e reinventar gêneros convencionais de escrita, de
entrevista e ao criar experimentos midiáticos. Entre esses últimos, os
programas de rádio compõem uma gama à parte de especial interesse, sendo
``A vocação de Gide'', ``Paul Valéry -- Em seu sexagésimo aniversário''
e ``Talentos parisienses'', exemplos desse tipo de produção midiática. A
Carta parisiense \versal{II}, ainda nesse âmbito, pouco estudada em comparação
com o ensaio sobre a obra de arte e sua reprodutibilidade técnica, que
marca a recepção de Benjamin no Brasil e conta com diversas traduções de
suas versões, pode surpreender o leitor, tanto em seu teor político,
quanto em seu posicionamento favorável à pintura na chave da
``politização da arte'', como podemos ler ao final: ``Eles sabem o que é
hoje útil em uma imagem: cada marca secreta ou aberta que mostre que o
fascismo encontrou no homem barreiras tão intransponíveis quanto no
globo terrestre''.

Os textos dessa coletânea apresentam uma estreita e tensa relação entre
política e estética, de forma a ampliar e oferecer chaves de compreensão
das senhas finais de seu ensaio de referência sobre a reprodutibilidade
da obra de arte: a ``politização da arte'' contra a ``estetização da
política''. Mas, não apenas: a crítica proustiana às ``pretensões da
burguesia'' em camuflar sua base material, a exposição da elite como, em
suas palavras, ``um clã de criminosos, uma gangue de conspiradores, com
a qual nenhum outro pode se comparar: a camorra dos consumidores'',
torna a ``análise de Proust do esnobismo'' algo ``muito mais importante
do que sua apoteose da arte, representa o auge de sua crítica social''.
Essas passagens são pouco lembradas em função dos profundos conceitos
que se encontram presentes no ensaio sobre Proust, um esquecimento que
diminui a importância da crítica política que Benjamin supõe existir em
sua obra. O humor, cuja função é parte da crítica proustiana, jamais é
desprezado por Benjamin, muito ao contrário. Nesse caso, o pequeno
comentário que é feito de um livro de ilustrações, ``O comerciante no
poeta'', guarda a mesma característica do humor, ao mesmo tempo que
desfaz, em sua interpretação, as caricaturas desenhadas por Pierre Mac
Orlan.

Por fim, Benjamin leitor assíduo de Proust, Valèry e Gide, encontra seu
lugar de crítico legítimo da literatura alemã e francesa, nesse ambiente
de largos conflitos, em condição de trânsito necessário, ou melhor dito,
de fuga. Destino trágico daquele que detém a palavra e pode ser
satírico. As últimas palavras da resenha ``Édipo ou o mito racional''
compõem com outras parábolas gideanas um quadro positivo de fuga, ao
confluir o referido drama com \emph{O filho pródigo} e \emph{Frutos da
terra}:

\begin{quote}
Um frequentador assíduo da \emph{Rotonde} não poderia ter"-se expressado
de forma mais desinibida a respeito da pergunta. É como se diante dele,
nas inextrincáveis relações de sua casa, todas as misérias domésticas da
pequena burguesia (aumentadas enormemente) fossem encontradas. Édipo
vira"-lhes as costas para seguir os rastros dos emancipados que tomaram a
dianteira: o irmão mais novo do \emph{Filho pródigo} e o andarilho de
\emph{Frutos da terra}. Édipo é o mais velho dos grandes que partem, que
receberam o aceno daquele que escreveu: ``é preciso sempre partir, não
importa de onde''.
\end{quote}

\asterisc

As traduções dos textos dessa coletânea, em parte inéditos em língua
portuguesa, seguem o caráter experimental de perto, buscando fidelidade
como critério, uma proximidade quase literal ao texto. Um trabalho de
equipe, que contou diretamente, desde a proposta, com Amon Pinho, com a
parceria de Pedro Hussak na concepção e organização da coletânea e com a
dedicada revisão de Francisco de Ambrosis Pinheiro Machado. No percurso
da tarefa da tradução, devo agradecer também ao professor e amigo Peter
Reinacher e aos alunos"-orientandos Mariana Andrade Santos, Fernando
Ferreira da Silva e Gilmário Guerreiro da Costa, cuja paciência em ler
em conjunto os esboços de tradução, auxiliaram a refletir sobre os
textos em vários aspectos e a submetê"-los a um constante trabalho de
revisão.
