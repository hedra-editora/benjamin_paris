\chapter{Édipo ou o mito racional\footnote[*]{``Oedipus oder Der vernünftige Mythos'', in \versal{GS II}, p. 391--395. Tradução de
  Pedro Hussak e Carla Milani Damião. Ensaio escrito
  por Benjamin em abril ou maio de 1932 para o programa de encenação da
  peça ``\OE dipe'', de Gide, em Darmstadt, sob a direção de Gustav Hartung,
  em junho do mesmo ano (cf. \versal{GS II}, p. 1147--1148). Publicado em
  \emph{Blätter des hessischen Landtheaters}, em abril de 1932. [\versal{N.~O.}]}}

Deve ter sido logo após a guerra que se ouviu falar sobre o experimento
teatral inglês ``Hamlet de casaca''.\footnote{``Hamlet im Frak'':
  encenada pela primeira vez em Viena, em 1926. \versal{[N.~O.]}} Naquela época,
debateu"-se muito sobre essa experimentação. Talvez aqui fosse suficiente
notar o paradoxo de que a peça seja excessivamente moderna para ser
modernizada. Inquestionavelmente, houve épocas em que se podia
empreender coisas semelhantes sem com isso ter objetivos conscientes em mente.
Sabe"-se que nas peças de Mistérios da Idade Média,\footnote{As peças de Mistério eram um tipo de drama muito comuns na Idade Média, ao lado das peças de milagres e moralidades. As peças normalmente eram baseadas em temas bíblicos. [\versal{N.~O.}]} assim como nas
pinturas de seu tempo, as personagens entravam em cena com figurinos de
época. Mas é certo que o mesmo procedimento, hoje, deva provir da mais
precisa reflexão artística para ser mais do que uma brincadeira esnobe.
Na verdade, pôde"-se agora acompanhar como nos últimos anos grandes
artistas --- ou ao menos reflexivos --- aplicaram tais ``modernizações''
tão bem na poesia quanto na música e na pintura. À tendência
representada pelas pinturas de Picasso por volta de 1927, pelo
\emph{Oedipus Rex} de Stravinsky e por Cocteau com \emph{Orfeu} deu"-se o
nome de neoclassicismo. Porém, não colocamos esse nome aqui para
associar Gide a essa tendência (contra a qual ele teria com razão
protestado), mas sim para indicar como os mais diferentes artistas
chegaram a, justamente em relação aos gregos, despi"-los\footnote{No original em alemão: \emph{Entkleidung}. [\versal{N.~T.}]} das suas roupas tradicionais, ou, se quisermos, transvesti"-los\footnote{No original em alemão: \emph{Verkleidung}. [\versal{N.~T.}]} no sentido de colocar neles roupas atuais.
Primeiramente, podiam iludir"-se sobre a vantagem de
obter para suas experimentações objetos conhecidos para a sua audiência, porém distantes da
esfera temática atual. Pois, trata"-se expressamente, em todos esses
casos, de experimentações de caráter construtivo, em certa medida, de
obras de estúdio. Em segundo lugar, nada poderia
interessar mais ao propósito construtivista do que concorrer com as obras
dos gregos,\footnote{No original em alemão: \emph{Griechentum}. [\versal{N.~T.}]} cuja legitimidade perdura através dos
séculos como cânone do natural e do orgânico. E, em terceiro lugar,
estava em jogo o intento, secreto ou público, de fazer uma genuína prova
histórico"-filosófica da eternidade dos gregos --- quer dizer, da sua
atualidade reiteradamente confirmada. Com essa terceira consideração,
entretanto, o observador já se encontra no cerne da última obra de André
Gide.\footnote{André Gide, \emph{\OE dipe}. Paris: Pléiade, 1931.
  Benjamin se refere à tradução alemã de Ernst Robert Curtius,
  \emph{Oedipus oder Der vernünftige Mythos}, que foi publicada no mesmo
  ano. \versal{[N.~T.]}} Em todo caso, logo ele perceberá que no ambiente deste Édipo a
obra ganha sua peculiaridade. Nela fala"-se do domingo, do recalque, de
Lorena, dos \emph{décadents} e das vestais. O poeta torna
impossível ao seu público agarrar qualquer detalhe do local ou da
situação; ele retira"-lhe até a ilusão, chamando logo nas primeiras
palavras o palco pelo seu nome. Em suma, quem quiser segui"-lo deve
``lançar"-se na água'', pegar as cristas e os vales da onda como vierem
do mar das sagas em movimento há dois mil anos, deixando"-se levar e
deixando"-se cair. Apenas assim sentirá o que esta cultura grega pode
ser para ele, ou ele para ela. E o que isso significa? Que
isso se pode encontrar no próprio Édipo, e, de todas as profundas alterações de significação ou de
encenação que a saga experimenta em Gide, esta é a mais intrigante de
todas: ``Mas eu entendo, sozinho entendi, que a única senha, com a qual
alguém poderia salvar"-se das garras\footnote{As traduções do texto
  de Gide foram feitas por Benjamin, que as ``dramatiza'' neste caso, ao
  dizer ``garras da Esfinge'', no original apenas ``da Esfinge''. \versal{[N.~O.]}} da
Esfinge, chama"-se `homem'. Embora certa coragem fosse necessária para
pronunciar essa palavra, eu já a tinha em prontidão antes que tivesse
escutado o enigma, e minha força reside no fato de que não quis saber de
nenhuma outra resposta, qualquer que fosse a pergunta.''\footnote{Gide, \emph{Oedipus oder Der vernünftige Mythos}, p. 62. \versal{[N.~T.]}}

Édipo conhecia antecipadamente a palavra que quebraria o poder da Esfinge,
assim como Gide também conhecia antecipadamente a palavra que
esvaneceu o horror da tragédia de Sófocles. Há mais de
doze anos foram publicadas suas ``Reflexões sobre a mitologia grega'', e lá
consta: ```Como foi possível acreditar nisso?' exclama Voltaire. E,
contudo, em primeiro lugar, é à razão --- e somente à razão --- que cada mito
volta"-se, e ninguém terá compreendido nada desse mito enquanto a razão primeiramente não
o admitir. A saga grega é fundamentalmente racional e precisamente por
isso pode"-se dizer, sem ser um mau Cristão, que ela é muito
mais fácil de compreender do que o ensinamento de Paulo.''\footnote{```Comment
  a"-t"-on pu croire à cela?', s'écrie Voltaire. Et pourtant chaque
  mythe, c'est à la raison d'abord et seulement qu'il s'adresse, et l'on
  n'a rien compris à ce mythe pendant que ne l'admet pas d'abord la
  raison.'' André Gide. \emph{Incidences}. Paris: \emph{Nouvelle Revue Française},
  1924, p.~62. \versal{[N.~T.]}} Bem entendido: em parte alguma o poeta afirma que a
\emph{ratio} teceu a saga grega, tampouco que o sentido grego do mito
repousou apenas nela. O mais importante antes consiste em saber como o
sentido atual ganha distância do antigo sentido e como essa distância do antigo
significado é apenas uma nova aproximação da própria saga, a partir da
qual o novo sentido oferece"-se inesgotavelmente sempre para novas
descobertas. Eis porque a saga grega é como o cântaro de Filémon, ``que
nenhuma sede esvazia quando se bebe em companhia de Júpiter''.\footnote{A história de Filémon aparece em \emph{Metamorfoses}, de Ovídio. \versal{[N.~O.]}} O
momento certo é também um Júpiter, e, por conseguinte, o neoclassicismo
hoje pode descobrir na saga aquilo que nela ainda não havia sido
encontrado: a construção, a lógica, a razão.

Paremos aqui para nos permitir um contra"-argumento de que, em vez de explicação,
um paradoxo verdadeiramente vertiginoso surgiu. É no lugar onde
ficava o palácio de Édipo --- a casa que, como nenhuma outra, foi rodeada
pela noite e pela escuridão, pelo incesto, parricídio, fatalidade e
culpa --- que se deve erguer hoje o templo da deusa da Razão? Como é
possível? O que aconteceu a Édipo nos vinte e três séculos, desde que
Sófocles colocou"-o primeiramente no palco grego até os dias atuais,
quando Gide apresenta"-o de novo no palco francês? Pouco. E o que
resultou desse pouco? Muito. \emph{Édipo ganhou a fala}. Pois o Édipo de
Sófocles é mudo, ou quase mudo. Cão rastreador de seu próprio rastro,
urrando em consequência dos maus tratos cometidos por suas próprias
mãos, ele não encontra nenhum lugar em sua fala para pensar, para refletir: muito
embora Édipo seja insaciável em pronunciar sempre, repetidamente, o
terrível.

\begin{verse}
Geraste"-me, conúbio, e germinaste,\\
Semeando o mesmo sêmen. Revelaste\\
pais, irmãos, filhos \qb{}--- tribo homossanguínea ---,\\
fêmeas, mulheres"-mães, o quanto houver\\
de mais abominável entre os homens.\\
(versos 1404--1408, p. 105--106)\footnote{Adotamos a tradução
  brasileira feita diretamente do grego por Trajano Vieira da tragédia
  de Sófocles, \emph{Édipo Rei} (São Paulo: Perspectiva, 2001). O
  original em alemão, citado por Benjamin, é uma tradução de Hölderlin:
  ``O Ehe, Ehe!/ Du pflanztest mich. Und da du mich gepflanzt,/
  So sandtest du denselben Saamen aus,/
  Und zeigtest Väter, Brüder, Kinder, ein/
  Verwandtes Blut, und Jungfraun, Weiber, Mütter,/
  Und was nur schändlichstes entstehet unter Menschen!'' \versal{[N.~T.]}}
\end{verse}

Mas é justamente esta fala que faz calar seu interior, e da mesma maneira ele
gostaria de ser semelhante à noite:

\begin{verse}
Impossível! Pudesse pôr no ouvido\\
Lacre auditivo, e eu não hesitaria\\
em isolar meu pobre corpo: surdo,\\
além de cego.\\
(versos 1386--1389, p. 105)\footnote{Na tradução de Hölderlin
  citada por Benjamin: ``Sondern wäre für den Quell,/ Der in dem Ohre
  tönt, ein Schloß, ich hielt es nicht,/ Ich schlösse meinen müheseelgen
  Leib,/ Daß blind ich wär' und taub.'' \versal{[N.~T.]}}
\end{verse}

E como ele poderia não emudecer? Como o pensamento poderia desfazer o
emaranhado que torna completamente impossível de decidir aquilo que o destrói: o
próprio crime, a sentença do oráculo de Apolo ou a maldição que ele
mesmo roga para o assassino de Laio? Aliás, essa mudez não apenas caracteriza
Édipo como também em geral os heróis da tragédia grega. Eis
porque os modernos pesquisadores continuam a deter"-se nela. ``O herói
trágico tem apenas uma linguagem que lhe corresponde perfeitamente: o
silêncio.''\footnote{Franz Rosenzweig. \emph{Der Stern der
  Erlösung}, vol.\,1. Edição de Karl Schlechta. Munique: Hanser, 1954, p. 81. \versal{[N.~O.]}}
Ou citando outro autor: ``Os heróis trágicos falam, de certo modo, mais
superficialmente do que atuam.''\footnote{Friedrich Nietzsche.
  \emph{Die Geburt der Tragödie}, 17. Werke in drei Bänden. Edição de
  Karl Schlechta, vol.~1. Munique: Hanser, 1954, p. 94. \versal{[N.~O.]}} Ou um terceiro: ``na
tragédia o pagão dá"-se conta de que ele é melhor que seus deuses, mas
esse reconhecimento rouba"-lhe a linguagem, ela permanece abafada. Sem
declarar"-se, ela tenta secretamente reunir seu poder\ldots{} não está em
questão que `a ordenação moral~do mundo' seja restaurada, mas sim que o
homem moral queira erguer"-se ainda mudo, ainda na menoridade\footnote{No original em alemão: \emph{unmündig} (literalmente ``sem boca''), que significa
  menor de idade, aquele que ainda não pode exercer diretamente atos da
  vida civil, moral ou juridicamente falar e responder por si. \versal{[N.~T.]}} ---
enquanto tal ele é chamado herói --- no estremecimento daquele mundo
pleno de sofrimentos. O paradoxo do nascimento do gênio na mudez moral,
na infantilidade moral é o sublime da tragédia.''\footnote{Cf.
  Walter Benjamin, ``Schicksal und Charakter'' {[}Destino e Caráter{]},
  publicado em 1921 (\versal{GS II}, p.~175); autocitação que faz também em
  \emph{Ursprung des deutschen Trauerspiels}, de 1928 (\versal{GS I}, p.~288--289),
  na tradução brasileira de Sérgio Paulo Rouanet, \emph{Origem do drama
  barroco alemão}. São Paulo: Brasiliense, 1984, p.~132--133. \versal{[N.~T.]}}

É apenas a partir daqui que se consegue reconhecer a ousadia da
tentativa de dotar o herói da tragédia com a fala. Aqui entra em cena o
que as grandiosas palavras têm a dizer sobre o ``destino'' que o poeta
escreveu no contexto já mencionado muito antes de ele tê"-las realizado
no ``Édipo'': ``com essa palavra repugnante atribui"-se muito mais ao
acaso do que lhe convém, sua trapaça brota em toda parte em que se
renuncia a um esclarecimento. Porém, afirmo que quanto mais se repele o
destino na saga, tanto mais instrutiva ela se torna.''\footnote{Gide, \emph{Incidences}, p. 81. \versal{[N.~T.]}} No final do segundo ato no drama de %manter essa nota assim?
Sófocles (que possui cinco atos ao todo), o papel do vidente Tirésias é
encerrado. Édipo necessitou de dois mil anos para, através de Gide,
enfrentá"-lo no grande debate no qual declara o que em Sófocles nunca
teria arriscado pensar. ``Este crime Deus impôs a mim. Ele o escondeu em
meu caminho. Ainda antes de meu nascimento, a armadilha estava posta,
sobre a qual deveria tropeçar, pois ou seu oráculo mentia, ou eu não
podia salvar"-me. Eu estava cercado.''\footnote{Gide, \emph{Oedipus oder Der vernünftige Mythos}, p.~81. {[}\versal{N.~T.}{]}}

Graças a tal superioridade involuntária do herói, o drama satírico em
Gide estabelece"-se no próprio lugar, ou ao menos nos arredores, do antigo
horror, como transparece nas palavras de Creonte e de vez em quando
também nas do coro. Essas nunca foram tão superiores do que na lição que
Édipo dá às crianças, cuja conversa ele escutava. Um frequentador
assíduo da \emph{Rotonde}\footnote{La Rotonde era um café de Montparnasse
  frequentado por artistas e intelectuais nos anos 1920 e 1930. \versal{[N.~O.]}} não
poderia ter se expressado de forma mais desinibida a respeito da
pergunta. É como se, diante dele, nas inextricáveis relações de sua
casa, todas as misérias domésticas da pequena"-burguesia (aumentadas
enormemente) fossem encontradas. Édipo vira"-lhes as costas para seguir
os rastros dos emancipados que tomaram a dianteira: o irmão mais novo do
\emph{Filho pródigo} e o andarilho de \emph{Frutos da terra}.\footnote{Dois romances de André Gide, respectivamente: \emph{Le retour de l'enfant
  prodigue}, de 1907; e \emph{Nourritures terrestres}, de 1897. Cf. Nota
  dos editores, \versal{GS II}, p. 1149. \versal{[N.~O.]}} Édipo é o
mais velho dos grandes que partem, que receberam o aceno daquele que
escreveu: ``\emph{Il faut toujours sortir n'importe d'où}''.\footnote{No original em francês: ``é preciso sempre partir, não importa de onde''.
  Benjamin cita aqui imprecisamente, talvez de memória, uma frase do
  prefácio de \emph{Les nourritures terrestres}, que seria: ``Et quand
  tu m'auras lu, jette ce livre --- et sors. Je voudrais qu'il t'eût
  donné le désir de sortir --- sortir de n'importe où''. Cf. Nota dos
  editores, \versal{GS II}, p.~1149. \versal{[N.~T.]}}
