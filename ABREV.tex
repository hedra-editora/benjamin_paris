%!TEX  root=./LIVRO.tex
\chapter{Abreviaturas}

\begin{table}[h]
\centering\small
% \caption{My caption}
% \label{my-label}
\begin{tabular}{rl}
\textit{G.S.} 		& \emph{Gesammelte Schriften} [Escritos Reunidos] \\
\textit{G.B.} 		& \emph{Gesammelte Briefe} [Correspondência Reunida] \\
\textit{N. de W.B.} & Nota de Walter Benjamin \\
\textit{N. E.} 		& \begin{tabular}[c]{@{}l@{}}Nota Editorial 
					  (Amon Pinho 
					  e Francisco Pinheiro Machado)\end{tabular} \\
\textit{N. T.}   & \begin{tabular}[c]{@{}l@{}}Nota do Tradutor 
					  (Carla Milani Damião ou\\ Pedro Hussak van Velthen Ramos) 
					  \end{tabular}
\end{tabular}
\end{table}



% G.S.	Gesammelte Schriften [Escritos Reunidos]
% G.B.	Gesammelte Briefe [Correspondência Reunida]
% N. de W.B. 	Nota de Walter Benjamin
% N. E.	Nota Editorial (Patrícia Lavelle, Amon Pinho e Francisco De\,P.\,Machado)
% N. da T.	Nota do Tradutor (Patricia Lavelle, Georg Otte ou Marcelo Backes)

% , de Walter Benjamin, 7 volumes, 14 tomos, mais suplementos. Edição de Rolf Tiedemann e Hermann Schweppenhäuser. Frankfurt a. M.: Suhrkamp, 1991;

% , de Walter Benjamin, 6 volumes. Edição de Christoph Gödde e Henri Lonitz. Frankfurt a. M.: Suhrkamp, 1995-2000;