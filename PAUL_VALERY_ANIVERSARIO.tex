\chapter{Paul Valéry (1931) - Em seu sexagésimo aniversário\footnote[*]{\emph{G.S.} II (I), p. 386-390. Tradução de Carla Milani Damião.}}

\emph{O langage chargé de sel, e paroles véritablement marines!}

~

Valèry, certa vez, quis tornar-se oficial naval. Os traços desse sonho
juvenil são ainda reconhecíveis naquele que ele se tornou.
Primeiramente, há a sua poesia, a plenitude das formas que a linguagem
extrai do pensamento, como o mar extrai da calmaria; e em segundo lugar,
há esse pensamento, orientado de cima abaixo matematicamente, que se
inclina sobre os fatos como sobre cartas náuticas e, sem agradar-se
vendo ``profundidades", já se dá por feliz por manter um curso sem
perigo. O mar e a matemática: estabelecem uma associação de ideias
encantadora em uma das passagens mais belas que ele escreveu, com o
narrador Sócrates relatando à Fedra a descoberta que fez à beira mar. O
achado é uma formação duvidosa - marfim ou mármore, ou um osso de animal
- que, como uma cabeça com as feições de Apolo, as ondas levaram para
beira do mar. E Sócrates se pergunta se essa seria obra das ondas ou do
artista; ele pondera : quanto tempo o oceano provavelmente precisaria
até que, dentro de bilhões de formas, um acaso gere justamente essa
forma; e quanto tempo precisa o artista, e ele pode bem dizer: "que um
artista vale mais que mil séculos ou cem mil ou muito mais. . . Aí
reside uma medida estranha de avaliação de obras de arte''. Se
devêssemos surpreender, com um \emph{ex-libris,} o sexagésimo
aniversário do autor dessa grande obra \emph{Eupalino ou o arquiteto}:
ele poderia mostrar um compasso enorme, uma perna ancorada firmemente no
fundo do mar e a outra estendida longe até o horizonte. Seria também uma
parábola para a envergadura deste homem. A tensão é a impressão
dominante de sua aparência física, é a expressão de sua cabeça, cujos
olhos baixos indicam um devaneio de imagens terrenas que permitem ao
homem determinar a rota de sua vida interior, quer de acordo com essas
imagens, quer de acordo com as estrelas. A solidão é a noite de onde
irradiam tais imagens, e dela Valéry tem uma longa experiência. Quando,
aos vinte e cinco anos, publicou seus primeiros poemas e os dois
primeiros ensaios, teve início a pausa de vinte anos de sua atuação
pública, a partir da qual ele ressurgiu tão brilhantemente com o poema
"A jovem Parca", em 1917. Oito anos depois, uma série de obras notáveis
​​e manobras engenhosas na sociedade lhe renderam ingresso na Academia
Francesa. Não sem uma fina malícia, foi-lhe designada a cadeira de
Anatole France. Valéry aparou o golpe com um discurso extremamente
elegante - a obrigatória \emph{laudatio} a seu antecessor - em que o
nome France não foi mencionado uma só vez. Além disso, este discurso
contém uma perspectiva sobre a profissão de escritor que é incomum o
suficiente para caracterizar o autor. Fala-se de um "Vale Josafat", no
qual a multidão de escritores, antigos e presentes, se aglomera: "Tudo o
que é novo perde-se em outro novo. Qualquer ilusão de ser genuíno
desaparece. A alma se entristece e, volta-se, em pensamentos, embora com
dor, porém com dor estranha, misturada com profunda compaixão e ironia,
volta-se para aquelas milhões de criaturas que empunham a pena, aqueles
inúmeros agentes do espírito, os quais, cada um a seu momento, se viu
como um criador livre, como a primeira causa motora, como o proprietário
de uma certeza inabalável, como a única fonte inconfundível, e ele que
passou seus dias em labuta e gastou os melhores momentos a fim de
tornar-se um diferenciado para toda eternidade, vê-se agora destruído
pela quantidade e devorado pela multidão crescente de seus iguais. Em
Valéry, o lugar dessa vontade totalmente vã de se diferir, é substituída
por outra - a vontade pela duração, pela duração da palavra escrita. A
duração da escrita, no entanto, é algo bastante diferente da
imortalidade daquele que escreve, a duração existiu em muitos casos sem
a última. É a duração, não a originalidade, que caracteriza o clássico
na literatura, e Valéry não se cansou de perseguir suas condições. "Um
escritor clássico", diz ele, "é um escritor que esconde suas associações
de ideias ou as absorve". Naqueles lugares onde o impulso fez o autor
arriscar-se, onde ele se achou acima do destino, não viu as lacunas, e
porque ele não as viu, não as preencheu - nesses lugares cresce o mofo
do envelhecimento.

É preciso autocrítica para perceber as lacunas e fronteiras do
pensamento. Valéry examina de maneira inquisitória a inteligência do
escritor, sobretudo do poeta, depois, exige uma ruptura com a opinião
generalizada de que ela é óbvia em quem escreve, e mais ainda com outra,
muito mais generalizada, de que ela não tem nada a dizer no poeta. Ele
mesmo tem uma e de uma maneira que não é óbvia de modo algum. Nada pode
ser mais desconcertante do que sua encarnação, o \emph{Senhor Teste}.
Desde o início de sua obra até a mais recente, reiteradamente, ele volta
a essa figura estranha em torno da qual se reuniu todo um círculo de
pequenos escritos: uma noite com Monsieur Teste, uma carta de sua
esposa, um prefácio e, como é conhecido, um diário de bordo. Monsieur
Teste - em alemão: Herr Kopf \footnote{N.T.: Em português: Senhor
  Cabeça''.}- é a personificação do intelecto, que lembra muito do Deus
tratado na teologia negativa de Nicolau de Cusa. Tudo o que podemos
saber de \emph{Teste} resulta em negação. O mais fascinante de sua
apresentação não está, por isso, em teoremas, mas nos truques de um modo
de comportamento que prejudique, o menos possível o não-ser e satisfaça
a máxima: "Cada emoção, cada sentimento é sinal de um erro na construção
e adaptação''. Mesmo que o Senhor Teste se sinta naturalmente humano,
ele levou a sabedoria de Valéry à sério, de que os pensamentos mais
importantes são aqueles que contradizem nossos sentimentos. Ele é por
isso também a negação do "humano": "Veja, o crepúsculo do impreciso está
entrando, e diante da porta está o reino do desumanizado, que vai surgir
da precisão, do rigor e da pureza nos assuntos dos humanos". Nada de
expansivo, patético, "humano", entra no raio dessa esquisita personagem
de Valéry, para a qual o pensamento representa a única substância da
qual o perfeito se deixa formar. Um de seus atributos é a continuidade.
Assim, também, ciência e a arte no espírito puro são um \emph{continuum}
por meio do qual o método de Leonardo (que aparece na primeira obra do
poeta, "Introdução ao Método de Leonardo da Vinci," como um precursor do
Senhor Teste) abre caminhos que não podem, de maneira alguma, ser mal
entendidos como limites. É o método que, em sua aplicação na poesia,
levou Valéry para o famoso conceito de \emph{poésie pure}, que
certamente não foi criado para ser arrastado durante meses por um abade
letrado\footnote{N.E.: Benjamin refere-se ao famoso discurso do Abade
  Henri Brémond na reunião das Cinco Academias francesas, ocorrido em
  1925, em torno da ideia de ``poésie pure'' (In: BREMOND. \emph{La
  poésie purê),} que gerou um largo debate, a querela da ``poésie
  purê'', pois tratava de apresentar Valéry como uma espécie de exemplo
  negativo/construtivo à concepção de poesia pura. Em
  \emph{Éclaircissements}, escrito após o discurso polêmico, Brémond
  justifica a escolha, entre outros argumentos, ao marcar a ambiguidade
  do poeta, da seguinte maneira: ``a perversidade do poeta {[}Valéry{]}
  que se nega; o esplendor da auréola que ele não consegue extinguir''.
  A referência de Benjamin ao abade, contudo, é irônica.} pelas revistas
literárias da França para fazê-lo extorquir sua confissão de identidade
com o conceito de oração. Reiteradamente e com estrondoso sucesso, o
próprio Valéry nomeou as estações individuais na história das teorias
poéticas - as teses de Poe, Baudelaire e Mallarmé -, nas quais o
construtivo e o musical da lírica tentaram delimitar suas habilidades
uma contra a outra, até que ela, em sua própria obra, entenda a si mesma
como interação perfeita de inteligência e voz - em reflexões cujo centro
é formado em suas obras-primas líricas - "Le Cimetière marin", "La jeune
parque", "Le serpent". As ideias de seus poemas elevam-se como ilhas no
mar da voz. É isso que separa essa poesia reflexiva
(\emph{Gedenkenlyrik}) de tudo o que chamamos assim em alemão: em nenhum
lugar a ideia se choca com a "vida" ou a "realidade" nela. O pensamento
não tem que lidar com mais nada além da voz: essa é a quintessência da
poesia pura (\emph{poésie pure}). "A lírica é aquele tipo de literatura
que tem como condição a voz em ação - a voz, como ela diretamente parte
ou como é despertada pelas coisas como vimos ou sentimos em sua
presença". E: "As exigências de uma estrita prosódia são o artifício, em
virtude do qual o discurso natural adquire as características de um
material resistente, estranho a nossa alma e surdo aos nossos desejos".
Esta é precisamente a peculiaridade da inteligência pura. Essa
\emph{intelligence pure}, no entanto, que entrou em alojamentos de
inverno sobre os picos inóspitos de uma poesia esotérica na obra de
Valéry, ainda é a mesma, sob liderança da qual a burguesia europeia, na
era dos descobrimentos, partiu para suas conquistas. A dúvida cartesiana
do conhecimento se aprofundou em Valéry de maneira quase aventureira e
ainda assim metódica para uma dúvida sobre as próprias perguntas: "A
gama de potências de acaso, dos deuses e destino nada mais são do que de
nossas falhas intelectuais. Se nós tivéssemos uma resposta para tudo --
quer dizer, uma resposta exata, então essas potências não existiriam ...
Sentimos isso também, e é por isso que, no final, voltamo-nos contra
nossas próprias perguntas. Isso deveria representar o começo. Devemos
formar em nosso interior uma pergunta que preceda todas as outras e
pergunte a cada uma para que servem''. A ligação retroativa específica
de tais pensamentos ao período heroico da burguesia europeia permite
dominar a surpresa com a qual encontramos novamente aqui, num dos postos
mais avançados do velho humanismo europeu, a ideia do progresso. Ou
seja, é a ideia convincente e autêntica: aquela do transferível nos
métodos que corresponde tão palpavelmente à noção de construção em
Valéry, quanto contraria a obsessão da inspiração. "A obra de arte",
disse um de seus intérpretes, "não é uma criação. É uma construção em
que a análise, o cálculo, o planejamento desempenha o papel principal".
Valéry comprovou a última virtude do processo metódico: levar o
pesquisador para além de si mesmo. Pois, quem é Monsieur Teste, senão o
indivíduo que já pronto para cruzar o limiar do desaparecimento
histórico, de novo, como uma sombra, se apronta para responder ao apelo,
para logo em seguida submergir imediatamente, onde ele não é mais
afetado por ninguém, em uma ordem, cuja aproximação Valéry circunscreve
da seguinte maneira: "Na era de Napoleão, a eletricidade tinha
aproximadamente a mesma importância atribuída ao cristianismo no tempo
de Tibério. Aos poucos, tornou-se óbvio que esta inervação geral do
mundo foi mais importante e mais capaz de mudar a vida futura do que
quaisquer eventos políticos de Ampére até os dias atuais''. O olhar que
ele dirige a este mundo vindouro não é mais aquele do oficial, mas
apenas do marinheiro que conhece a meteorologia, que está sentindo a
grande tempestade se aproximar e reconheceu bem demais as condições
transformadas do curso do mundo - "crescente precisão e exatidão,
crescente potência" - para não saber que diante deles até mesmo "os
pensamentos mais profundos de um Maquiavel ou Richelieu tem hoje apenas
a confiabilidade e o valor das dicas para a bolsa de valores". Assim,
ele fica " de pé lá, o homem no promontório do pensamento, mantendo um
olhar atento, o mais nítido que pode, para os limites das coisas ou da
capacidade de visão".
