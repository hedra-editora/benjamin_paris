Anotações para o ensaio sobre Paul Valèry\footnote{\textsuperscript{*}
  \emph{G.S.} II-3, p. 1145-1146. Tradução de Carla M. Damião. Os
  editores alemães Tiedemann e Schweppenhäuser cometam o material da
  seguinte maneira: ``Uma folha com \emph{brouillons} do ensaio Paul
  Valéry - o único trabalho preliminar que existe no espólio de
  Benjamin'' (p.1145).}

A ideia de inatividade -- em Senhor Teste - é a crítica mais decidida do
mundo de Valéry a si mesmo.

Dificuldade de distinguir a ociosidade humanista da desumana. O
compromisso de Valéry com o esnobismo.

O caráter de fuga de seu pensamento: a matemática e o mar; mundos puros
da forma, alienados do mundo interior da práxis.

Os pensamentos universais de Valéry. Há, de fato, uma área em seus
pensamentos, onde eles rebentam em terra firme, quase, sentiria tentado
a dizer, em terra prometida.

O desolado clichê que os franceses reivindicam, \emph{raison} e
\emph{clarté,} como virtudes nacionais poderiam ganhar um pouco de vida
sob uma observação de Valéry.

Em Valéry espreita sempre um materialismo inescrupuloso, como os
enciclopedistas o dão a conhecer.

\{Uma avaliação válida de Valéry exige que a inteligência do escritor,
em particular do poeta, seja perseguida com fúria inquisitiva; exige que
ela rompa com a crença generalizada de que a inteligência, no caso dos
escritores, é óbvia. Valéry possui uma {[}inteligência{]} que não é
óbvia; a outra é uma variante da falta de inteligência\}.

Existe uma conexão entre a escassez metodológica e a sobriedade do
pensador e escritor Valéry e a crueldade com que o poeta faz do nada o
atributo da perfeição.

Arquitetura e dança - são eles os mais transparentes diante do nada?

Em Valéry, no entanto, a intenção metódica em sua aplicação à poesia
levou apenas à ideia de uma \emph{poésie pure}; mesmo que ele tenha
reconhecido claramente os vínculos precisos entre poesia e ciência,
parece que ele não foi capaz de realizar uma continuidade igualmente
rigorosa: a da poesia e da literatura - portanto, de sua própria práxis
nas mesmas.

A poesia de Valéry - um jogo muito preciso de referência recíproca entre
voz e inteligência, as ideias de seus poemas emergem como ilhas do mar
da voz. Isto é o que distingue essa lírica reflexiva \footnote{N.T.: No
  original, \emph{Gedankenlyrik}.} de tudo o que em alemão denominamos
assim: em nenhum lugar nela a ideia se encontra com a ``vida'' ou com a
realidade. O pensamento não tem a ver com mais ninguém do que com a voz:
isso e nada mais é a quintessência da \emph{poésie pure}. Em outras
palavras: se a determinação da lírica é a \emph{poésie pure}, então esta
tem por sua vez a ver exclusivamente com a \emph{intelligence}
\emph{pure}.

Leonardo e Pascal representam para Valéry o esplendor e a miséria do
homem de pensamento. No \emph{corpus} geral de suas obras, o confronto
apaixonado com o último é ainda mais frequente do que a adesão sem
reservas ao primeiro. A ``Introdução ao método de Leonardo'' coloca os
dois em comparação.

A produção de Valéry é caracterizada pela tarefa, - cada vez mais
difícil de assumir e por fim irresolúvel -, de harmonizar certos
conhecimentos com o uso de privilégios específicos.
