\chapter{Paul Valéry na \emph{École Normale}\footnote[*]{GS IV (1), p. 479-480. Tradução de Carla Milani Damião.}}

Devemos pensar nos internatos da era de Metternich
(\emph{Vormärz}),\footnote{O termo \emph{Vormärz} refere-se ao período
  histórico da Alemanha marcado pelas Revoluções de Março de 1848.
  Período conhecido como Era de Metternich. {[}N. E.{]}} no sul da
Alemanha, para termos uma ideia dos espaços sóbrios da \emph{École
Normale}. Napoleão fundou esse Instituto para uma elite, a fim de lhe
garantir toda liberdade e independência material em seus estudos. Nessa
escola, em 1911, Norbert von Hellingrath, o inesquecível editor de
Hölderlin, falecido ainda jovem, foi professor de alemão, que também
assegurou ali o lugar da língua alemã. Seu bibliotecário recém falecido,
Lucien Herr, tradutor da correspondência entre Goethe e Schiller, foi um
dos melhores especialistas do movimento intelectual alemão. Grande parte
dos cientistas franceses surgiu dessa escola. Pasteur, Taine, Fustel de
Coulanges e muitos outros estão gravados no painel de honra de um
``salão nobre''. A gravura de ouro acima dele é o único adorno do
pequeno, escuro e baixo recinto. Lá, Valéry ocupa a tribuna por meia
hora.

Lento, muito discretamente, ele caminha até ela. Uma vontade
arquitetônica envolveu este corpo, seu gesto está para o bailarino como
o som de seus versos para a música, e a elegância dá a aparição mil
facetas geométricas. Imediatamente uma contradição impressiona e
fascina: por mais brilhante que seja essa face bem cultivada e rigorosa,
o porte espiritualmente pleno da figura que envelhece é ainda capaz de
impactar as pessoas, mesmo que olhar e voz recusam-se-lhe. O olhar é
aguçado como um caçador; mira, porém, ctonicamente desviado, enviesado
para baixo e para dentro. A voz soando com precisão, mas audível apenas
em conjuntos. Ela exige, para ser ouvida, adivinhação, como um texto
para ser compreendido. Nem mesmo a fama, a idade e a sabedoria, ela põe
na balança, ``para influenciar na orientação'' dos 60 ou 70 jovens.
Valéry, a quem algo canônico do ``poeta'', que ainda hoje permanece em
vigor, foi muito tardiamente como que por si mesmo atribuído, nunca o
procurou conquistar através de ``posicionamentos'' em relação aos
assuntos de seu povo, por um gesto de líder. Ele -- que recentemente
tornou-se um dos ``Imortais'' -- não faz isso nem mesmo hoje. E por mais
que ele busque, com precisão, distinguir-se do simbolismo -- o rigor de
Mallarmé, quando não a ousadia, continua vivo nele. Por isso, também o
tom crítico de fundo é tão significativo, rompendo de vez em quando, ao
contar de memória, a grande época do simbolismo.

Há quarenta anos, a grande preocupação de todos eles chamava-se: música.
Literalmente esmagado (``littéralement écrasé'') saía todos os domingos
do Concerto Lamoureux no Champs-Elysées, após ter se entregue às grandes
\emph{Ouverturen} de Wagner. O que podemos criar que se possa equiparar
a elas? Assim ecoava desesperadamente o grande ensaio de Baudelaire
sobre Tannhäuser, para uma geração de poetas mais novos. A música possui
tons, escala e modo: ela consegue construir. O que é, no entanto,
construção na poesia? Quase sempre um simples contornar a estrutura
lógica. Os simbolistas procuram reproduzir linguística e foneticamente a
construção de sinfonias. E depois que para Mallarmé as obras-primas
desse estilo foram bem-sucedidas, ele dá um passo adiante. Recorre à
escrita para competir com a música. Então, um dia, ele apresenta para
Valéry, como o primeiro leitor, o manuscrito do ``coup de dés''. ``Veja
e diga se estou louco!'' (Conhece-se esse livro pela edição póstuma de
1914. Um volume in-quarto de algumas páginas. Aparentemente aleatórias,
com distâncias muito consideráveis, palavras estão distribuídas em
cambiantes tipos de letras sobre as páginas.) Mallarmé, cuja imersão
estrita no meio da construção cristalina de sua literatura certamente
tradicionalista viu a verdadeira imagem do que estava por vir, processou
aqui pela primeira vez (como um poeta puro) o poder gráfico do anúncio
no tipo de letra. Assim, a poesia absoluta no extremo revirou no oposto
aparente, o que ela refuta para os moderantistas, para o pensador só
confirma. Para Valéry, talvez ainda não completamente: ``O dedo
provavelmente pode atravessar a chama, mas não viver nela.''
