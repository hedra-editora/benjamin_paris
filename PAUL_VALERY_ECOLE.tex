\chapter{Paul Valéry na \emph{École Normale}\footnote[*]{``Paul Valéry in der École Normal'', in \versal{GS IV}-1, p. 479--480. Tradução de Carla Milani Damião. Publicado originalmente na revista \emph{Die literarische Welt} em 13 de agosto de 1926. [\versal{N.~O.}]}}
\hedramarkboth{Valéry na École Normale}{}

Devemos pensar nos internatos da Era de Metternich
(\emph{Vormärz}),\footnote{O termo \emph{Vormärz} refere"-se ao período
  histórico da Alemanha marcado pelas Revoluções de Março de 1848.
  Período conhecido como Era de Metternich. [\versal{N.~O.}]} no sul da
Alemanha, para termos uma ideia dos espaços sóbrios da \emph{École
Normale}. Napoleão fundou esse Instituto para uma elite, a fim de lhe
garantir toda liberdade e independência material em seus estudos. Nessa
escola, em 1911, Norbert von Hellingrath, o inesquecível editor de
Hölderlin, falecido ainda jovem, foi professor de alemão, que também
assegurou ali o lugar da língua alemã. Seu bibliotecário recém"-falecido,
Lucien Herr, tradutor da correspondência entre Goethe e Schiller, foi um
dos melhores especialistas do movimento intelectual alemão. Grande parte
dos cientistas franceses surgiu dessa escola. Pasteur, Taine, Fustel de
Coulanges e muitos outros estão gravados no painel de honra de um
``salão nobre''. A gravura de ouro acima dele é o único ornamento neste local
pequeno, escuro e baixo. Lá, Valéry ocupa a cátedra por meia
hora.

Lentamente, ele caminha até ela com discrição. Uma vontade
arquitetônica envolveu este corpo, seu gesto está para o bailarino como
o som de seus versos para a música, e a elegância confere à aparição mil
facetas geométricas. Imediatamente uma contradição impressiona e
fascina: por mais brilhante que seja essa face bem cultivada e austera,
o porte espiritualmente pleno da figura que envelhece é ainda capaz de
impactar as pessoas, mesmo que olhar e voz recusem"-se"-lhe. O olhar é
aguçado como o de um caçador; mira, porém, ctonicamente desviado, enviesado
para baixo e para dentro. A voz ressoando com precisão, deixava"-se ouvir apenas
indiretamente. Para ser ouvida, ela exige adivinhação, como um texto
para ser compreendido. Ele nem mesmo sobrepõe a fama, a idade e a sabedoria,
``para influenciar na orientação'' dos sessenta ou setenta jovens.
Valéry, a quem algo canônico do ``poeta'', que ainda hoje permanece em
vigor, foi muito tardiamente por si mesmo atribuído, ele nunca
procurou conquistar isso através de ``posicionamentos'' em relação aos
assuntos de seu povo, por um gesto de líder. Ele --- que recentemente
tornou"-se um dos ``Imortais'' --- não faz isso nem mesmo hoje. E por mais
que ele busque, precisamente, distinguir"-se do simbolismo, sobrevive nele, senão a audácia, o rigor de
Mallarmé. Por isso, também o
tom crítico de fundo é tão significativo, rompendo de vez em quando, ao
contar de memória, a grande época do simbolismo.

Há quarenta anos, a grande preocupação de todos eles chamava-se: música.
Todos os domingos ia ao Concerto Lamoureux no Champs"-Élysées, e após ter se entregue às 
grandes \emph{Ouvertures} de Wagner, saía literalmente destruído (``\emph{littéralement %verificar Ouvertures
écrasé}''). Será que podemos criar algo que se possa equiparar
a elas? Assim ecoava angustiadamente o grande ensaio de Baudelaire
sobre o \emph{Tannhäuser}, para uma geração de poetas mais novos. A música possui
sons, escala e tons: ela é capaz de construir. O que é, no entanto,
construção na poesia? Quase sempre um simples contornar a estrutura
lógica. Os simbolistas procuram reproduzir linguística e foneticamente a
construção de sinfonias. E Mallarmé, após ter realizado as obras"-primas
desse estilo, dá um passo adiante. Recorre à
escrita para competir com a música. Então, um dia, ele apresenta para
Valéry, como o primeiro leitor, o manuscrito do ``Coup de dés''. ``Veja
e diga se estou louco!'' (Esse livro é conhecido pela edição póstuma de
1914. É um volume de poucas páginas. Aparentemente aleatórias,
com distâncias muito consideráveis, as palavras estão distribuídas em
diferentes tipos de letras sobre as páginas). Mallarmé, cujo mergulho
rigoroso no meio da construção cristalina de sua obra certamente
tradicionalista viu a verdadeira imagem do que estava por vir, processou
ali, pela primeira vez (como um poeta puro), o poder gráfico do anúncio
no tipo de letra. Assim, a poesia absoluta em seu extremo é revertida em seu oposto
aparente, e isso que ela refuta para os modernos, serve apenas para confirmar ao pensador. Para Valéry, talvez ainda não completamente: ``O dedo
pode talvez deslizar pela chama, mas não pode viver nela.''
