\chapter{Cabeças parisienses\footnote[*]{``Pariser Köpfe'', in \versal{GS VII}-1, p.
  279-286. Tradução Pedro Hussak van Velthen Ramos. Palestra radiofônica apresentada na rádio de Frankfurt \textsc{am} --- \emph{Südwestdeutsche Rundfunk} --- em 23 de janeiro de 1930. O texto tem por base as anotações do ``Diário parisiense'', escrito entre 1929 e 1930. {[}\versal{N.~E.}{]}}}

Para alguém que tenha partido em viagem, serão claros os diferentes
graus de estranheza e de familiaridade, proximidade e distância,
resolução e relutância que ele experimenta em relação às cidades. Graças
a Deus, muitos graus separam a existência de alguém que viaja por prazer
--- ou um turista --- de uma pessoa que é moradora e trabalha. Certamente, a
classificação das pessoas entre aquelas que em uma cidade gastam
dinheiro e as que recebem é bastante justificada e, muito mais do que em outras cidades, também é verdadeira para um grande centro de turismo e
diversão como é Paris. Entretanto, o escritor, em todo caso --- e isso para ele é uma das suas maiores sortes ---, está dispensado disso. Para ele, desde que ele encontre alguma concentração, qualquer lugar
onde tenha vivido por algum tempo, torna"-se uma cidade para
trabalhar. E, depois de longa ausência, talvez fique admirado --- pelo menos para mim foi inesperado --- ao notar o quanto rapidamente são reconstruídos austeros hábitos de vida e de trabalho, mesmo no âmbito de uma estadia fugaz e pouco sobrecarregada de programas. Portanto, tenho menos a falar"-lhes sobre a novidade da vida teatral e artística do que de constelações fortuitas da vida cotidiana, sobretudo de encontros e pessoas, sobre quem há pouco de novo e muito de velho, que me fascinaram.
Talvez não haja nenhuma coincidência feliz maior para um reencontro com
a cidade do que ter vivido e aprendido lá por muito tempo, de estar
ausente ainda por mais tempo e então depois de muitos projetos de viagem
fracassados, acordar nela quase atônito numa manhã. Aliás, para mim, foi
um belo consolo, mesmo que um tanto esnobe, descobrir pela leitura de
jornal que minha ausência meio involuntária na cidade coincidiu quase em
ano e dia com a ausência forçada de um de seus residentes mais
interessantes. Léon Daudet, filho do famoso autor de
\emph{Tartarin},\footnote{Benjamin refere"-se ao romance de Alphonse
  Daudet intitulado \emph{Tartarin de Tarascon}, de 1872. [\versal{N. E.}]} redator"-chefe
da \emph{royal}\footnote{Em francês no original: real. [\versal{N. T.}]} \emph{Action française}, que há dois anos e meio graças a um golpe genial dos
\emph{Camelots du Roi}\footnote{\emph{Camelots du Roi},
  oficialmente \emph{Fédération nationale des Camelots du Roi} era uma
  organização juvenil de extrema"-direita do movimento militante
  integralista \emph{Action française} de 1908 a 1936, conhecida por
  participar de muitas manifestações de direita na França nas décadas de
  1920 e 1930. [\versal{N. E.}]} foi retirado da prisão e depois fugiu para Bélgica ---
este Léon Daudet, de quem se supôs que o governo iria perdoá"-lo após
oito dias, obteve somente agora a permissão de voltar do exílio. Os
intelectuais repetidas vezes exigiram energicamente o seu perdão e
compreende"-se que os manifestos com os quais se expressaram a favor
deste fanático radical de direita traziam, dentre outros, os nomes dos
mais significativos autores de orientação esquerdista, pois Léon Daudet
tem não apenas o mais significativo dos méritos em relação à literatura
francesa --- assim ele é e permanece como o autêntico descobridor de
Marcel Proust no sentido de que dentre todos os seus mais antigos
tímidos admiradores, ele foi o único que interveio a favor dele
publicamente, assegurando"-lhe assim o prêmio Goncourt ---, como também
tem o mérito muito particular em relação à cidade de ter sido o único a
ter a ideia de fazer de sua própria biografia um monumento de Paris. Ele
intitulou sua autobiografia de \emph{Paris vécu},\footnote{Em francês no original: \emph{Paris vivida}. [\versal{N. E.}]} baseada não em um esquema cronológico,
mas topográfico. Narrou o que cada um dos
\emph{arrondissements}\footnote{Em francês no original: divisões
  administrativas da cidade de Paris. [\versal{N. T.}]} deu"-lhe desde seus primeiros dias
em Paris até hoje. Para compreender este livro inteiramente, é
necessário conhecer a vida singular dos \emph{arrondissements} de Paris
que é tão rica e teimosa a ponto de que esses \emph{arrondissements} são semelhantes
a muitas cidades de província. Bem sabemos que é possível observar grandes
diferenças folclóricas nos diferentes bairros de todas cidades do mundo.
Mas onde mais do que em Paris a autoconsciência de um
\emph{arrondissement} qualquer, provinciano e inteiramente pequeno"-burguês poderia ir tão longe a ponto de originar um jornal semanal,
\emph{Echos du quatorzième}, que se apresenta já há dez anos como a voz
do pacato bairro que se estende entre o \emph{Parc Montsouris} e a
\emph{Gare Montparnasse}. O bairro que, aliás, deu asilo a Lenin por anos
na fantasmagórica rua nomeada \emph{Rue de la Tombe"-Issoire}. Mas basta
sobre Lenin e Daudet. Tem"-se em vista a publicação do segundo tomo de
suas \emph{Memórias}, \emph{Rive Gauche} um \emph{pendant}\footnote{Em francês no original: um anexo. [\versal{N. T.}]} da \emph{Rive Droite}, que pode ser
indicado a todos amantes desta cidade como um de seus documentos mais
vivazes.

Não falaremos muito sobre livros, sobretudo não os comentaremos, mas ao
menos não queremos deixar escapar uma promessa. Referimo"-nos ao novo
romance de André Gide, \emph{Robert}, que, diante dos olhos surpresos e
dilacerados dos parisienses, começou a ser publicado na \emph{Revue
Hebdomadaire}. Deve"-se saber que, na França, Gide conecta ao mesmo tempo
a reputação de um grande autor com a de um desmancha prazeres, e que
\emph{Si le grain ne meurt}, sua autobiografia (recentemente publicada
em alemão com o título \emph{Stirb und werde}\footnote{A tradução
  literal do título em alemão é: \emph{Morra e venha a ser}. A tradução
  brasileira ateve"-se à literalidade: \emph{Se o grão não morre}. [\versal{N. E.}]})
ofende o pai de família da mesma forma que suas grandes reportagens
coloniais \emph{Voyage au Congo} e \emph{Le retour du Tchad}\footnote{André Gide, \emph{Retorno de Chade} (1928) e \emph{Viagem ao Congo} (1927). [\versal{N. E.}]} ofendem
o cidadão médio francês. A \emph{Revue Hebdomadaire} é, entretanto, a
leitura semanal desses mesmos pais de família e desses cidadãos médios. O
senhor Le Grix\footnote{‎François Le Grix, diretor da revista
  \emph{Hebdomadaire}. [\versal{N. E.}]} também acompanhou o novo romance de Gide com uma
nota de redação que inclui não menos que dezoito páginas. O francês
médio, devemos sabê"-lo, não nutre nenhum interesse pela discussão de
problemas sexuais --- e menos ainda por problemas tais como aqueles
introduzidos por Gide de modo tão especial. ``\emph{Il en est encore}'',
como acidentalmente disse"-me o biógrafo de Proust Léon Pierre"-Quint,
``\emph{toute aux histoires de jupons dans le genre de la} \emph{`Vie
Parisienne' et du} `\emph{Sourire}'".\footnote{Em francês no
  original. ``Ele está ainda completamente nas histórias de anáguas no
  gênero da \emph{Vie Parisienne} e do \emph{Sourire}''. Quint refere"-se a duas
  revistas francesas: a primeira, \emph{Vie Parisienne}
  (1863-1970), foi uma revista cultural que se voltou ao público
  masculino, com ilustrações eróticas femininas, no início do século \versal{XX};
  e a segunda, \emph{Sourire} (1989-1940), uma revista semanal
  humorística e ilustrada. [\versal{N. T.}]} Precisamente entre os pais de família e
cidadãos médios, entre os sólidos franceses, não são poucos os que
consideram Gide um segundo Marquês de Sade. Pode"-se mesmo retirar algum
proveito racional deste julgamento, caso tenhamos presente por um
instante a característica de Sade oferecida há pouco tempo por um jovem
ensaísta francês. Ele escreve:

\begin{quote} 
O que de outro a obra de Sade ensina a reconhecer senão o quanto um espírito verdadeiramente revolucionário é estranho à ideia do amor? Na medida em que seus escritos não são representações de recalques, como seria natural para um prisioneiro; na medida em que eles provêm da intenção de ofender --- o que eu não acredito no caso de Sade, pois seria uma bastante tola aspiração para um prisioneiro da Bastilha ---, e na medida em que tais motivos não estão em jogo, suas obras brotam de uma negação revolucionária que se desenrolou até consequências lógicas extremas. Qual seria pois a utilidade de um protesto contra os poderosos, dado que se tenha aceitado o domínio da natureza sobre a existência humana, com tudo de revoltante que isso implica? Como se o ``amor normal'' não fosse o mais repulsivo de todos os preconceitos! Como se a procriação fosse algo de diferente da mais desprezível forma de subscrever o desenho fundamental do universo! Como se as leis da natureza, às quais o amor se submete, não fossem mais tirânicas e odiosas do que as leis da sociedade! O significado metafísico do sadismo repousa na esperança de que a revolta da humanidade possa atingir uma intensidade tão violenta que forçaria a natureza a mudar suas leis, e que diante da decisão das mulheres de parar de tolerar a injustiça da gravidez e os perigos e dor de dar à luz, a natureza seria compelida a encontrar outras maneiras de garantir a sobrevivência da raça humana na Terra. A força que diz não à família ou ao estado deve também dizer não a Deus; e assim como as regulações de funcionários e padres, a antiga lei do Gênesis também deve ser quebrada: ``do suor do teu rosto, comerás teu pão, na dor parirás teus filhos''. Pois o que constitui o crime de Adão e Eva não é o fato de que eles provocaram esta lei, mas o fato de que eles a suportaram. 
\end{quote}

Essas frases impressionantes provêm de um escrito do jovem Emmanuel
Berl.\footnote{Emmanuel Berl (1892-1976), ensaísta francês,
  jornalista, filósofo, ligado a Bergson, Proust e aos surrealistas. {[}\versal{N.~E.}{]}}
Elas são retiradas do livro intitulado \emph{Mort de la pensée %Seria Morte do pensamento burguês, certo?
bourgeoise}.\footnote{Em francês no original: \emph{Morte ao pensamento burguês}. {[}\versal{N.~T.}{]}} Se a  produção ensaística francesa possui uma relevância na Europa, e seus escritos críticos sobressaem"-se tanto
particularmente em relação aos nossos, isso ocorre graças a figuras
como Julien Brenda, como Alain Chartier, como Emmanuel Berl. Fui visitar
Berl e depois de uma conversa de horas, tive uma impressão bastante
clara do modo de pensar e de ser desse homem. Assegurei"-lhe que seus
escritos são significativos também para a vanguarda da inteligência
alemã e notei que ele pertence ao tipo de pessoa que quer conversar apenas sobre o seu tema favorito, para depois, sem tolerar tantas interrupções, falar o que tem a dizer de memória. Para ele, trata"-se, sobretudo, de
continuar sua obra polêmica, expulsando a pseudoreligiosidade da
burguesia dos seus últimos esconderijos. Descobre"-os, entretanto, nem no
catolicismo com suas hierarquias e sacramentos nem no Estado, mas no
individualismo, na crença naquilo que é incomparável, na imortalidade do
indivíduo singular, no convencimento de que a própria interioridade é o
cenário de uma ação trágica única, jamais repetida. Ele identifica a forma
mais na moda desta convicção no culto do inconsciente. Eu saberia, mesmo que ele não tivesse também me assegurado, que ele tem Freud ao seu lado na luta fanática que ele declarou contra este culto. E olhando
para \emph{Le grand jeu}, a revista de alguns membros dissidentes do
grupo, que eu recém"-adquirira, disse: ``\emph{Tout ça ce sont des
séminaristes}''.\footnote{Em francês no original: ``Todos estes são
  seminaristas''. {[}\versal{N. T.}{]}} E  nesse momento, lanço algumas insinuações curiosas ao estilo de
vida destas jovens pessoas: o \emph{refus},\footnote{Em francês no original: a recusa. [\versal{N. T.}]} como diz Berl, uma palavra que podemos traduzir como ``sabotagem''.
Recusar uma entrevista, refutar uma colaboração, negar uma foto, tudo
isso eles tomam como prova de seu talento. De modo muito espirituoso,
Berl conecta isso com a enraizada tendência à ascese, tão típica dos
parisienses. Por outro lado, ainda assombra aqui a ideia de um gênio incompreendido,
que nós estamos a ponto de eliminar radicalmente.
Ouço"-o e não o contradigo. Entretanto, a atitude destes jovens não me
resulta tão completamente incompreensível. Falo para mim mesmo que há
muitos procedimentos para se ter sucesso como literato e que poucos
dentre eles têm um mínimo que ver com a literatura. Um campeão nesta
arte: Jean Cocteau. Não há mesmo em Paris muitos autores que sabem
permanecer na lembrança do público mesmo sem escrever como Cocteau.
Como, ainda recentemente, em um tipo de escrito panfletário no programa do
recém"-inaugurado \emph{Théâtre Pigalle}, construído com enorme esforço
pelo Baron Rothschild para uma atriz. Em Paris ele ficou popularmente
conhecido como \emph{Théâtre de la monnaie}.\footnote{Em francês no original: Teatro da moeda. [\versal{N. T.}]} O seu interior preserva seu caráter
por meio da contraposição entre partes construtivas, principalmente
metálicas ou de vidro e feixes de luz multicoloridos e variados, sob
cujo brilho eles destacam"-se. No entreato, o vestíbulo com seus produtos
e barracas de livros, flores e discos oferece uma imagem muito brilhante
e peculiar para um público, ainda sempre vestido de uma maneira solene
de acordo com as convenções parisienses. De fato, é incerto o quanto
isso deve ser atribuído ao contraste com as imagens poeirentas das
\emph{Histoires de France} de Guitry\footnote{\emph{Histoires de
  France} é uma peça de Sacha Guitry em três atos e doze pinturas
  criadas no \emph{Théâtre Pigalle} em 1929. [\versal{N. E.}]} que, com teimosas
inscrições alexandrinas, desenrolou"-se no interior do teatro. ``A grande
utilidade das obras de Cocteau'', assim foi escrito recentemente em um
jornal parisiense, ``consiste --- excetuando obviamente seu valor
literário --- em sua habilidade de nomear bares que, de outra maneira, sem
seu protetorado provavelmente estariam atolados na banalidade. O primeiro foi
\emph{Bœuf sur le toit},\footnote{Em francês no original: Carne no telhado. [\versal{N. T.}]} depois veio o \emph{Grand écart}\footnote{Em francês no original: Grande lacuna. [\versal{N. T.}]} e o mais novo é
\emph{Enfants terribles} que teve uma deslumbrante inauguração.'' Na
realidade, todos esses são ao mesmo tempo títulos das obras de Cocteau.
Isso ainda é aceitável. Duvidoso é o gosto com o qual se quis acreditar
retirar da obra de Rimbaud o nome de um pequeno bar mundano na praça do
Odeon: \emph{Le bâteau ivre}, o barco bêbado. Efetivamente, lá dentro há
pontes de comando, vigias, tubos de som, muitos objetos de latão, muita
laca branca e a proprietária da empresa, a Princesa d'Erlanger que fez
todos os esforços para estar à altura do nome que escolheu. A última
moda é que as \emph{Boîtes de nuit}\footnote{Em francês no original: Clubes noturnos. [\versal{N. T.}]} sejam mantidas pelas damas da aristocracia.
Além disso, dado que o \emph{gin fizz} custa 20 francos, a aristocracia
pode de passagem ainda fazer negócios --- e mesmo com a consciência bem
tranquila, pois um grande número dos que são inspirados por seus
drinks ``intelectuais'' são escritores. Assim, o estabelecimento aumenta o patrimônio intelectual da nação. Aliás, não tenho razão pessoal de estar insatisfeito
com o \emph{Bâteau ivre} e com a princesa que o dirige, pois ali me
encontro, muito depois da meia"-noite, com Léon"-Paul Fargue, que raramente se deixa ver, ardendo em brasa, como que emergindo da casa das
caldeiras.\footnote{L.~-P. Fargue (1876-1947) foi poeta e
  ensaísta francês. [\versal{N.~T.}]} Não é nada fácil apresentar esse homem.
Poder"-se"-ia dizer, por exemplo, que ele poussui um bela barba grossa,
que ele, entretanto, raspa de um dia para o outro quando lhe dá na
telha. Poder"-se"-ia dizer também que ele é proprietário de uma fábrica de
Majólica\footnote{Faiança italiana do Renascimento, inspirada a
  princípio na tradição hispano"-mourisca. [\versal{N. E.}]} bem estabelecida. Quando ele
subtamente surgiu diante de mim, tive tempo apenas de sussurrar a quem
estava ao meu lado: ``o maior lírico da França''. Talvez eu tenha sido
um pouco precipitado. Este posto deve ser reservado para Valéry.
Excetuando o fato de que Fargue de fato seja um grande lírico,
descobrimos nessa noite que era um dos contadores de história dos mais
cativantes. Mal soube que me ocupei muito de Marcel Proust e colocou
como um ponto de honra evocar para nós a imagem mais colorida e
contraditória do seu velho amigo. Isso foi não apenas a fisiognomia do
homem que admiravelmente renasceu na voz de Fargue; não apenas a risada
alta e exaltada do jovem Proust, do leão dos salões que, sacudindo todo
o corpo, pressionava a boca aberta com as mãos vestidas com luvas
brancas, enquanto seu monóculo quadrado, amarrado a uma grande fita
preta, balançava diante dele; não apenas o Proust doente vivendo em um
quarto que mal pode ser distinguido de um armazém de móveis de uma casa
de leilões, em uma cama que não foi feita por dias, uma cama que mais
parece uma caverna cheia de manuscritos, papéis escritos e em branco,
material indispensável para poder escrever, livros amontoados uns sobre
os outros, presos na fresta entre a cama e a parede, empilhados sobre a
mesa de cabeceira. Ele não apenas evocou este Proust, como também
delineou a história de uma amizade de vinte anos, as manifestações de
afetuosa ternura, as explosões de insana desconfiança, aquele \emph{Vous
m'avez trahi à propos de tout et de rien}\footnote{Em francês no original: ``Você me traiu sobre tudo e sobre nada''. [\versal{N. T.}]} ---, sem esquecer
da sua notável descrição do jantar (e naturalmente também da sua conduta
no próprio jantar) que ele ofereceu para Marcel Proust e James Joyce que,
justamente nesta ocasião, encontraram"-se pela primeira e última vez.
``Manter viva a conversa'', disse Fargue, ``significa para mim levantar
uma carga de cinquenta quilos. Além disso, por precaução convidei duas
belas moças para amenizar o impacto do encontro. Mas isso não impediu
que Joyce, saindo de sua companhia, jurasse em alto e bom som nunca mais
colocar os pés em uma sala onde ele possa correr o risco de encontrar
com essa figura''. E Fargue imitou o horror que havia feito tremer o
irlandês, quando Proust reiterou com olhos lacerados a respeito de uma
alteza imperial ou principesca ``\emph{C'était ma première altesse}''.\footnote{Em francês no original: ``Foi minha primeira alteza''. [\versal{N. T.}]} Este primeiro Proust do final dos
anos 1890 estava no início de um caminho cujo percurso ele mesmo ainda
não poderia prever. Naquela época, ele procurava a identidade no homem que lhe
aparecia como o verdadeiro elemento divinizante. Assim iniciava
a maior destruição do conceito de personalidade conhecida pela nova
literatura. Permanecemos juntos sob uma pequena turbina de lembranças
e máximas até que, às três horas, colocaram"-nos para fora. Ainda não
transcorreram quarenta e oito horas de minha última noite em Paris, com
a qual eu quero concluir aqui, deixei emergir em mim a imagem de Proust
em um espelho muito diferente. O espelho de Albertine, caso nos seja
permitido nomear assim um homem que está junto aos seus amigos e junto a
todos os parisienses conhecedores de Proust, chama"-se Monsieur Albert.
Esse espelho não teria sido tanto o que o Monsieur Albert tem a contar
sobre Proust; não tudo o que ele contou"-me de melhor era novidade e
ainda menos destinado a ser posteriormente divulgado. No entanto, nesse
homem mesmo há ainda algo que fornece, como num espelho, o reflexo do
escritor. Enfim, a discrição que o Monsieur Albert possui tanto na fala
como na apresentação, revela mais o antigo servidor do Príncipe Orloff,
o futuro camareiro do príncipe von Radziwill, do que o atual
proprietário do bar \emph{Trois colonnes},\footnote{Em francês no original: Três colunas. [\versal{N. T.}]} próximo à Praça da Bastilha.
Monsieur Albert quis mostrar"-me as honras desse estabelecimento, mas eu
preferi segurá"-lo para um café no bar mais nobre onde havíamos jantado
de maneira excelente e ouvir a agradável inflexão com a qual ele evocava
a lembrança das primeiras caminhadas noturnas no \emph{Boulevard
Haussman} ao lado do poeta que acompanhava o efeito mutante da luz da
lua respectivamente com os mais apropriados versos de Vigny, Hugo,
Lamartine ou Mallarmé. Paris não me deu nesta semana nenhuma imagem mais
atraente do que aquela que soube suscitar em mim essas palavras de
Monsieur Albert.
