\chapter{Talentos parisienses\footnote[*]{\versal{GS VII}-1, p.
  279-286. Tradução Pedro Hussak van Velthen Ramos.}}

Para alguém que tenha partido em viagem, serão claros os diferentes
graus de estranheza e de familiaridade, proximidade e distância,
resolução e relutância que ele experimenta em relação às cidades. Graças
a Deus, muitos graus separam a existência de alguém que viaja por prazer
--- ou um turista --- de uma pessoa que é moradora e trabalha. Certamente, a
classificação das pessoas entre aquelas que em uma cidade gastam
dinheiro e as que recebem é bastante justificada e, de forma muito mais
extensiva, também é mais verdadeira para um grande centro de turismo e
diversão como é Paris do que para outras cidades. No entanto, o escritor
está, em todo caso, dispensado desta classificação, o que, para ele, é
uma de suas grandes chances. Com alguma concentração, qualquer lugar
onde ele tenha vivido por algum tempo, torna"-se uma cidade para
trabalhar. E talvez se admire --- em todo caso, para mim foi inesperado
--- como, depois de longa ausência, reconstroem"-se austeros hábitos de
vida e de trabalho, mesmo durante uma permanência transitória e
programaticamente pouco carregada. Portanto, tenho menos a falar aos
senhores sobre a novidade da vida teatral e artística do que de
constelações fortuitas da vida cotidiana, sobretudo de encontros e
pessoas que me tocaram sobre quem há pouco de novo e muito de velho.
Talvez não haja nenhuma coincidência feliz maior para um reencontro com
a cidade do que ter vivido e aprendido lá por muito tempo, de estar
ausente ainda por mais tempo e então depois de muitos projetos de viagem
fracassados, acordar nela quase atônito numa manhã. Aliás, para mim, foi
um belo consolo, mesmo que um tanto esnobe, descobrir pela leitura de
jornal que minha ausência meio involuntária na cidade coincidiu quase em
ano e dia com a ausência forçada de um de seus residentes mais
interessantes. Léon Daudet, filho do famoso autor de
\emph{Tartarin},\footnote{Benjamin refere"-se ao romance de Alphonse
  Daudet intitulado \emph{Tartarin de Tarascon}, de 1872. [\versal{N. E.}]} redator"-chefe
da \emph{royal Action française},\footnote{Em francês no original. [\versal{N. T.}]} que há dois anos e meio graças a um golpe genial dos
\emph{Camelots du Roi}\footnote{\emph{Camelots du Roi},
  oficialmente \emph{Fédération nationale des Camelots du Roi} era uma
  organização juvenil de extrema"-direita do movimento militante
  integralista \emph{Action Française} de 1908 a 1936, conhecida por
  participar de muitas manifestações de direita na França nas décadas de
  1920 e 1930. [\versal{N. E.}]} foi retirado da prisão e depois fugiu para Bélgica ---
este Léon Daudet, de quem se supôs que o governo iria perdoá"-lo após
oito dias, obteve somente agora a permissão de voltar do exílio. Os
intelectuais repetidas vezes exigiram energicamente o seu perdão e
compreende"-se que os manifestos com os quais se expressaram a favor
deste fanático radical de direita, traziam, dentre outros, os nomes dos
mais significativos autores de orientação esquerdista. Pois, Léon Daudet
tem não apenas o mais significativo dos méritos em relação à literatura
francesa --- assim ele é e permanece como o autêntico descobridor de
Marcel Proust no sentido de que dentre todos os seus mais antigos
tímidos admiradores, ele foi o único que interveio a favor dele
publicamente, assegurando"-lhe assim o prêmio Gongourt ---, como também
tem o mérito muito particular em relação à cidade de ter sido o único a
ter a ideia de fazer de sua própria biografia um monumento de Paris. Ele
intitulou sua autobiografia de \emph{Paris vécu}.\footnote{Paris
  vivida. [\versal{N. E.}]} Para ele, o que está na base não é o esquema cronológico,
mas topográfico. Narrou o que cada um dos
\emph{arrondissement}\footnote{Em francês no original: divisões
  administrativas da cidade de Paris. [\versal{N. T.}]} deu"-lhe desde seus primeiros dias
em Paris até hoje. Para compreender este livro inteiramente, é
necessário conhecer a vida singular dos \emph{arrondissements} de Paris
que é tão rica e teimosa como se estes \emph{arrondissements} fossem
muitas cidades de província. Bem sabemos ser possível observar grandes
diferenças folclóricas nos diferentes bairros de todas cidades do mundo.
Mas onde mais do que em Paris a autoconsciência de um
\emph{arrondissement} qualquer, provinciano e completamente pequeno
burguês poderia ir tão longe {[}a ponto de originar um jornal semanal %Colchetes não fecham
\emph{Echos du quatorzième} que se apresenta já há dez anos como a voz
do pacato bairro que se estende entre o \emph{Parc Montsouris} e a
\emph{Gare Montparnasse}. O bairro aliás que deu asilo a Lenin por anos
na fantasmagórica rua nomeada \emph{Rue de la Tombe"-Issoire}. Mas basta
sobre Lenin e Daudet. Tem"-se em vista a publicação do segundo tomo de
suas \emph{Memórias}, \emph{Rive Gauche} um \emph{pendant}\footnote{Em francês no original: um anexo. [\versal{N. T.}]} da \emph{Rive Droite}, que pode ser
indicado a todos amantes desta cidade como um de seus documentos mais
vivazes. %Assim mesmo, um e da sem itálico?

Não falaremos muito sobre livros, sobretudo não os comentaremos, mas ao
menos não queremos deixar escapar uma promessa. Referimo"-nos ao novo
romance de André Gide, \emph{Robert}, que, diante dos olhos surpresos e
dilacerados dos parisienses, começou a ser publicado na \emph{Revue
Hebdomadaire}. Deve"-se saber que, na França, Gide conecta ao mesmo tempo
a reputação de um grande autor com a de um desmancha prazeres, e que
\emph{Si le grain ne meurt}, sua autobiografia (recentemente publicada
em alemão com o título \emph{Stirb und werde}\footnote{A tradução
  literal do título em alemão é: \emph{Morra e venha a ser}. A tradução
  brasileira ateve"-se à literalidade: \emph{Se o grão não morre}. [\versal{N. E.}]})
ofende o pai de família da mesma forma que suas grandes reportagens
coloniais \emph{Voyage au Congo} e \emph{Le retour du Tchad}\footnote{André Gide, \emph{Retorno de Chade} (1928) e \emph{Viagem ao Congo} (1927). [\versal{N. E.}]} ofendem
o cidadão médio francês. A \emph{Revue Hebdomadaire} é, entretanto, a
leitura semanal destes mesmos pais de família e cidadãos médios. O
senhor Le Grix\footnote{‎François Le Grix, diretor da revista
  \emph{Hebdomadaire}. [\versal{N. E.}]} também acompanhou o novo romance de Gide com uma
nota de redação que inclui não menos que dezoito páginas. O francês
médio, devemos sabê"-lo, não tem grande interesse pela discussão de
problemas sexuais --- e agora menos por problemas tais como aqueles
levantados por Gide de modo tão especial. ``\emph{Il en est encore}'',
como acidentalmente disse"-me o biógrafo de Proust Léon Pierre"-Quint,
``\emph{toute aux histoires de jupons dans le genre de la} \emph{`Vie
Parisienne' et du} `\emph{Sourire}'".\footnote{Em francês no
  original. ``Ele está ainda completamente nas histórias de anáguas no
  gênero da `Vie Parisienne' e do `Sourire'". Quint refere"-se a duas
  revistas francesas: a primeira, \emph{Vie Parisienne}
  (1863-1970), foi uma revista cultural que se voltou ao público
  masculino, com ilustrações eróticas femininas, no início do século \versal{XX};
  e a segunda, \emph{Sourire} (1989-1940), uma revista semanal
  humorística e ilustrada. [\versal{N. T.}]} Precisamente entre os pais de família e
cidadãos médios, entre os sólidos franceses, não são poucos os que
consideram Gide um segundo Marquês de Sade. Pode"-se mesmo retirar algum
proveito racional deste julgamento, caso tenhamos presente por um
instante a característica de Sade oferecida há pouco tempo por um jovem
ensaísta francês. Ele escreve:

\begin{quote}
O que a obra de Sade nos ensina senão reconhecer a extensão à qual uma
mente verdadeiramente revolucionária é alienada da ideia do amor? Na
medida em que seus escritos não são simples repressões, como seria
natural para um prisioneiro, e na medida em que eles não provém do
desejo de ofender --- no que eu não acredito no caso de Sade, pois seria
uma bastante tola aspiração para um prisioneiro da Bastilha --- na medida
em que tais motivos não estão envolvidos, suas obras têm suas raízes em
uma negação revolucionária desenvolvida até a sua consequência lógica
extrema. Qual seria pois a utilidade de um protesto contra os poderosos,
dado que se tenha aceitado o domínio da natureza sobre a existência
humana, com todas as indignidades que isso implica? Como se o ``amor
normal'' não fosse o mais repulsivo de todos os preconceitos! Como se a
concepção fosse algo diferente da mais desprezível forma de subscrever o
desenho fundamental do universo! Como se as leis da natureza, às quais o
amor se submete, não fossem mais tirânicas e odiosas do que as leis da
sociedade! O significado metafísico do sadismo repousa na esperança de
que a revolta da humanidade pode atingir uma intensidade tão violenta
que forçaria a natureza a mudar suas leis, e que diante da decisão das
mulheres de parar de tolerar a injustiça da gravidez e os perigos e dor
de dar à luz, a natureza encontrar"-se"-ia compelida a encontrar outras
maneiras de garantir a sobrevivência da raça humana na Terra. A força
que diz não à família ou ao estado deve também dizer não a Deus; e assim
como as regulações de funcionários e padres, a antiga lei do Genesis
também deve ser quebrada: ``do suor do teu rosto, comerás teu pão, na
dor parirás teus filhos''. Pois o que constitui o crime de Adão e Eva
não é o fato de que eles provocaram esta lei, mas que eles a suportaram.
\end{quote}

Estas frases impressionantes provêm de um escrito do jovem Emmanuel
Berl.\footnote{Emmanuel Berl (1892-1976), ensaísta francês,
  jornalista, filósofo, ligado a Bergson, Proust e aos surrealistas. [\versal{N. E.}]}
Elas são retiradas do livro intitulado \emph{Mort de la pensée
bourgeoise}. Se a produção ensaística francesa encontra"-se em elevado
nível europeu, e seus escritos críticos sobressaem"-se tanto
particularmente em relação aos nossos, isso acontece graças a figuras
como Julien Brenda, como Alain Chartier, como Emmanuel Berl. Fui visitar
Berl e depois de uma conversa de horas, tive uma impressão bastante
clara do modo de pensar e de ser deste homem. Assegurei"-lhe que seus
escritos são significativos também para a vanguarda da inteligência
alemã e notei que ele pertence ao tipo de pessoa que quer ser levado
unicamente pelo seu tema preferido para então trazer à memória o que tem
a dizer sem tolerar muita interrupção. Para ele, trata"-se, sobretudo, de
continuar sua obra polêmica, expulsando a pseudoreligiosidade da
burguesia dos seus últimos esconderijos. Descobre"-os, entretanto, nem no
catolicismo com suas hierarquias e sacramentos nem no Estado, mas no
individualismo, na crença naquilo que é incomparável, na imortalidade do
indivíduo singular, o convencimento de que a própria interioridade é o
cenário de uma ação trágica única, jamais repetida. Ele avista a forma
mais na moda desta convicção no culto do inconsciente. Que ele tenha
Freud do seu lado na luta fanática que ele declarou contra este culto ---
ou seja, contra o surrealismo, que o celebra ---, isso é algo que eu
saberia mesmo que ele não me tivesse assegurado. E lançando uma vista
para \emph{Le grand jeu}, a revista de alguns membros dissidentes do
grupo, que eu recém adquirira: ``\emph{Tout ça ce sont des
séminaristes}''.\footnote{No original em francês: ``Todos estes são
  seminaristas''. [\versal{N. T.}]} E agora algumas insinuações curiosas ao estilo de
vida destas jovens pessoas: o \emph{refus},\footnote{No original em
  francês: ``a recusa''. [\versal{N. T.}]} como diz Berl. Podemos traduzir: sabotagem.
Recusar uma entrevista, refutar uma colaboração, negar uma foto, tudo
isso eles tomam como prova de seu talento. De modo muito espirituoso,
Berl conecta isso com a enraizada tendência à ascese, típica dos
parisienses. Por outro lado, ainda assombra aqui a ideia, que nós
estamos a ponto de eliminar radicalmente, de um gênio incompreendido.
Ouço"-o e não o contradigo. Entretanto, a atitude destes jovens não me
resulta tão completamente incompreensível. Falo para mim mesmo que há
muitos procedimentos para se ter sucesso como literato e que poucos
dentre eles têm um mínimo que ver com a literatura. Um campeão nesta
arte: Jean Cocteau. Não há muitos autores mesmo em Paris que sabem
permanecer na lembrança do público mesmo sem escrever como Cocteau.
Ainda recentemente em um tipo de escrito panfletário no programa do
recém"-inaugurado \emph{Théatre Pigalle}, construído com enorme esforço
pelo Baron Rothschild para uma atriz. Em Paris ele ficou popularmente
conhecido como \emph{Théatre de la monnaie}.\footnote{No original
  em francês: ``Teatro da moeda''. [\versal{N. T.}]} O seu interior preserva seu caráter
por meio da contraposição entre partes construtivas, principalmente
metálicas ou de vidro e feixes de luz multicoloridos e variados, sob
cujo brilho eles destacam"-se. No entreato, o vestíbulo com seus produtos
e barracas de livros, flores e discos oferece uma imagem muito brilhante
e peculiar para um público, ainda sempre vestido de uma maneira solene
de acordo com as convenções parisienses. De fato, é incerto o quanto
isso deve ser atribuído ao contraste com as imagens poeirentas das
\emph{Histoires de France} de Guitry\footnote{\emph{Histoires de
  France} é uma peça de Sacha Guitry em três atos e doze pinturas
  criadas no \emph{Théâtre Pigalle} em 1929. [\versal{N. E.}]} que, com teimosas
inscrições alexandrinas, desenrolou"-se no interior do teatro. ``A grande
utilidade das obras de Cocteau'', assim foi escrito recentemente em um
jornal parisiense, ``consiste --- excetuando obviamente seu valor
literário --- em sua habilidade de nomear segundo elas os bares que sem
seu protetorado presumivelmente atolasse na banalidade. O primeiro foi
\emph{Bœuf sur le toit,} depois veio o \emph{Grand écart}\footnote{No original em francês: ``Grande lacuna''. [\versal{N. T.}]} e o mais novo é
\emph{Enfants terribles} que teve uma deslumbrante inauguração''. Na
realidade, todos estes são ao mesmo tempo títulos das obras de Cocteau.
Isso ainda é aceitável. Duvidoso é o gosto com o qual se quis acreditar
retirar da obra de Rimbaud o nome de um pequeno bar mundano na praça do
Odeon: \emph{Le bâteau ivre}, o barco bêbado. Efetivamente, lá dentro há
pontes de comando, vigias, tubos de som, muitos objetos de latão, muita
laca branca e a proprietária da empresa, a Princesa d'Erlanger que fez
todos os esforços para estar à altura do nome que escolheu. A última
moda é que as \emph{Boîtes de nuit}\footnote{No original em
  francês: ``Clube noturno''. [\versal{N. T.}]} sejam mantidas pelas damas da aristocracia.
Além disso, dado que o \emph{gin fizz} custa 20 francos, a aristocracia
pode de passagem ainda fazer dinheiro --- e mesmo com a consciência bem
tranquila, pois um grande número dos que são inspirados por seus
coquetéis são escritores. Assim, o estabelecimento aumenta os bens
culturais da nação. Aliás, não tenho razão pessoal de estar insatisfeito
com o \emph{Bâteau Ivre} e com a princesa que o dirige, pois ali me
encontro com o raramente visível Léon"-Paul Fargue muito depois da
meia"-noite, ardendo em brasa, como que emergindo da casa das
caldeiras.\footnote{Léon"-Paul Fargue (1876-1947) foi poeta e
  ensaísta francês. [\versal{N. T.}]} Não é nada fácil apresentar este homem.
Poder"-se"-ia dizer, por exemplo, que ele poussui um bela barba grossa,
que ele, entretanto, raspa de um dia para o outro quando lhe dá na
telha. Poder"-se"-ia dizer também que ele é proprietário de uma fábrica de
Majólica\footnote{Faiança italiana do Renascimento, inspirada a
  princípio na tradição hispano"-mourisca. [\versal{N. E.}]} bem estabelecida. Quando ele
subtamente surgiu diante de mim, tive tempo apenas de sussurrar a quem
estava ao meu lado: ``o maior lírico da França''. Talvez eu tenha sido
um pouco precipitado. Este posto deve ser reservado para Valéry.
Excetuando o fato de que Fargue de fato seja um grande lírico,
descobrimos nesta noite que era um dos contadores de história mais
cativantes. Mal soube que eu me ocupei muito de Marcel Proust e colocou
como um ponto de honra evocar para nós a imagem mais colorida e
contraditória do seu velho amigo. Isso foi não apenas a fisiognomia do
homem que admiravelmente renasceu na voz de Fargue; não apenas a risada
alta e exaltada do jovem Proust, do leão dos salões que, sacudindo todo
o corpo, pressionava a boca aberta com as mãos vestidas com luvas
brancas, enquanto seu monóculo quadrado, amarrado a uma grande fita
preta, balançava diante dele; não apenas o Proust doente vivendo em um
quarto que mal pode ser distinguido de um armazém de móveis de uma casa
de leilões, em um cama que não foi feita por dias, uma cama que mais
parece uma caverna cheia de manuscritos, papéis escritos e em branco,
material indispensável para poder escrever, livros amontoados uns sobre
os outros, presos na fresta entre a cama e a parede, empilhados sobre a
mesa de cabeceira. Ele não apenas evocou este Proust, como também
delineou a história de uma amizade de vinte anos, as manifestações de
afetuosa ternura, as explosões de insana desconfiança, aquele \emph{Vous
m'avez trahi à propos de tout et de rien}\footnote{No original em
  francês: ``Você me traiu sobre tudo e sobre nada''. [\versal{N. T.}]} ---, sem esquecer
da sua notável descrição do jantar (e naturalmente também da sua conduta
no próprio jantar) que ele ofereceu para Marcel Proust e James Joyce que
justamente nesta ocasião encontraram"-se pela primeira e última vez.
``Manter viva a conversa'', disse Fargue, ``significa para mim levantar
uma carga de cinquenta quilos. Além disso, por precaução convidei duas
belas moças para amenizar o impacto do encontro. Mas isso não impediu
que Joyce, saindo de sua companhia, jurasse em alto e bom som nunca mais
colocar os pés em uma sala onde ele possa correr o risco de encontrar
com essa figura''. E Fargue imitou o horror que havia feito tremer o
irlandês, quando Proust reiterou com olhos lacerados a respeito de uma
alteza imperial ou principesca ``\emph{C'était ma première altesse}''
(``Foi minha primeira alteza''). Este primeiro Proust do final dos
anos 1890 estava no início de um caminho cujo percurso ele mesmo ainda
não poderia prever. Naquela época, ele procurava a identidade no homem.
Esta lhe aparecia como o verdadeiro elemento divinizante. Assim iniciava
a maior destruição do conceito de personalidade conhecida pela nova
literatura. Permanecemos juntos sob uma pequena turbina de lembranças
e máximas até que, às três horas, colocaram"-nos para fora. Ainda não
transcorreram quarenta e oito horas de minha última noite em Paris, com
a qual eu quero concluir aqui, deixei emergir em mim a imagem de Proust
em um espelho muito diferente. O espelho de Albertine, se nos é
permitido nomear assim um homem que está junto aos seus amigos e junto a
todos os parisienses conhecedores de Proust, chama"-se Monsieur Albert.
Este espelho não teria sido tanto o que o Monsieur Albert sabe contar
sobre Proust; não tudo o que ele contou"-me de melhor era novidade e
ainda menos determinado para posterior divulgação. No entanto, neste
homem mesmo há ainda algo que fornece, como num espelho, o reflexo do
escritor. Enfim, a discrição que o Monsieur Albert possui tanto na fala
como na apresentação, revela mais o antigo servidor do Príncipe Orloff,
o futuro camareiro do príncipe von Radziwill, do que o atual
proprietário do bar \emph{Trois colonnes}, próximo à Praça da Bastilha.
Monsieur Albert quis mostrar"-me as honras deste estabelecimento, mas eu
preferi segurá"-lo para um café no bar mais nobre onde havíamos jantado
de maneira excelente e ouvir a agradável inflexão com o qual ele evocava
a lembrança das primeiras caminhadas noturnas no \emph{Boulevard
Haussman} ao lado do poeta que acompanhava o efeito mutante da luz da
lua respectivamente com os mais apropriados versos de Vigny, Hugo,
Lamartine ou Malarmé. Paris não me deu nesta semana nenhuma imagem mais
atraente do que aquela que soube suscitar em mim estas palavras de
Monsieur Albert.
