\chapter{Diário parisiense\footnote[*]{``Pariser Tagebuch'', in \versal{GS IV}-1, p.
  567--587. Tradução de Pedro Hussak. Texto publicado na revista \emph{Die literarische Welt}, abril"-junho de 1930. [\versal{N.~O.}]}}

\begin{flushright}
\emph{30 de dezembro de 1929}
\end{flushright}

Mal se entra na cidade, e já se sente
presenteado. Inútil o propósito de não escrever sobre isso.
Reconstruo o dia anterior da mesma forma que as crianças dispõem
novamente os presentes sobre a mesa na manhã de natal. Isso também é um
modo de agradecer. Aliás, estou atendo"-me aos meus planos de algum dia
fazer mais do que isso. Desta vez, entretanto, justamente estes planos
proíbem"-me --- assim como proíbe"-me a prudência que eu devo guardar para
cada trabalho --- de abandonar minha força de vontade e entregar"-me à
cidade. Pela primeira vez esquivo"-me dela: retiro"-me ao encontro para o
qual a solidão, essa velha alcoviteira, convidou"-me, organizando as
coisas de modo que eu, em muitos dias, não visse a cidade diante dos
parisienses. Obviamente não é fácil ignorar esta cidade! Tão fácil quanto
ignorar a saúde e a felicidade. É inacreditável o quão pouco insistente
ela é. Não há provavelmente outra cidade onde se é menos notado
do que Berlim. Nisso comparece o espírito organizador e
técnico que, para o bem ou para o mal, domina"-a. Em Paris sucede o
contrário. Precisa"-se ter habitado em Paris por muito tempo para saber o
quanto a própria rua é um \emph{interieur} habitual, desgastado;
o quanto não vemos quotidianamente mesmo nas partes mais
familiares, como é mais decisivo, mais do que em qualquer outro lugar, mudar de
calçada, atravessando da direita para a esquerda. De onde vem essa discrição
que se ajusta às necessidades e capacidades da pessoa mais modesta?
Talvez de uma interpenetração, muito estranha para nós, entre uma
mentalidade conservadora e metropolitana. Não apenas Aragon, que
escreveu aquele livro, é um \emph{Paysan de Paris}, mas sobretudo o
\emph{concierge}, o \emph{marchad de quatre saisons}, mesmo o
\emph{flic}:\footnote{Louis Aragon (1897--1992) escreveu
  \emph{Le paysan de Paris} (\emph{O camponês de Paris}) no momento em que André Breton estava
  escrevendo o seu \emph{Manifeste du surréalisme}. A descrição
  minuciosa da \emph{Passage de l'Opera}, uma das passagens parisienses
  alinhada com restaurantes e pequenas lojas é um documento"-chave do
  surrealismo. Sua tentativa de liberar a imaginação através de um foco
  poético no cotidiano, muitas vezes objetos comerciais, fala diretamente
  a Benjamin que se refere a isso repetidamente no início do
  seu trabalho sobre o \emph{Passagen"-werk} no final dos anos 1920 e
  começo dos anos 1930. Um \emph{concierge} é um porteiro, \emph{marchand de quatre saisons} é um
  vendedor de frutas e vegetais, e \emph{flic} é o policial. \versal{[N.~O.]}} todas
aquelas pessoas que cultivam seu quarteirão, contínua e pacificamente
como os camponeses. Certamente a história da construção da cidade não
foi menos movimentada, nem menos cheia de atos violentos do que outra.
Mas assim como a natureza cura as rachaduras de velhos castelos com
arbustos verdes, a numerosa burguesia que reside no local pacificou as
fraturas da grande cidade. Se muitas praças afastadas abraçam tão
intimamente o espaço --- como se uma convenção de casas as tivessem
fundado --- então o sentido da pacificidade e da duração que as criou
está sendo transferido para os seus habitantes por séculos. E tudo isso
ressoa na saudação com a qual o velho caixa da minha casa de câmbio
recebeu"-me hoje depois de um longo intervalo: \emph{Vous avez été un
moment absent}\footnote{Em francês no original: ``Você esteve sumido por um tempo''. \versal{[N.~T.]}} --- um puxão com o qual ele laçou a minha
ausência, como um saco no qual ele poderia me entregar a poupança de
três anos.

\begin{flushright}
\emph{6 de janeiro de 1930}
\end{flushright}

Nos primeiros dias de janeiro eu vi Aragon,
Desnos, Green, Fargue.\footnote{Robert Desnos (1900--1945), poeta
  francês, foi um dos últimos a aderir ao surrealismo. Ele morreu em um
  campo de concentração em \emph{Theresienstadt}, deportado por causa de
  suas atividades na \emph{Resistence}. Julien Green (1900--1998) é um
  escritor francês de origem americana. Léon"-Paul Fargue (1876--1947) é
  poeta, ensaísta e crítico e contribuiu com a \emph{Nouvelle Revue
  Française} e editou (com Paul Valéry e Valery Larbaud) a revista
  \emph{Commerce} na qual ele promoveu um grande número de autores
  surrealistas. \versal{[N.~O.]}} Fargue apareceu no \emph{Bateau ivre}.\footnote{Em francês no original: Barco bêbado. \versal{[N.~T.]}} Lá dentro há
pontes de comando, vigias, tubos de som, muito latão, muita coisa em
esmalte branco. A moda mais nova é que damas da aristocracia possuam
\emph{boîtes de nuit}.\footnote{Em francês no original: casas noturnas. \versal{[N.~T.]}} Este pertence à princesa d'Erlanger. Além disso, dado que
o \emph{gin fizz} custa 20 francos, a aristocracia pode de passagem
ainda fazer negócios --- e mesmo com a consciência bem tranquila, pois um
grande número dos que são inspirados por seus drinks ``intelectuais'' são escritores.
Assim, o estabelecimento aumenta o patrimônio intelectual da nação. Ali me
encontro com Léon"-Paul Fargue muito depois da meia"-noite, ardendo em
brasa, como que emergindo da casa das caldeiras. Quando ele surgiu
subitamente tive apenas tempo de segredar a Dausse que estava sentado
perto de mim: ``O maior lírico vivo da França''. Excetuando o fato de
que Fargue de fato seja um grande lírico, descobrimos nessa noite que
era um dos contadores de história dos mais cativantes. Mal soube que me
ocupei muito de Marcel Proust e colocou como um ponto de honra evocar
para nós a imagem mais colorida e contraditória do seu velho amigo. Isso
foi não apenas a fisiognomia do homem que admiravelmente renasceu na voz
de Fargue; não apenas a risada alta e exaltada do jovem Proust, do leão
dos salões que, sacudindo todo o corpo, pressionava a boca aberta com as
mãos vestidas com luvas brancas, enquanto seu monóculo quadrado,
amarrado a uma grande fita preta, balançava diante dele; não apenas o
Proust doente, vivendo em um quarto que mal pode ser distinguido de um
armazém de móveis de uma casa de leilões, em um cama que não foi feita
por dias, uma cama que mais parece uma caverna cheia de manuscritos,
papéis escritos e em branco, material indispensável para poder escrever,
livros amontoados uns sobre os outros, presos na fresta entre a cama e a
parede, empilhados sobre a mesa de cabeceira. Ele não apenas evocou este
Proust, como também delineou a história de uma amizade de vinte anos, as
manifestações de afetuosa ternura, as explosões de insana desconfiança,
aquele \emph{Vous m'avez trahi à propos de tout et de rien},\footnote{Em francês no original: ``Você me traiu sobre tudo e sobre nada''. \versal{[N.~T.]}}
sem esquecer da sua notável descrição do jantar (e naturalmente também
da sua conduta no próprio jantar) que ele ofereceu para Marcel Proust e
James Joyce que justamente nesta ocasião encontraram"-se pela primeira e
última vez. ``Manter viva a conversa'', disse Fargue, ``significa para
mim levantar uma carga de cinquenta quilos. Além disso, por precaução
convidei duas belas moças para amenizar o impacto do encontro. Mas isso
não impediu que Joyce, saindo de sua companhia, jurasse em alto e bom
som nunca mais colocar os pés em uma sala onde ele possa correr o risco
de encontrar com essa figura''. E Fargue imitou o horror que havia feito
tremer o irlandês, quando Proust reiterou com olhos lacerados a respeito
de uma alteza imperial ou principesca: \emph{C'était ma première
altesse}.\footnote{Em francês no original: ``Foi minha primeira alteza''. \versal{[N.~T.]}} Esse primeiro Proust do final dos anos 1890 estava no
início de um caminho cujo percurso ele mesmo ainda não poderia prever.
Naquela época, ele procurava a identidade no homem. Essa lhe aparecia
como o verdadeiro elemento divinizante. Assim começou o maior destruidor
da ideia de personalidade que a literatura recente conhece. ``Fargue'',
escreve Léon Pierre"-Quint\footnote{Léon Pierre"-Quint (1895--1958) foi
  um crítico francês e amigo de Benjamin. Foi autor do primeiro grande
  estudo sobre a obra de Proust (1925). \versal{[N.~O.]}} em novembro de 1929:

\begin{quote}
é uma daquelas pessoas que escrevem como falam; e fala produzindo
constantemente obras que permanecem não escritas --- talvez por
indolência, talvez por desprezo pela escrita. Ele não poderia
expressar"-se de outra forma senão no lampejo intelectual, no jogo de
palavras, que se revezam tão casual quanto interminavelmente. Ama Paris
como uma criança: seus pequenos Cafés esquecidos, seus bares, as ruas e
a vida noturna que nunca termina. Ele deve ter uma saúde esplêndida e
uma natureza altamente resistente. Durante o dia, trabalha como
industrial e à noite passeia. Está sempre acompanhado de mulheres
elegantes, americanas. Este homem próximo aos cinquenta anos de idade,
como se não pudesse ser de outra maneira, leva à noite uma vida de
gigolô, dominando a todos que ele encontra pelo encanto do seu discurso.
\end{quote}

Conheci"-o exatamente assim, e assim permanecemos juntos sob um pequeno
fogo de artifício de lembranças e máximas até que nos colocassem para
fora às 3h da manhã.

\begin{flushright}
\emph{9 de janeiro}
\end{flushright}

Jouhandeau.\footnote{Marcel Jouhandeau
  (1888--1979) foi autor de numerosos trabalhos autobiográficos baseado
  em ficção. A tradução de Benjamin \emph{Mademoiselle Zéline ou bonheur
  de Dieu à l'usage d'une vieille demioselle} (1924) de Jouhandeau foi
  publicado na antologia de escritos franceses contemporâneos
  \emph{Neue französische Erzähler} em 1930; a tradução do conto de
  Jouhandeau \emph{Le marié du village} foi publicada em 1931 na
  \emph{Europäishe Revue}. Em 1917, Jouhandeau trabalhou no conhecido
  bordel masculino de Albert Le Cuziat que foi frequentado por Proust. \versal{[N.~O.]}}
O espaço onde me recebeu é a mais perfeita interpenetração entre um
ateliê e uma cela de monge. Uma inquebrantável série de janelas corre ao
longo de duas paredes. Além disso, há uma claraboia no teto. Cortinas
verdes pesadas por toda parte. Duas mesas, cada uma das quais se podia
considerar com igual justiça como mesas de trabalho. Diante delas,
cadeiras como que perdidas no espaço. Cinco horas da tarde e a luz vem
de uma pequena coroa e de um alto candeeiro. Conversa sobre a magia das
condições de trabalho. Jouhandeau discursa sobre as forças inspiradoras
da luz que chega da direita. E depois muitos elementos autobiográficos.
Aos treze, quatorze anos sofreu a influência decisiva de duas amáveis
irmãs que viviam na escola de freiras carmelitas da cidade natal dele. O
catolicismo, que anteriormente não era nada além de objeto de educação e
instrução, agarrou"-o a partir daquele momento. A vista de um crucifixo de
porcelana sobre a cama na minha primeira olhada no espaço revela que o
catolicismo tem um significado maior para ele. Confesso"-lhe, entretanto,
que após a leitura do seu primeiro livro, continuo completamente sem
saber se ele descreve o catolicismo como crente ou apenas como um
viajante pesquisador, um \emph{explorateur}. Essa expressão agradou"-lhe
muito. Como ele sentou"-se diante de mim, com uma presença muito
distinta, mas um tanto frágil, falando de modo penetrante e vivaz com
uma voz suave, tive pena de aproximar"-me dele apenas agora, no momento
em que tantas das forças que ameaçam seu mundo tenham sido despertadas
em mim. Continuou a falar da sua vida, particularmente sobre a noite ---
era a que seguia ao funeral de Dérouledès\footnote{Paul Déroulède
  (1846--1914) foi um poeta e dramaturgo que foi presidente da liga
  francesa de patriotas e apoiador ativo da campanha populista e
  antirrepublicana de Georges Boulanger. \versal{[N.~O.]}} --- em que ele queimou seu
trabalho inteiro: uma infindável quantidade de notas e especulações que,
ao fim, apareciam"-lhe como um obstáculo no caminho para uma verdadeira
vida. Apenas a partir de então sua produção começou a perder o caráter
lírico"-especulativo. Apenas a partir de então formou"-se o mundo dos
personagens que propriamente, como Jouhandeau contou"-me, provém
completamente de uma rua de sua cidade natal onde ele habitava. Para
ele, é importante caracterizar o mundo desses personagens: um cosmos
cuja lei manifesta"-se apenas a partir de um ponto central. Esse ponto
central é Godeau, um Santo Antônio revivido, cujos diabos, prostitutas e
bestas provêm da teologia especulativa. --- Além disso: ``o que mais me
cativou no catolicismo foram as heresias''. Cada indivíduo é, para ele,
um herético. Apaixonante para ele eram imensas desfigurações individuais
do catolicismo. Frequentemente seus personagens, dos quais muitos dos
que ele conhece ainda não entraram em seus livros, permanecem diante
dele já há muito tempo antes de tornarem"-se palpáveis para que ele
pudesse expô"-los. Frequentemente leva muito tempo até que um pequeno
gesto ou mudança neles revele"-lhe a sua particular e mais peculiar
heresia. Digo"-lhe da estupenda e abstrusa jocosidade de seus personagens
cuja dispersão maneja não com objetos do uso diário --- facas ou garfos,
fósforos ou lápis ---, mas com dogmas, fórmulas mágicas e iluminações.
Minha expressão \emph{jouets menaçants}\footnote{Em francês no original: brinquedos
  ameaçadores. \versal{[N.~T.]}} muito lhe agrada. Ermeline e
Noëmie Bodeau são mencionadas. E Mademoiselle Zéline, cuja história eu
gostaria de contar alegoricamente através da imagem do vício que não a
seduz, mas ao contrário, agarra"-a pela nuca e empurra"-a atordoada para o
caminho da virtude. Evidentemente, disse"-lhe meus pensamentos sobre a
imponente e inspiradora descrição da loucura em \emph{Marié du village}.
Um crítico comparou o autor a Blake. Creio que é correto reconhecer,
quando se pensa na crueldade com a qual Jouhandeau expõe seus
personagens à experiência religiosa, um abandono expresso nos contornos
abruptos de suas sentenças. \emph{Vos personnages sont tout le temps à
l'abri de rien}.\footnote{Em francês no original: ``Seus personagens não estão a salvo em
  nenhum momento''. \versal{[N.~T.]}} O fim da nossa conversa
foi marcado pela passagem que ele mostrou"-me na bela edição de luxo de
seu \emph{Monsieur Godeau intime}. Ele mesmo descreveu"-a como o ponto
crucial do livro, e é o momento em que se fala da estadia de Deus no
inferno e da Sua luta com ele.

\bigskip

\begin{flushright}
\emph{11 de janeiro}
\end{flushright}

Café da manhã com Quint. Ele está planejando um
livro sobre Gide. Ele destaca o quanto os últimos livros de Gide
desconcertaram o público e o quão modesto foi seu sucesso nas livrarias,
com a exceção de \emph{L'école des femmes}.\footnote{\emph{L'école
  des femmes} é o romance de Gide sobre a desilusão da mulher no
  casamento, publicado em 1929. \versal{[N.~O.]}} O público francês não nutre nenhum
interesse no debate sobre questões sexuais e permanece ainda próximo aos
\emph{retroussés}\footnote{Em francês no original: saias levantadas. \versal{[N.~T.]}} do \emph{Le Sourire} e \emph{La Vie Parisienne} do
que ao fenômeno Oscar Wilde. De minha parte cabe dizer que a mais
notável fraqueza na discussão de Gide sobre a homossexualidade é a sua
tentativa de estabelecê"-la absolutamente como puro fenômeno natural, em
vez de, como Proust, tomar a sociologia como ponto crucial para o estudo
dessa inclinação. Mas também isso é conectado com cada constituição do
homem que parece, a meu juízo, ter cada vez mais sua fórmula no
contraste da sua puberdade atrasada, sua natureza atormentada e
dilacerada por um lado, e a linha pura, severa e figurativa dos seus
escritos, por outro.

\begin{flushright}
\emph{16 de janeiro}
\end{flushright}

\emph{Théâtre des Champs"-Élysées}. \emph{Amphitryon 38} de
Giraudoux, a única peça teatral que atualmente vale a pena ir em Paris
desde que o talentoso Pitoëff ocupou seu palco com uma apresentação de
\emph{Les criminels}.\footnote{O drama de Jean Giraudoux em três
  atos teve sua \emph{première} na \emph{Comédie des Champs"-Élysées} em
  novembro de 1929 e fui publicado no mesmo ano. \versal{[N.~O.]}} O 38 significa o
trigésimo oitavo tratamento deste material. Basta transformar apenas um
pouco essa palavra para abarcar o essencial. Na realidade, Giraudoux
contemplou a lenda como um material incrivelmente precioso que não
perdeu seu valor mesmo passando por tantas mãos. Seu valor foi
reforçado por um toque de esplendor antigo, estabelecendo então ao
escritor a tarefa, atualmente na moda, de encontrar um novo estilo
elegante, exibindo"-a de maneira inesperada. Quando comparada com
\emph{Orfeu} de Cocteau,\footnote{A ``tragédia'' em um ato de
  Jean Cocteau, exibida no teatro Georges Pitoëff em junho de 1926 e
  publicada em 1927. \versal{[N.~O.]}} também uma reelaboração de um objeto antigo,
nota"-se a maneira como Cocteau constrói o mito de acordo com os mais
novos princípios arquitetônicos. Giraudoux, entretanto, compreendeu como
renová"-lo de acordo com a moda. Ficamos com vontade de estabelecer a
equação Cocteau e Corbusier = Giraudoux e Lanvin.\footnote{O suíço Le
  Corbusier (pseudônimo de Charles"-Edouard Jeanneret"-Gris, 1887--1965) ganhou
  fama como nome fundamental da arquitetura modernista europeia. Jeanne
  Lanvin (1867--1946) com a sua casa \emph{haute couture} na Faubourg
  Saint"-Honoré foi uma estilista da elite da moda em Paris. \versal{[N.~O.]}}

De fato, a grande casa da moda de Lanvin proveu os figurinos e Valentine
Tessier,\footnote{Valentine Tessier (1892--1981) foi a atriz
  escolhida para as primeiras apresentações das peças de Giraudoux. Ela
  participou de vários filmes como o \emph{Madame Bovary} de Jean
  Renoir. \versal{[N.~O.]}} a atriz que interpreta Alcmene, representa um papel no qual
rufos, faixas, babados e fichus de seus vestidos são parceiros com ao
menos tanto talento e vivacidade quanto Mercúrio, Sósia, Zeus e
Anfitrión. Aceitemos que a moral, que se insinua tão virtuosa e sedutora
ao espectador, conduza a questão da fidelidade conjugal contra todo
refinamento olímpico do erotismo, e então teremos abarcado, em uma
palavra, a tendência eminentemente francesa do todo. Quão pensativo volta"-se
para casa em uma dessas suaves noites de inverno, e sente"-se um
tanto mais próximo das forças que fizeram com que esta cidade por
séculos tenha dedicado à moda a mais vasta organização intelectual e
econômica. Então leva"-se também de Giraudoux a certeza de que a moda
veste não apenas as mulheres como também as musas.

\begin{flushright}
\emph{18 de janeiro}
\end{flushright}

Berl.\footnote{Emmanuel Berl (1892--1976) foi um
  ensaísta francês. Em ambos textos citados aqui (1930), ele denuncia a
  petrificação sociocultural que ele vê como resultado da cultura
  burguesa moderna. \versal{[N.~O.]}} Este método primitivo ainda é o melhor: antes de
visitar um desconhecido, ler por uma meia hora seus escritos. Não foi em
vão. Em \emph{La mort de la pensée bourgeoise},\footnote{Em francês no original: \emph{Morte do pensamento burguês}. {[}\versal{N.~T}{]}} deparei"-me com a seguinte
passagem que iluminava não apenas Berl antecipadamente, como também
retrospectivamente minha conversa com Quint sobre Lautréamont. Esta
passagem sobre sadismo:

\begin{quote} 
O que de outro a obra de Sade ensina a reconhecer senão o quanto um espírito verdadeiramente revolucionário é estranho à ideia do amor? Na medida em que seus escritos não são representações de recalques, como seria natural para um prisioneiro; na medida em que eles provêm da intenção de ofender --- o que eu não acredito no caso de Sade, pois seria uma bastante tola aspiração para um prisioneiro da Bastilha ---, e na medida em que tais motivos não estão em jogo, suas obras brotam de uma negação revolucionária que se desenrolou até consequências lógicas extremas. Qual seria pois a utilidade de um protesto contra os poderosos, dado que se tenha aceitado o domínio da natureza sobre a existência humana, com tudo de revoltante que isso implica? Como se o ``amor normal'' não fosse o mais repulsivo de todos os preconceitos! Como se a procriação fosse algo de diferente da mais desprezível forma de subscrever o desenho fundamental do universo! Como se as leis da natureza, às quais o amor se submete, não fossem mais tirânicas e odiosas do que as leis da sociedade! O significado metafísico do sadismo repousa na esperança de que a revolta da humanidade possa atingir uma intensidade tão violenta que forçaria a natureza a mudar suas leis, e que diante da decisão das mulheres de parar de tolerar a injustiça da gravidez e os perigos e dor de dar à luz, a natureza seria compelida a encontrar outras maneiras de garantir a sobrevivência da raça humana na Terra. A força que diz não à família ou ao estado deve também dizer não a Deus; e assim como as regulações de funcionários e padres, a antiga lei do Gênesis também deve ser quebrada: ``do suor do teu rosto, comerás teu pão, na dor parirás teus filhos''. Pois o que constitui o crime de Adão e Eva não é o fato de que eles provocaram esta lei, mas o fato de que eles a suportaram.
\end{quote}

E agora no quarto em que o autor habita: assentos baixos exceto uma
cadeira de escritório. No lado estreito da direita um aquário que pode
ser iluminado. Acima do aquário, uma pintura de Picabia. Todas as paredes
são revestidas de verde e emolduradas com listras douradas. As estantes
revestidas com couro verde: Courier, Sainte"-Beuve, Balzac, Staël,
\emph{As mil e um noites}, Goethe, Heine, Agrippa d'Aubigné e Meilhac e
Halévy.\footnote{Paul"-Louis Courier (1772--1825) é um panfletário
  francês cheio de ironia e inventividade. Foi um oponente ardoroso da
  restauração dos Bourbons em 1814. Charles"-Augustin Sainte"-Beuve
  (1804--1869) foi um influente crítico, romancista e poeta, conhecido
  por haver delineado um método histórico que enfatiza a psicologia
  autoral e o papel do meio social do autor. Madame de Staël (nascida
  Germanie Necker, 1766--1817) alcançou a fama após a publicação de
  \emph{De l'Allemagne}, uma pesquisa abrangente da atividade cultural
  na Alemanha. O livro abriu uma ponte crucial entre o Romantismo alemão
  e a cultura francesa pós"-revolucionária. Théodore Agrippa d'Aubigné
  (1552--1630) foi um historiador francês e campeão da reforma
  protestante. Henri Meilhac (1831--1897) foi um dramaturgo francês. Foi
  colaborador de Ludovic Halévy (1834--1908) em uma série de libretos de
  opereta, muitos dos quais feitos para a música de Jacques Offenbach. \versal{[N.~O.]}}
Não é difícil notar que ele pertence ao tipo de pessoa que quer conversar apenas sobre o seu tema favorito, para depois, sem tolerar tantas interrupções, falar o que tem a dizer de memória. Para
ele, trata"-se, sobretudo, de continuar sua obra polêmica, expulsando a
pseudoreligiosidade da burguesia dos seus últimos esconderijos.
Descobre"-os, entretanto, nem no catolicismo com suas hierarquias e
sacramentos nem no Estado, mas no individualismo, na crença naquilo que
é incomparável, na imortalidade do indivíduo singular, o convencimento
de que a própria interioridade é o cenário de uma ação trágica única,
jamais repetida. Ele identifica a forma mais na moda desta convicção no
culto do inconsciente. Eu saberia, mesmo que ele não tivesse também me assegurado, que ele tem Freud ao seu lado na luta fanática que ele declarou contra este culto. E olhando para \emph{Le grand jeu},\footnote{Em francês no original: \emph{O grande jogo}. {[}\versal{N.~T.}{]}} a revista de
alguns membros dissidentes do grupo, que eu recém"-adquirira: ``eles são
seminaristas, nada além disso''. Agora apenas algumas insinuações
curiosas ao estilo de vida destas jovens pessoas: o seu \emph{refus},
como diz Berl, uma palavra que podemos traduzir como ``sabotagem''. Recusar uma entrevista,
refutar uma colaboração, negar uma foto, tudo isso eles tomam como prova
de seu talento. De modo muito espirituoso, Berl conecta isso com a
enraizada tendência à ascese, tão típica dos parisienses. Por outro lado,
ainda assombra aqui a ideia de um gênio incompreendido, que nós estamos a ponto de eliminar
radicalmente. O \emph{raté}\footnote{Em francês no original:
  ``ideia perdida'' ou ``fracassado''. \versal{[N.~T.]}} ainda tem
uma auréola e o esnobismo está a ponto de dourá"-la novamente. Por
exemplo, os financistas, que fundaram um clube para a aquisição de
quadros e que se comprometeram a não colocar estas aquisições novamente
no mercado por dez anos. A regra principal: não se deve pagar nada além
de 500 francos por nenhum quadro. Qualquer um que tenha custado mais do
que isso, já obteve sucesso; quem já tenha tido sucesso não vale nada.
Escuto"-o e não o contradigo. Mas não considero nem a atitude dos jovens
nem destes velhos esnobes totalmente incompreensível. Afinal de contas,
quantos procedimentos existem para que se tenha sucesso como artista e
quão poucos deles têm a mais remota relação com a arte!


\begin{flushright}
\emph{21 de janeiro}
\end{flushright}

Monsieur Albert.\footnote{Albert Le Cuziat
  serviu a Proust como uma importante fonte no submundo homossexual de
  Paris. Le Cuziat abriu seu bordel
  masculino no Hotel Marigny, 11, rue de l'Arcade em 1916. \versal{[N.~O.]}}
D{[}ausse{]} encontrou"-me de manhã no Hotel e pede"-me para reservar a
noite. Busca"-me às sete para apresentar"-me ao Monsieur Albert. Vai, como
disse, visitar o Monsieur Albert no seu \emph{établissement}.\footnote{Em francês no original: estabelecimento. {[}\versal{N.~T.}{]}} Descreve"-o
como um lugar extraordinário. Agora, este \emph{établissement} --- um
banho público no Quartier Saint"-Lazare --- é notável. Mas longe de ser
pitoresco. Os vícios graves, genuínos --- em outras palavras o
socialmente perigosos --- dão mostras de modéstia, evitam naturalmente
toda aparência de que se trata de um negócio, podem ter mesmo algo de
comovente. Proust, o amigo do Monsieur Albert, provavelmente estava a
par de tudo isso. Por isso, a atmosfera deste banho público é difícil de
descrever. Por exemplo, vizinho de porta com a família, mas com a
família pelas costas, como todos os vícios genuínos. O verdadeiramente
estranho, e isto pela noite inteira, é a familiar intimidade destes
jovens, em nenhuma contraposição com sua admirável \emph{franchise}.\footnote{Em francês no original: franqueza. [\versal{N.~T.}]} Em
todo caso, aqueles que eu vi ali têm, no modo mais extravagante e
precioso de apresentar"-se, ainda uma ingenuidade, uma rebeldia juvenil,
uma vivacidade e teimosia que me lembrou secretamente do meu de
internato. --- Primeiramente, o pátio que era preciso atravessar: uma
paisagem de piso de pedra e paz. Poucas janelas, que se podem ver do lado de
fora, iluminadas. Mas há luz atrás do vidro fosco do escritório de Albert e
em uma mansarda à esquerda que se ergue aos céus em forma de ameias. Chegamos
em três: Hessel e eu tivemos que admitir ser apresentados por Dausse como
tradutores de Proust. Confirmou"-se de modo surpreendente o que Hessel
dissera"-me sobre ele alguns dias antes: ele assemelha"-se a uma divindade
marinha, misturando"-se com tudo, fugindo de tudo. Aliás, isso fica mais
nítido nos figurinos de bonecas de porcelana. Porcelana é o material
mais alcoviteiro para casais enamorados. Dausse apresentou"-me como um %Benjamin que foi apresentado como um deus? Se não seria, apresentou-o, certo?
alcoviteiro deus fluvial de porcelana. O papel de protagonista em um
conjunto de figurantes ficou reservado a M{[}aurice{]}
S{[}achs{]}.\footnote{Maurice Sachs (pseudônimo de Maurice
  Ettinghausen, 1906--1945) é um romancista francês e ensaísta satírico.
  Ganhou notoriedade como resultado da sua vida caótica e escandalosa.
  Converteu"-se do judaísmo para o catolicismo,
  foi deportado e assassinado em uma prisão em Hamburgo no
  fim da Segunda Guerra. \versal{[N.~O.]}} Graças à sua vivacidade e à precisão bem
ensaiada de suas anedotas, ele contribuiu sobretudo para tornar
suspeitos para mim certos episódios e informações que se seguiram. E mal
ele havia entrado no carro comigo e com Hessel, começou a desdobrar como
que um inventário ou catálogo de amostras das principais histórias de
Albert, de modo que acreditei descobrir com algum desconforto também
aqui os sinais de uma \emph{tournée des grands"-ducs}\footnote{Em francês no original: expressão que poderia ser traduzida como ``noitada''. {[}\versal{N.~T.}{]}} avançada. Atrás do
vidro opaco, o espaço de recepção foi vedado por cortinas contra todo
aposento contíguo onde possa haver cenas indecentes. E o Monsieur Albert
atrás do balcão ou da bilheteria --- em resumo um \emph{arrangement}\footnote{Em francês no original: arranjos. {[}\versal{N.~T.}{]}} de
esponjas de banho, perfumes, \emph{pochettes surprise},\footnote{Em francês no original: caixinhas de surpresa. {[}\versal{N.~T.}{]}} bilhetes para
banho e bonecas em pose vulgar. Muito polido, muito discreto na maneira
de cumprimentar, e agradavelmente ocupado ao mesmo tempo com o resto do
trabalho do dia. Proust, se eu bem me lembro, conheceu"-o em 1912.
Naquele tempo ele não tinha mais que vinte anos. E vendo sua aparência
atual, pode"-se ter uma ideia de como ele devia ser bonito naquele tempo
em que foi valete do Príncipe de Radziwill, e antes disso do Príncipe
Orloff. A completa interpenetração do mais alto servilismo com extrema
decisão que caracteriza o lacaio --- como se a casta dos senhores não
tivesse prazer em ordenar seres que eles não sujeitam pelo comando ---
entrou em seus traços graças a uma certa fermentação, de modo que, por
vezes, ele parece um professor de ginástica. O programa da noite foi
ambicioso. Em todo caso, pensamos, depois do jantar, em reforçar a nova
amizade com uma visita ao segundo \emph{établissement} do Monsieur
Albert, o \emph{Bal des trois colonnes}.\footnote{Em francês no original: O Baile das três colunas. {[}\versal{N.~T.}{]}} Por algum momento, talvez
apenas por delicadeza, tivemos a impressão de estarmos incertos sobre
onde jantar. Então, rapidamente estávamos de acordo sobre o local na
\emph{rue Vaugirard}, onde Hessel e eu uma vez passeamos há três anos,
sem arriscar"-nos a espiar lá dentro. Hoje, havia ali alguns jovens
admiravelmente belos. Dentre eles um presumidamente autêntico príncipe
indiano que suscitou um interesse tão vivo em Maurice Sachs que ele
esqueceu de realizar seu propósito de levar o Monsieur Albert a fazer
confidências particularmente profundas. Mas também não estou certo se
elas interessar"-me"-iam mais do que alguns comentários muito secundários,
quase involuntários que acessoriamente entraram na nossa conversa. Pois
não me interessa muito saber o que aconteceria se alguém fizesse uso da
paixão de Proust --- algumas delas certamente assemelham"-se a uma famosa
cena com a Mademoiselle Vinteuil --- para a interpretação da sua obra.
Mas, ao contrário, para mim, a obra de um Proust parece conter o indício
de características gerais do sadismo, mesmo que muito secretas. E neste
caso, eu adoto como ponto de partida a ousadia de Proust na análise dos
acontecimentos mais triviais. Também da sua curiosidade que lhe é muito
próxima. Sabemos pela experiência que a curiosidade, sob a forma de uma
questão repetitiva, provando constantemente o mesmo conteúdo, pode
tornar"-se um instrumento nas mãos do sádico, o mesmo instrumento tão
inocente nas mãos da criança. A relação de Proust com a existência
possui algo dessa curiosidade sádica. Há passagens nas quais ele com
suas questões leva, de certo modo, a vida ao extremo, outras nas quais
ele arma"-se diante de um fato do coração como um professor sádico diante
de uma criança intimidada para obrigá"-la, com gestos ambíguos, um puxão
e beliscões, algo entre acariciar e atormentar, a revelar um segredo
suspeito que talvez não seja real. Em todo caso, as duas grandes paixões
deste homem, a curiosidade e o sadismo, convergem em não poder encontrar
qualquer aquietamento em nenhum resultado, em farejar em cada segredo
encaixotado um segredo menor, e nele um segredo ainda menor e assim por
diante até o infinito, com o efeito de que quanto mais seu tamanho
diminui, mais cresce em importância o que foi descoberto. --- Tudo isso
passava pela minha cabeça enquanto Monsieur Albert esboçava"-me o
desenvolvimento do seu conhecimento. Sabe"-se que Proust, pouco tempo
depois que eles se conheceram, equipou para ele uma \emph{Maison de
rendez"-vous}.\footnote{Em francês no original:  Casa para encontros. {[}\versal{N.~T.}{]}} Esta casa foi para o escritor ao mesmo tempo um
\emph{pied"-à"-terre}\footnote{Em francês no original: ``pé no chão''. Trata"-se de um pequeno apartamento ou casa, usada ou como segunda residência temporária ou como local para trabalhar. {[}\versal{N.~T.}{]}} e um laboratório. Aqui se instruiu,
provavelmente também pelo que via com os próprios olhos a respeito de
todas as especialidades de homossexualidade. Ali foram feitas as
observações que ele mais tarde utilizou na descrição de Charlus
atado. Ali Proust depositou os móveis da sua defunta tia, apenas
para lamentar em \emph{À l'ombre des jeunes filles en fleur}\footnote{Em francês no original: \emph{À sombra das moças em flor}, 1918. {[}\versal{N.~T.}{]}} o seu
indecoroso fim como \emph{ameublement}\footnote{Em francês no original: mobiliário. {[}\versal{N.~T.}{]}} de um bordel. --- Estava tarde, eu
tinha toda dificuldade de filtrar a voz fracamente articulada do
Monsieur Albert do ruído de um gramofone que era suplementado o tempo
todo com novos discos por uma beldade elegíaca que não podia dançar, pois
tinha um buraco na calça, e que estava ofendida pela eficaz rivalidade
com o príncipe indiano. Estávamos muito cansados para dar ao Monsieur
Albert a oportunidade de retribuir a hospitalidade \emph{chez lui}\footnote{Em francês no original: na casa dele. {[}\versal{N.~T.}{]}} --- ou
seja, no \emph{Trois colonnes}. Dausse conduziu"-nos para casa em seu carro.

\begin{flushright}
\emph{26 de Janeiro}
\end{flushright}

Félix Bertaux.\footnote{Félix Bertaux (1881--1948) foi tradutor e germanista francês. Benjamin cita seu livro:  \emph{Panorama de la littérature allemande contemporaine} de 1928. Félix é pai de Pierre Bertaux, também germanista e testemunha da entrevista que Benjamin fez com Gide. Seu diário com anotações do encontro em 1928 mostra que ele intermediou a conversa entre Benjamin e Gide (Cf. \versal{GS VII}--2, p.~257--269, p.~617--624). \versal{[N.~O.]}} Apesar de sua aparência sulista,
ele é da Lorena. Mas provavelmente possui sangue francês. Esta graciosa
história vem da época em que foi um jovem professor em Poitiers: a
\emph{Action française} organizou ali --- isso foi antes da
guerra\footnote{\emph{Action française} foi um movimento %repetir nota sobre a action?
  nacionalista e politicamente de direita de antes da Segunda Guerra.
  Fundado por Charles Maurras e Léon Daudet em 1898 dentro do
  contexto maior do caso Dreyfus, seu combate contra a democracia
  parlamentar, seu catolicismo militante e antissemitismo foram
  divulgados no seu jornal, igualmente chamado de \emph{L'action
  française}. \versal{[N.~O.]}} --- um encontro noturno. Bertaux apareceu com alguns
amigos dissidentes. Depois do discurso de propaganda do relator, o
presidente convidou aqueles que pensavam diferente para expressar
publicamente sua opinião. Grande foi a surpresa de Bertaux quando
percebeu que ninguém se apresentou. Com rápida decisão, saltou do seu
lugar no fundo da sala e depois de atravessar o tumultuoso espaço com
passos gigantes até o pódio, ocupou a tribuna por quarenta e cinco
minutos. Seu discurso provoca neste público de simpatizantes da
\emph{Action française} uma completa mudança. \emph{Il a renversé la
salle}\footnote{Em francês no original: expressão que significa que o locutor reverteu a posição inicial da maioria dos presentes em uma reunião. {[}\versal{N.~T.}{]}} como se costuma dizer. E o epílogo foi no dia seguinte em sua
classe. Ele estava a ponto de escrever alguma coisa no quadro negro e
deixou cair ao lado um pedaço de giz. Dez jovens jogaram"-se dos seus
bancos para pegá"-los. Surpreendido, Bertaux quis observar melhor a cena
e agora apenas aparentemente por engano, deixa novamente cair alguma
coisa. Desta vez, vinte jovens levantaram"-se, tal era a impetuosa
simpatia que o seu discurso havia conquistado na noite anterior. Ele
podia ter feito a mais brilhante carreira política, tornando"-se
prefeito ou deputado. Preferiu, entretanto, trabalhar nas suas horas vagas
em um léxico francês"-alemão que o ocupou pelos últimos quinze anos e que
é muito esperado. Nossa conversa girou em torno de Proust.
Proust e Gide --- no ano de 1919, a sua ordem de importância poderia ser
momentaneamente colocada em dúvida. Mas não para Bertaux, mesmo na
época. As objeções e reservas canônicas: que Proust não enriqueceu a
\emph{substance humaine},\footnote{Em francês no original: a substância humana. {[}\versal{N.~T.}{]}} que ele não elevou e refinou o conceito de
\emph{humanité}.\footnote{Em francês no original: a humanidade. {[}\versal{N.~T.}{]}} Bertaux cita a comparação com a qual Gide caracteriza o
procedimento de Proust perante o dia a dia: um pedaço de queijo
observado ao microscópio. Pode"-se dizer a si mesmo: \emph{Eh bien, c'est
ça ce que nous mangeons tous les jours?!}\footnote{Em francês no original: ``Bem, é isso
  que nós comemos todos os dias ?!'' \versal{[N.~T.]}} --- Replico
que certamente ele não elevou a humanidade, mas apenas analisou"-a. A sua
grandeza moral, entretanto, permanece em um campo completamente
diferente. Ele fez seu assunto, com uma paixão desconhecida por qualquer
outro escritor antes dele, a fidelidade às coisas que cruzaram a nossa
vida. Fidelidade a uma tarde, uma árvore, uma mancha solar sobre a
tapeçaria, fidelidade aos vestidos, móveis, aos perfumes ou paisagens
(eu devia ter notado que a descoberta do último volume no qual o caminho
para Guermantes e Méseglise entrelaçam"-se --- representa alegoricamente a
mais alta moral que Proust dispunha). Concedo que Proust, no fundo,
\emph{peut"-être se range du côté de la mort}.\footnote{Em francês no original: ``ele se
  coloca do lado da morte''. \versal{[N.~T.]}} \emph{Para além
do princípio de prazer}, a genial obra tardia de Freud, é provavelmente
o comentário fundamental de Proust. Meu companheiro quer admitir tudo
isso, mas permanece extremamente reservado, \emph{se range du côté de la
santé}.\footnote{Em francês no original: ``ele se coloca do lado da
  saúde''. \versal{[N.~T.]}} Contento"-me com a indicação de que
se Proust, em sua obra, procurou e amou alguma saúde, em todo caso,
não é aquela vigorosa do homem adulto, mas a indefesa e frágil da
criança.

\begin{flushright}
\emph{4 de fevereiro}
\end{flushright}

Adrienne Monnier.\footnote{Esta livraria de
  Adrienne Monnier (1892--1968) chamada na verdade de \emph{Maison des
  amis des livres} estava localizada em frente da Sylvia Beach's
  Shakespeare and Company, na 7, rue de l'Odéon. As duas lojas serviram
  de fórum para artistas modernistas como Paul Valéry, André Gide e
  James Joyce. \versal{[N.~O.]}} Pouco depois das três, abro a porta de \emph{Aux amis
des livres}, \emph{7, rue de l'Odéon}. Sinto uma certa diferença de
outras livrarias. Certamente, não deve ser um sebo. Adrienne Monnier
parece ocupar"-se apenas com novos livros. Mas a aparência é ainda menos
multicolorida, menos movimentada ou distraída do que em outras lojas.
Sobre a ampla mesa domina uma tonalidade quente e pálida de marfim. Ela
vem talvez da proteção transparente de muitos livros daqui --- todos em
moderna primeira edição em impressão de luxo. Dirijo"-me para a mulher que me está próxima. Tive a sensação de que caso ela fosse Adrienne Monnier, isso significaria o grande desapontamento da minha fugidia e superficial expectativa de conhecer uma linda jovem. Uma mulher corpulenta, cabelos
loiros, com olhos cinza"-azuis muito claros, totalmente vestida em um
resistente tecido cinza com corte semelhante ao de freiras. A frente do
vestido é coberta com botões ornamentados, com debrum fora de moda. É
ela! Imediatamente senti estar diante de umas das pessoas das quais
nunca se pode mostrar respeito suficiente, e que sem contar
aparentemente com tal respeito, todavia não recusará ou minimizará tal
respeito por nenhum instante. Notável é que esta mulher, como ela me
disse, tenha cruzado o caminho de Rilke, que viveu em Paris por tanto
tempo, apenas duas vezes. Posso pensar que ele deveria ter manifestado a
mais profunda inclinação por um ser de tal pureza rústica, por tal
criatura cômica de mosteiro. Aliás, ela fala muito bem dele e diz ``ele
parece ter deixado a todos que o conheceram um pouco esta sensação:
estar profundamente envolvido com tudo o que faziam''. E durante a sua
vida pôde dar"-lhes essa sensação totalmente sem palavras, graças à sua
mera existência. Estamos sentados no seu escritório estreito, coberto de
livros logo no primeiro plano da loja. Naturalmente, nosso primeiro
tópico de conversa é \versal{I.M.S.};\footnote{\versal{I.M.S.} era o pseudônimo com o
  qual Adrienne Monnier publicou alguns dos seus textos em prosa. \versal{[N.~O.]}} em
seguida o assunto versou sobre as virgens sábias e as virgens tolas. Ela
fala de diferentes formas de \emph{vierge sage} --- aquela de Estrasburgo
que a inspirou a escrever a peça que eu li, e aquela da Notre"-Dame de
Paris ``\emph{qui est si désabusée, si bourgeoise, si parisienne --- ça
vous rappelle ces épouses qui ont appris à se faire à leur mari et qui
ont cette façon de dire: Mais oui, mon ami; qui pensent un peu plus
loin}''.\footnote{Em francês no original: ``Que é tão desiludida, tão burguesa, tão
  parisiense --- isto lembra estas esposas que aprenderam a se acostumar
  aos maridos e que têm esse modo de dizer, `Mas sim, meu amigo'; que
  pensam um pouco mais longe''. \versal{[N.~T.]}} ``E agora''
disse"-lhe Paulhan quando ele conheceu a \emph{vierge sage}, ``a senhora
vai escrever"-nos uma `\emph{vierge folle}'". Mas não! Há apenas uma
\emph{vierge sage}, embora elas estivessem reunidas em sete, enquanto
que as \emph{vierge folle} são muitas, isso seria o conjunto inteiro.
Aliás, sua \emph{Servante en colère}\footnote{Em francês no original: \emph{Criada encolerizada}. {[}\versal{N.~T.}{]}} foi já justamente uma \emph{vierge
folle}. E então ela disse algo que me daria a impressão de estar na
essência do seu mundo, mesmo que eu não tivesse lido nada das suas
coisas. Ela discursava sobre a integridade das sábias, e sobre como
nesta integridade não pode ser ignorada uma aparência, um vislumbre de
hipocrisia. As virgens sábias e as tolas --- evidentemente o que ela fala
não é no sentido da igreja, mas na verdade distingue"-se apenas
minimamente --- são infinitamente próximas entre si. Enquanto falava,
suas mãos movimentaram"-se com um balanço tão insinuante e sedutor que eu
vi, na minha fantasia, um portal rodeado por todas as quatorzes virgens,
repousando como pedestais de pedra movente, e um balanço contínuo de uma
para a outra que expressava o sentido de suas palavras de forma mais
perfeita. De forma totalmente natural chegamos assim a falar de
\emph{coincidentia oppositorum}, sem que este termo tenha sido
explicitamente mencionado. Quis dirigir a conversa para Gide, de quem eu
sabia que ela foi ou é próxima, já que aqui nestes espaços aconteceu o
célebre leilão no qual Gide confirmou sem nenhuma dúvida a separação com
aqueles amigos que recusaram a seguir o seu \emph{Corydon},\footnote{O \emph{Corydon: quatre dialogues socratiques} de Gide foi publicado
  anonimamente em trechos em 1911 e na versão completa em 1920. O texto
  apareceu com o nome do autor apenas em 1924. Ele representa a
  exploração mais explícita da sua homossexualidade. Por isso, esta
  publicação custou"-lhe muitos amigos. \versal{[N.~O.]}} leiloando exemplares com a
dedicatória para eles na sua livraria. ``Não, sob nenhuma
circunstância'' --- ela não quer saber de mencionar Gide aqui. É verdade
que Gide possui a resposta afirmativa e a negativa, mas uma após a outra
e no tempo, não na grande unidade, na grande calma que se pode encontrar
primeiramente ainda junto aos místicos. Neste momento, ela menciona
Breton, e isso realmente interessa"-me. Ela diz que Breton é um caráter
tenso, explosivo, em cuja proximidade a vida é impossível:
``\emph{quelqu'un d'inviable, comme nous disons}''.\footnote{Em francês no original: ``alguém inviável, como nós dizemos''. \versal{[N.~T.]}} Mas como
aparece"-lhe extraordinário o primeiro Manifesto surrealista, e como
ainda é possível achar coisas esplêndidas no segundo. Confesso"-lhe que
reparei sobretudo na virada ocultista do segundo, em um sentido
desagradável. Ela também não quer saber da veneração das cartomantes na
\emph{Lettre aux voyantes}\footnote{A \emph{Lettre aux voyantes}
  acompanhou uma nova edição do \emph{Manifeste du surréalisme},
  incluindo um frontispício de Max Ernst publicado pela edição Kra em
  1929. \versal{[N.~O.]}} de Breton. Ela lembra"-se ainda do tempo em que Breton com
Apollinaire apareceram"-lhe na butique e de como Breton era submisso a
Apollinaire, obediente aos seus mínimos acenos. ``\emph{Breton faites
cela, cherchez ceci}''.\footnote{Em francês no original: ``Breton faça isso, procure
  aquilo''. \versal{[N.~T.]}} Muitos clientes surgiram. Não quis
forçar esta situação no canto da mesa, temo também estar atrapalhando
o seu trabalho. Mas depois fui novamente cativado pelo modo com o qual
ela lidou com a minha velha idiossincrasia de defender veementemente fotografias de obras de arte. A princípio, ela parece surpresa com minha
afirmação sobre como é mais fácil ``desfrutar'' um quadro --- sobretudo uma
escultura, e mesmo a arquitetura --- fotograficamente do que na
realidade. Mas quando eu continuei e chamei essa forma de se ocupar com
arte de pobre e enervante, ela teimou: ``as grandes criações'', ela disse,
``não podem ser vistas como a obra de um pessoa. São formas coletivas,
tão poderosas que a condição de desfrutá"-las consiste em simplesmente
reduzi"-las. No fundo, os métodos de reprodução mecânica são uma técnica
de redução, ajudando os humanos a obter o grau de domínio sobre a obra,
sem o que não se pode fruí"-la''. E assim eu troco uma foto da
\emph{vierge sage} de Estrasburgo que ela prometera"-me no início do
encontro por uma teoria da reprodução que talvez me fosse ainda mais
valiosa.


\begin{flushright}
\emph{7 de fevereiro}
\end{flushright}

\emph{Mélo} de Bernstein no Ginásio.\footnote{A obra do dramaturgo francês Henri Bernstein (1876--1953) --- peças como
  \emph{Félix} (1926) e \emph{Mélo} (1929) --- oferece um registro dos
  dilemas morais e eróticos da França burguesa do início do século \versal{XX}. \versal{[N.~O.]}}
Pode"-se encontrar a arquitetura desse teatro atualmente apenas ainda nas
fachadas de velhos cassinos de cidades termais. E quando é iluminada
do início ao fim no estilo da Grande Ópera de Paris --- de
modo que todos os raios e feixes projetados tornem"-se ribaltas das quais
holofotes colocam as colunas de \emph{primadonnas} em luz correta, ao
passo que o escuro do céu noturno forma um arco em nichos sobre elas ---,
a cena exterior da fachada supera consideravelmente a do palco no
interior.


\begin{flushright}
\emph{10 de fevereiro}
\end{flushright}

Adrienne Monnier pela segunda vez. Nesse ínterim,
a topografia da \emph{rue de l'Odéon} tornou"-se mais familiar. Eu lera
no seu perdido livro de poesia \emph{Visages} os belos versos para sua
amiga Sylvia Beach que possuía uma pequena livraria em frente da dela,
onde o inglês Joyce experimentou a movimentada história da sua primeira
aparição. ``\emph{Déjà}'', ela escreve ---``\emph{Déjà midi nous voit,
l'une en face de l'autre/ Debout devant nos seuils, au niveau de la rue/ Doux fleuve de soleil qui porte sur ses bords/ Nos
Libraires}''.\footnote{Em francês no original: ``Já a tarde nos vê, uma em frente da outra/
  em pé diante da porta, ao nível da rua/ rio suave de sol que incide
  sobre suas beiras/nossas livrarias''. \versal{[N.~T.]}} Ela
entra na loja por volta das 15h30, exatamente um minuto depois de mim,
envolta em uma manta de lã cinza, com um ar de avó cossaca, tímida e
muito determinada. E mal nos sentamos ainda no mesmo lugar que antes,
ela acena para alguém através da janela. Ele entra. É Fargue. E como
trata"-se de um dia encantador em Paris, um pouco frio com sol, e nós
todos sentíamo"-nos plenos com a beleza do lado de fora, eu também
contei como vagueei por Saint"-Sulpice, e Fargue confidenciou que
gostaria de morar ali. Neste momento ambos iniciaram um maravilhoso
dueto em torno dos padres de Saint"-Sulpice, um dueto que reverberou sua
velha amizade. Fargue diz \emph{Ces grands corbeaux civilisés}\footnote{Em francês no original:
  ``Estes grandes corvos civilizados''. \versal{[N.~T.]}} e
Monnier: ``toda vez que atravesso a praça, é tão bonito, mas sobre ela
sopra um vento gelado no inverno e os padres saem da igreja. Eu sinto
como se a tormenta devesse pegar suas batinas e levantá"-las ao ar''.
Deve"-se ter visto aquele negro vai e vem, parar e fluir ali no bairro, e
como este silêncio sacerdotal mesclava"-se com o puro silêncio de muitas
livrarias para ter a chave deste incomparável bairro e sobretudo a praça
que é seu coração. Depois ambos retomaram suas desavenças amigáveis que
devem vir de longa data. Fargue paga por sua preguiça: ele não escreve
nada. ``\emph{Que voulez"-vous, j'ai pitié de ce que je fais. Ces pauvres
mots, je les vois, à peine écrits, qu'ils s'en vont, traînant, boitant,
vers le Père"-Lachaise}''.\footnote{Em francês no original: ``O que vocês querem, sinto"-me
  aflito pelo que escrevi. Vejo estas pobres palavras, escritas com
  dificuldade, como elas vão embora arrastando"-se, claudicantes para o
  cemitério de Père"-Lachaise''. \versal{[N.~T.]}} Nossa conversa
seguinte é dominada pela tradução de Joyce que ela publicou. Ela
comunicou as circunstâncias que a levaram a um tão surpreendente
sucesso. Alguma menção a Proust: ela fala sobre a repugnância que sua
transfiguração da alta sociedade despertou"-lhe, do protesto rebelde que
lhe impedia de amar Proust, e então com fanatismo, quase com ódio, de
Albertine que em uma medida tão absurda que é \emph{ce garçon du Ritz}\footnote{Em francês no original: ``Essa rapaz do Ritz''. {[}\versal{N.~T.}{]}}
--- Albert --- cujo corpo atarracado e andar masculino ela sentia sempre
em Albertine. A moral de Proust impedia de amá"-lo, mas também somente de
querer amá"-lo. O que ela disse, permite me falar sobre as dificuldades
que Proust encontra com o leitor alemão; o quanto essas dificuldades
exigem estudos introdutórios sobre o escritor para serem superadas, o
quão pouco há de tais estudos entre nós e, no fundo, mesmo na França. Sua
surpresa com respeito a essa afirmação é ocasião suficiente para
indicar"-lhe brevemente minha imagem de uma interpretação de Proust.
Expliquei"-lhe que o que ainda resta por descobrir não é o lado
psicológico, nem a tendência analítica, mas o elemento metafísico de
seus trabalhos. As cem portas que abrem o acesso ao seu mundo restam
inexploradas: a ideia do envelhecer, o parentesco entre humanos e
plantas, sua imagem do século dezenove, sua sensibilidade para o que
está mofado, resíduos e coisas do gênero. E de como eu fiquei muito
convencido de que para compreender Proust, deve"-se em geral partir do
fato de que seu objeto é o lado reverso: \emph{le revers --- moins du
monde que de la vie même}.\footnote{Em francês no original: ``O reverso --- menos do mundo do que da própria vida''. \versal{[N.~T.]}}

Depois, na mesma tarde, conversa com Audisio, a quem estou em
condições de dizer muitas coisas sobre seu \emph{Héliotrope}. Isso lhe
alegrou.\footnote{Gabriel Audisio (1900--1978) é poeta e ensaísta
  francês e era funcionário público tanto na Argélia quanto na França.
  Sua obra celebra a luz e o sol do mundo mediterrâneo. Seu romance
  \emph{Héliotrope} foi publicado em 1928. \versal{[N.~O.]}} Ele conta as circunstâncias
sob as quais ele escreveu o livro, o que nos levou a falar sobre
trabalhar no clima do Sul. Estamos totalmente de acordo sobre o absurdo
da opinião de que o sol do Sul seja um inimigo da concentração
necessária para o trabalho intelectual. Audisio revelou seu plano
de escrever uma \emph{Defense du soleil}.\footnote{Em francês no original: ``Defesa do sol''. {[}\versal{N.~T.}{]}} Entramos em uma reflexão sobre os
diferentes tipos de contemplação mística que nasce sob o nórdico céu da
meia"-noite e sob o céu sulino do meio"-dia. Jean Paul por um lado e a
mística oriental, por outro.\footnote{Jean Paul Richter (1793--1825)
  escreveu uma série de romances extravagantes e imaginativos que, na sua
  combinação de fantasia e realismo, continua a desafiar as
  categorizações. \versal{[N.~O.]}} A atitude romântica do nórdico, que tenta adaptar"-se
ao infinito tecendo o seu mundo de sonhos, e o rigor dos sulistas que,
de modo bastante desafiador, entra em concorrência com a infinitude do
azul do meio"-dia para criar algo igualmente duradouro. Essa conversa
leva"-me a pensar na minha temporada em Capri, em pleno julho, quando
escrevi as primeiras quarenta páginas do livro sobre o barroco alemão:
não tinha nada além de uma pena, tinta, papel, uma cadeira, uma mesa e o
calor do meio"-dia. Essa competição com duração infinita, cuja imagem
evoca urgentemente o céu do meio"-dia do Sul como o céu noturno evoca
aquela de um espaço infinito, perfaz o caráter reservado e construtivo
da mística do Sul que encontra sua expressão arquitetônica, por exemplo,
nos templos Sufi.

\begin{flushright}
\emph{11 de fevereiro}
\end{flushright}

Café da manhã com Quint. Não teria conversado
tanto diretamente sobre a mais recente situação do surrealismo, se
tivesse me dado conta de que a editora Kra --- da qual Quint é diretor ---
publicou o segundo Manifesto surrealista. Para Quint, a parte que diz
respeito ao confronto com os primeiros membros do grupo é a mais fraca.
Aliás, eu mal precisaria ter a necessidade da sua confirmação de que o
movimento original alcançou seu fim. Mas é o momento exato para colocar
com segurança retrospectivamente alguns fatos. E o primeiro e sobretudo
mais glorioso: o surrealismo reagiu violentamente --- o que na França
testemunha a saúde de seus intelectuais --- àquela mistura de poesia e
jornalismo que começou a tornar"-se a fórmula da atividade literária na
Alemanha. Ele deu ao pensamento da \emph{poésie pure},\footnote{Em francês no original: poesia pura. {[}\versal{N.~T.}{]}} que ameaçou
infiltrar"-se na academia, uma ênfase demagógica quase política.
Reencontrou a grande tradição da literatura esotérica que na verdade
está longe da \emph{art pour art}\footnote{Em francês no original: arte pela arte. {[}\versal{N.~T.}{]}} que para os escritores é uma prática
secreta salutar, uma receita de escrita. Ele intuiu a íntima
inter"-relação existente entre diletantismo e corrupção que formou a base
do jornalismo. Com paixão anárquica, ele tornou impossível o conceito de
nível, de média decente, na literatura. E daqui prosseguiu em seus
esforços de sabotagem a regiões sempre mais amplas da vida pública e
privada, até chegar finalmente à correção política desta atitude: sua força de
separação, ou melhor de eliminação, torna"-se evidente e eficiente. Mas
também estamos de acordo quanto ao fato de que uma das grandes
limitações do movimento foi a preguiça de seus líderes. Se encontrarmos
Breton e fizermos a pergunta sobre o que ele faz, responderá assim: ``\emph{Rien.
Que voulez"-vous que l'on fasse?}''.\footnote{Em francês no original: ``Nada. O que você espera que eu faça?''. \versal{[N.~T.]}} Há muita afetação ali, mas
também muito de verdadeiro. E, sobretudo, Breton não presta contas com o
fato de que ser aplicado à parte da cultura burguesa"-filistina tem um
lado mágico.

Antes disso, um passeio de despedida pelos \emph{Champs"-Él-\\ysées}. A superfície
plana dos dias é como um espelho que agora, quando eu chego às suas
bordas, resplandece em todas as cores do prisma. Juntamente com o frio,
chega também a primavera, e enquanto descemos o \emph{Champs"-Élysées} como um
declive de montanha nevada com o pulso batendo e bochechas coradas,
paramos de repente diante de um pequeno gramado atrás do \emph{Théâtre
Marigny}, onde pressentimos a primavera. Atrás do \emph{Théâtre Marigny} algo
muito imponente está sendo construído: uma cerca alta e verde o rodeia,
e atrás dela erguem"-se andaimes. Ali eu vi essa magia: uma árvore que
lançava suas raízes no lugar cercado, alongando lateralmente para fora
seus ramos desfolhados um pouco sobre a cerca. A esses ramos, adapta"-se a
rígida cerca com rendas vazadas tão intimamente como uma gargantilha em
uma moça. No meio da massiva marcenaria da obra, essa silhueta estava
desenhada como que da mão segura de uma alfaiate. O sol chegava
brilhando da direita, e como o cume da Torre Eiffel --- o meu símbolo
preferido de prepotência --- quase que se dissolve no esplendor da luz, ele
era então a imagem da reconciliação e da purificação, que a primavera
evoca. O brilho do sol dissolveu os palácios de aluguel que de longe
deixavam entrever a \emph{rue La Boétie} e a \emph{rue Montaigne},
tornando"-se poderosa tinta pastel de cor marrom dourado que o Demiurgo
espalhou sobre a palheta que é Paris. E enquanto caminhava senti meus
pensamentos confundirem"-se de modo tão caleidoscópico --- a cada passo
uma nova constelação, velhos elementos desapareciam e novos
desconhecidos vinham tropeçando; muitas figuras, mas se uma delas
estiver presa, ela se chamará ``uma frase'' --- então, entre milhares
delas, formei também a que eu esperei tantos anos --- a frase que
envolveu completamente todo o milagre que a Madeleine (a verdadeira, não
a proustiana) foi para mim desde o primeiro momento: no inverno a
Madeleine é uma grande estufa que com sua sombra aquece a \emph{rue
Royale}.
