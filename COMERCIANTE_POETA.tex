\chapter{O comerciante no poeta\footnote[*]{\versal{GS} vol. \versal{III}, p.~46-48. Tradução de Carla Milani Damião a partir do
  original alemão. Resenha originalmente publicada na revista
  \emph{Literarische Welt} em 15/10/1926. [\versal{N. E.}]}}

\epigraph{Todo autor tem um jeito próprio de vender sua mercadoria; --- de minha
parte, não gostaria de, diante da morte, ficar numa loja escura,
regateando e pechinchando alguns tostões a mais ou a menos.}{Sterne, \emph{Tristram Shandy}, \versal{I}, 9.\footnotemark}

\footnotetext{A   passagem, no texto original em inglês, não corresponde integralmente à
  citação da obra de Sterne feita por Benjamin: ``--- Every author has a
  way of his own in bringing his points to bear; --- for my own part, as I
  hate chaffering and higgling for a few guineas in a dark entry; --- I
  resolved within myself, from the very beginning, to deal squarely and
  openly with your Great Folks in this affair, and try whether I should
  not come off the better by it.'' Sterne, Laurence. \emph{The Life and
  Opinions of Tristram Shandy}, Gentleman (p. 8). Edição do Kindle.
  {[}\versal{N. T.}{]}}
"
É mais ou menos aceitável incluir o poeta na condição de produtor entre
os ``produtores'', as ``classes produtoras''. Obviamente, devemos ignorar
quanta mesquinharia e impudência se escondem sob a imagem do
``trabalhador espiritual'' (como percevejos de fogo sob uma pedra). Mas o
fato de que os poetas sejam retratados como comerciantes é novo, tudo
menos do que frase feita e um modo de falar, sob o qual nesse momento em
Paris --- esta escola única de boa conduta na crítica --- tenta"-se criar
uma variação elegante e adequada da ``característica'' usual de poetas.

Não é sabido que o poeta tenha realmente mais do comerciante do que se
gostaria de admitir --- às vezes mais do que do produtor? Sem dúvida, há o
suficiente daqueles que, como comerciantes grandes ou pequenos, vendem
tecidos antiquíssimos, nobres ou novidades da moda para as pessoas e,
além disso, utilizam todo o aparato do comerciante: o prefácio
publicitário e a decoração da vitrine dos pequenos capítulos, o ``eu''
servil atrás do balcão e os cálculos de tensão, o descanso do domingo
após cada sexta ideia e o que recebe os pagamentos. Os escritores, no
entanto, têm mais a ganhar com essa visão do que com uma mística da
produção que na maioria das vezes corresponde à do taverneiro.

Tudo isso não está escrito no livro do qual se trata. Porque este tem
a vantagem de não ter texto. \emph{Em breve \ldots{} 62 lojas literárias,}
apresentadas por Pierre Mac Orlan; Henri Guilac, arquiteto; Simon Kra,
empreendedor. (``\emph{Prochainement ouverture \ldots{} de 62 boutiques
littéraires} presentées par Pierre Mac Orlan'').\footnote{Trata"-se do
  livro editado por Simon Kra em 1925, ilustrado pelo desenhista Henri
  Guilac, colorido por Jacomet, com apresentação de Pierre Mac Orlan.
  São 62 caricaturas de escritores franceses como vendedores, tendo o
  nome de obras associadas a lojas com toldo como as de um mercado e
  fachadas. {[}\versal{N. E.}{]}} A capa do livro mostra isso pincelado sobre uma
parede verde de madeira, dito em alemão: Henri Guilac desenhou este
livro, Pierre Mac Orlan prefaciou e Simon Kra publicou. As imagens
apresentam 62 poetas franceses em frente de suas lojas imaginárias. Todo
alemão, nesse caso, esperaria uma sátira fulminante. Decepcioná"-lo é o
tipicamente parisiense neste livro. Porque nessas páginas, todas
coloridas à mão de maneira muito limpa e em cores vibrantes, há uma
\emph{candeur},\footnote{Em francês no original: candura. {[}\versal{N. E.}{]}}
uma ternura que deve torná"-las um puro prazer para quase todos os 62 que
são por ela afetados. Eles, à frente da porta na espera de seus
clientes, olham através da vitrine ou se inclinam sobre o balcão. Quão
óbvio, no entanto, que ninguém apareça! E isso já na França! Quão
desertas não pareceriam essas lojas para nós alemães! Pintar clientes
também não teria dado certo ou cada milhar da tiragem deveria ter sido
representado por um homenzinho que compra? Seja como for, a rua está
vazia. Gide criou para si uma \emph{Delicatesse} com o seu trabalho
juvenil, os \emph{Nourritures Terrestres},\footnote{\emph{Frutos da
  terra,} de 1897. {[}\versal{N. E.}{]}} que vende vinhos das \emph{Caves
du Vatican}.\footnote{\emph{Os subterrâneos do vaticano}, de 1914.
  {[}\versal{N. E.}{]}} Paul Morand, como proxeneta, posta"-se na entrada de um
estabelecimento duvidoso, cuja lanterna vermelha indica \emph{Ouvert la
nuit}.\footnote{\emph{Aberto à noite}, livro de 1922. {[}\versal{N.E.}{]}} Lê"-se
``F. Carco'' --- especialista em romances de Apache --- em uma marquise
verde, em cuja escassa proteção \emph{Rien qu'une femme}\footnote{\emph{Nada
  mais que uma mulher,} de 1921. {[}\versal{N. E.}{]}} mostra seus seios na
janela. Casa após casa enfileiram"-se nesta cidade literária paradisíaca:
loja de malas (Colette), perfumaria, loja de câmbio, padaria,
restaurante ao ar livre (Eugène Montfort) e agência de viagens (Charles
Vildrac). Ao final, passamos pelo \emph{banlieue},\footnote{Em francês no original: arredores,
  subúrbio. {[}\versal{N. E.}{]}} onde se encontra toda
uma feira de barracas, uma quermesse com uma tenda de loteria, um
gabinete de curiosidades anatômicas, um estande de charlatão, uma
barraca de arremessar bolas (com o delgado Jean Cocteau como dono), uma
barraca com livros antigos \emph{Les livres du Temp},\footnote{\emph{O
  livro do tempo}, 3 volumes, entre 1913-1930. {[}\versal{N. E.}{]}} em frente à
qual é posto Paul Souday, o crítico literário do \emph{Temps}.

Ouvimos sobre um plano antigo e abandonado para realmente construir
feiras de mercado literários e dessa maneira plantar o próprio poeta
nelas. Com Mac Orlan lamentamos que algo assim não tenha acontecido na
\emph{Exposition des Arts et Métiers}. Com certeza, tem razão o
prefácio, no qual avisa aos escritores que eles desconhecem até que
ponto o que fazem parece ser irrelevante para o povo e que um dia eles
teriam que pagar por isso.

Tal brincadeira engenhosa com as coisas da literatura poderia mudar
isso, se, com todo o charme que possui, não permanecesse tão particular
e tão isolada. Por isso, devemos, silenciosamente, alegrarmo"-nos por
ela, porque a andorinha, que sozinha não faz verão, é o bicho de
estimação de nossa época.
