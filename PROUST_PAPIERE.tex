%Texto com marcações com chaves! Repensar

\chapter*{Proust-Papiere\\ \emph{documentos sobre Proust}\footnote[*]{``Proust-Papiere'', in \versal{GS II}-3, p. 1048-1065. Tradução de Carla M. Damião. Documentos, anotações ou materiais sobre o ensaio ``Imagem de Proust''. Segundo os editores alemães (\versal{GS II}-3, p.~1047), a maior parte desses documentos foi encontrada num envelope, no dorso do qual se lia: ``Proust"-Papiere''. Os textos foram editados na ordem em que se encontravam, contendo algumas repetições, correções e reescritas de trechos. Importante notar que nesse agrupamento de textos é notável perceber que os editores mantiveram, dos manuscritos, inúmeras exclusões de trechos riscados pelo próprio Benjamin. O uso de chaves \{\} sinaliza quais são os trechos. Ao final de cada manuscrito a sigla ``Ms'' indica o número do manuscrito no arquivo de origem, na época dessa edição, Benjamin"-Archiv Theodor Adorno sediado em Frankfurt a.~\versal{M}. atualmente no Walter Benjamin"-Archiv, Akademie der Künste em Berlim. Há também o uso de colchetes [] para eventuais comentários dos editores para irregularidades nos textos.}}
\addcontentsline{toc}{chapter}{Proust-Papiere: \emph{documentos sobre Proust}}
\hedramarkboth{Proust-Papiere: documentos sobre Proust}{}

\section{Materiais relativos ao ensaio sobre Proust}


\{Em conexão com o relacionamento íntimo de Proust com os Bibescos, sua
linguagem secreta,\footnote{A linguagem secreta refere"-se, por exemplo, \label{bibescos}
  ao uso de palíndromos com os nomes próprios de qual Marcel era Lecram
  e os Bibescos eram Ocsebib. Proust era próximo do príncipe Antoine
  Bibesco, que lhe teria servido de modelo para a personagem Robert de
  Saint"-Loup.} a ineficiência de sua correspondência através de cartas.
A atitude de Proust diante da sociedade mundana era de uma atenção
detetivesca e de curiosidade. Ela era para ele um clã de criminosos,
organização de conspiradores, com a qual nenhuma outra poderia
comparar"-se. Ela era, para ele, a Camorra dos consumidores. Ela exclui
do seu mundo tudo o que tem participação na produção, exige pelo menos
que essa participação se esconda vergonhosa e graciosamente atrás do
gesticular que os perfeitos \emph{professionels}\footnote{Em francês no original:
  profissionais. \versal{[N. T.]}} do consumo o exibem. O estudo do esnobismo deveria
ganhar, para Proust, um significado incomparável, porque essa atitude
não significa nada a não ser a observação consequente, organizada,
reforçada da existência do ponto de vista quimicamente puro do
consumidor. E, por causa dessa encenação fantástica e satânica, tanto a
mais remota, quanto a mais primitiva lembrança das forças produtivas da
natureza deveria ser banida; por isso, para ele, até no amor as ligações
pervertidas\footnote{No ensaio ``Imagem de Proust'', a palavra
  é substituída de ``ligações pervertidas'' (\emph{die pervertierten Bindungen}) para ``ligações invertidas'' (\emph{die
    invertierte Bindungen}). \versal{[N. E.]}} eram mais úteis do que as normais. Em algum
lugar da obra, quando ele está prestes a evocar particularmente de modo
intenso o mundo de Sodoma e Gomorra, ele declara, mais tarde, querer
falar sobre a necessidade da qual ele capturou justamente este assunto.
Um programa que ele, naturalmente, nunca cumpriu. Porque explicar"-se
justamente nesse lugar teria significado, para ele, em retirar a própria
obra de seus batentes. Sodoma e Gomorra são, de fato, os parafusos da
dobradiça com os quais a porta para o inferno da existência puramente de
zangão, de gozo absoluto está, em seus batentes, fixada. Certamente um
inferno: porque isso é o último resultado para análise irreverente dos
prazeres, que sua \emph{œuvre}\footnote{Em francês no original:
  obra. \versal{[N. T.]}} executa: o gozo absoluto, quimicamente puro da existência é uma
conexão extraordinariamente fugaz de sofrimento, dor, humilhação e
doença, e seu aroma é decepção.\}

\{Proust não se cansou do treinamento que o contato nesse círculo
mundano de criminosos exigia. De fato, ele praticou o tempo todo, sem
ter que violar muito sua natureza, para torná"-la tão flexível, engenhosa
e impermeável como ele haveria de se tornar em razão de sua tarefa.
Mais tarde, a mistificação, a cerimoniosidade, tornaram"-se um elemento de
sua própria natureza, que suas cartas às vezes se tornam um sistema
inteiro de parênteses e não apenas em sentido gramatical: cartas que de
maneira infinitamente espirituosa e versátil revelam mesmo assim um
certo parentesco com um esquema antigo: ``Muito honrada senhora! Acabo
de perceber que ontem esqueci minha bengala em sua casa, peço a senhora
que a entregue ao portador dessa mensagem. \versal{P.S.}: Perdoe o incômodo
desajeitado. Acabei de encontrá"-la''. Quão inventivo era ele em apuros:
certa vez ele apareceu tarde da noite na casa da duquesa
Clermont"-Tonnerre. Ele condiciona sua permanência se alguém trouxesse
seus medicamentos de casa. E, então, ele envia o camareiro e dá uma
longa descrição da região da casa e por último: ``O senhor não tem como
errar, a única janela no Boulevard Haussmann onde ainda há luz''. Tudo,
menos o número. Tente conseguir o endereço de um bordel numa cidade
estrangeira e --- uma vez que tenha recebido as informações mais longas ---
menos a rua e número da casa, então entenderá de que se fala aqui
{[}acrescentado posteriormente{]}:\footnote{Observação dos editores
  alemães Tiedemann e Schweppenhäuser, \versal{GS II}-3, p. 1051. \versal{[N. E.]}} e
como isso está conectado por seu amor pelo cerimonial. Sua veneração por
Saint"-Simon, e não por último, seu ``francesismo''.\footnote{No original em alemão: \emph{Franzosentume}. [\versal{N.~T.}]}
(A observação reveladora é da duquesa Cl{[}ermont"-Tonnerre{]} que na
obra de Proust não existe nenhum único estrangeiro {[}fim do
acréscimo{]}. É preciso deixar passar muita água embaixo da ponte até
que se perceba o quão difícil é experenciar tanta coisa que
aparentemente se deveria compartilhar em algumas poucas palavras. Pode
ser isso mesmo, só que essas palavras fazem parte de um vocabulário
secreto determinado por castas e reduzidas situações e não pode ser
entendido por quem está de fora.\}

\{As investidas de Proust contra a amizade devem ser basicamente entendidas
da seguinte forma: o que lhe parece ser o preço da existência, como a
essência do prazer, é algo que pode ser bem comunicado, mas não
compartilhado no sentido vital. Refere"-se totalmente à solidão e, assim,
apresenta o contraste fundamental com os prazeres e alegrias que provêm
da esfera produtiva. O que aparece tão irritante e caprichoso em tantas
anedotas é a combinação de uma intensidade sem precedentes de conversa
com uma distância intransponível do parceiro. Por um momento, vamos
imaginar a felicidade de caminhar ao lado do poeta, ser seu acompanhante
num passeio. Então aprenderemos: nunca houve alguém que pudesse nos
mostrar as coisas como ele fazia. O apontar de seu dedo é inigualável. Mas
há um outro gesto no caminhar amigavelmente juntos e na conversação: o toque. Este
gesto não é mais estranho a ninguém do que a Proust. Ele também não pode
tocar seu leitor, não o poderia por nada nesse mundo. Se alguém quiser
colocar as coisas nesta escala --- entre os que apontam e os que
tocam --- então Proust ficaria em um extremo, e no outro extremo ficaria
Péguy. No fundo, é isso o que Ramón Fernandez entendeu brilhantemente
(também isso é um contraste extremo com Péguy: ``\emph{La profondeur, ou
plutôt l'imensité, est toujours du côté de lui"-même, non du côté
d'autrui}''\footnote{Em francês no original: ``A profundidade, ou melhor, a imensidão,
  está sempre do lado de si mesmo, não do lado de outrem''. \versal{[N. T.]}}). Além
dessa felicidade embriagante do mostrar, da magia das imagens, não há em
Proust espaço para a felicidade física, para a embriaguez puramente
física. Ele é entre os grandes escritores épicos um dos poucos que não
conseguem fazer com que seus heróis comam. Marcel Brion fez, quando nós
falávamos, um comentário espirituoso, ao dizer que isso acontece porque não
existem prazeres realmente perversos à mesa e Proust não tinha as
características de Huysmans, que encontrou prazer na descrição de
alimentos ruins. Em tais observações lúdicas, no entanto, tilintam as
chaves das câmaras mais secretas da obra.\}

\begin{flushright}
\emph{\footnotesize{Cópia do original: Arquivo Benjamin,\\ Ms 428, \versal{GS II}-3, p. 1048-1050.}}
\end{flushright}

\{Mas, para concluir esta série de críticos e apresentadores de Proust
com Proust mesmo, como apresentador e crítico, devemos falar brevemente
daquilo que o ocupou ao lado de sua obra principal, como um
\emph{chroniqueur}, como jornalista, como crítico, mas sempre como o
gênio que era. Então, especialmente dos \emph{Pastiches et mélanges} e
da antologia póstuma \emph{Chroniques}. Até que ponto essas duas
funções --- de cronista e crítico --- foram capazes nele de permear"-se
mutuamente, o que mostra a primeira das duas obras de modo mais
surpreendente. Um caso criminal qualquer, do início deste século,
forneceu"-lhe o motivo para uma série de nove capítulos, que foi
tratado respectivamente no estilo de Balzac, Flaubert, Sainte"-Beuve,
Henri de Régnier, dos Goncourt, de Michelet, Faguet, Renan e,
finalmente, no estilo de seu favorito: Saint"-Simon. Não seria suficiente
falar aqui de maestria da observação. É graça obtida do ser atingido, um
abalo tão profundo e instantâneo como um raio através de um corpo
de linguagem, que a crítica, em resposta a ele, aparece com força
criativa na forma de uma paródia. A capacidade mimética e crítica não
podem mais ser separadas aqui. Assim podemos ao menos dizer. Mas nós não
queremos por isso negligenciar outros olhares talvez mais notáveis para
este fenomenal desempenho artístico.\}

\{Proust desenvolveu não só o vício da bajulação num grau eminente ---
pode"-se dizer teológico ---, mas também o da curiosidade. Em seus lábios
havia um reflexo do sorriso que perpassa nos arcos de algumas das
catedrais que tanto amava, passando pelos lábios das virgens tolas como
fogo alastrado. É o sorriso da curiosidade. Foi a curiosidade que fez dele, no
fundo, um parodista tão grandioso? Assim saberíamos, ao mesmo
tempo, o que deveríamos pensar sobre a palavra ``parodista'' neste lugar.
Não muito. Pois mesmo que se faça justiça à sua \emph{malice} abissal,
isso ainda passa ao largo do que existe de amargo, selvagem e mordaz nesses relatos. É
o mimetismo do curioso, que foi o artifício engenhoso dessa série, mas ao mesmo tempo um momento de todo o seu processo
criativo. Em Proust, a paixão pela flora --- ou melhor pela vida vegetativa ---,
não pode ser levada suficientemente a sério. Em seu perímetro está o
mimetismo e como muitos outros {[}lados?{]}\footnote{Observação dos
  editores alemães Tiedemann e Schweppenhäuser, \versal{GS II}-3,
  p. 1051. \versal{[N. E.]}} dessa esfera de vida, é extremamente típico para o
procedimento de Proust. Seus conhecimentos mais exatos, mais evidentes
estão pousados sob seus objetos como insetos nas folhas, brotos e
ramos, e não traem nada de sua existência, até que um salto, um bater
de asas, um pulo mostrem ao observador assustado que aqui uma vida própria,
imprevisível, esgueirou"-se discretamente num mundo
estranho. O verdadeiro leitor de Proust é continuamente sacudido por
pequenos sustos. Ele encontra aqui, nessas paródias como um jogo com
``estilos'', aquilo que já o afetou de modo completamente diferente como
uma luta pela existência desse espírito na cobertura de folhagem da
sociedade.

A paródia tem valor catártico. ``Quando terminamos um livro, nós queremos
não apenas viver mais tempo com suas personagens, Madame de Beauseant ou
Frederic Moreau, mas até mesmo nossa voz interior quer continuar a falar
em sua linguagem, visto já ter sido disciplinada ao longo da leitura no
ritmo de um Balzac, um Flaubert. Então devemos nos submeter a ela por um
momento e, para que o tom reverbere, pisar no pedal; ou
seja, nada mais do que imitar intencionalmente para depois voltar a
sermos originais de novo e não imitarmos despropositadamente pelo resto
de nossas vidas''. É o que diz o ensaio \emph{À propos du style de Flaubert}\footnote{Em francês no original: \emph{Sobre o estilo de Flaubert}. [\versal{N.~T.}]} que se encontra em \emph{Chroniques} junto com outras críticas literárias, glosas sobre os
salões parisienses, descrições de paisagens rurais (ambos são estudos
preliminares para as cenas da obra de sua vida).\}

Certamente Proust poderia ter completado com o desempenho sintético de
suas \emph{Pastiches}, como ele o fez no caso de Flaubert. Mas, para esse
grande crítico, a forma da crítica e também os objetos da literatura
ficavam na última fileira. Caso contrário, não teríamos ficado
confinados a suas observações sobre Ruskin, Flaubert, Baudelaire,
algumas páginas sobre a condessa de Noailles e pouco mais. \emph{Non
multa sed multum}.\footnote{Em latim no original: ``Não muitos,
  mas o bastante''. \versal{[N. T.]}} Assim, não temos --- para permanecer um pouco mais
no ensaio sobre Flaubert ---, em nenhum ensaio literário de observância
materialista histórica um comentário tão profundo, que anuncia tanto o
método, como na apresentação de partes de frases de Flaubert ``como
materiais pesados que suas frases erguem, em ritmo intermitente de
escavadora, para, em seguida, deixá"-los cair de novo''. Aliás, este
trabalho crítico oferece intencionalmente conclusões determinantes sobre
o próprio trabalho de Proust, especialmente no que diz respeito ao
tratamento dos tempos. Aqui se aprende que é uma lacuna (``\emph{un
blanc}'') o trecho que Proust mais admira em toda \emph{L'Éducation
sentimentale}. Teria sido porque ele reconheceu nela o espaço de sua
futura obra? \{Proust não era mais um desconhecido quando escreveu isso.
Um ano e meio depois, \emph{À propos de Baudelaire} escrito na
elevada altura da fama e na baixa altura do leito da morte, de modo muito
surpreendente, e, certamente muito maravilho em sua concordância
maçônica no sofrimento, nas suas \emph{défaillances de la
mémoire},\footnote{Em francês no original: falhas de memória. \versal{[N. T.]}}
com a tagarelice daquele que repousa, com o \emph{détachement}\footnote{Em francês no original: desprendimento. \versal{[N. T.]}} em
relação ao tema levado ao extremo daquele que quer falar mais uma
vez ainda, não importa sobre o quê. E como esta exaustão é entretecida em
tudo que o Proust saudável teve de mais maligno, de mais astuto. Aqui
aparece --- frente ao Baudelaire morto --- uma atitude que determinou a
economia de Proust também no tato com seus contemporâneos: uma ternura
tão exuberante, adivinhatória que o revés ao sarcasmo aparecia como
um reflexo inevitável da exaustão e, aparentemente, mal podendo ser
atribuído moralmente ao próprio poeta.\} {[}Léon Pierre"-{]}Quint
{[}\emph{Marcel Proust. Sa vie, son \oe uvre}, Paris, 1925{]} p.~113.

\begin{flushright}
\emph{\footnotesize{Cópia do original: Arquivo Benjamin,\\ Ms 429 r, \versal{GS II}-3, p. 1050-1052.}}
\end{flushright}

\{Este \emph{Baudelaire} foi sua última publicação. No ano de sua
escrita, em 1922, ele morreu de asma nervosa logo após a conclusão da
obra principal. Os médicos sentiam"-se impotentes diante desse
sofrimento ao longo de sua vida, bem diferente do poeta que parece ter
usado a doença a seu serviço de modo muito planejado. Ele foi --- para
começar com o mais superficial --- um verdadeiro diretor de sua doença.
Por meses, ele combina, com ironia devastadora, a imagem de um devoto que
lhe enviara flores, com um cheiro insuportável para ele. Alerta os
amigos com os tempos e pausas de seu sofrimento como um czar alertava os
boiardos;\footnote{Boiardo era o título atribuído aos membros da
  aristocracia russa do século \versal{X} ao \versal{XVII}, tratava"-se de classe social
  dominante, de proprietários de terra, na qual trabalhavam os mujiques. \versal{[N. E.]}}
e ansiado e temido pelos amigos era o momento quando o poeta, de
repente, muito depois da meia"-noite, aparecia em um salão --- \emph{brisé
de fatigue}\footnote{Em francês no original: morto de cansaço. \versal{[N. T.]}}
--- e só por cinco minutos, como declarava, para depois permanecer até o
amanhecer acinzentado, cansado demais para levantar"-se, cansado demais
até para interromper seu discurso. O autor de cartas é incansável e não
encontra fim em extrair desse sofrimento os efeitos mais impensáveis. ``O
chiado de meu respirar soa mais alto do que minha pena e do que um banho
que alguém prepara no andar abaixo de mim{[}.''{]} Mas não só isso. Nem
mesmo que a doença o tenha arrancado da existência mundana. Não, essa
asma entrou em sua arte, se não foi a sua arte que a criou. Sua sintaxe
reproduz ritmicamente, passo a passo, esse medo de sufocamento. Assim
podem ser interpretados especialmente aqueles traços que Leo Spitzer
{[}\emph{Stilstudien}, Vol. 2, Munique, 1928, p. 365-497{]} destacou como
``elementos retardatários'' em um estudo digno de ser lido sobre a
linguagem em Proust. E sua reflexão irônica, filosófica, didática é
sempre a tomada de fôlego com o qual o pesadelo da lembrança alivia seu
coração. Em uma escala maior, é, porém, a morte, sempre presente em seus
últimos anos e justamente no trabalho, a última crise asmática
sufocante. A estilística fisiológica levaria ao âmago dessa obra. {[}({]}\footnote{Parênteses aberto e não fechado ou por Benjamin ou pelos
  editores Tiedemann e Schweppenhäuser, \versal{GS II}-3, p. 1053. \versal{[N. T.]}}
Ninguém, portanto, que conheça a tenacidade particular com a qual lembranças são
guardadas no sentido do olfato, poderá declarar como um acaso, a
hipersensibilidade de Proust aos odores. Certamente, a maioria das
lembranças pelas quais procuramos, aparecem como imagens visuais à nossa
frente. E mesmo as formações da \emph{mémoire involontaire} que ascendem
livremente, ainda são imagens visuais em boa parte isoladas, apenas
enigmaticamente presentes. Por isso, para nos entregarmos
conscientemente a esse movimento mais íntimo da linguagem deste poeta,
precisamos aproximar de nós mesmos uma camada especial e mais profunda
desta rememoração involuntária,\footnote{No original em alemão:
\emph{unwirküllichen Eingedenken}. \versal{[N. T.]}} na qual os momentos de
lembrança\footnote{No original em alemão: \emph{Erinnerung}. \versal{[N. T.]}} nos dão notícia
não mais individualmente como imagens, mas sem imagens e sem forma,\}
indeterminada e pesadamente de um todo, assim como o peso da rede dá ao
pescador notícia de sua captura. O odor: esse é o sentido de peso do que
no ocorrido{[}\footnote{No original a frase está incompleta,
  dificultando a compreensão. Verificar a imagem completa em ``Imagem de Proust''. \versal{[N. T.]}} sic{]} do pescador no mar do \emph{Temps
perdu}. E essas frases são todo o jogo muscular do corpo inteligível,
contendo todo seu indescritível esforço para içar essa captura. \{O
quão íntima e profunda era a simbiose desse processo particular de
criação e deste sofrimento particular, mostra"-se além disso no fato
de que nunca em Proust se esbarra naquele heroico ``apesar de
tudo'',\footnote{No original em alemão: \emph{Dennoch}. \versal{[N. T.]}} com o qual em outros
casos pessoas criativas levantam"-se contra seu sofrimento. E, portanto, por outro lado,
pode"-se dizer: uma cumplicidade tão profunda com o mundo,
como era a de Proust, teria inequivocamente que desembocar em uma
suficiência comum e inerte em qualquer outra base, a não ser na de um
sofrimento tão profundo e tão incessante\}. Não existe, pois, um
sofrimento tão revoltante do indivíduo, nem mesmo uma injustiça social
tão gritante, no caminho da qual essa obra colocaria um ``apesar de
tudo'' ou um ``não''? Ao contrário: uma concordância apaixonada da
existência, até mesmo em sua forma mais triste e bestial; e, inseparável
disso, um olhar que no curso das coisas mundanas encontra uma justiça
que nenhum céu seria capaz de superar. É apenas em tal arco de justiça
tão dolorosamente estendido que a compaixão de Proust recai
sobre as pessoas de seu ``\emph{temps perdu}'' e o mecanismo histérico
que atravessa esse arco, ocasionalmente, fazia com que ele caísse em uma gargalhada estridente e frenética ao ler os manuscritos.

\{Marx mostrou como a consciência de classe da burguesia, no auge de seu
desenvolvimento, entra numa contradição inextricável consigo mesma. A
esta observação, alinha"-se György Lukács {[}Cf. \emph{História e
consciência de classe}, Berlim, 1923, p.~737{]}, quando ele diz: ``Esta
posição da burguesia reflete"-se historicamente no fato de que ela não
derrotou ainda seu antecessor, o feudalismo, quando o novo inimigo, o
proletariado, já aparecia''. Mas o surgimento do proletariado também
muda a posição estratégica na frente de batalha contra o feudalismo. A
burguesia tem que procurar um acordo a qualquer custo para encontrar
proteção nas posições do feudalismo, menos do proletariado que está
avançando do que da voz de sua própria consciência de classe. Essa é a
posição da obra de Proust. Seus problemas decorrem de uma sociedade
saturada, mas as respostas às quais ele chega, são subversivas.\}


\begin{flushright}
\emph{\footnotesize{Cópia do original: Arquivo Walter Benjamin,\\ Ms 430, \versal{GS II}-3, p. 1052-1054.}}
\end{flushright}

\{Pode"-se dizer: uma cumplicidade tão profunda com o mundo deveria ter
resultado inequivocamente em uma suficiência comum e inerte, em qualquer
outra base, a não ser na de um sofrimento tão profundo e
incessante. Não existe, pois, um sofrimento tão revoltado do ser
humano individual, nem uma injustiça social tão gritante, à qual Proust
não contrapôs um simples irascível ``não'', ou um corajoso ``apesar de %acho melhor mantermos as apas para melhor indicar que se tratam de pensamentos de Proust
tudo''? Ao contrário: encontramos em todos os lugares uma profunda
concordância com a vida, mesmo em sua forma mais bestial e triste; e,
inseparável disso, uma análise que no curso da justiça
mudana encontra uma perfeição na qual nenhuma justiça celeste poderia
superá"-la. Apenas nesse arco tão dolorosa e amplamente estendido da
justiça, a compaixão de Proust deixa"-se descer sobre as pessoas de seu
\emph{temps perdu} e o mecanismo histérico que, ocasionalmente, quando
ele lia seus manuscritos, fazia com que ele caísse em uma gargalhada
estridente e frenética, é uma expressão disso {[}Formulação variada:
atravessa este arco{]}\footnote{Observação dos editores Tiedemann e
  Schweppenhäuser, \versal{GS II}-3, p. 1054, 1º§. \versal{[N. T.]}}.\}

\{Alguém deve ter passado por muitas experiências e ter deixado muito
vento passar pelas narinas\footnote{A expressão ``\emph{sich Wind um die
  Nase wehen lassen}'' equivale a ``conhecer o mundo e a vida'', o que se
  relaciona ao sentido de experiência da frase. \versal{[N. T.]}} para entender aos
poucos quão dificilmente tanta coisa pode ser experimentada que, mesmo
assim, deixar"-se"-ia aparentemente comunicar em poucas palavras. E assim
também pode ser, só que estas palavras muitas vezes pertencem a um
vocabulário secreto, determinado de acordo com a casta e a categoria e %O vocabulário é secreto mas é compreensível para quem está de fora? Um pouco incogruente...
não são incompreensíveis para quem está de fora. Procure encontrar o
endereço de um bordel em uma cidade desconhecida, e após obter a
informação mais longa (qualquer coisa, exceto o nome da rua e o número da
casa), então entenderá o que isso significa. Chegar a saber algo
inteligível sobre aquilo que acontece em países estrangeiros, em clubes
políticos, em sociedades religiosas, não é mais fácil.

Para Proust, a sociedade era uma corporação dessas. Sua linguagem
secreta com os Bibescos. Seu misticismo da homossexualidade. Sua
veneração pelo cerimonial e por Saint"-Simon.\}

No livro de Spitzer, aparece uma dependência um pouco embaraçosa e uma
falta de pensamentos próprios.

É claro que é totalmente absurdo procurar paralelos entre Proust e o
expressionismo. Especialmente como Spitzer procura comparar a disparidade
das coisas, dos motivos, dos estados que ele elenca em suas enumerações,
com a individualização ``acentuada'', ``falada{[}?{]}'' do expressionismo.

O trabalho e a meticulosidade das referências não são proporcionais aos
resultados muito brutos e gerais. As exceções são, acima de tudo, as
passagens sobre o conjuntivo, o \emph{a c i} {[}sic{]}. Aqui é
importante a referência ao tremor da estrutura da frase por esta forma e
também a iluminação de seu caráter arcaico.

Em quase todos os lugares nos faz falta a justificação das
características linguísticas do verdadeiro centro dessa obra. Pois isso
não é de modo algum ``psicologia'', mas a lembrança,\footnote{No
  original em alemão: \emph{Erinnerung}. \versal{[N. T.]}} o martírio da rememoração.\footnote{No original em alemão: \emph{Eingendenken}. \versal{[N. T.]}}

\begin{flushright}
\emph{\footnotesize{Cópia do original: Arquivo Walter Benjamin,\\ Ms 432, \versal{GS II}-3, p. 1054-1055.}}
\end{flushright}

\{É bastante surpreendente e, claro, muito maravilhoso como esse ensaio
sobre Baudelaire foi escrito a partir do leito de enfermo. Com este
acordo maçônico no sofrimento, estas \emph{défaillances de la mémoire},\footnote{Em francês no original: falhas da memória. \versal{[N. T.]}} esta
tagarelice do que é \emph{dormant},\footnote{Em francês no
  original: repousa. \versal{[N. T.]}} com esse \emph{détachement}\footnote{Em francês no original: desprendimento. \versal{[N. T.]}} do tema levado ao
extremo, que tem alguém que só quer falar mais uma vez, não importa
sobre o quê. E como esta exaustão é entretecida com tudo o que o
saudável Proust teve de mais maligno, astuto, aparece aqui --- frente ao
Baudelaire morto --- uma atitude que determinou a economia de Proust
também no trato com seus contemporâneos: uma ternura tão exuberante e
divinatória, que o revés ao sarcasmo aparece como um reflexo inevitável
da exaustão e --- aparentemente --- mal podendo ser moralmente atribuído ao poeta mesmo.\}

\{O que aparece em tantas anedotas como algo caprichoso, irritante, é a
conexão de intensidade tão incomparável da conversa com uma distância
intransponível do parceiro para o qual ele se dirige. Por um
momento, vamos imaginar a felicidade de caminhar ao lado do poeta, ser seu acompanhante num
passeio. Então aprenderemos: nunca houve alguém que pudesse nos
mostrar as coisas como ele o fez. O apontar de seu dedo é inigualável. Mas
há um outro gesto no caminhar amigavelmente junto e na conversação: o
toque. Não há autor para o qual este gesto seja mais distante do que em
Proust. Ele não poderia tocar seu leitor, por nada no mundo. Se alguém
quiser ordenar os poetas nessa escala --- entre aqueles que apontam e os
que tocam ---, então Proust ficaria num extremo, e no outro extremo,
Péguy. \emph{Contra a amizade}.\}

\{Mas essa tagarelice --- se se quiser realmente dizer assim --- é apenas
reflexo de um caráter mais profundo, constitutivo de sua obra. Seu
editor Gallimard relatou a maneira pela qual Proust costumava tratar
as provas do livro, que levava os editores ao desespero. Elas voltavam
cheias de anotações. Mas nem um único erro ortográfico havia sido
eliminado; todo o espaço disponível era preenchido com textos novos. Se
Proust era tagarela dessa maneira, então apenas uma lei de seu próprio
mundo é revelada. Isto é: lembrança.\footnote{No original em alemão: \emph{Erinnerung}. {[}\versal{N. T.}{]}} A lembrança, no entanto, não é, em
princípio, passível de ser concluída. Um acontecimento vivido é finito,
limitado; um acontecimento lembrado é ilimitado.\}

\{Os médicos sentiam"-se impotentes diante dessa ``asma nervosa''. Ao
contrário, o poeta, parece tê"-la colocado de modo planejado a seu
serviço. Não é apenas isso, que a doença o tenha arrancado da existência
mundana. Não, essa asma entrou em sua arte, ou foi sua arte que a criou.
Sua sintaxe reproduz ritmicamente, passo a passo, esse medo do
sufocamento. E sua reflexão irônica, filosófica, didática é sempre o
desafogo que faz com que o pesadelo de lembranças desintegre"-se em seu
peito. Uma estilística fisiológica levaria ao centro desse processo
criativo. Assim, ninguém que conheça a tenacidade especial com a qual as
lembranças são guardadas no sentido do olfato, poderá considerar
coincidência as idiossincrasias osmológicas\footnote{Osmologia ou,
  no original, \emph{Osphrasiologie} (adjetivado:
  \emph{osphrasilogisch}), corresponde à doutrina ou ciência que
  investiga odores e cheiros relacionados à memória. \versal{[N. E.]}} de Proust). Quão
íntima era a simbiose desse processo de criação particular e desse
sofrimento particular, comprova"-se talvez também no fato de nele, em
Proust, nunca se encontrar aquele heroico ``apesar de tudo'', com o qual em outros casos pessoas criativas levantam"-se contra seu sofrimento.

Proust lamenta. Mas não é, no fundo, o lamento que ele levanta no
trabalho de servil de sua obra?\}

Sobre o prazer de ler extratos de Proust antologicamente como elas estão
escritas em Spitzer. Antologia de Proust.

As observações de Spitzer são muitas vezes infrutíferas porque são
``esteticamente'' orientadas. O que adianta quando se faz analogia da
técnica proustiana --- sobre a qual se fala, {[}\emph{Estudos de estilo},
vol. 2{]} p. 394 ---, com a técnica wagneriana dos motivos condutores
(\emph{Leitmotive}).\footnote{\emph{Leitmotiv} é uma técnica de
  composição introduzida por Richard Wagner. Trata"-se do uso de um ou
  mais temas que se repetem em passagens de óperas no tocante a uma
  personagem ou a um assunto. \versal{[N. E.]}} É claro que tais alusões causam, ao
contrário, o choque que toda a verdadeira experiência na vida nos dá,
que entra em nossa casa pela escada do fundo, em uma hora incomum, como
um visitante não convidado.

Proust Swann I 13\footnote{Anotação do próprio texto. {[}\versal{N. T.}{]}} -- sobre o aspecto cinematográfico de seu trabalho.

\emph{Soit que} em Proust.\footnote{Em francês no original: ``Seja
  que em Proust''. \versal{[N. T.]}}

A afinidade que a visão do mundo de Proust tem com a do cerimonial,
expressa"-se tanto no fato de que Saint"-Simon, para ele, era o cume do
desempenho literário, quanto como na relação que uma aristocrata como a
duquesa Clermont"-Tonnerre estabelece com sua obra.

\{Quase nunca houve um autor do qual poder"-se"-ia dizer com a mesma exatidão
como em Proust, em que lugar de sua obra se localiza que antes dele não
existia, onde o absolutamente novo se destaca tão inconfundivelmente do
complexo geral. Esse é o grande pretexto que este autor dá ao crítico,
ele só tem que usá"-lo. É óbvio: não apenas a análise psicológica, não apenas a crítica
social, ou o poder de observação, são inequivocamente proustianos. Em
tudo isso, existem inúmeros pontos de contato com os romancistas
anteriores, especialmente os ingleses. A \emph{signet}\footnote{Em francês no original: marca. \versal{[N. T.]}} de sua obra, oculta nas dobras de
seu texto (\emph{textum} = tecido), é a lembrança. Em outras palavras, o
que definitivamente não estava lá antes de Proust, é que alguém rompeu o
compartimento secreto do ``humor'',\footnote{No original em alemão:
  \emph{Stimmung}. \versal{[N. T.]}} e conseguiu se apropriar do que estava dentro (até
agora apenas um odor saiu dele): este desordenado, este acumulado que
nós mesmos tínhamos esquecido, fielmente guardados no inconsciente, e
que agora simplesmente domina aquele que está diante dessas coisas,
como\} o homem é domado pela visão de uma gaveta que está até a borda
cheia de brinquedos inúteis e esquecidos. Esse prazer de brincar da vida
verdadeira, da qual apenas a lembrança nos conta, isso devemos procurar
em Proust e fazer dele o ponto focal da observação.


\begin{flushright}
\emph{\footnotesize{Cópia do original: Arquivo Walter Benjamin,\\ Ms 439, \versal{GS II}-3, p. 1055-1057.}}
\end{flushright}

Em Grenoble, no século passado, existia um restaurante \emph{Au temps
perdu}. Aquele que mandou pintar a placa do restaurante era um
predecessor sentimental de Marcel Proust? Quis ele convidar os passantes
a perder seu tempo em seu estabelecimento ou quis ele muito mais achar
seu tempo perdido no fundo do copo, como Proust o achou no fundo da
famosa \emph{tasse de thé}, da qual um dia emergiu sua juventude em
Combray e Swann para serem eternizados. Uma embriaguez está também aqui
no início, certamente, aquela de um sistema nervoso infinitamente
refinado, por meio do qual um aroma seria o suficiente para abalá"-lo e
transferi"-lo para tempos distantes.

Numa coisa ele distinguiu"-se de tudo que nós chamamos por este nome.
Aquele que no minuto inesquecível uma vez o experenciou, a ele
permaneceu fiel, e subjugou sua existência a uma disciplina, que colocou
todas suas forças a serviço do mais intenso aumento e aproveitamento
daquela experiência de uma tarde.

A biografia desse homem é tão significativa porque mostra como aqui, com
extravagância e irreverência raras, uma vida retirou suas leis
completamente das necessidades de seu processo criativo. O infortúnio
grotesco, que o entendimento de
sua criação encontrou na Alemanha, com o qual nós reiteradamente
teremos que lidar, vinha, de uma parte, do fato de que o caminho orgânico
mais próximo não foi percorrido: apresentar a vida de um dos
contemporâneos mais estranhos que temos.

Mas isso Léon Pierre"-Quint fez de maneira exemplar na França e lá também
não era fácil conferir o devido lugar a Proust. Por isso, seria um bom
sinal se aqui primeiramente acontecesse a coisa mais próxima: traduzir
essas coisas para o alemão. Daí talvez a tradução {[}?{]}\footnote{Interrogação sobre a palavra manuscrita anotada pelos editores
  Tiedemann e Schweppenhäuser, \versal{GS II}-3, p. 1058. \versal{[N. T.]}} até ficasse
rápida. Um propósito tangencial das exposições seguintes é apresentar o
quanto poder"-se"-ia ganhar por meio da característica do homem, senão um
entendimento mais profundo, ao menos um interesse vivaz pela obra de
Proust.


\begin{flushright}
\emph{\footnotesize{Cópia do original: Arquivo Walter Benjamin,\\ Ms 437, \versal{GS II}-3, p. 1057-1058.}}
\end{flushright}

\pagebreak

\section{{[}Anotações do ensaio sobre Proust{]}}

A verificação dos versos em Proust. Sua maneira mais arbitrária de
escolhê"-los.

\{O quão inventivo ele era em dificuldades, ``la seule fenêtre
éclairée''.\footnote{Em francês no original: ``a única janela iluminada''. \versal{[N. T.]}}
Clermont"-Tonnerre: {[}Robert de{]} Montesquiou {[}et Marcel Proust,
Paris, 1925{]}\}

Ritz, Grand"-Hotel Balbec: ``simplification de travail'',\footnote{Em francês no original: ``simplificação do trabalho''. \versal{[N. T.]}} op. cit.

``papier fait exprès à Londres''.\footnote{Em francês no original: ``papel adequado
  Londres''. \versal{[N. T.]}} {[}Clermont"-Tonnerre{]} Montesquiou, op. cit.

\{Ela inveja as domésticas porque elas podem satisfazer suas
curiosidades\}

Singeries\footnote{Em francês no original: caricaturas. {[}\versal{N. T.}{]}} {[}Clermont"-Tonnerre:{]} Montesquiou {[}op. cit\ldots{}{]}

A respeito da palavra de Barrès: imbricação mútua de soberania e
servidão.

As marés dos vícios. Diferentes épocas, diferentes vícios. Aqueles que
quase foram extintos em nossa época. Entre esses, a bajulação. Os olhos
bajuladores de Proust, pessoas da catedral.\footnote{No original em alemão: \emph{Kathedralenmenschen}. [\versal{N.~T.}]}

\{Proust como o professor do lembrar\}\footnote{No original em alemão: \emph{Proust als Lehrmeister des Erinnerns}. {[}\versal{N.~T.}{]}}

\{Na observação da sociedade deve"-se chegar a falar do preconceito
alemão contra o \emph{milieu}\footnote{Em francês no original: meio. [\versal{N.~T.}]} aristocrático de Proust.\}

\{A função da felicidade no mundo de Proust\}

\{quando ele saía, ele estava enfiado em seu casaco de pele, como em uma
casa, feroz e perturbado\}

\medskip

\noindent{}\emph{Enseigne}\footnote{Em francês no original: ensinar. [\versal{N.~T.}]} ``Au temps perdu''

\noindent{}\{Proust na Alemanha\}

\noindent{}\{O que Pierre"-Quint fez por ele\}

\noindent{}\{\emph{Procédé}\footnote{Em francês no original: procedimento. [\versal{N.~T.}]} de Proust\}

\noindent{}\emph{Mémorie involontaire} e felicidade

\noindent{}\emph{Poète persan dans une loge de portière}\footnote{Em francês no original: ``Poeta
  persa em um alojamento de porteiro''. \versal{[N. T.]}}

\noindent{}A sociedade como um mundo de consumo

\noindent{}Feudalismo e Burguesia

\noindent{}Os vícios de Proust

\noindent{}Função de sua solidão

\noindent{}Função de sua doença

\noindent{}Proust e o teto da Capela Sistina

\medskip

\{A morte, na qual ele por último sempre teve que pensar e que se tornou
para ele uma condição de produção como o último sufocamento asmático.\}

Retração dos caminhos perto de Combray em conexão com a ideia de
envelhecimento.

Léon Pierre"-Quint vai longe demais em suas ressalvas a respeito da
mística de Proust.

Muito corretamente Quint chamou a atenção como os jovens autores acham
suspeita a mística da arte de Proust. Importante é a perspectiva para os
mais variados {[}interrompido{]}\footnote{Sinalização dos editores
  Tiedemann e Schweppenhäuser sobre a interrupção do manuscrito,
  \versal{GS II}-3, p. 1059. \versal{[N. T.]}}

Assim como em quadros antigos de visitação, Maria comove"-nos,
{[}aqueles{]} em que ela visivelmente carrega a criancinha debaixo do
coração, Proust sabe apresentar"-nos as fases e os momentos da
existência, sempre com a criancinha, a imagenzinha no ventre materno. E
como ele a enobrece. ``E quando se tinha mandado por ela buscar\ldots{}''
\emph{Côté de Guermantes I}

Proust também achou uma das maiores fórmulas do amor: ``posséder à lui
seul désirs d'une femme''\footnote{Em francês no original:
  ``Possuir apenas desejos de uma mulher''. \versal{[N. T.]}}

\begin{flushright}
\emph{\footnotesize{Cópia do original: Arquivo Benjamin,\\ Ms 439, \versal{GS II}-3, p. 1058-1060.}}
\end{flushright}

\begin{flushleft}
\hspace*{-1.7cm}\includegraphics[width=13.5cm]{./diagrama.pdf}
\end{flushleft}

\begin{flushright}
\emph{\footnotesize{Cópia do original: Arquivo Benjamin,\\ Ms 431, \versal{GS II}-3, p. 1060.}}
\end{flushright}

A cerimonisidade\footnote{No manuscrito Ms 438, lê"-se ``Prousts
  Umständlichkeit''. Na edição de Tiedemann e Schwenppenhäuser, lê"-se:
  „Prousts Unverständlichkeit”. Traduzimos ``Umständlichkeit'' por
  ``cerimoniosidade'', a fim de seguir a ideia desenvolvida na sequência
  do texto. \versal{[N. T.]}} de Proust. Quão inventivo ele era quando estava em apuros.
``La seule fenetre éclairée''\footnote{Em francês no original: ``A
  única janela iluminada''. \versal{[N. T.]}} {[}Clermont"-Tonnerre:{]} Montesquiou
{[}Proust, Paris 1925{]} 136. A maneira como ele escreveu na cama.
\{Como isso tem conexão com seu senso de cerimonial, seu amor por
Saint"-Simon, sua intolerância, seu francesismo intransigente. A duquesa
Clermont"-Tonnerre faz a observação reveladora que na obra de Proust não
aparece um único estrangeiro.\} De Proust ``papier fait exprès à
Londres''\footnote{Em francês no original: ``papel feito %Logo acima traduzem de outra forma a mesma frase...
  propositalmente em Londres''. \versal{[N. T.]}} Montesquiou. Em tudo isso há algo muito
diferente do alheamento romântico do mundo. Talvez ele fosse alheio ao
mundo, mas de uma maneira teimosa, que muitas vezes beira o sádico e
sempre mantém o contato mais íntimo com o mundo particular em que ele
viveu. Qual é o preconceito principal que se opõe a Proust na Alemanha.
Como o \emph{milieu} de Proust deve ser apreendido na Alemanha.
Primeiro, seu livro não é um romance social no sentido de que um
indivíduo opina sobre o acontecido. Sua obra é a ``socialização''
literária do eu. Como a sociedade coloca o eu em operação --- e qual
sociedade coloca o Eu em operação. Como ela o faz: através de uma
destruição que ocorre na memória.\footnote{No original em alemão: \emph{Gedächtnis}. [\versal{N.~T.}]} Qual o faz: uma
sociedade burguesa declinante, que está sendo derrotada pelos poderes
invictos do feudalismo.

Processo desta conquista da burguesia pelos poderes do feudalismo. O
ócio burguês. A burguesia fracassa por causa do seu ócio, de sua
reflexão.

\{Devemos tentar principalmente entender o que está por trás do
fato de que Proust não fornece o desenvolvimento da essência do Faubourg
Saint"-Germain apenas para destruir esta casta.\} \{Citar a frase de
Lukács.\} Hochdorf {[}?{]}. Toth {[}?{]}

Os vícios de Proust: tagarelice, curiosidade e bajulação.

Especialmente o último é um vício arcaico que não tem mais uma base
clara na sociedade de hoje.

{[}notas subsequentes à lápis:{]} Sua profunda assimilação por meio do
feudalismo é o tema sociológico do livro e sua fuga de regresso ao
feudalismo, do qual esse havia se emancipado.

E este cômico, o poeta descobre, não por último em todas as pretensões
sociais da burguesia.

\begin{flushright}
\emph{\footnotesize{Cópia do original: Arquivo Walter Benjamin,\\ Ms 438,\versal{GS II}-3, p. 1058-1060.}}
\end{flushright}

\{A lembrança como Penélope\footnote{No original em alemão: \emph{Die Erinnerung als Penelope}. {[}\versal{N. T.}{]}} --- o dia não acrescenta algo ao tecido dela.
O que ele traz ele toma de volta com a outra mão. Ele desfaz
o que foi tecido. E toda noite, a lembrança começa a tecer de novo.
Então, a obra de Proust é realmente um tapete. Já em 1914, Francis de
Miomandre escreveu {[}?{]}: ``Tout vint sur le même plan''\}\footnote{Em francês no original: ``Tudo veio no mesmo plano''. \versal{[N. T.]}}

Relação entre a \emph{ingénuité} de Proust, infantilidade, e aquela
vontade fanática de felicidade que o impulsiona até o fundo para de %para de?
todas as coisas. A mais alta função criativa do hedonismo de Proust
parece não ter sido ainda reconhecida.

\{Montesquiou compara a obra de Proust com o \emph{Defilee}\footnote{Em francês no original: desfile. \versal{[N. T.]}} dos insetos naquele romance de
Salomão.\}

Além disso, deve ser observado o nexo de uma certa cerimoniosidade
indesejada com sua intensidade criativa. Ser cerimonioso significa,
entre outras coisas, dificultar as coisas para si mesmo e isso pode ser
significativo para uma obra.

Os livros de Clermont"-Tonnerre: suas lembranças, secas e borbulhantes
como um bom champanhe, são boas e muito habilmente armazenadas.

Montesquiou era um homem que olhou para os momentos de sua existência
como joias e para quem a vida e o espírito de seus contemporâneos,
finalmente também Proust, basicamente, eram apenas bom o suficiente para
reunir estas pedras preciosas como uma moldura. Clermont"-Tonnerre: R{[}obert{]} d{[}e{]} M{[}ontesquiou e Marcel Proust, a. a. O.{]} p
183 sobre seus poemas de guerra. O \emph{déclin}\footnote{Em francês no original: declínio. \versal{[N. T.]}} é muito bem descrito por
Montesquiou: vem à tona que ele é uma das existências marcadas desde o
início com o estigma da morte solitária.

Ao considerar sua poesia, gostaríamos de dizer que no declínio desta
classe (a feudal) algo de sua borra é derramada.

Clermont"-Tonnerre sobre a poesia de Montesquiou ``encouragé par son
insuccès''\footnote{Em francês no original: ``Encorajado pelo seu insucesso''. \versal{[N. T.]}} {[}a. a. O.p. 183{]}

Citar p 192 sobre as \emph{Memórias} (\emph{Memoiren}) de Montesquiou

\emph{``Le Don Juan de la haine}''.\footnote{Em francês no original: ``O Don Juan do ódio''. \versal{[N. T.]}}

\{Clermont"-Tonnerre: \emph{les romans de Proust ne mentionnent pas un
seul étranger}\}\footnote{Em francês no original: ``Os romances
  de Proust não mencionam um único estrangeiro''. \versal{[N. T.]}}

Aquele Proust, o frequentador assíduo do Ritz

\{\emph{La mer ``jamais la même''}\footnote{Em francês no original: ``o mar'', ``nunca o mesmo''. \versal{[N. T.]}} --- e os dioramas com sua
mudança de iluminação, o que faz com que o dia passe tão rápido diante
do espectador quanto ele passa para o leitor em Proust. Aqui, a mais
baixa e mais alta forma de mímesis, dão as mãos.\}


\begin{flushright}
\emph{\footnotesize{Cópia do original: Arquivo Walter Benjamin,\\ Ms 433, \versal{GS II}-3, p. 1060-1062.}}
\end{flushright}

Proust --- como ele aquece com cartas

A felicidade em seus olhos, a sorte ``no'' jogo, ``no'' amor

O \emph{coupe}\footnote{Em francês no original: o corte. \versal{[N. T.]}} do
estilo da autora {[}provavelmente Elisabeth de Clermont"-Tonnerre{]}\footnote{Observação dos editores Tiedemann e
  Schwepenhäuser, p. 1062. \versal{[N. E.]}} Predominância da literatura biográfica e
anedótica no primeiro período.

Oposição com a doutrina muito antiga de que na felicidade os contornos
das coisas tornam"-se turvos

A homossexualidade de Proust

\begin{flushright}
\emph{\footnotesize{Cópia do original: Arquivo Walter Benjamin,\\ Ms 434,\versal{GS II}-3, p. 1062.}}
\end{flushright}

\section{{[}Notas sobre Proust e Baudelaire{]}\protect\footnote{\uppercase{A}nterior a esses
  últimos fragmentos das anotações de \uppercase{B}enjamin sobre \uppercase{P}roust,
  encontram"-se comentários dos editores \uppercase{T}iedemann e \uppercase{S}chweppenhäuser
  (\versal{GS II}-3, p. 1062) a respeito do material --- \uppercase{P}rolegômena ---
  que poderia ter servido a um segundo ensaio sobre \uppercase{P}roust: ``\uppercase{A}
  compilação dos documentos sobre \uppercase{P}roust contém, além disso, duas folhas
  (\uppercase{M}s 426 f.), nas quais \uppercase{B}enjamin anotou correções do ensaio
  `\uppercase{I}magem de \uppercase{P}roust'. \uppercase{E}m `\uppercase{P}rolegômena a \uppercase{P}roust
  \versal{II}', o significado do \versal{II} não é claro: o mais improvável é que isto
  quer dizer prolegômena sobre o segundo volume da `\emph{\uppercase{R}echerche}';
  possivelmente as notas são a respeito de um segundo ensaio planejado
  sobre \uppercase{P}roust; no entanto, os editores estão inclinados a pensar em um
  trabalho preparatório para o ensaio de 1929 (o \uppercase{P}rolegômeno \versal{I} estaria,
  portanto, perdido). \uppercase{F}inalmente, a curta palestra sobre \uppercase{P}roust é um
  texto independente, porém fragmentário, provavelmente de 1932''. \versal{[N. E.]}}}

\{Porque ele inclui toda a existência. Seu lembrar espontâneo\footnote{No original em alemão: \emph{spontanen Erinnern}. \versal{[N. T.]}} e assim surge da escuridão das
nuvens da semelhança onde ela é mais esvanecedora --- na imagem
desfigurada do que envelhece---\}

O mundo das \emph{correspondences} de Baudelaire, as regiões onduladas
da semelhança onde os toxicomaníacos\ldots{} estão em casa. Dessa escuridão
das nuvens da semelhança, onde ela se forma da maneira mais escura, do
envelhecimento, rompe o fértil, ressoa a água fértil, rejuvenescente nas
gotas da qual o instante se reflete magicamente.

O envelhecimento como processo terrível no cosmos da semelhança. Este
cosmos torna acessível as \emph{correspondances} de Baudelaire.

Mas da escuridão das nuvens, vem como chuva o poder fértil da lembrança,
nas gostas da qual o mundo se reflete.

A lembrança rejuvenesce.

A solidão para a qual ela leva, é a organização de um mundo, no qual
Proust nunca introduz{[}iu{]} mais claramente do que nesta imagem:
\emph{Côté du Swann}, \emph{Côté des Guermantes}.

Não só o tempo é encontrado novamente, mas a proximidade.

O verdadeiro empenho de Proust direciona"-se ao curso do tempo, em sua
face mais verdadeira, menos redigida {[}?{]} que ele no cosmos

O verdadeiro interesse de Proust se dirige à passagem do tempo em sua
forma banal {[}?{]}, na sua forma entrecruzada no espaço.

O curso do tempo no mundo desfigurado do estado da semelhança, o
verdadeiro mundo de Proust.

O envelhecimento

Neste cosmos aparecem as correspondências baudelairianas.

E nelas o mundo se rejuvenesce, inebriante

Pois não só a eternidade é banida em tempo, mas também a distância na
proximidade

A imagem de Combray

A solidão como embriaguês


\begin{flushright}
\emph{\footnotesize{Cópia do original: Arquivo Walter Benjamin,\\ Ms 1962, \versal{GS II}-3, p. 1063.}}
\end{flushright}

\section{Prolegômenos para Proust II}

Com tantas analogias do mundo botânico, quase nada sobre animais. A
respeito do universo das plantas em Proust comparar livros {[}como{]} os
de {[}Raoul H.{]} Francé ou \emph{Blumen} {[}Berlin, 1928{]} de Th{[}eodor{]} Lessing.

Ensaios de Raphael Cor no \emph{Mercure de France} 15 de julho de 1924
{[}?{]} 15 de maio de 1926 {[}tomo 188, pp. 46-55: \emph{Marcel Proust et
la jeune littérature}{]} / 15 de maio de 1928 {[}tomo 204, pp. 55-74:
\emph{Marcel Proust ou l'indépendant. Réflexions sur le ``Temps
retrouvé}'',{]} /

Dialética em Proust: seus problemas decorrem de uma sociedade saturada,
mas as respostas às quais ele chega, são subversivas. Ou: a obra é
legível apenas em longas viagens marítimas, durante uma convalescença,
etc. Por outro lado, a quintessência da obra se encontra em todas as
páginas.

Louis de Robert {[}\emph{Comment débuta Marcel Proust}, Paris 1925{]} e
{[}Robert{]} Dreyfus destaca, com razão, o tom epistolar que tem toda
sua obra. O que resulta disso? Dreyfus {[}\emph{Souvenirs sur Marcel
Proust}, Paris, 1926{]} p 202

\begin{flushright}
\emph{\footnotesize{Cópia do original: Arquivo Walter Benjamin,\\ Ms 717, \versal{GS II}-3, p. 1064.}}
\end{flushright}

De uma curta palestra sobre Proust proferida em meu 40º aniversário

Sobre o conhecimento da \emph{mémoire involontaire}: suas imagens vêm
sem serem evocadas, trata"-se nela {[}\emph{mémoire
involontaire}{]} muito mais de imagens que nunca vimos, antes de
lembrarmos delas. Isto fica mais nítido naquelas imagens, nas quais nós
podemos ser vistos --- exatamente como em alguns sonhos. Nós ficamos em
frente de nós mesmos, do mesmo jeito como provavelmente estávamos no
passado primevo, em algum lugar lá, mas nunca diante de nosso olhar. E
são justamente as imagens mais importantes que chegamos a ver, aquelas
que são reveladas na câmera escura do momento vivido. Poder"-se"-ia dizer
que em nossos momentos mais profundos, como naqueles pacotinhos de
cigarro, foi nos dada uma pequena imagenzinha, uma foto de nós mesmos. E
aquela ``vida toda'' que passa, como ouvimos frequentemente dizer,
diante de moribundos ou de pessoas em perigo de vida, consiste
exatamente dessas pequenas imagens. Elas apresentam uma sequência rápida
como aqueles cadernos, que precedem o cinematógrafo, nos quais nós
podíamos admirar, na infância, um boxeador, um nadador e um jogador de
tênis praticando suas artes.

O hedonismo de Proust não pode ser entendido sem a ideia de
representação. Proust mesmo viu"-se como o representante dos pobres e
deserdados no prazer. Ele está completamente impregnado pela obrigação
--- isto é uma segunda coisa --- de realmente vivenciar não apenas o
prazer para todos, mas também o prazer em todo lugar e em tudo no qual
ele é reivindicado. O propósito incondicional de salvar o prazer, de
justificá"-lo, de realmente achá"-lo onde ele comumente é apenas fingido,
é uma paixão de Proust, que vai muito mais fundo e é {[}palavra não
decifrada{]},\footnote{Observação dos editores Tiedemann e
  Schweppenhäuser, \versal{GS II}-3, p. 1065. \versal{[N. E.]}} do que suas análises
desilusionistas. Daí sua especial fixação no esnobismo, do qual ele quer
pegar o que tem de prazer autêntico dentro dele --- um tesouro, que os
membros da sociedade parecem ser menos capazes de resgatar.

\begin{flushright}
\emph{\footnotesize{Cópia do original: Arquivo Benjamin,\\ Ms 754 f., \versal{GS II}-3, p. 1064-1065.}} \enlargethispage{\baselineskip}
\end{flushright}
