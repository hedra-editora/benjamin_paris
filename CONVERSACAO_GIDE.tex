\chapter{Conversação com André Gide\footnote[*]{``Gespräch mit André Gide'', in \versal{GS IV}-1, p. 502--509. Tradução de Carla Milani Damião. Benjamin
  foi escolhido pelo jornal \emph{Deutschland Allgemeine Zeitung} e pela
  revista \emph{Literarische Welt} para entrevistar André Gide em sua
  visita a Berlim. Do encontro resultaram dois artigos: ``André Gide
  und Deutschland. Gespräch mit dem Dichter'' (\versal{GS IV}-1, p. 497--502),
  publicado no jornal em 29/01/1928, e ``Gespräch mit André Gide''
  (\versal{GS IV}-1, p. 502--509), publicado na revista em 17/02/1928. Essa
  tradução corresponde ao artigo publicado na revista. Benjamin tinha 35
  anos à data, e Gide, 59 anos. Em carta a Scholem, de 30/01/1928,
  Benjamin diz esperar que com a publicação da entrevista sua situação
  em Paris melhore. [\versal{N.~O.}]}}
\hedramarkboth{Conversação com Gide}{}

É agradável conversar com André Gide em seu quarto de hotel. Sei que ele possui uma casa de campo em Cuverville e um
apartamento em Paris e seria uma impressão certamente inesquecível
encontrá"-lo entre seus livros, nos lugares onde ele concebeu e realizou
sua grande obra. Mas não compararia isto ao encontro com esse grande
viajante em meio à sua bagagem, \emph{omnia sua secum portans},\footnote{Em latim no original: ``portador de todas suas coisas consigo''. \versal{[N.~T.]}} em
estado de prontidão defensiva, à luz clara da manhã~em seu amplo quarto
de hotel na \emph{Potsdamer Platz}. Podemos admitir que a entrevista,\footnote{\emph{Interview} no texto original.
  Benjamin teoriza sobre \emph{Gespräch} (conversa ou conversação) no
  texto de juventude intitulado \emph{Metafísica da juventude. A
  conversação} (\emph{Metaphysik der Jugend. Das Gespräch)},
  \versal{GS II}-1, p.~91--96. \versal{[N.~T.]}} uma forma que diplomatas, economistas, gente de cinema, criaram
para si, não é, à primeira vista, o meio pelo qual o poeta\footnote{No original em alemão: \emph{Dichter}. [\versal{N.~T.}]} --- o mais diferenciado entre os viventes --- se dá a
reconhecer. Se observarmos melhor, a entrevista parece de fato
diferente. Fala e resposta articulam o pensamento gideano como um raio
de luz. Eu o comparo a uma fortaleza: tão inalcançável visualmente em
sua construção, com circunvalações, em recuos e bastiões avançados,
sobretudo tão rigoroso na forma e tão perfeito na construção de sua
funcionalidade dialética.

Mesmo o último dos diletantes sabe que é perigoso e que deve arcar com consequências
quem fizer
registros nas proximidades de fortalezas. Papel e lápis tiveram que ser
deixados de lado, e se as palavras seguintes forem autênticas, elas o são
graças à agudeza da suave e entusiasmada voz que as gerou.

Quase não fiz a Gide perguntas que normalmente, mais por rotina do que
por interesse, surgem em uma entrevista. Pois ele próprio se basta como entrevistador e entrevistado, tal como ele estava
sentado à minha frente em um degrau da sacada interna do quarto,
recostado na almofada de uma poltrona, com um \emph{foulard}\footnote{Em francês no original: lenço de seda. \versal{[N.~T.]}} marrom envolto no pescoço e as mãos estendidas ora
sobre o tapete, ora entrelaçadas e recolhidas sobre os joelhos.
De vez em
quando, caso uma de minhas raras perguntas suscitasse seu interesse, seu
olhar saltava dos reluzentes óculos de aro de tartaruga e recaia sobre
mim. É fascinante observar seu rosto, mesmo que seja apenas para
acompanhar o jogo alternante entre malícia e bondade, as quais
estaríamos tentados a dizer que ambas habitam as mesmas rugas,
dividindo"-se fraternalmente em sua expressão facial. Não são os piores
momentos, quando a pura alegria provinda de uma anedota maliciosa
ilumina suas feições.

Não existe hoje nenhum poeta europeu que tenha acolhido a fama de forma
tão inóspita, quando enfim essa o alcançou no final de seus quarenta anos.
Não existe francês que tenha se entrincheirado mais firmemente contra a
Academia Francesa. Gide e D'Annunzio --- precisamos somente colocar esses
nomes lado a lado para reconhecer o que se é capaz de fazer a favor ou
contra a fama. ``Como~o senhor encara sua fama?'' E, então, Gide conta o
quão pouco a procurou, a quem agradeceu por tê"-la um dia, ainda assim,
encontrado; e como ele dela defendeu"-se.

Até 1914 ele estava firmemente convicto de que seria lido apenas após
sua morte. Não se tratava de resignação, mas de confiança na duração e
força de sua obra. ``Desde que comecei a escrever, Keats, Baudelaire,
Rimbaud, foram para mim um modelo: pois eu queria, como eles, dever meu
nome apenas à minha obra e a nada mais.'' Uma vez que um poeta ocupe
esse posto, não é raro, então, que um inimigo intervenha e lhe sirva de
burro de Balaão.\footnote{Benjamin refere"-se a uma passagem bíblica do profeta Balaão e de seu burro ou jumenta (as versões, em português, variam). Balaão açoitava"-o sem perceber que o burro era capaz de ver o anjo que ele não via. Expressão que significa que o adversário presta"-lhe um grande favor ao criticá"-lo. \versal{[N.~T.]}} Este foi, para Gide, Henri Béraud, o romancista.\footnote{No original em alemão: \emph{der Romancier}. \versal{[N.~T.]}} Ao leitor francês de jornal, ele tanto afirmou que nada
havia de mais ignorante, enfadonho e depravado do que os livros de André
Gide, até que por fim as pessoas ficassem atentas e perguntassem: Quem é
de fato esse André Gide, que, por nenhum preço, deve ser lido por
pessoas decentes? Quando certa vez, após muitos anos, Béraud escreveu,
num de seus ímpetos, que de todas as pessoas este Gide era, além de
tudo, um ingrato com seus benfeitores, então, o poeta, para enfraquecer
essa áspera reprovação, enviou a Béraud a mais bela caixa de chocolate
\emph{Pihan.} Junto um cartãozinho com as seguintes palavras:
``\emph{Non, non, je ne suis pas un ingrat}''.\footnote{Em francês no original: ``Não,
  não, eu não sou um ingrato''. \versal{[N.~T.]}}

O que mais contrariava os adversários do jovem Gide era a constatação de
que no exterior ele era considerado mais notável do que eles próprios.
Isso passaria uma impressão completamente falsa, assim pensavam.
E, de fato, seus livros teriam dado uma impressão mais exata do tipo médio de romances fabricados na França. Gide foi
desde cedo traduzido entre nós e mantém uma relação de amizade com seus
primeiros tradutores, com Rilke até sua morte, com Kassner e Blei ainda
hoje. Assim chegamos à questão atual da tradução. O próprio Gide
tornou Conrad\footnote{Trata"-se do escritor britânico Joseph
  Conrad (1857--1924). \versal{[N.~O.]}} conhecido e apreciado ao traduzi"-lo, como a Shakespeare. Sabíamos de sua magistral tradução de
\emph{Antônio e Cleópatra}. Há pouco tempo, Pitöeff, diretor do
\emph{Théâtre de l'Art}, pediu a ele que traduzisse \emph{Hamlet}.
``O primeiro ato custou"-me meses. Quando ficou pronto, escrevi a
Pitöeff: não consigo mais,/ extenua"-me em demasia.'' ``Mas o senhor
publicará o primeiro ato?'' ``Talvez, não sei. Neste momento ele se
perdeu em algum lugar entre meus papéis em Paris ou Cuverville. Viajo
tanto que não consigo por nada em ordem.''\footnote{As citações
  estão em sequência no texto original, sem indicar os autores. \versal{[N.~T.]}} Não por acaso ele conduz a conversa para Proust. Ele está a
par do empreendimento de tradução para o alemão, também conhece o lado
obscuro de sua história.\footnote{O empreendimento de traduzir a
  obra de Proust \emph{Em busca do tempo perdido}, cuja incumbência
  inicialmente seria de Benjamin e Franz Hessel, havia fracassado
  naquele momento. \versal{[N.~T.]}} Sua esperança por um desfecho
favorável dessa história é amável. E como sabemos, por experiência, que todos
que se aproximam de Proust passaram por fases, arrisco
perguntar sobre sua própria relação\ldots{} Ela não é uma exceção a essa lei.
O jovem Gide testemunhou o momento inesquecível em que Proust, o
brilhante \emph{Causeur},\footnote{Em francês no original: conversador. \versal{[N.~T.]}} começou
a despontar nos salões. ``Quando nós nos encontrávamos socialmente, eu o
tinha como o \emph{snob} mais empedernido. Penso que ele não me
estimou de outro modo. Nenhum de nós pressentiu a estreita amizade que
iria nos ligar.'' E quando um dia chegou um metro de pilha de cadernos
no escritório da editora da \versal{\emph{NRF},}\footnote{A revista
  \emph{Nouvelle Revue Française}, na qual Gide teve um importante papel
  intelectual, como atesta a obra de Auguste Anglés, \emph{André Gide et
  le premier groupe de la} Nouvelle Revue Française. \emph{Une inquiète
  maturité, 1913--1914}. Paris: Éditions Gallimard, 1986. \versal{[N.~T.]}} foi
primeiramente desconcertante. Gide não arriscou imediatamente imergir
naquele mundo. Mas quando começou, logo sucumbiu a seu fascínio. Desde
então, Proust é para ele um dos maiores entre todos os pioneiros dessa
mais nova conquista do espírito: a psicologia.

Também esse termo, quando se conversa com Gide, é de novo uma porta que
leva a uma dessas imensas galerias nas quais corremos o risco de quase nos
perder. A psicologia é a causa do ocaso do teatro. O drama psicológico,
sua morte. A psicologia é o domínio do diferenciado, do que isola e
desconcerta. O teatro é o domínio da unanimidade, da solidariedade, da
realização. Amor, inimizade, fidelidade, ciúme, coragem e ódio --- no
teatro são todas as partes de uma constelação de contornos previsíveis,
dadas de antemão. O contrário daquilo que é para a psicologia, cuja
compreensão descobre no amor o ódio, na coragem, a covardia. ``\emph{Le
théâtre c'est un terrain banal}.''\footnote{Em francês no original: ``O teatro é um terreno banal''. \versal{[N.~T.]}}

Voltamos a Proust. Gide esboça a descrição que já vinha se tornando
clássica, desse quarto de enfermo, desse ser adoentado que, naquele aposento
permanentemente escuro, o qual, para impedir o ruído, era forrado
com cortiça ao redor --- mesmo as janelas eram cobertas com almofadas ---,
recebia raros visitantes e, sobre sua cama, sem suporte para escrever,
rodeado por uma pilha de papéis cheios de rabiscos, escrevia, escrevia e
mesmo suas correções ao invés de lê"-las, cobria"-as com mais frases,
``\emph{bien plus que Balzac''}.\footnote{Em francês no original: ``Ainda mais do que
  Balzac''. \versal{[N.~T.]}} Apesar de sua admiração, Gide constata: ``Não tive
contato com suas personagens. \emph{Vanité}\footnote{Em francês no original: vaidade. \versal{[N.~T.]}} --- essa era a matéria da qual elas eram feitas. Acredito que
Proust deixou muita coisa que não conseguiu expressar, brotos que nunca
conseguiram se abrir. Em sua obra tardia prevaleceu certa ironia sobre a
moral e o religioso, o que nos primeiros escritos era perceptível''.
Parece também que o poeta reconheceria em uma característica de sua
técnica, de sua composição, uma ambiguidade da essência proustiana, às
vezes ocultada pela ironia. ``Fala"-se de Proust como um grande
psicólogo. Ele seguramente o foi. Porém, se insistimos em lembrar com
tanta frequência quão artisticamente ele seria capaz de mostrar a
mudança de suas principais personagens ao longo de sua vida, então não
nos damos conta talvez de uma coisa: cada uma de suas personagens, até a
mais inferior, é trabalhada segundo um modelo. Esse modelo, no entanto,
não permaneceu sempre o mesmo. Para Charlus, por exemplo, havia
certamente no mínimo dois modelos; ao Charlus da última época servia um
modelo muito diferente daquele orgulhoso da primeira época.'' Gide fala
de \emph{surimpression},\footnote{Em francês no original: impressão sobre impressão. \versal{[N.~T.]}} de um \emph{fondu}.\footnote{Em francês no original: fusão. \versal{[N.~T.]}} Como em um filme, uma personagem se transforma sucessivamente
noutra.

Ao final de uma pausa, Gide diz: ``Eu vim para proferir uma
\emph{conférence},\footnote{Em francês no original: conferência. \versal{[N.~T.]}} mas a vida
berlinense não me deixou fazer em paz aquilo que efetivamente havia
planejado. Voltarei e então trarei comigo minha \emph{conférence}. Mas
hoje gostaria de lhe contar algo sobre minha relação com a língua alemã.
Depois de um longo, intensivo e exclusivo envolvimento com a língua
alemã --- que ocorreu durante os anos de minha amizade com Pierre Louÿs,
quando lemos juntos o segundo \emph{Fausto} --- deixei de lado minhas
coisas alemãs durante dez anos. A língua inglesa prendeu toda minha
atenção. No ano passado, então, no Congo, voltei a abrir novamente um
livro alemão, \emph{As afinidades eletivas}.\footnote{Gide
  refere"-e à obra de Goethe \emph{As afinidades eletivas} (\emph{Die
  Wahlverwandschaften}), sobre a qual Benjamin escreveu um longo ensaio.
  Segundo a única testemunha presente nessa entrevista, Pierre Bertaux,
  Benjamin havia enviado seu ensaio a Gide, pois ele teria perguntado se
  Gide o havia lido. Essas observações foram feitas em carta de Bertaux
  a respeito desse encontro (cf. Walter Benjamin, \versal{GS VII},
  p. 257--269). \versal{[N.~T.]}} Então fiz uma estranha descoberta, minha leitura após esses
dez anos de pausa não estava pior, mas estava melhor. Não foi'' --- e aqui
Gide insiste --- ``o parentesco entre o alemão e o inglês que tornava a
leitura mais fácil para mim. Não, mas justamente porque havia me
separado de minha própria língua materna, recebi o \emph{élan} de
apoderar"-me de outra língua estrangeira. No aprendizado de línguas o
mais importante não é qual língua se aprende; abandonar a sua é o
decisivo. Só assim, também, a entendemos verdadeiramente''. Gide cita
uma frase da descrição da viagem do navegante de Bougainville: ``Quando
nós deixamos a ilha, demos"-lhe o nome de \emph{Ile du Salut}'',\footnote{Em francês no original: Ilha da Salvação. \versal{[N.~T.]}} ao que ele emenda com a maravilhosa
frase: ``\emph{Ce n'est qu'en quittant une chose que nous la
nommons}''.\footnote{Em francês no original: ``Apenas quando deixamos uma coisa é que a
  nomeamos''. \versal{[N.~T.]}}

``Se eu'', prossegue, ``influenciei a geração que me sucedeu em alguma
coisa, foi no fato de que agora os franceses começam a mostrar interesse
por terras e línguas estrangeiras, onde antes reinava a indiferença e a
indolência. Leia a \emph{Voyage de Sparte}\footnote{Em francês no original: \emph{Viagem
  a Esparta}. \versal{[N.~T.]}} de Barrès e o senhor saberá o que quero dizer. O que
Barrès via na Grécia é a França, e, onde ele não via a França, nada
queria ter visto.'' Chegamos, pois, de súbito, a um dos temas gideanos
prediletos: Barrès. Sua crítica ao \emph{Déracinés}\footnote{Em francês no original: \emph{Os desenraizados}. \versal{[N.~T.]}} de Barrès, escrita já há trinta anos, foi uma firme recusa dessa
epopeia do enraizamento. Foi a confissão magistral do homem que não quer
fazer valer o nacionalismo saturado e reconhece as características do
povo francês apenas onde elas em si encerram o espaço de tensão da
história europeia e da família de povos europeus.

``Os desenraizados'' --- Gide tem só um amável escárnio para uma metáfora
poética que passa tão completamente ao lado da verdadeira natureza. ``Eu
sempre disse que é uma pena que Barrès tenha a botânica contra si. Como
se a árvore se fechasse e não mais alcançasse por seus rebentos, com
todos seus ramos, a vasta atmosfera. É uma desgraça quando os poetas não
têm a menor noção de ciências naturais.'' Diante de mim, está um homem
que escreveu certa vez: ``Só quero lidar com a natureza. Uma carroça
de legumes transporta mais verdade do que o mais belo período de
Cícero''. Esse círculo de imagens ainda prende Gide: ``Há pouco falava
de Proust, de quantos de seus brotos permaneceram fechados. Comigo foi
diferente. Eu quero que tudo o que aprendi venha à luz do dia, encontre
sua forma. Isso possui talvez uma desvantagem. Minha obra tem algo de
matagal, do qual dificilmente desprendem"-se meus traços determinantes.
Aqui eu aguardo com paciência. \emph{Je n'écris que pour être
relu}.\footnote{Em francês no original: ``Eu escrevo apenas para
  ser relido''. [\versal{N.~T.}]} Eu conto com o tempo depois da minha morte. Somente
a morte expulsará a figura do poeta da obra. Então a unidade da obra
será inconfundível. Entretanto, não deixei que ela se tornasse fácil
para mim. Sei que existem poetas que, desde o início, aspiram apenas
limitar"-se cada vez mais. Um homem como Jules Renard não se tornou o que
é por desdobramento, mas pelo mais descarado corte de seus
brotos.\footnote{No original em alemão: \emph{Trieb}. Parece existir um jogo
  que pode ser expresso nas possíveis traduções como broto, rebento ou
  pulsão. \versal{[N.~T.]}} Isso não é pouco. O senhor conhece seus diários? Um dos
documentos mais interessantes\ldots{} Mas algo que às vezes pode adquirir
traços grotescos. Comigo é bem diferente. Sei o quanto foi
atormentador meu primeiro contato com os livros de Stendhal, de que maneira
hostil esse universo me tocou num primeiro momento. Por isso mesmo me
senti por ele apaixonado. Mais tarde aprendi bastante com Stendhal''.
Gide sempre foi um grande aprendiz. Se observarmos mais de perto, isso
talvez o tenha decisivamente poupado, de forma mais determinante, da
influência estrangeira do que um obstinado recuo poderia ter feito. O
mais ``influenciado'' é o inerte, ao passo que o aprendiz chega, mais
cedo ou mais tarde, a apoderar"-se do que na criação estranha lhe é útil,
para incorporá"-lo como técnica em sua obra. Nesse sentido existem poucos
autores que mais tenham aprendido, com tanta dedicação, do que Gide.
``Fui em cada direção que tomei até o mais extremo, para então com a
mesma decisão tomar a direção oposta.'' Essa negação, em princípio, de
toda boa medida,\footnote{No original em alemão: ``\emph{Die goldene Mitte}'', expressão que
  significa meio termo, termo médio entre dois extremos. \versal{[N.~T.]}} essa
confissão em favor dos extremos, o que é, senão a dialética, não como
método de um intelecto, mas como sopro de vida e paixão desse homem.
Gide não quer aparentemente me contradizer, quando aqui suponho o motivo
de todos os mal"-entendidos e certas inimizades que o atingem. Ele
continua explicando: ``Muitos têm por certo que não faço outra coisa do
que desenhar a mim mesmo, e então quando meus livros põem em jogo as
mais diferentes personagens, eles concluem com sua `perspicácia': quão
sem caráter, vacilante e não confiável esse autor deve ser''.

``Integrar'', essa é a paixão de Gide no pensar e no expor. O interesse
crescente pela ``natureza'' --- conhecido por muitos grandes autores como
um direcionamento da vida na maturidade --- significa para ele:
o mundo também nos extremos é ainda completo, saudável e natural. E o
que leva a esses extremos não é a curiosidade ou o entusiasmo
apologético, mas um alto discernimento dialético.

Pode"-se dizer sobre esse homem: ele seria o ``\emph{poète des cas
exceptionnels}'', o poeta de casos excepcionais. Gide: ``\emph{Bien
entendu,} é assim mesmo. Mas por quê? Nós encontramos dia a dia
comportamentos e caráteres que através de sua mera existência tiram de
circulação nossas velhas normas. Uma boa parte de nossas decisões, as
mais banais ou as mais extraordinárias, escapam da avaliação ética
tradicional. E por isso é necessário acolher primeiramente tais casos,
com precisão, sem covardia ou cinismo''. Qualquer coisa que Gide tenha
escrito no estudo desses assuntos em romances como \emph{Les
faux"-monnayeurs},\footnote{Em francês no original: \emph{Os moedeiros falsos}. \versal{[N.~T.]}}
em ensaios, em sua importante autobiografia \emph{Si le grain ne meurt},\footnote{Em francês no original: \emph{Se o grão não morre.} \versal{[N.~T.]}} seus inimigos lhe
perdoariam, contanto que nisso tivesse apenas aquela pequena dose de
cinismo que sempre concilia tanto os \emph{snobs} quanto os
pequeno"-burgueses sobre tudo. O que os exaspera não é o ``imoral'', mas a
seriedade. Desta Gide é, contudo, inalienável, apesar de toda a malícia
de sua conversação e de toda ironia soberana que emerge em
\emph{Prométhée mal enchaîné},\footnote{Em francês no original: \emph{Prometeu mal
  acorrentado}. \versal{[N.~T.]}} em \emph{Nourritures terrestres}\footnote{Em francês no original: \emph{Frutos da terra}. \versal{[N.~T.]}} e em \emph{Caves du Vatican}.\footnote{Em francês no original: \emph{Subterrâneos do Vaticano}. \versal{[N.~T.]}} Ele é, como recentemente declarou Willy Haas,\footnote{Willy Haas (1891--1973), fundador da revista \emph{Die
  Literarische Welt}, para a qual Benjamin escrevia. \versal{[N.~O.]}} nesse momento, o
último francês da têmpera de Pascal. Na linhagem dos moralistas
franceses que prossegue com La Bruyère, La Rochefoucauld, Vauvenargues,
nenhum lhe é mais aparentado do que o próprio Pascal. Um homem a quem,
no século \versal{XVII}, teriam chamado de ``\emph{cas particulier}'', de doente,
se a terminologia clínica superficial de nossa época fosse conhecida. É
justamente por isso que Gide aparece, com Pascal, na linhagem dos
grandes educadores da França. Para os alemães, fechados em si mesmos,
recolhidos, encorujados, aquele sempre será o modelo, a figura educadora
pura que põe em relevo o tipo alemão, como hoje Hofmannsthal e Borchardt
tentam fazer. Para os franceses, entretanto, ricos em caráter popular
e multiplamente diferenciados em ramificações genealógicas, padronizados
em suas virtudes nacionais e literárias de maneira mais rígida e
precária do que qualquer outro povo, o grande caso de exceção à moral
esclarecida é a mais alta instância educadora. Este é Gide. Este
semblante, no qual às vezes o grande poeta mais se esconde do que se
revela, contrapõe, inabalavelmente, sua fronte\footnote{O termo
  \emph{front}, em francês no texto original, relacionado ao semblante de
  Gide, sugere uma fusão com a imagem inicial do texto com o forte ou
  fortaleza. Ao concluir com essa imagem, Benjamin utiliza a ambiguidade
  da palavra em francês que coincide em sentido no português: a fronte
  como testa e o fronte como fileira militar no campo de batalha. \versal{[N.~T.]}}
ameaçadoramente concentrada à indiferença moral e à suficiência branda.
