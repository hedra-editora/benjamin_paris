\chapter{Carta parisiense \versal{I}\\ \emph{André Gide e seu novo adversário}\footnote[*]{``Pariser Brief \textsc{i}. André Gide und sein neuer Gegner'', in
  \versal{GS III}, p.\,482--495. Tradução de Carla Milani Damião e Pedro Hussak. Publicada originalmente na revista \emph{Das Wort} em novembro de 1936. [\versal{N.~O.}]}}

\hedramarkboth{Carta parisiense I}{}

Uma memorável frase de Renan: ``Somente aquele que pode ter certeza de
que o que ele escreve permanece sem efeito possui liberdade de
pensamento''. Assim citou Gide. Se essa frase procede, o autor de
\emph{Nouvelles pages de journal}\footnote{André Gide.
  \emph{Nouvelles pages de journal (1931--1935)}. Paris: Gallimard, 1936. \versal{[N.}~de~\versal{W.B.]}}
dispõe de tão pouca liberdade de pensamento quanto seu oponente
Thierry Maulnier.\footnote{Thierry Maulnier. \emph{Mythes
  socialistes}. Paris: Gallimard, 1936. [\versal{N.}~de~\versal{W.B.}]} Ambos sabem claramente quais
são os efeitos de sua escrita e escrevem para provocar
efeitos. Se dedicamos"-lhes o mesmo interesse, justificamos em
fazê"-lo menos pela importância do mais jovem do que pela decisão pela
qual ele ocupou seu lugar em relação a um Gide e contra ele. No
momento em que Gide adere ao comunismo, ele passa a ter que lidar com
os fascistas.

Não que outros já não tivessem questionado Gide. Seu caminho foi seguido
atentamente desde 1897, quando, em um famoso artigo escrito no
\emph{Ermitage}, ele opôs"-se a Barrès, que se colocou àquela época, com
as \emph{Déracinés}, a serviço do nacionalismo.\footnote{Hoje, Gide
  pode remeter a este artigo. No volume do diário citado lê"-se: ``Não
  foi Barrès o apologista de um certo tipo de justiça que~se manifesta
  hoje como aquela de Hitler? E não foi fácil prever que no momento no
  qual qualquer outro fosse apoderar"-se dessas belas teorias,
  voltar"-se"-ia contra nós mesmos?'' [\versal{N.}~de~\versal{W.B.}]} Mais tarde, o desenvolvimento
religioso do Gide protestante foi literariamente seguido e
precisamente por ninguém menos do que seu amigo, o crítico católico
Charles Du Bos. O fato de que o \emph{Corydon} de Gide, que apresenta a pederastia
em suas condições e analogias históricas naturais, produza uma
tempestade não é difícil de entender. Em 1931, em torno dos 60 anos,
Gide estava acostumado a ser contestado quando ele descreveu seu caminho
para o comunismo no primeiro volume de seus \emph{Diários}.

A literatura burguesa reagiu a esse volume com uma profusão de glosas e
polêmicas. O fato de que o \emph{Écho de Paris} (próximo ao \emph{Croix de Feu}), sob a pena de François Mauriac, retorne três vezes a esse
livro pode dar ideia do alvoroço que Gide provocou. O debate foi muito
extenso, e também muito exacerbado para manter inteiramente o nível. Ele
teve seu ápice intelectual na \emph{Union pour la vérité}, em que Gide
discursou e respondeu perguntas de um círculo de escritores
importantes.\footnote{O debate foi publicado em Paris em 1935 sob o
  título \emph{André Gide et notre temps}. [\versal{N.}~de~\versal{W.B.}]} Ele não havia ainda se
acalmado, quando as \emph{Nouvelles pages de journal} foram publicadas
nesse ano.

Na medida em que o próprio Gide determinou a discussão, reiteradamente
ela versou sobre a questão de até que ponto ele permanece fiel a si
mesmo com essa sua mudança ou se realiza uma ruptura com o mundo mental
de sua idade madura. Gide poderia invocar --- e assim o fez no primeiro
volume de seus diários --- a paixão com a qual ele desde sempre fez sua a
causa do indivíduo; uma causa que ele reconhece ter, hoje, no comunismo seu
competente advogado. O novo volume dos diários contém vários apontamentos que
permitem reconhecer, no desenvolvimento de Gide, uma continuidade oculta, mas não menos importante. Gide alude a essa continuidade quando pensa
que a ``apologia da pobreza'' (p. 167) percorre todo o seu trabalho.
Ela encontrou expressão mais variada, estendendo"-se da inesquecível obra
juvenil \emph{O retorno do filho pródigo}\footnote{O livro foi
  publicado em tradução alemã de Rilke pela editora Insel. [\versal{N.}~de~\versal{W.B.}]} até a mais recente, \emph{Nouvelles nourritures},\footnote{André Gide,
  \emph{Les nouvelles nourritures}. {[}Paris{]} (1935). [\versal{N.}~de~\versal{W.B.}]} na qual
se pode ler: ``Por mim, tomei aversão a qualquer posse exclusiva.
Encontro minha felicidade em doar\footnote{No original em
  francês, Gide diz \emph{don}, dom. Benjamin traduz por \emph{Fortgeben},
  doar. \versal{[N.~T.]}} e a morte não vai me tirar das mãos grande coisa. De tudo o que
ela vai me privar, serei privado mais de bens esparsos, naturais, que
escapam a ser tomados e comum a todos. Quanto ao resto, eu prefiro a
refeição de albergue à mesa bem servida em casa, o jardim público ao
mais belo parque enclausurado entre muros, o livro que eu não tenho medo
de levar para uma caminhada à edição mais rara, e, se eu devesse
contemplar uma obra de arte sozinho, quanto mais ela fosse bela, mais
minha tristeza na contemplação iria sobrepujar meu contentamento'' (p.
61).\footnote{A citação remete a uma avaliação de Gide após seu
  engajamento ao partido comunista. Mesmo que Benjamin não cite
  diretamente o escrito de Gide \emph{De volta da \versal{URSS}} (1936),
  fortemente crítico ao stalinismo, ao citar essa passagem de \emph{Les
  nouvelles nourritures} (1935),\,ele
  acaba por indiretamente inserir as críticas de Gide ao stalinismo. [\versal{N.\,O.}]}
  %Leia"-se na citação
  %original: ``Tous les arguments de ma raison ne me retiendront pas sur
  %la pente du communisme¹ {[}¹\versal{N.}~de~\versal{W.B.}: Sur cette pente, qui %m'apparaît une
  %montée, ma raison a rejoint mon cœur. Que dis"-je? Ma raison
  %aujourd'hui l'y précède. Et si parfois je souffre de voir certains
  %communistes n'être que des théoriciens, me paraît aujourd'hui tout
  %aussi grave cette autre erreur qui tend à faire du communisme une
  %affaire de sentiment. (Mars 1935.){]}. Et ce qui me paraît une erreur,
  %c'est d'exiger de celui qui possède la distribution de ses biens; mais
  %quelle chimère que d'attendre, de celui qui possède, un renoncement
  %volontaire à des biens --- auxquels son âme reste attachée. \emph{Pour
  %moi j'ai pris en aversion toute possession exclusive; c'est de don
  %qu'est fait mon bonheur, et la mort ne me retirera des mains pas
  %grand'chose. Ce dont elle me privera le plus c'est des biens épars,
  %naturels, échappant à la prise et communs à tous; d'eux surtout je me
  %suis soûlé. Quant au reste, je préfère le repas d'auberge à la table
  %la mieux servie, le jardin public au plus beau parc enclos de murs, le
  %livre que je ne crains pas d'emmener en promenade à l'édition}
  %\emph{la plus rare, et, si je devais être seul à pouvoir contempler
  %une œuvre d'art, plus elle serait belle et plus l'emporterait
  %sur la joie ma tristesse}''.
 % (\emph{https://bit.ly/2OQsCyl}, p. 27). Em itálico, a parte citada por Benjamin. %Tradução: ``Todos os
  %argumentos da minha razão não vão me manter na encosta do
  %comunismo² [²\versal{N.}~de~\versal{W.B.}: Sobre essa encosta, que me parece uma subida,
  %minha razão se juntou ao meu coração. O que estou dizendo? A minha
  %razão hoje precede"-lhe. E se às vezes eu sofro em ver alguns
  %comunistas sendo apenas teóricos, hoje parece"-me tão grave este outro
 % erro que tende a fazer do comunismo uma questão de sentimento. (março de 1935.)]. E o %que me parece um erro consiste em exigir daquele
  %que tem posses a distribuição de seus bens. Mas que quimera esperar
  %daquele que possui, uma renúncia voluntária aos bens aos quais sua
  %alma permanece ligada. Por mim, tomei aversão a qualquer posse
  %exclusiva. É pelo dom que minha felicidade é feita e a morte não vai
  %me tirar das mãos grande coisa. Do que ela vai me privar mais são bens
  %esparsos, naturais, que escapam a ser tomados e comum a todos;
  %sobretudo deles eu me embebedei. Quanto ao resto, eu prefiro a
  %refeição de albergue à mesa bem servida, o jardim público ao mais belo
  %parque enclausurado entre muros, o livro que eu não tenho medo de
  %levar para uma caminhada à edição mais rara, e, se eu estivesse só
  %para poder contemplar uma obra de arte, mais ela seria bela e mais ela
  %conduziria para a alegria a minha tristeza''. \versal{[N. E.]}}

Para a ``apologia da pobreza'', Gide encontrou as formas mais diversas.
No fundo, todas elas coincidem com o desdobramento daquela pobreza que,
segundo o jovem Marx (o autor da \emph{Sagrada família}), a sociedade
tem a tarefa de tornar visível de modo não alterado. Para Gide, todas elas
aparecem como variedades da necessidade que o homem tem do homem. Se
Gide no curso de sua atividade criativa interessou"-se por muitas formas
de fraqueza, se ele em seu estudo sobre Dostoiévski, que em muitos
aspectos é um autorretrato, coloca como ponto central a fraqueza como
uma ``insuficiência da carne, uma inquietude, uma anomalia'', então ele
tem sempre que lidar com aquela fraqueza digna de uma extrema
participação que liga o humano ao humano.

Às vezes, o próprio Gide gosta de trazer à luz essa fraqueza. Mas o que
lhe determina a isso não é fraqueza, mas antes o cálculo. Ele dirige"-se para
este incógnito porque poderá ensiná"-lo algo sobre o mundo e sobre os
homens. E assim escreveu em maio de 1935: ``Pode"-se explicar a renúncia
de Tolstói à atividade artística pela redução de suas forças criativas.
Se uma segunda Anna Karênina tivesse se formado em seu interior,
ter"-se"-ia ocupado --- há bons motivos para supô"-lo --- menos com os
\emph{Dukhobors}\footnote{Grupo religioso cristão não ortodoxo da Rússia. [\versal{N.~O.}]} e comentado de forma menos depreciativa sobre a arte.
Mas ele sentiu que estava no fim de sua carreira literária: o impulso
poético não inflava mais o seu pensamento [\ldots{}]. Se hoje me ocupo de
questões sociais, é também porque o demônio da criação retira"-se de mim.
Essas perguntas só ocupam lugar porque esse último já o desocupou. Por
que eu deveria me superestimar? Por que não constatar em mim mesmo
aquilo que considerei em Tolstói incondicionalmente como um fenômeno de
decadência?'' (\emph{La Nouvelle Revue Française}, maio de 1935, p.
665).

Não queremos contradizer o autor aqui. Tampouco levantar a questão sobre
se as forças criativas não conhecem um sono passageiro (o próprio Gide
diz isso em suas \emph{Nouvelles pages}); se elas não podem trabalhar de
forma totalmente não"-endemoniada (as \emph{Nouvelles nourritures}
mostram isso); se elas não colidem com barreiras históricas (os
\emph{Faux"-monnayeurs}, de Gide, sugerem isso para o romance). Deixemos
que Gide, em seu incógnito, faça um encontro instrutivo. É o encontro
com Maulnier, que na \emph{Action française} cita essas afirmações de
Gide: ``Nenhum elogio ou censura pode acrescentar nada a essas linhas
estranhas. Acreditamos que não há quase nenhum exemplo de um criador
que se coloca em evidência com uma tal confissão. Achamos também que a
perspicácia, a modéstia e a coragem sem reservas para consigo mesmo, que
servem de base para um diagnóstico tão impiedoso, têm o direito ao nosso
respeito. Mas não podemos nos limitar a mostrar respeito aqui. Essa
franqueza trágica é rica em explicações que não temos o direito de
ocultar.''

Com essas afirmações, Maulnier começa uma crítica abrangente a Gide. Trata"-se de
uma crítica que lança muita luz sobre a posição fascista e,
especialmente, sobre o conceito de cultura do fascismo. Ter traído e
abandonado a ``cultura'' em favor do comunismo --- esta é a acusação que
Maulnier levanta contra as últimas obras de Gide.

A elaboração do conceito de cultura parece pertencer a um estágio
inicial do fascismo. Em todo caso, na Alemanha foi assim. De modo
imperdoável, a crítica revolucionária alemã, antes de 1930, deixou de
dedicar a atenção necessária às ideologias de Gottfried Benn ou Arnolt
Bronnen. Assim como esses foram os precursores do fascismo alemão, da mesma
forma Maulnier, hoje, não fosse pelo \emph{Front populaire}, deveria ser
contado entre aqueles do fascismo francês. Ele provavelmente não
escapará, em nenhum caso, ao rápido esquecimento. Pois quanto mais o
fascismo fortalece"-se, menos ele necessita de inteligências
qualificadas, precisamente na especialidade de Maulnier. As melhores
perspectivas são abertas às naturezas subalternas. O fascismo procura
cúmplices para o Ministério da Propaganda. Eis o motivo pelo qual Benn e Bronnen
foram exonerados.

A reação representada por Maulnier é uma reação especificamente fascista
e diferente daquela católica de um Claudel, daquela reação burguesa de
um Bordeaux, daquela mundana de um Morand, ou daquela filistina de um
Bedel. Ele encontra seus companheiros principalmente na geração mais
jovem.\footnote{Cf. Pierre Drieu La Rochelle. \emph{Socialisme
  fasciste}. Paris: 1934. [\versal{N.}~de~\versal{W.B.}]} Na geração mais velha, os fascistas
decisivos como Léon Daudet ou Louis Bertrand são casos isolados. O que
faz de Maulnier um fascista é a percepção de que a posição dos
privilegiados pode ser afirmada apenas pela violência. Representar a soma
de seus privilégios como ``a cultura'' --- assim ele avista a sua
particular tarefa. Subentende"-se, por conseguinte, que ele apresenta
como algo inconcebível que exista uma cultura não baseada no privilégio. E o
\emph{Leitmotiv} de seus ensaios é demonstrar que o destino da cultura
ocidental está indissoluvelmente ligado ao da classe dominante.

Maulnier não é um político. Volta"-se aos intelectuais, não às massas. A
convenção que prevalece entre os primeiros interdiz (na França ainda) o
chamado à violência bruta. Maulnier é compelido a uma particular cautela
quando apela à violência bruta. Ele pode, a dizer a verdade, apenas
preparar esse apelo. Faz isso de modo muito hábil quando proclama que
cabe a uma ``síntese da ação'' constranger a realidade interna e a
externa a unir"-se, mesmo quando uma ``síntese dialética'' permanece impossível (p.
19). Um pouco mais claramente, explica"-se que ele declara"-se de acordo
com a censura direcionada à civilização capitalista (que é sempre o
combate fictício dos fascistas) de não ter obtido a força ``de reconhecer
a insolubilidade'' dos problemas materiais e espirituais que
a nossa época erigiu (p. 8).

Hoje, a necessidade de não dar nenhum argumento contra os
privilegiados coloca o escritor, e sobretudo o teórico, em dificuldades
incomuns. Maulnier tem a coragem de dar uma solução rápida para essas
dificuldades. Em parte, elas são de natureza moral. O advogado do
fascismo tem muito a ganhar, se varrer do caminho os critérios morais.
De mais a mais, ele não se mostra muito exigente na escolha de seus
meios. Trata"-se de um trabalho rude --- o conceito não pode colocar luvas
para executá"-lo. Ele agarra firme, e dessa maneira: ``A civilização
{[}\ldots{}{]} é a instituição e a ordem de artifícios e ficções que
condicionam qualquer relação entre os homens, o sistema de convenções
úteis, a hierarquia artificiosa necessária para a vida em toda a sua
magnitude e indispensabilidade. A civilização é a mentira {[}\ldots{}{]}. Quem
não está disposto {[}\ldots{}{]} a reconhecer nessa mentira a condição
fundamental de todo progresso humano e grandeza humana admite ser um
adversário da própria civilização. É preciso escolher entre civilização
e sinceridade'' (p. 210). Assim afirma Maulnier no ensaio dirigido
contra Gide de sua coletânea. É em torno desse adágio de miserável
esplendor que os paradoxos desgastados de Oscar Wilde prestam"-se há
muito tempo, e pode"-se regressar facilmente até o ``declínio das
mentiras'' dele.

Alguém poderia perceber, pela primeira vez, como frutos diferentes às
vezes possuem as sementes de uma mesma vida. O mesmo homem que vê seu
esteticismo, a parte mais corruptível de sua produção, recebido pelo
fascismo, deu, no momento em que se coloca com seu desprezo contra a
sociedade com a qual ele se divertiu durante toda a vida, ao jovem
André Gide um exemplo que determinou sua vida posterior.\footnote{A
  importância que Wilde tinha para Gide é testemunhada pelo seu
  \emph{Elogio fúnebre de Wilde}, de 1910 {[}Oscar Wilde, \emph{Mercure
  de France}, 1910{]}. [\versal{N.}~de~\versal{W.B.}]} Em segundo lugar, entenderíamos quão
profundamente a ideologia fascista está comprometida com o decadentismo
e esteticismo e por que ela encontra pioneiros entre os artistas
extremistas tanto na França quanto na Alemanha ou na Itália.

Qual destinação a arte deve esperar de uma civilização erguida sobre a
mentira? Expressará, em sua esfera mais estrita, as contradições não resolvidas da civilização --- e insolúveis enquanto for preservado o sistema da propriedade.
A contradição na arte fascista é, não diferentemente da economia
fascista ou do Estado fascista, uma contradição entre teoria e
\emph{práxis}. A teoria da arte fascista tem os traços do puro
esteticismo: a arte é apenas uma das máscaras atrás das quais ``nada
está'', como Maulnier formula, ``além da natureza animal do homem, o
animal humano nu e despojado de tudo de Lucrécio'' (p. 209). Essa arte é
reservada aos sábios, à elite, ``que é beneficiária de toda civilização
na qual ela'', como Maulnier diz muito claramente, ``representa os
parasitas, os herdeiros e a flor inútil'' (p. 211). Assim parece ser o
caso na teoria. A \emph{práxis} fascista oferece uma imagem diferente.
A arte fascista é uma arte de propaganda. Seus consumidores não são os
sábios, mas, totalmente ao contrário, os que se enganam. Além disso, no
presente eles não são poucos, mas muitos ou, ao menos, muito numerosos.
Portanto, é evidente que as características dessa arte não correspondem
inteiramente àquelas manifestas em um esteticismo decadente. Nunca o
decadentismo voltou sua atenção para a arte monumental. A ligação da
teoria decadentista da arte com sua \emph{práxis} monumental ficou
reservada ao fascismo. Nada é mais instrutivo do que esse cruzamento
em si mesmo contraditório.

O caráter monumental da arte fascista está conectado ao seu caráter de
massa. Mas esse não é de modo algum direto. Nem toda arte de massa é uma
arte monumental: a dos contos de Hebel para o calendário dos camponeses
é tão pouco monumental quanto a opereta de Lehár. Se a arte de massa fascista
é uma arte monumental --- e ela o é até dentro do estilo literário ---,
então isso tem um significado particular.

A arte fascista é uma arte de propaganda. Portanto, é executada para
massas. Além disso, a propaganda fascista deve penetrar em toda a vida
social. A arte fascista é, portanto, executada não apenas \emph{para} as
massas, como também \emph{pelas} massas. Isso parece supor que a massa
tem que lidar consigo mesma nessa arte, compreende"-se a si mesma, ela é a
dona da casa: dona dos seus teatros e dos seus estádios, dona dos seus
estúdios de cinema e das suas editoras. Todos sabem que não é o caso.
Nesses lugares, quem domina muito mais é a ``elite''. E ela não quer a
autoconsciência da massa na arte. Pois então a arte teria que ser uma
arte da classe proletária pela qual a realidade do trabalho assalariado
e da exploração chegaria ao seu direito, isto é, ao caminho de sua
abolição. Mas com isso a elite seria prejudicada.

O fascismo, por conseguinte, tem interesse ​​em limitar o caráter
funcional da arte de tal maneira que não se deve recear de sua parte
nenhuma influência transformadora da posição de classe do proletariado
--- que constitui a maior parte daqueles que ela atinge e uma parte menor
dos quadros que a executam. A esse interesse político"-artístico serve a
``forma monumental''. E isso acontece de duas maneiras: em primeiro
lugar, ela bajula a ordem econômica pacífica existente, apresentando"-a
em termos de seus ``traços eternos'', isto é, intransponíveis. O
Terceiro Reich calcula o tempo em milênios. Em segundo lugar, ela
desloca os executores, bem como os receptores, para um feitiço
no qual eles mesmos devem aparecer como monumentais, ou
seja, incapazes de ações refletidas e autônomas.\footnote{Não há
  um efeito enfeitiçante apenas na estilização fascista das artes de
  massa (compare os desfiles alemães com os russos), mas também no
  quadro das diferentes ``comunidades'' e ``frentes'' em que elas
  ocorrem. [\versal{N.}~de~\versal{W.B.}]} A arte aumenta assim as energias sugestivas de seu efeito às
custas dos intelectuais e dos esclarecidos. A perpetuação das condições
existentes ocorre na arte fascista através da paralisia dos homens
(executores ou receptores) que poderiam alterar essas condições. Apenas
com a atitude que o feitiço impõe"-lhe --- assim ensina o fascismo --- a
massa chega enfim à sua expressão.

O material com o qual o fascismo constrói seus monumentos, que considera
perene como o bronze, é, sobretudo, o assim chamado material humano.
Nesses monumentos, a elite imortaliza seu domínio. E é apenas graças a
esses monumentos que o material humano encontra sua forma. Diante do
olhar dos senhores fascistas --- que como vimos abrange milênios ---, é
ínfima a diferença entre os escravos que erigiram as pirâmides com
blocos de pedra e as massas de proletários que formam blocos nas praças
e nos campos exercício em frente do \emph{Führer}. Então, pode"-se
entender bem Maulnier quando ele coloca juntos ``construtores e
soldados'' como expoentes da Elite (melhor Gide, quando reconhece na
nova construção monumental romana a presença de um ``jornalismo
arquitetônico'' {[}\emph{Nouvelles pages}, página 85{]}).

O esteticismo de Maulnier não é, como foi sugerido, um ponto de vista
improvisado que o fascismo adota apenas no debate de questões históricas
da arte. O fascismo depende desse ponto de vista em todos os lugares,
quando quer aproximar"-se da aparência sem se comprometer com a
realidade. Um modo de ver as coisas que evita o valor funcional da arte
também será recomendável quando houver interesse de remover do campo
visual o caráter funcional de um fenômeno. Isso, como pode ser visto no
caso de Maulnier, é em uma importante medida o caso da técnica. O motivo
é facilmente compreensível. O desenvolvimento das forças produtivas, que
ao lado do proletariado compreende também a técnica, provocou a crise
que impulsiona a socialização dos meios de produção. Em primeiro lugar,
essa crise é, portanto, uma função da técnica. Quem pensa em resolvê"-la
de maneira irracional e violenta, preservando os privilégios, tem todo o
interesse em tornar o caráter funcional da técnica tão irreconhecível
quanto possível.

Há dois caminhos a seguir. Eles levam a direções opostas, mas são
determinados por ideias afins: a saber, precisamente ideias estéticas.
Encontramos o primeiro caminho em Georges Duhamel.\footnote{Georges
  Duhamel, \emph{Scènes de la vie future}, Paris 1930, e
  \emph{L'humaniste et l'automate}, Paris 1933. [\versal{N.}~de~\versal{W.B.}]} Ele tende
decididamente a deixar de lado o papel da máquina no processo de
produção e a criticá"-la levando em conta os vários inconvenientes e
desvantagens ligados ao uso, próprio ou outro, das máquinas pelo
homem privado. Duhamel chega a um juízo circunspecto sobre o automóvel,
a uma rejeição resoluta do filme, a uma proposta meio brincalhona e meio séria, segundo a qual todas as invenções seriam banidas pelo
Estado por cinco anos. O proletário revolta"-se contra o empresário; o
pequeno"-burguês contra a máquina. Duhamel toma partido contra a máquina
em nome da arte. É compreensível que para o fascismo as coisas sejam um
pouco diferentes. A grande mentalidade burguesa de seus chefes
deixou seu rastro nos intelectuais que estavam à sua disposição. Um
deles era Marinetti. A princípio, ele sentiu instintivamente que uma
visão ``futurista'' da máquina era útil ao imperialismo. Marinetti começou
como um \emph{bruitist},\footnote{\emph{Bruit}, do francês, som ou
  ruído. O bruitismo foi um movimento que partiu de considerações
  iniciais de Marinetti e Ferruccio Busoni, em um manifesto de 11 de
  março de 1913, ``A arte do ruído'', de Luigi
  Russolo. Segundo este, um novo conceito musical desvinculado da
  concepção tradicional de música havia surgido, igualando combinações
  instrumentais clássicas de sons com ruídos de máquinas. \versal{[N.~O.]}} proclamando
o ruído (a atividade improdutiva da máquina) como sua atividade mais
significativa. Ele acabou por tornar"-se membro da academia real,
confessando ter encontrado na guerra da Etiópia a realização de seus
sonhos futuristas de juventude.\footnote{Cf. O manifesto de
  Marinetti sobre a guerra da Etiópia. [\versal{N.}~de~\versal{W.B.}]} Sem se dar conta disso,
Maulnier refere"-se a ele quando, contra o ``novo humanismo'' de Górki,
declara que o que constitui o principal valor das descobertas técnicas e
científicas não era ``tanto seu resultado e sua possível utilidade
{[}\ldots{}{]} quanto {[}\ldots{}{]} seu valor poético'' (p. 77). ``Marinetti ---
escreve Maulnier --- inebriou"-se com o nível das máquinas, com seu
movimento, seu aço, sua precisão, seu ruído, sua velocidade --- em suma,
com tudo o que pode ser considerado como um valor em si na máquina e que
não participa de seu caráter instrumental {[}\ldots{}{]}. Limitou"-se a e
deliberadamente a mantê"-la em seu aspecto inutilizável, isto é, estético''
(p. 84).

Maulnier considera que essa posição está tão fundamentada que ele não
tem escrúpulos em citar, como uma curiosidade, as frases em que
Maiakóvski trata da visão de Marinetti da máquina. Maiakóvski fala
a linguagem saudável do senso comum: ``A era da máquina não exige hinos
em seu louvor; exige ser dominada pelo interesse da humanidade. O aço
dos arranha"-céus não exige imersão contemplativa, mas uma recuperação
decisiva da construção de moradias\ldots{} Não vamos buscar o ruído, mas
organizar os silêncios. Nós poetas queremos poder falar nos vagões de
trem'' (p. 83). A atitude digna de Maiakóvski, porquanto reservada e
sóbria, é incompatível com o esforço de retirar da técnica um aspecto
``monumental''. Ela guarda um testemunho conclusivo contra a afirmação
de Maulnier de que o coletivismo russo fez do ``engenheiro um senhor
espiritual'' (p. 79). Trata"-se de uma reinterpretação tecnocrática que
falsifica a formação politécnica do cidadão soviético,
transformando"-a em trabalho servil tecnocrático e supervisionado.
E é também uma reinterpretação tecnocrática em um outro sentido: trata"-se de
uma reinterpretação próxima exatamente dos tecnocratas.

Ora, ninguém mais decididamente do que Maulnier recusará a acusação de
pensar tecnocraticamente. Esse modo de pensar parecer"-lhe"-á, em vez
disso, incompatível com o artístico. À primeira vista, sua definição de
arte poderia dar"-lhe razão. Ela diz: ``A verdadeira missão da arte é
tornar inutilizável objetos e criaturas'' (p. 86). Mas não nos
contentemos com a primeira vista. Vejamos as coisas mais de perto!
Dentre as artes, existe uma que satisfaz a definição de Maulnier de uma
maneira particularmente exata. Essa arte é a arte da guerra. Ela
incorpora a ideia da arte fascista tanto pelo uso monumental do material
humano quanto pela aplicação de toda a técnica inteiramente desvinculada
de propósitos banais. O lado poético da técnica, que o fascista joga
contra o prosaico, de que os russos, em sua opinião, fazem muito caso, é
o seu lado homicida. Nesse sentido, o significado da sentença: ``Tudo o
que é primitivo, espontâneo, inocente, é"-nos somente por isso odioso''
(p. 213) alcança plenamente sua validade.

Essa frase pode ser encontrada na última seção do ensaio em que Maulnier
polemiza com Gide. A capacidade de provocar reações reveladoras não
merece reconhecimento? Gide não encarnou a figura ideal que ele evoca em
seu \emph{Diário}, em 28 de março de 1935: o \emph{inquiéteur} --- aquele
que provoca inquietude? De fato, ele tornou"-se o porta"-voz daqueles que
perturbam o autor fascista como nunca antes.

Essas são as massas, e na verdade as massa leitoras. ``Através dos esforços
gigantescos para o benefício de todos os estágios de ensino, eliminando
qualquer barreira entre os vários níveis de formação\ldots{} através da
surpreendentemente rápida redução do analfabetismo\ldots{}, com o apelo
imediato à invenção literária de todos, e até mesmo das crianças\ldots{} por
meio de tudo isso vocês'' --- assim falou Jean"-Richard Bloch no Congresso
dos Escritores de Paris, em 1935, dirigindo"-se aos representantes da
União Soviética --- ``oferecem ao escritor\ldots{} o presente mais
maravilhoso que ele sonhou: presentear"-lhe um público de 170 milhões de
leitores''.

Para o escritor fascista, é um ``presente de grego''. Para a elite, da qual Maulnier vem em socorro, é impensável um gozo artístico que não esteja protegido por todos os lados contra os elementos perturbadores por meio do monopólio da formação cultural. A abolição do monopólio da formação cultural em si seria para
Maulnier assustador o suficiente. E agora Górki diz"-lhe que é
precisamente a arte que é chamada a participar dessa abolição,
dizendo"-lhe~que na literatura soviética não existe diferença fundamental
entre um livro de divulgação científica e um livro de valor artístico. E
com essa proposição, por meio dos divulgadores mais modernos da literatura
ocidental, um Frank, um De Griffin, um Eddington, um Neurath --- Maulnier %notas?
não pode fazer nada de melhor do que inseri"-la em sua descrição da
``barbárie'', ``à qual Górki colocou"-se a serviço'' (p. 78).

Maulnier também não cede um palmo de sua ideia de apresentar a cultura
como a soma dos privilégios. Talvez com essa maneira de ser apresentada ele não faça
uma bela figura. Mas, ao procurar a confrontação da cultura imperialista com a da
Rússia soviética, Maulnier deve resignar"-se com isso. Ele não pode mudar
que o caráter de consumo da primeira contraponha"-se ao caráter produtivo da segunda. A ênfase extenuante na criatividade, que nos é
familiar a partir do debate cultural, tem antes de tudo a tarefa
principal de desviar a atenção sobre quão pouco o produto gerado de
forma tão ``criativa'' vem em benefício do processo produtivo, mas
recai exclusivamente no consumo. O imperialismo determinou uma situação \enlargethispage{\baselineskip}
na qual a poesia, celebrada como ``divina'', compartilha legitimamente tal elogio com o bolo de sobremesa.

Maulnier não pode renunciar a qualquer preço à ``criatividade''. ``O
homem'', ele escreve, ``fabrica uma coisa para usá"-la; mas cria para
criar'' (p. 86). A formação politécnica dos soviéticos demonstra quão falaciosa é a separação morta e a"-dialética entre
criação e fabricação que está na base da estética da criatividade.
Essa formação é
igualmente capaz tanto de conduzir para um trabalho criativo o operário
fabril, no âmbito do plano de produção que ele domina, de uma comunidade
produtiva que sustenta sua existência, de um modo de produção que ele
pode melhorar, como induzir o escritor --- pela precisão das tarefas que
lhe são colocadas, ou seja pelo público específico que o garante --- a
uma produção que, graças à prestação de contas que o produtor pode dar
de seu procedimento, tem direito ao título honorífico de fabricado. E
precisamente o escritor deveria lembrar que a palavra ``texto'' --- de
tecedura: \emph{textum} --- já foi um tal nome honorífico. Com a futura
educação politécnica do homem diante de seus olhos, ele não ficará
impassível diante do porta"-voz da elite, que lhe narra que, ``para uma
sociedade coletivista, aqueles momentos fugazes em que o homem pode
escapar de uma existência que, como nos tempos cinzentos, é quase
inteiramente dedicada ao sustento {[}\ldots{}{]} serão considerados deserção''
(p. 80). A quem deveria agradecer o homem, se esses momentos eram tão
fugazes? À elite. Quem tem interesse em tornar o mesmo trabalho digno da
humanidade? O proletariado.

Em sua construção, ele pode certamente dispensar o que
Maulnier nomeia de ``privilégios da interioridade'' (p.\,5), mas nunca
o que sente e descreve esses privilégios como Gide o faz (em 8 de
março de 1935): ``Hoje me vem à consciência, de modo opressivo e
profundo no interior, o sentimento de uma inferioridade: nunca precisei
ganhar o pão, nunca trabalhei sob a pressão da necessidade. Mas sempre
amei tanto o trabalho que minha felicidade não teria sido afetada por
isso. Além disso, quero ir além do seguinte: chegará um momento em que
será considerado uma falha não ter conhecido tal trabalho. A imaginação
mais rica não pode substituí-lo; a instrução que ele transmite nunca
poderá ser recuperada. Chegará o momento em que o cidadão vai sentir"-se
inferior ao trabalhador simples. E, para alguns, este tempo já chegou''
(\emph{Nouvelles pages}, p. 164f).

Para Maulnier, ainda mais alarmante do que o fato de haver um público de
170 milhões de leitores no Leste, é que existam escritores que moram na
França pensando nisso. André Gide dedicou seu último livro \emph{Les
nouvelles nourritures} aos jovens leitores da União Soviética. O
primeiro parágrafo desse livro diz: 

\begin{quote}
Tu que virás quando eu não puder mais ouvir os sons da terra e meus
lábios não beberem mais o seu orvalho --- tu que mais tarde, talvez, me
lerá ---, escrevo estas páginas para ti; pois talvez o fato de viver não te
surpreenda o bastante; tu não serás dominado pela maravilha entorpecente
que é a tua vida. Parece"-me às vezes que será com a minha sede que tu
beberás,\,e o que faz com que tu te curves sobre outra criatura que tu\,acaricias é meu próprio desejo, hoje!\,(\emph{Nouvelles nourritures}, p.\,9).\footnote{Em francês no original: ``Toi qui viendras lorsque je
  n'entendrai plus les bruits de la terre et que mes lèvres ne boiront
  plus sa rosée --- toi qui, plus tard, peut"-être me liras --- c'est pour
  toi que j'écris ces pages; car tu ne t'étonnes peut"-être pas assez de
  vivre ; tu n'admires pas comme il faudrait ce miracle étourdissant
  qu'est ta vie. Il me semble parfois que c'est avec ma soif que tu vas %\enlargethispage{\baselineskip}
  boire, et que ce qui te penche sur cet autre être que tu caresses,
  c'est déjà mon propre désir.'' \versal{[N.~T.]}} %\enlargethispage{-\baselineskip}
\end{quote}

