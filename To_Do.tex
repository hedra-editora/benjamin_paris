TO_DO

Padronizações


++ Bater minuciosamente padrão do 1º volume com esse (abreviaturas, folha de rosto, página de cŕeditos, disposição dos títulos no sumário...)

++ Padronizar: títulos de livro vem no original em francês? Então todas as ocorrências devem aparecer como tal

++ .'' X ''.

++ : C.A. X : c.b.

----------------------------------------------------------------------------------------
COMERCIANTE NO POETA:
+ Manter menção à edição Kindle?


+ Manter a acentuação do português da época da tradução de Milliet?: sôbre, tôdo(a)(as), amigàvelmente, vêzes, êle, sòzinho, êsse, vôo, apóia? Se for o caso de manter, acrescentar [sic] após os termos?

CARTA PARISIENSE ++ p.208: manter apenas trecho traduzido? Não vejo a necessidade de manter o trecho original. Idem na página 222. Por que trazer os trechos originais em francês? + l. 508: A footnote com o trecho original em francês está gerando uma viúva da footnote na página seguinte. Talvez suprimir o trecho e deixar indicado: "No original o trecho está em francês. Tradução de...", como adotaram nos parágrafos de Gide que Benjamin escreve em traduções alemães. Além de solucionar a viúva, deixa padronizado com o outro parágrafo.


--------------------------------------------------------------------

+ Benjamin concebe algo capaz de indicar uma nova concepção


+ Os vícios graves, genuínos --- em outras palavras o
socialmente perigosos ---

+ pavimentada com cubos de pedra

+o porte espiritualmente pleno da figura que envelhece é ainda capaz de
impactar as pessoas, mesmo que olhar e voz recusem"-se