TO_DO

GERAL:

- Benjamin, quando se refere a livros, usa aspas, e não itálico. Mantemos aspas para ficar próximo à anotação original, ou mudamos para itálico para adequar ao nosso padrão editorial?



1ª parte----------------------------------------

COMERCIANTE_POETA:

l.42: Não tem itálico mesmo a apresentação por Pierre Mac Orlan:``\emph{Prochainement ouverture \ldots{} de 62 boutiques littéraires} presentées par Pierre Mac Orlan''?

TALENTOS_PARISIENTES:

l.65: colchete que não fecha;

l.73: assim mesmo, esse ``um'' sem itálico: \emph{Rive Gauche} um \emph{pendant}?

DIARIO_PARISIENSE:

l.310: grande parte do texto anterior está replicado nas páginas do diário (longa passagem sobre Fargue, a citação sobre Sade, apartamento e relação de Berl com os surrealistas)

l.510: ``perfeito'' mesmo ou ``prefeito''?

l.526: ``notado que a descoberta''... acho que sem esse ``que'' faz mais sentido

l.810: a informação sobre a publicação do Diário estava ao final do texto. Achei estranho, joguei provisoriamente para rodapé, mas podemos pensar em alternativas juntos.


2ª parte----------------------------------------

PAUL_VALERY_ANIVERSARIO:

l.44: Aspas sem fechar

l.150: Em ``quanto contraria a obsessão da inspiração'' é ``quanto'' mesmo ou ``quando''? O ``quanto'' me causou estranhamento na frase;