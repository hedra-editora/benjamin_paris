TO_DO

GERAL:

+ Benjamin, quando se refere a livros, usa aspas, e não itálico. Mantemos aspas para ficar próximo à anotação original, ou mudamos para itálico para adequar ao nosso padrão editorial?

+ No primeiro volume da coleção, todas as notas de rodapé dos capítulos indicam o título original do texto. Recuperar esses títulos para essa publicação?

+ Padronizar ``No original em francês:'' X ``Em francês no original:'' E, igualmente, padronizar se o termo traduzido depois dos dois pontos vem em c.a. ou c.b.

+ Padronizar: o nome do ensaio vai entre aspas ou itálico, quando referenciado ao longo do texto?

+ No primeio volume da coleção, termos originais em alemão estão em itálico ou aspas, como nesse volume?

1ª parte----------------------------------------

COMERCIANTE_POETA:

+ l.42: Não tem itálico mesmo a apresentação por Pierre Mac Orlan:``\emph{Prochainement ouverture \ldots{} de 62 boutiques littéraires} presentées par Pierre Mac Orlan''?

TALENTOS_PARISIENTES:

+ l.65: colchete que não fecha;

+ l.73: assim mesmo, esse ``um'' sem itálico: \emph{Rive Gauche} um \emph{pendant}?

DIARIO_PARISIENSE:

+ l.310: grande parte do texto anterior está replicado nas páginas do diário (longa passagem sobre Fargue, a citação sobre Sade, apartamento e relação de Berl com os surrealistas)

+ l.510: ``perfeito'' mesmo ou ``prefeito''?

+ l.526: ``notado que a descoberta''... acho que sem esse ``que'' faz mais sentido

+ l.810: a informação sobre a publicação do Diário estava ao final do texto. Achei estranho, joguei provisoriamente para rodapé, mas podemos pensar em alternativas juntos.


2ª parte----------------------------------------

PAUL_VALERY_ANIVERSARIO:

+ l.44: Aspas sem fechar

+ l.150: Em ``quanto contraria a obsessão da inspiração'' é ``quanto'' mesmo ou ``quando''? O ``quanto'' me causou estranhamento na frase;

ANOTACOES_PAUL_VALERY

+ l.27: parágrafo entre chaves ``{}''. É assim mesmo? Não é entre parênteses?

3ª parte----------------------------------------

IMAGEM_PROUST:

+ l.59: Nota sem indicação se é do autor ou tradutor (assumi que seja do tradudor pelo conteúdo);

+ l.209: Há uma numeração de secion II, mas no início do texto não o I. Iniciar o texto como a sessão I?

PROUST_PAPIERE

+ Pensar em uma forma melhor de deixar os subtítulos referentes ao conteúdo do material que se seguirá (como "Materiais relativos ao ensaio sobre Proust")

+ É assim mesmo, todos os parágrafos entre chaves? Se for o caso, indicar o que significam as chaves em nota no início do capítulo

+ Estou deixando como footnote as indicações ao final dos trechos das páginas a que se referem no Arquivo Walter Benjamin. Pensar em solução melhor... do jeito que estava, no meio do texto, fica muito esquisito... Talvez deixar em itálico, em fonte um pouco menor, recuado à direita...

+ l.65: parêntesis mesmo?

+ l.83: início de parênteses sem fechamento

+ l.124: "com Proust mesmo" --> É com Proust, ou como Proust?

+ l.176: Indicar de quem é a anotação entre co+ lchetes [do piano]?

+ l.252: Não seria "parênteses" aberto e não fechado, ao invés de "parágrafo"?

+ l.278: "teria inequivocamente desembocar" --> acrescentar um "que"? --> "teria inequivocamente [que] desembocar"

+ l.705: chave sem abertura

>>>>>OLhar atentamente para esse capítulo, que tem muita coisa esquisita de pontuação para tentar reproduzir a escrita fragmentária do caderno de anotações de Benjamin<<<<<<

+ l.865: Parênteses vazio