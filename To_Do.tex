TO_DO

----------------------------------------------------------------------------------------
Padronizações

++ Bater minuciosamente padrão do 1º volume com esse (abreviaturas, folha de rosto, página de cŕeditos, disposição dos títulos no sumário...)

++ Repensar título. Sugestões "A Paris do século XX"; "A Paris do século XX e o crítico literário"; "Paris e o lugar do crítico literário"; "Proust, Gide, Valéry e o lugar do crítico literário"; "A nova literatura francesa e o lugar do crítico literário"

++ Expressões francesas que aparecem mais de uma vez no texto: remeter o restante das notas à primeira nota com a tradução, em vez de ficar replicando toda vez a tradução?

++ .'' X ''.

++ : C.A. X c.b.

++ Padronizar: títulos de livro vem no original em francês? Então todas as ocorrências devem aparecer como tal

++ Verificar, de acordo com o primeio volume da coleção, se termos originais, em alemão e francês, estão em itálico ou aspas. Padronizar <<<<ITÁLICO
+ Itálico x aspas em termos em francês e alemão [em notas e no texto corrido]

+++ No primeiro volume da coleção, todas as notas de rodapé dos capítulos indicam o título original do texto. Recuperar esses títulos para essa publicação:

++ REPENSAR REPETIÇÃO DE NOTAS

----------------------------------------------------------------------------------------

+ FALTA INDICAR O TÍTULO ORIGINAL DO TEXTO: <<<IMPORTANTE>>>
-- Paul Valéry – Em seu sexagésimo aniversário;
-- Folha de anotações para o ensaio sobre Paul Valéry;

+ TÍTULOS ADICIONADOS POR MIM, VERIFICAR:
-- O Comerciante no poeta (Der Kaufmann im Dichter)
-- Cabeças parisienses (Pariser Köpfe)
-- Diário parisiense (Pariser Tagebuch)
-- Paul Valéry na École Normale (Paul Valéry in der École Normal)
-- Édipo ou o mito racional (Oedipus oder Der vernünftige Mythos)
-- Carta parisiense I (Pariser Brief 1. André Gide und sein neuer Gegne)
-- Carta parisiense II (Pariser Brief 2. Malerei und Photographie)

+ FALTA INDICAR LOCAL/DATA DA PRIMEIRA PUBLICAÇÃO:

-- Cabeças parisienses; 
-- Paul Valéry na École Normale;
-- Paul Valéry – Em seu sexagésimo aniversário;
-- Imagem de Proust; 
-- Carta parisiense I;
-- Carta parisiense II;

++ Grande-burguesia x grande burguesia || pequena-burguesia x pequena burguesia?

++++++++++++++++++++++++++++++++++++++++++++++++++++++++++++++++++++++++++++
OBSERVAÇÕES AOS TRADUTORES/ORGANIZADORES:

APRESENTAÇÃO:

++ p. 14: "Da entrevista com Gide realizada em 1929" --> seria 1928, como aparece na nota de rodapé?

++ p. 14: ...ao "Diário de Paris" --> não seria melhor "Diário parisiense", como foi traduzido no volume?

++ p. 14: "o interesse é marcado pela preocupação com a recepção dos escritores" --> achei a frase esquisita, esse "interesse marcado pela preocupação", talvez substituir por "ressalta-se a preocupação..."?

++ p. 17: "Benjamin, no entanto, não chegou de traduzir a obra" --> achei estranho também, não seria "não chegou a traduzir", ou "não chegou a terminar de traduzir"?

++ p. 18: nota de rodapé nº13: Falta ano de edição da primeira obra mencionada de Pierre-Quint. Também estranhei, na nota 12, a obra "André Gide. Sa vie, son œuvre" aparecer como pela editora "Librarie Stock, 1952" e, na nota 18, a mesma obra aparecer pela editora "Éditions Stock, 1933". É isso mesmo?

++ p. 20: "as dificuldades materiais que Benjamin se vê forçado a enfrentar em seu estilo" --> é "estilo" mesmo? ou seria "exílio"?

++ p. 23: "Reciprocidade demonstrada por Benjamin ao sugerir a publicação do livro de Pierre-Quint sobre Gide na Alemanha" --> achei a frase um pouco solta, sem um conectivo com a frase anterior...

++ p. 24: nota de rodapé º24: achei a nota um pouco prolixa, repetindo referências bibliográficas que já aparecem ao longo da "Apresentação". Entendo que a função seja fazer um levantamento bibliográfico da relação de Benjamin com Paris, mas creio que, editorialmente pensando, seja escuso repetir referências que já aparecem citadas anteriormente no texto...

1ª PARTE----------------------------------------

TRÊS FRANCESES:

++ p. 33: nota de rodapé nº4: "Benjamin repete a expressão de Gide sobre Baudelaire: “magicien ès lettres”" --> não encontrei essa citação de Gide sobre Baudelaire, mas essa é a frase com a qual Baudelaire abre as "Flores do mal", em referência a Gautier...

COMERCIANTE NO POETA:

++ p. 38: "por Pierre Mac Orlan; Henri Guilac, arquiteto; Simon Kra, empreendedor.
(“par Pierre Mac Orlan”)" --> precisamos deixar essa informação entre parênteses, uma vez que ela já vem traduzida anteriormente [por Pierre Mac Orlan...]? Achei redundante

CABEÇAS PARISIENSES:

++ p. 45: "O que de outro a obra de Sade ensina a reconhecer senão a extensão à qual um espírito verdadeiramente revolucionária" --> seria "revolucionário", para concordar com "espírito", não?

++ p. 46: A tradução mais precisa para "Mort de la pensée bourgeoise" seria "Morte do pensamento burguês", em vez de "Morte ao pensamento burguês", não?

++ p. 50: "“C’était ma première altesse” (“Foi minha primeira alteza”)" --> no original, Benjamin traz a citação em francês e, logo em seguida, a tradução em alemão entre parênteses? Se for nossa tradução, precisa ir para nota de rodapé para manter padrão;

++ p. 51: "ouvir a agradável inflexão com o qual ele evocava" --> seria "com a qual", não?

DIÁRIO PARISIENSE:

++ p. 54: "Certamente a história da construção da cidade não foi menos movimentada, não menos cheia de atos" --> não seria melhor "não foi menos movimentada, nem menos cheia..."?

++ p. 57: "“Frague”, escreve Léon Pierre-Quin..." --> "Frague" é propositalmente grafado errado, certo?

++ p. 62: "Anfítrion" mesmo? Não "Anfitrión" ou "Amphitryon"?

++ p. 65: "que eu recém-adquirira “eles são seminaristas, nada além disso”" --> manter a citação original em francês, e a tradução no rodapé, como adotado no texto anterior?


++ p. 67: adicionar nota com a tradução de "franchise"?

++ P. 67: "Dausse apresentou-me como um alcoviteiro deus fluvial de porcelana" --> é isso mesmo, Benjamin que foi apresentado como um "alcoviteiro deus"?

++ p. 68: adicionar notas com a tradução para "arrangement", "pochettes surprise" e "Bal des trois colonnes"?

++ p. 70: adicionar nota com a tradução para "Maison de rendez-vous"?

++ p. 71: adicionar nota sobre "Félix Bertaux"?

++ p. 72: "Nossa conversa girou sobretudo em torno a Proust" --> não seria melhor "em torno de Proust"?

++ p. 72: "eu devia ter notado que a descoberta do último volume no qual o caminho" --> não faz mais sentido sem esse -que, "eu devia ter notado a descoberta..."?

++ p. 76: "temo também de estar atrapalhando o seu trabalho" --> não seria melhor sem o -de, "temo também estar atrapalhando"?

++ p. 80: "menos do mundo do que a própria vida" --> não seria melhor "menos do mundo do que da própria vida" (moins du monde que de la vie même)?

2ª PARTE----------------------------------------


ANOTACOES_PAUL_VALERY
<<<<l.27: parágrafo entre chaves ``{}''. -- REPENSAR NOTAÇÃO>>>>


3ª PARTE----------------------------------------


PROUST_PAPIERE

++ Repensar o uso das chaves para indicar trechos suprimidos

++ De quem é a primeira nota de rodapé?

++ Notas explicando que os colchetes são acréscimos dos editores alemães são desnecessárias, não? uma vez que isso já vem explicado na primeira nota do texto;

4ª PARTE----------------------------------------

GIDE_ALEMANHA

++ l. 132: "torna-se algo raro". Não seria tornar-se algo raro?


VOCACAO_GIDE

+ Manter a acentuação do português da época da tradução de Milliet?: sôbre, tôdo(a)(as), amigàvelmente, vêzes, êle, sòzinho, êsse, vôo, apóia? Se for o caso de manter, acrescentar [sic] após os termos?

++ p. 192: "Certamente, o grande poeta André Gide mostrou-se como o
grande homem, muito mais tarde, ao menos ele tornou-se
perceptível a si mesmo muito tarde." --> frase esquisita


5ª PARTE----------------------------------------

CARTA_PARISIENSE_I

++ p.208: manter apenas trecho traduzido? Não vejo a necessidade de manter o trecho original. Idem na página 222. Por que trazer os trechos originais em francês?

+ l. 508: A footnote com o trecho original em francês está gerando uma viúva da footnote na página seguinte. Talvez suprimir o trecho e deixar indicado: "No original o trecho está em francês. Tradução de...", como adotaram nos parágrafos de Gide que Benjamin escreve em traduções alemães. Além de solucionar a viúva, deixa padronizado com o outro parágrafo.
