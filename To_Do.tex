TO_DO

++ Bater minuciosamente padrão do 1º volume com esse (abreviaturas, folha de rosto, página de cŕeditos, disposição dos títulos no sumário...)

++ Repensar título. Sugestões "A Paris do século XX"; "A Paris do século XX e o crítico literário"; "Paris e o lugar do crítico literário"; "Proust, Gide, Valéry e o lugar do crítico literário"; "A nova literatura francesa e o lugar do crítico literário"

++ Expressões francesas que aparecem mais de uma vez no texto: remeter o restante das notas à primeira nota com a tradução, em vez de ficar replicando toda vez a tradução?

++ .'' X ''.

++ : C.A. X c.b.

++ Itálico x aspas em termos em francês e alemão

+ FALTA INDICAR O TÍTULO ORIGINAL DO TEXTO: <<<IMPORTANTE>>>
-- Paul Valéry – Em seu sexagésimo aniversário;
-- Folha de anotações para o ensaio sobre Paul Valéry;

+ TÍTULOS ADICIONADOS POR MIM, VERIFICAR:
https://de.wikisource.org/wiki/Walter_Benjamin/Werkverzeichnis
-- O Comerciante no poeta (Der Kaufmann im Dichter)
-- Cabeças parisienses (Pariser Köpfe)
-- Diário parisiense (Pariser Tagebuch)
-- Paul Valéry na École Normale (Paul Valéry in der École Normal)


+ FALTA INDICAR LOCAL/DATA DA PRIMEIRA PUBLICAÇÃO:

-- Cabeças parisienses; 
-- Paul Valéry na École Normale;
-- Paul Valéry – Em seu sexagésimo aniversário;
-- Imagem de Proust; 
++++++++++++++++++++++++++++++++++++++++++++++++++++++++++++++++++++++++++++
GERAL:

+ Benjamin, quando se refere a livros, usa aspas, e não itálico. Mantemos aspas para ficar próximo à anotação original, ou mudamos para itálico para adequar ao nosso padrão editorial?

+ No primeiro volume da coleção, todas as notas de rodapé dos capítulos indicam o título original do texto. Recuperar esses títulos para essa publicação?

+ Padronizar: o nome do ensaio vai entre aspas ou itálico, quando referenciado ao longo do texto?

+ No primeio volume da coleção, termos originais em alemão estão em itálico ou aspas, como nesse volume?

+ Padronizar se os títulos de livros vem em caixa-alta ou caixa-baixa.

+ Padronizar Tolstói e Dostoiévsky

1ª PARTE----------------------------------------

DIÁRIO PARISIENSE:
++++ l.483: ``notado que a descoberta''... acho que sem esse ``que'' faz mais sentido


2ª PARTE----------------------------------------


ANOTACOES_PAUL_VALERY

++ l.27: parágrafo entre chaves ``{}''. -- REPENSAR NOTAÇÃO

3ª PARTE----------------------------------------

IMAGEM_PROUST:

+ l.59: Nota sem indicação se é do autor ou tradutor (assumi que seja do tradudor pelo conteúdo);

+ l.209: Há uma numeração de secion II, mas no início do texto não o I. Iniciar o texto como a sessão I?

PROUST_PAPIERE

+ Pensar em uma forma melhor de deixar os subtítulos referentes ao conteúdo do material que se seguirá (como "Materiais relativos ao ensaio sobre Proust")

+ É assim mesmo, todos os parágrafos entre chaves? Se for o caso, indicar o que significam as chaves em nota no início do capítulo

+ Estou deixando como footnote as indicações ao final dos trechos das páginas a que se referem no Arquivo Walter Benjamin. Pensar em solução melhor... do jeito que estava, no meio do texto, fica muito esquisito... Talvez deixar em itálico, em fonte um pouco menor, recuado à direita...

+ l.65: parêntesis mesmo?

+ l.83: início de parênteses sem fechamento

+ l.124: "com Proust mesmo" --> É com Proust, ou como Proust?

+ l.176: Indicar de quem é a anotação entre co+ lchetes [do piano]?

+ l.252: Não seria "parênteses" aberto e não fechado, ao invés de "parágrafo"?

+ l.278: "teria inequivocamente desembocar" --> acrescentar um "que"? --> "teria inequivocamente [que] desembocar"

+ l.705: chave sem abertura

>>>>>OLhar atentamente para esse capítulo, que tem muita coisa esquisita de pontuação para tentar reproduzir a escrita fragmentária do caderno de anotações de Benjamin<<<<<<

+ l.865: Parênteses vazio

4ª PARTE----------------------------------------

GIDE_ALEMANHA

++ Padronizar As Afinidades eletivas por As afinidades eletivas.

+ l. 132: "torna-se algo raro". Não seria tornar-se algo raro?

+ l. 133: Aspas sem abertura.

CONVERSACAO_GIDe

+ l. 35: "é perigoso tem consequências". Falta alguma coisa entre "perigoso" e "tem consequências"

+l. 223: Parênteses vazio

VOCACAO_GIDE

+ Manter a acentuação do português da época da tradução de Milliet?: sôbre, tôdo(a)(as), amigàvelmente, vêzes, êle, sòzinho, êsse, vôo?

+ l. 462: É assim mesmo, pecatto e peccato?

EDIPO

+ l.36: cânon mesmo, e não cânone?

+ l.178: Aspas sem fechamento.


5ª PARTE----------------------------------------

CARTA_PARISIENSE_I

+ l. 508: A footnote com o trecho original em francês está gerando uma viúva da footnote na página seguinte. Talvez suprimir o trecho e deixar indicado: "No original o trecho está em francês. Tradução de...", como adotaram nos parágrafos de Gide que Benjamin escreve em traduções alemães. Além de solucionar a viúva, deixa padronizado com o outro parágrafo.


CARTA_PARISIENSE_II

+ l. 22: sem referência à autoria da footnote [N. E., T. A.]



----------------------------------------------------------------------
++ Nome de ensaios vem entre aspas