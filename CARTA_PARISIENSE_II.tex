\chapter{Carta parisiense \versal{II}\\
\emph{Pintura e fotografia}\footnote[*]{``Pariser Brief 2. Malerei und Photographie'', in
  \versal{GS III}, p. 495--507. Tradução de Pedro Hussak. Publicada originalmente na revista \emph{Das Wort} em novembro/dezembro de 1936. [\versal{N.~O.}]}}

\hedramarkboth{Carta parisiense II}{}

Quando passeamos em Paris aos domingos e feriados com um clima tolerável
nos bairros de Montparnasse ou Montmartre, esbarramos nas ruas espaçosas
aqui e ali em \emph{paravents},\footnote{Em francês no original:
 para"-vento. \versal{[N.~T.]}} combinados uns ao lado dos outros ou em pequenos
labirintos, nos quais são penduradas determinadas pinturas para serem
vendidas. Encontram"-se ali os temas que combinam com a boa sala:
naturezas"-mortas, marinas, nus, pinturas de gênero e \emph{intérieurs}.
O pintor, não raro trajando em um estilo romântico um chapéu com largas abas e
um casaco de veludo, permanece sentado em uma banqueta ao lado de seus
quadros. Sua arte dirige"-se à família burguesa que ali passeia,
talvez mais impressionada por sua presença ou por sua veste
imponente do que pelos quadros expostos. Mas provavelmente estaríamos
superestimando o espírito comercial desses pintores, se supuséssemos que
eles colocam sua pessoa a serviço da atração da clientela.

Naturalmente, não foram esses os pintores que estavam em evidência nos
grandes debates travados recentemente em torno da situação da
pintura.\footnote{\emph{Entretiens, l'art et la réalité. L'art et
  l'état}. {[}Com a contribuição de Mario Alvera, Daniel Baud"-Bovy,
  Emilio Bodrero entre outros.{]} Paris: Institut internationale de
  Coopération intellectuelle, 1935. --- \emph{La querelle du réalisme.
  Deux débats par l'Association des peintures et sculptures de la maison
  de la culture}. {[}Com contribuições de Lurçat, Granaire entre
  outros.{]} Paris: Editions socialistes internationales, 1936. [\versal{N.}~de~\versal{W.B.}]} Pois eles
têm relação com a pintura como arte apenas na medida em
que também sua produção é cada vez mais destinada ao mercado em sentido
geral. Todavia, esses distintos pintores não têm a necessidade de estar
em pessoa no mercado. Eles têm à sua disposição \emph{marchands} e os
salões de arte. Ainda assim, seus colegas ambulantes oferecem algo de
diferente do que a pintura no estado da sua mais profunda degradação.
Eles revelam como um talento mediano de lidar com pincéis e palhetas
tornou"-se comum. E foi nessa medida que eles apesar de tudo ocuparam o seu
lugar nos debates mencionados. Esse lugar foi"-lhes concedido por André
Lhote, ao dizer: ``quem, hoje em dia, interessa"-se por pintura vai começar
mais cedo ou mais tarde a pintar\ldots{} A partir do dia, porém, em que um
amador começar ele mesmo a pintar, a pintura cessará de exercer sobre ele o
mesmo fascínio religioso que ela exerce sobre o leigo''
(\emph{Entretiens}, p. 39). Se concebêssemos uma época na qual fosse
possível a alguém interessar"-se por pintura sem também ter a ideia de pintar, voltaríamos então à época das corporações de
artesãos. E assim como ocorre frequentemente que o destino do liberal --- Lhote é
no melhor sentido um espírito liberal --- seja o de que o fascista conclua seu
raciocínio, ouvimos de Alexandre Cingria que a decadência começou com a
eliminação do sistema corporativo de artesãos, ou seja, com a Revolução
Francesa. Após essa eliminação, os artistas teriam desprezado toda
disciplina e se comportado ``como animais selvagens''
(\emph{Entretiens}, p. 96). E quanto ao seu público, os burgueses,
``após o ano de 1789 ter"-lhes retirado de uma ordem construída
politicamente na hierarquia, e espiritualmente na primazia de valores
intelectuais'', ``perderam aos poucos a compreensão da forma de produção
desinteressante, mentirosa, amoral e inútil para as leis artísticas'' (\emph{Entretiens}, p. 97).

É evidente que o fascismo manifestou"-se abertamente no congresso de
Veneza. O fato de que esse congresso tenha acontecido na Itália é tão
claramente notável quanto o fato de que o parisiense tenha
sido convocado pela \emph{Maison de la Culture}. Isso quanto à índole
desses eventos. De resto, quem estudar mais de perto os
discursos vai deparar"-se no congresso de Veneza (que evidentemente foi um
evento internacional) com reflexões maduras e ponderadas sobre a
situação da arte, ao passo que, por outro lado, todos
os participantes do congresso de Paris não conseguiram deixar os
debates totalmente livres de caminhos já trilhados. É significativo, ao
menos, que dois dos mais importantes palestrantes venezianos, nomeadamente Lhote e Le Corbusier, tenham
participado do congresso de Paris~e se sentido em casa nessa atmosfera.
O primeiro aproveitou a oportunidade para
rever o evento de Veneza. ``Éramos sessenta reunidos
para\ldots{}'', disse ele, ``enxergar algo de mais claro nestas questões.
Eu não gostaria de ousar afirmar que um único dentre nós tenha
efetivamente tido sucesso nisso'' (\emph{La querelle}, p. 93).

O fato de que em Veneza a União Soviética não estava representada, e a
Alemanha apenas por uma pessoa, mesmo que tenha sido Thomas Mann, é
lastimável. No entanto, seria errado supor que, por causa disso,
posições mais avançadas tenham ficado totalmente órfãs. Escandinavos
como Johnny Roosval, austríacos como Hans Tietze, para não mencionar os
franceses já citados, sustentaram"-nas pelo menos em parte.\footnote{Por
  outro lado, deparamo"-nos em Veneza com restos de caráter formalista de
  museu de épocas perdidas de pensamento. Assim, definia por exemplo
  Salvador de Madariaga: ``A verdadeira arte é o produto de uma possível
  combinação do pensamento com o espaço em diferentes relações; e arte
  falsa é o resultado de tal combinação, na qual o pensamento prejudica
  a obra de arte'' (\emph{Entretiens}, p. 160). [\versal{N.}~de~\versal{W.B.}]} Em Paris, a vanguarda, composta
em igual proporção de pintores e escritores,
obteve, em todo caso, a primazia. Enfatizou"-se, desse modo, quão
necessário é recuperar para a pintura uma comunicação inteligente com a
palavra falada e a escrita.

A teoria da pintura dissociou"-se da própria pintura e tornou"-se, na
qualidade de disciplina especializada, assunto da crítica de arte. O que
está no fundo dessa divisão do trabalho é o desaparecimento da
solidariedade que uma vez ligou a pintura às aspirações do público.
Courbet foi talvez o último pintor no qual essa solidariedade
manifestou"-se. A teoria da sua pintura deu resposta não apenas aos
problemas picturais. Nos impressionistas, o \emph{argot}\footnote{Em francês no original: jargão. \versal{[N.~T.]}} dos ateliês reprimiu a própria teoria,
e a partir disso decorreu uma evolução constante até o
estágio em que um observador inteligente e informado podia chegar à tese
de que a pintura ``tornou"-se um assunto totalmente esotérico e
antiquado, e que o interesse por ela e por seus problemas\ldots{} não
existira mais''. Ela ``seria quase um resquício de um período passado, e
submeter"-se a ele\ldots{} seria um fracasso pessoal''.\footnote{Hermann Broch, \emph{James
  Joyce und die Gegenwart}. {[}James Joyce e o presente{]}. Discurso
  sobre o 50º Aniversário de Joyce, Wien"-Leipzig"-Zürich 1936, p. 24. [\versal{N.}~de~\versal{W.B.}]} A
culpa por tais concepções não é tanto da pintura quanto da crítica de
arte que apenas aparentemente serve ao público, pois na verdade está a
serviço do comércio de arte. Ela não utiliza nenhum conceito, mas apenas
um \emph{slang}\footnote{Em inglês no original: gíria. Note"-se que
  nos dois casos --- \emph{argot} e \emph{slang} --- Benjamin utiliza termos em
  outra língua para expressar o que seria o correspondente a jargão,
  gíria ou calão. \versal{[N.~T.]}} que muda de \emph{saison à saison}.\footnote{Em francês no original: de temporada a temporada. \versal{[N.~T.]}} Não é um acaso que
o mais representativo dos críticos de arte parisienses por anos,
Waldemar George, surgiu como fascista em Veneza. O jargão esnobe dele
valerá apenas enquanto durarem as formas atuais do negócio de arte.
Compreende"-se por que ele chegou ao ponto de esperar a salvação da
pintura francesa por um ``\emph{Führer}'' que deveria vir (Cf.
\emph{Entretiens}, p. 71).

O interesse do debate veneziano reside no esforço daqueles que se empenham por
uma apresentação sem complacências da crise da pintura. Esse é particularmente o caso
de Lhote. Sua constatação de que ``estamos diante da questão do quadro
\emph{útil}'' (\emph{Entretiens}, p. 47) indica onde devemos
procurar o ponto de Arquimedes do debate. Lhote é tão pintor quanto
teórico. Como pintor, deriva de Cézanne; como teórico, trabalha no âmbito
da \emph{Nouvelle Revue Française}.\footnote{\emph{Nouvelle Revue
  Française} -- \versal{NRF} (1909--) é uma revista de cunho literário e
  crítico, que já em seus primórdios foi dirigida por Gaston Gallimard e
  por André Gide. \versal{[N.~O.]}} Ele não se situa de forma alguma na ala extrema
esquerda, de modo que não foi somente lá que se sentiu a obrigação de refletir sobre
a ``utilidade'' de um quadro. O conceito de uso não pode, sendo leal a si
mesmo, focar na utilidade que o quadro tem para a pintura ou para a
fruição artística (ao contrário, deve"-se decidir sobre a utilidade da pintura e da fruição artística justamente com a ajuda desse conceito). Aliás, é impossível captar de modo
suficientemente amplo o conceito de utilidade. Obstruir"-se"-ia todo
caminho para essa reflexão, caso consideremos a utilidade imediata de uma
obra apenas em função do seu tema. A história mostra que frequentemente a pintura
cumpriu tarefas sociais gerais por efeitos indiretos.
O historiador da arte vienense Tietze alude a esses efeitos quando define
a utilidade de um quadro desta forma: ``A arte ajuda a compreender a realidade\ldots{}
Os primeiros artistas que impuseram à humanidade as primeiras convenções
da percepção visual prestaram"-lhe um serviço semelhante àquele dos
gênios da pré"-história que formaram as primeiras palavras''
(\emph{Entretiens}, p. 34). Lhote segue a mesma linha, mas através do tempo
histórico. Ele observa que cada nova técnica está na base de uma nova
ótica. ``Sabemos quais delírios acompanharam a invenção da perspectiva,
a descoberta decisiva da Renascença. Paolo Uccello, como o primeiro a
encontrar suas leis, mal pôde conter seu entusiasmo, de modo que acordou
sua esposa no meio da noite para dar"-lhe a maravilhosa notícia. Eu
poderia'', continua Lhote, ``elucidar as diferentes etapas do
desenvolvimento da percepção visual dos primitivos até hoje com o
simples exemplo de um prato. O primitivo o desenharia, como uma
criança, em forma de círculo; na época da Renascença, em forma oval; e,
finalmente na modernidade, cujo exemplo poderia ser Cézanne\ldots{} como
uma extraordinária figura complexa, a partir da qual poderíamos
imaginar a parte inferior da figura oval achatada e um de seus lados
inchado'' (\emph{Entretiens}, p. 38). Se a utilidade de tais aquisições
pictóricas --- talvez se pudesse objetar --- não servisse à percepção, mas
apenas à sua reprodução mais ou menos fiel, então essa utilidade seria
legitimada em um campo fora da arte. Pois tal reprodução
possui impacto no nível da produção e da formação cultural da sociedade
por meio de numerosos canais, como o desenho comercial, a
publicidade, as imagens populares e as ilustrações científicas.

O conceito elementar que se pode fazer da utilidade de um quadro foi
expandido consideravelmente pela fotografia. Essa forma expandida é o
atual estatuto desse conceito. O ápice do debate atual é alcançado quando a
fotografia foi incluída na sua análise para esclarecer sua relação com a pintura. Se esse debate não aconteceu em Veneza, Aragon preencheu essa lacuna em Paris. Foi
preciso, como ele narrou depois, certa \emph{courage}\footnote{Em
  francês no original: coragem. \versal{[N.~T.]}} para isso. Uma parte dos
pintores presentes
considerou uma ofensa basear reflexões sobre a história da pintura na
história da fotografia. ``Imagine'', concluiu Aragon, ``um físico que se
sente insultado porque lhe falam de química''.\footnote{Louis Aragon. Le
  réalisme à l'ordre du jour. In: \emph{Commune}, sept. 1936, 4, série
  37, p. 23. [\versal{N.}~de~\versal{W.B.}]}

A história da fotografia começou a ser pesquisada há oito ou dez anos.
Temos uma quantidade de trabalhos, na maioria ilustrados, sobre suas
origens e seus primeiros mestres.\footnote{Cf. Entre outros Helmut
  Theodor Bossert e Heinrich Guttmann, \emph{Aus der Frühzeit der
  Photographie 1840--1870}, Frankfurt am Main, 1930; Camille Recht,
  \emph{Die alte Photographie}, Paris, 1931; Heinrich Schwarz,
  \emph{David Octavius Hill, der Meister der Photographie}, Leipzig,
  1931; além disso, duas importantes fontes: Disdéri, \emph{Manuel
  opératoire de photographie}, Paris, 1853; Nadar, \emph{Quand
  j'étais photographe}, Paris, 1900. [\versal{N.}~de~\versal{W.B.}]} Mas foi reservada a uma das mais recentes
publicações tratar desse assunto em relação à história da
pintura. O fato de que ela tenha sido ensaiada no espírito do materialismo
dialético confirma novamente os aspectos altamente
originais que este método pode abrir. O estudo de Gisèle Freund \emph{La
photographie en France au dix"-neuvième siècle}\footnote{Gisèle Freund,
  \emph{La photographie en France au dix"-neuvième siècle}, Paris, 1936. A
  autora, uma emigrante alemã, foi laureada com este trabalho na
  Sorbonne. Quem assistiu à discussão pública que concluiu a prova deve
  ter tido uma forte impressão da ampla visão e liberalidade dos
  examinadores. Uma objeção metódica contra o livro meritório pode ser
  levantada. ``Quanto maior'', escreve a autora, ``é o gênio do artista,
  tanto melhor sua obra reflete, e mesmo na força da originalidade da
  sua forma, as tendências da sociedade que lhe é contemporânea''
  (Freund, p. 4). O que sobressai de questionável nessa afirmação não é
  sua tentativa de circunscrever as consequências artísticas de um
  trabalho com referência à estrutura social de seu tempo de origem.
  Questionável é apenas a pressuposição de que esta estrutura
  apresenta"-se sempre sob o mesmo aspecto. Na verdade, o seu aspecto
  poderia mudar com as diversas épocas que atraem seu olhar de volta
  para elas. E então definir o significado de uma obra de arte com olhar
  para a estrutura social de seu tempo original equivale bastante a
  determiná"-la sob a base da história de seus efeitos. Por exemplo, a
  poesia de Dante trouxe à luz tal capacidade para o século \versal{XIII}, a obra
  de Shakespeare para o período elisabetano. A clarificação deste
  problema metodológico é tanto mais importante em quanto a fórmula de
  Freund reconduz diretamente a uma posição que encontrou sua
  expressão mais drástica e ao mesmo tempo mais problemática em
  Plekhanov, que dizia: ``Quanto mais um escritor é grande, tanto maior é
  a força e evidência com a qual o caráter da sua obra depende do
  caráter da sua época, \emph{ou em outras palavras} (ênfase dos
  relatores): quanto menos é possível encontrar, na sua obra aquele
  elemento que poderia ser chamado de `pessoal'" (Georgi Plekhanov,
  ``Les jugements de Lanson sur Balzac et Corneille'' {[}Os juízos de
  Lanson sobre Balzac e Corneille{]}, in: \emph{Commune}, dezembro de
  1934, 2, série 16, p. 306. [\versal{N.}~de~\versal{W.B.}]} apresenta a ascensão da fotografia em
relação à da burguesia e exemplifica essa relação de uma maneira
particularmente feliz na história do retrato. Partindo da técnica do
retrato mais difundida no \emph{ancien régime}, as valiosas miniaturas
de marfim, a ensaísta apresenta diferentes procedimentos que, por volta de
1780, ou seja, setenta anos antes da descoberta da fotografia, tiveram
como meta a aceleração, barateamento e, consequentemente, uma propagação
mais ampla da produção de retratos. Sua descrição do
``fisiognotraço'',\footnote{No texto original:
  \emph{Physiognotrace}. \versal{[N.~T.]}} como uma técnica intermediária
entre o retrato em
miniatura e o registro fotográfico, tem o valor de uma descoberta. A
ensaísta mostra posteriormente como o desenvolvimento técnico alcança seu
padrão adequado ao do desenvolvimento da sociedade na fotografia, pois é através dela que o retrato torna"-se
acessível a largas camadas da burguesia. Ela expõe como os miniaturistas
tornaram"-se dentre os pintores as primeiras vítimas da fotografia.
Finalmente, ela relata a controvérsia teórica entre pintura e
fotografia em meados do século.

No campo da teoria, a controvérsia entre a fotografia e a pintura
concentrava"-se em torno da questão de saber se a fotografia seria uma arte.
A ensaísta chama~a atenção para a peculiar constelação que emerge com a
resposta a essa pergunta. Ela constata quão elevado era o nível
artístico de um bom número dos primeiros fotógrafos que trabalhavam sem
pretensões artísticas, expondo seu trabalho apenas a um restrito
círculo de amigos. ``A pretensão de tornar a fotografia uma arte foi
erguida justamente por aqueles que faziam da fotografia um negócio''
(Freund, p. 49). Em outros termos: a pretensão de que a
fotografia seja uma arte é simultânea à sua emergência como mercadoria.

Essa circunstância tem sua ironia dialética: o procedimento, que depois
foi determinante para colocar o próprio conceito de obra de arte
em questão, pois ao mesmo tempo que ela por meio de sua reprodução força seu
caráter de mercadoria, designa"-se como técnica artística.\footnote{Uma
  constelação analogamente irônica no mesmo campo é a seguinte. A câmera
  fotográfica é um aparelho altamente estandardizado, e então não é mais
  favorável do que um moinho à expressão de características nacionais
  particulares na forma de seu produto. Torna a produção da imagem
  independente das convenções e dos estilos nacionais em uma medida até
  então desconhecida. Por esse motivo, inquietaram"-se os teóricos que
  juraram fidelidade a tais convenções e estilos. A reação não demorou.
  Já em 1859 dizia"-se, no comentário a uma exposição fotográfica: ``o
  caráter nacional particular vem {[}\ldots{}{]} de forma palpável com toda
  evidência nas obras de vários países {[}\ldots{}{]}. Um fotografo francês
  {[}\ldots{}{]} não poderá mais ser confundido com um colega inglês'' (Louis
    Figuier, \emph{La photographie au salon de 1859} {[}A fotografia no
  salão de 1859{]}, Paris, 1960, p. 5). Setenta anos depois, no congresso
  de Veneza, Margherita Sarfatti diz exatamente a mesma coisa: ``um bom
  retrato fotográfico revelar"-nos"-á antes de mais nada a nacionalidade,
  não da pessoa fotografada, mas do fotógrafo'' (\emph{Entretiens}, p. 87). [\versal{N.}~de~\versal{W.B.}]}
Esse desenvolvimento posterior começa com Disdéri, que sabia que a
fotografia era uma mercadoria, mas que esse atributo ela compartilha com
todos produtos de nossa sociedade (também a pintura é uma mercadoria).
Além disso, Disdéri reconheceu qual serviço a fotografia é capaz de
prestar à economia de mercado. Ele foi o primeiro a usar o procedimento
fotográfico para lançar, no processo de circulação do mercado, bens que antes tinham
sido mais ou menos retirados dele, a começar pelas obras de
arte. Disdéri teve a ideia astuciosa de obter o monopólio estatal das
reproduções da coleção do Louvre. Desde então, a fotografia tornou cada vez mais vendáveis
numerosos segmentos do campo da percepção óptica. Ela
conquistou para a circulação mercadológica objetos que não
existiam nela anteriormente.

Mas esse desenvolvimento sai do âmbito que Gisèle Freund delimitou
para si. Ela lida, sobretudo, com a época na qual a fotografia iniciou
sua marcha triunfal, a época do \emph{juste milieu}. A ensaísta
descreve o seu ponto de vista estético, e ela, na sua apresentação, tem algo mais do que um valor anedótico quando explica que
um dos mestres mais festejados daquela época considerava como um objetivo
elevado da pintura a exata representação das escamas de peixe. Essa
escola viu seus ideais realizarem"-se do dia para noite pela fotografia.
Um pintor contemporâneo, Galimard, revela isso ingenuamente quando, em
uma reportagem sobre os quadros de Meissonier, escreve: ``o público não
vai contradizer"-nos, se expressarmos nossa admiração pelo pintor
refinado que\ldots{} este ano nos presenteou com um quadro que não seria
inferior em termos de exatidão às imagens de daguerreótipos''.\footnote{Auguste Galimard,
  \emph{Examen du salon de 1849}, Paris o. J., p. 95. [\versal{N.}~de~\versal{W.B.}]} A pintura
do \emph{juste milieu} só esperava ir a reboque da
fotografia. Por isso, não é de se estranhar que ela não tivesse
significado nada, ou ao menos nada de bom, para o desenvolvimento do
ofício da fotografia. Onde quer que encontremos esse ofício sob a sua
influência, topamos com a tentativa de fotógrafos, com a ajuda de
cenários e figurantes reunidos nos seus ateliês, de imitar os
pintores de temas históricos que, sob a ordem de Louis"-Philippe, estavam decorando
o palácio de Versailles com afrescos na época. Sem a menor hesitação fotografa"-se o
escultor Calímaco inventando o capitel coríntio olhando
uma folha de acanto; compunham a cena de ``Leonardo'' pintando a ``Mona
Lisa'' e fotografaram"-na. A pintura do \emph{juste mileu} encontrou em
Courbet seu adversário, pois com ele a relação entre o pintor e o fotógrafo
inverteu"-se por um certo tempo. Seu quadro mais famoso, \emph{La vague},
``A onda'', equivale à descoberta de um assunto fotográfico pela
pintura. Na época de Courbet, não se conhecia o registro do plano geral
e do instantâneo. Sua pintura mostra o caminho para eles, equipando uma
expedição para um mundo de formas e estruturas que apenas muitos
lustros\footnote{No original: \emph{Lustren}, do latim \emph{lustrum},
  período de tempo de cinco anos que marcava o intervalo entre um
  sacrifício de purificação e outro na Roma Antiga. \versal{[N.~T.]}} mais tarde
conseguiu fixar"-se sobre chapas fotográficas.

O lugar especial que Courbet ocupa deve"-se ao fato de que ele foi o
último que pôde tentar ultrapassar a fotografia. Os que vieram depois buscaram
escapar dela, a começar pelos impressionistas. Pintado, o quadro escapa
do esboço desenhado, e com isso elimina de certa forma a
concorrência com a câmera fotográfica. Prova disso é que a fotografia, na
virada do século, tentou, por seu lado, imitar os impressionistas. Ela
recorreu à emulsão de goma;\footnote{Em alemão no original:
  \emph{Gummidrucken}, emulsão de goma ou goma bicromatada, técnica que
  corresponde a um processo de sensibilização do papel com dicromato de
  sódio e uma goma, exposto em seguida diretamente à luz solar. \versal{[N.~T.]}} e
sabe"-se o quanto ela decaiu com esse procedimento. Aragon captou
agudamente o contexto: ``Os pintores\ldots{} viram no aparato fotográfico um
concorrente\ldots{} eles tentaram fazer diferentemente dele. Essa foi sua
grande ideia. Mas não reconhecer uma importante conquista da humanidade\ldots{} deveria levá"-los naturalmente a um comportamento reacionário. Os
pintores tornaram"-se com~o tempo --- isso vale mais para os mais
talentosos --- \ldots{} verdadeiros ignorantes.\footnote{\emph{La querelle}, p.
  64. Cf. A tese maligna de Dérain: ``o maior perigo para a arte é o
  excesso de cultura, O verdaderio artista é um homem sem cultura''
  (\emph{La querelle}, p. 163). [\versal{N.}~de~\versal{W.B.}]}

Aragon ocupou"-se das questões deixadas de lado pela mais recente história da
pintura em um escrito de 1930 intitulado \emph{La peinture au
défi}.\footnote{Louis Aragon. \emph{La peinture au défi}, Paris 1930. [\versal{N.}~de~\versal{W.B.}]}
Quem desafia é a fotografia. O escrito refere"-se à mudança que levou a
pintura, que até então vinha evitando o enfrentamento com a fotografia, a
encará"-la de frente. Aragon explica a maneira como a pintura fez isso,
reportando"-se aos trabalhos de seus amigos surrealistas à época. Eles usaram
diferentes procedimentos: ``grudava"-se um elemento fotográfico sobre uma
pintura ou um desenho; ou acrescentava"-se desenhando ou pintando sobre uma
fotografia'' (Aragon, p.~22). Aragon enumera ainda outros
procedimentos, tais como a utilização das reproduções que por recortes
ganham uma forma diferente daquilo que foi representado (pode"-se assim
recortar uma locomotiva de uma folha na qual está impressa uma rosa).
Aragon acreditava que esse procedimento, no qual é possível reconhecer uma
relação com o dadaísmo, era uma garantia da energia revolucionária
da nova arte que ele confrontava com a arte tradicional: ``A pintura
manteve"-se por muito tempo em uma posição confortável; ela adula o homem de gosto
que paga por ela. Ela é artigo de luxo\ldots{} Ora,
reconhece"-se nesses novos experimentos a possibilidade de que os pintores
consigam libertar"-se de sua domesticação pelo dinheiro. A
colagem é pobre. Por muito tempo ainda não se reconhecerá o seu valor'' (Aragon, p.~19).

Mas isso foi em 1930. Hoje, Aragon não escreveria mais essas frases. A
tentativa dos surrealistas de lidar com a fotografia ``artisticamente''
falhou. Eles também cometeram o mesmo erro dos fotógrafos de arte
comercial cujo credo pequeno"-burguês deu título à famosa
coleção de fotografia de Renger"-Patzsch ``\emph{Die Welt ist schön}''.\footnote{``O mundo é belo''. [\versal{N.~T.}]} Eles compreenderam mal o impacto social da
fotografia e por conseguinte a importância da legenda que, como pavio, conduz a
centelha crítica até o amontoado de imagens (como se pode ver melhor em
Heartfield). Aragon, por fim, ocupou"-se de Heartfield;\footnote{Louis
  Aragon, John Heartfield et la beauté révolutionnaire. In: \emph{Commune},
  maio/1935, 2, p. 21. [\versal{N.}~de~\versal{W.B.}]} e ele também aproveitou a oportunidade
para apontar o elemento crítico na fotografia. Hoje, ele avista esse
elemento até na obra aparentemente formalista de um virtuoso com a câmera
como Man Ray. Ele argumentou no debate parisiense que, com~Man Ray, a
fotografia conseguiu reproduzir a maneira de pintar dos pintores mais
modernos. ``Quem não conhecesse os pintores aos quais Man Ray alude não
poderia apreciar totalmente sua produção'' (\emph{La querelle}, p. 60).

Poderemos abandonar esta história cheia de tensão do encontro entre
pintura e fotografia por meio da amável fórmula que Lhote mantém à
disposição? Parecia"-lhe incontestável ``que a substituição muito
debatida da pintura pela fotografia pudesse encontrar seu lugar na
execução daquilo que poderíamos caracterizar como `assuntos correntes'. À
pintura, entretanto, resta então o misterioso domínio eternamente
intocável do puramente humano'' (\emph{La querelle}, p. 102).
Infelizmente essa construção não é nada mais do que uma armadilha que
se arma nas costas do pensador liberal e o entrega indefeso ao fascismo.
Muito mais longe alcançou o olhar do pintor de ideias canhestro, Antoine
Wiertz, que escreveu há quase 100 anos por ocasião da primeira exposição
mundial de fotografia: ``Há poucos anos surgiu entre nós uma máquina, a
glória da nossa geração, que diariamente causa admiração em nosso
pensamento e espanto aos nossos olhos. Antes que mais de um século tenha
passado, esta máquina será o pincel, a palheta, as cores, a habilidade,
a experiência, a paciência, a agilidade, a precisão, o colorido, o
verniz, o modelo, a conclusão, o extrato da pintura\ldots{} Que não se
acredite que o Daguerreótipo mate a arte\ldots{} Quando o Daguerreótipo, esta
criança gigante, tiver crescido, quando toda sua arte e força tiverem se
desenvolvido, então o Gênio rapidamente vai colocar a mão na sua nuca e
gritará: Por aqui! Você pertence a mim agora! Nós vamos trabalhar
juntos''.\footnote{A. I. Wiertz, \OE uvres littéraires, Paris, 1870, p. 309. [\versal{N.}~de~\versal{W.B.}]}
Quem tiver diante de si as grandes pinturas de Wiertz saberá que o Gênio
sobre quem ele fala é político. Ele considera que será no lampejo de uma
grande inspiração social que pintura e fotografia irão fundir"-se. Há uma
verdade contida nesta profecia, apenas que não é nas obras, mas nos
grandes mestres que se consumou tal fusão. Eles pertencem à geração de
Heartfield e transformaram"-se pela política de pintores a fotógrafos.

Essa mesma geração produziu pintores como George Grosz ou Otto Dix que
trabalharam visando ao mesmo objetivo. A pintura não perdeu sua função.
Basta não deformarmos nossa visão sobre elas, como o faz, por
exemplo, Christian Gaillard. ``Para que as lutas sociais sejam'', assim
diz, ``o assunto das minhas \emph{\OE uvres}, então eu deveria ser
visualmente tocado por elas'' (\emph{La querelle}, p. 190). Trata"-se de
uma formulação muito problemática para os estados fascistas contemporâneos, em cujas cidades e aldeias reina a
``calma e a ordem''. Gaillard não
deveria fazer a experiência do processo inverso? Sua comoção social não
resultará em inspiração visual? Assim aconteceu com os grandes
caricaturistas, cuja sabedoria política de sua percepção fisionômica
sedimentou"-se não menos profundamente quanto a experiência do sentido do
tato está em relação à percepção espacial. Mestres como Bosh, Hogarth,
Goya, Daumier indicaram o caminho. ``Dentre as obras mais importantes da
pintura'', escreve René Crevel, recentemente falecido, ``precisou"-se
contar sempre aquelas que, justamente por exibirem uma
decomposição, levantavam uma acusação contra aqueles que eram responsáveis por isso. De Grünewald
até Dali, do Cristo apodrecido ao Burro apodrecido\ldots{}\footnote{Um quadro de
  Dali. [\versal{N.}~de~\versal{W.B.}]} A pintura soube sempre\ldots{} descobrir novas verdades que não
eram verdades apenas da pintura'' (\emph{La querelle}, p. 154).

É da natureza da situação da Europa ocidental que a pintura tenha um efeito destruidor e purificador justamente quando se ocupa soberanamente das suas questões.
Talvez isso não surja tão claramente em um
país\footnote{Ainda por ocasião da grande mostra de Cézanne, o jornal
  parisiense \emph{Choc} se colocou a tarefa de concluir com o ``Bluff'' de
  Cézanne. Esta mostra foi organizada pelo governo de esquerda da França
  ``para desenhar o sentido artístico do próprio povo na sujeira''.
  Assim a crítica. Aliás há pintores que forneceram para todos os casos.
  Estão de acordo com Raoul Dufy, que escreve que se fosse alemão e
  devesse celebrar o triunfo de Hitler, faria à maneira de certos
  pintores do Medievo que pintaram imagens religiosas sem ser
  crentes'' (Cf. \emph{La querelle}, p. 187). [\versal{N.}~de~\versal{W.B.}]} que ainda tenha liberdade
democrática como em um país onde o fascismo está no comando. Ali há
pintores cuja pintura está proibida. (E a proibição raramente refere"-se
ao tema, mas na maioria das vezes ao modo de pintar dos artistas. Tão
profundamente, assim, o fascismo é atingido pela forma dela enxergar a realidade). A
polícia vai a esses pintores para controlar se não pintaram nada
desde a última \emph{Razzia}, a revista policial. Eles trabalham à noite
com as janelas cobertas por panos. Para eles, é bem pequena a tentação
de pintar ``segundo a natureza''. Também a pálida paisagem de seus
quadros, povoada de fantasmas ou monstros, não imita a
natureza, mas o Estado classista. Em Veneza não se falou desses
pintores, tampouco em Paris, infelizmente. Eles sabem o que é hoje útil
em um quadro: cada marca secreta ou visível que mostre que o fascismo
encontrou no homem barreiras tão intransponíveis quanto as que encontrou no globo
terrestre.
