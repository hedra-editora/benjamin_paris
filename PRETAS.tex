%!TEX root=./LIVRO.tex 

\textbf{Walter Benedix Schönflies Benjamin} nasceu no dia 15 de julho de 1892, em Berlim,
primogênito de uma família abastada de origem judaica. Entre 1912 e 1915, estudou Filosofia
em Freiburg e Berlim, tendo"-se engajado, até 1914, no Movimento da Juventude. Ainda em
1915, ano em que inicia a sua amizade com Gerhard Scholem, muda"-se de Berlim para
Munique, aonde prossegue com os seus estudos universitários. Casa"-se em 1917 com Dora
Sophie Pollak e, tempos de guerra, instalam"-se na Suíça. Em Berna, onde nasce seu filho Stefan
Rafael, e conhece Ernst Bloch, doutora"-se em 1919 com a dissertação \emph{O conceito de crítica de
arte no romantismo alemão}. Retorna a Berlim e, em função da crise do pós"-guerra, começa a
enfrentar problemas financeiros que perdurarão. Publica, em 1922, o ensaio ``\emph{As Afinidades
eletivas} de Goethe''. Durante os anos de preparação de sua tese de livre docência, \emph{Origem do
drama barroco alemão}, conhece Theodor Adorno, Siegfried Kracauer e Asja Lacis, aproxima"-se do materialismo histórico e das vanguardas artísticas. Recusada pela Universidade de
Frankfurt em 1925, a tese foi publicada em 1928, quando também sai o livro \emph{Rua de mão única}.
Nesse ínterim, traduz Proust, colabora em jornais e revistas e inicia seu projeto acerca das
Passagens parisienses, do qual se ocupou, com interrupções, por todos os anos seguintes,
deixando"-o, no entanto, inacabado. Em 1929 conhece Bertold Brecht, e começa a trabalhar na
produção e locução de peças radiofônicas que, posteriormente, se desdobram na elaboração de
suas memórias \emph{Crônica berlinense e Infância em Berlim por volta de 1900}. Em 1933, com a
ascensão do regime nazista, exila"-se, morando sobretudo em Paris, e passa a colaborar com o
Instituto de Pesquisa Social, que estava sob a direção de Max Horkheimer. Entre seus escritos
mais conhecidos, desse período em diante, estão: ``A obra de arte na época de sua
reprodutibilidade técnica'', ``O contador de histórias: considerações sobre a obra de Nikolai
Leskov'', ``A Paris do segundo império em Baudelaire'', ``Sobre alguns temas em Baudelaire'' e
``Sobre o conceito de história''. Este último, redigido meses antes de Walter Benjamin, em
pernoite no município espanhol de Portbou, vendo como fracassada sua tentativa de fuga do
avanço das forças nazistas, suicidar"-se a 26 de setembro de 1940.


%(1892-1940) foi um filósofo, crítico literário, tradutor (de Baudelaire, Proust e Balzac, entre outros) e também um ficcionista alemão. Estudou filosofia num ambiente dominado pelo neokantismo, em Berlim, Freiburg, Munique e Berna, onde defendeu tese de doutorado sobre os primeiros românticos alemães. Durante o seu exílio em Paris, nos anos trinta, foi ligado ao Instituto de Pesquisa Social, embrião da chamada Escola de Frankfurt. Entre seus interlocutores e amigos, encontram"-se personalidades marcantes do século~\versal{XX} como Theodor W. Adorno, Hannah Arendt, Bertolt Brecht e Gershom Scholem.

%\textbf{\titulo} reúne textos de teor literário"-crítico, filosófico,
%artístico, político e biográfico referentes a uma década (1926"-36) do percurso de Walter
%Benjamin entre Alemanha e França, e ao seu ``lugar'' como crítico literário alemão e refugiado
%político no debate literário francês do período entreguerras. Lugar desde o qual delineia o que
%definiu como o ``triângulo equilátero'' da moderna literatura francesa, formado por Marcel
%Proust, André Gide e Paul Valéry, num momento em que a fama póstuma ainda não os
%distinguia. Entre os muitos textos ainda inéditos no Brasil que compõem o presente volume, as
%\emph{Cartas Parisienses} fornecem elementos do contexto político e literário, bem como uma
%apresentação conflituosa entre fotografia e pintura que sugere uma revisão dos ensaios (mais
%conhecidos) sobre fotografia e cinema. O rico material proporcionado pelas ``Anotações ao
%ensaio \emph{Imagem de Proust}'' convida a um aprofundamento dos estudos acerca da importância do
%escritor francês para Benjamin. Muito interessante, ainda, notar"-se, entre outras tematizações de
%especial relevo, certa digressão nos escritos, especialmente no ``Diário parisiense'', sobre a
%homossexualidade de Gide e de Proust, muitas décadas antes da discussão ganhar a dimensão de
%nossos dias.

\textbf{Carla Milani Damião} é professora da Faculdade de Filosofia e dos Programas de Pós"-graduação em Filosofia e em Arte e Cultura Visual da Universidade Federal de Goiás (\versal{UFG}).
Entre outras publicações, é autora do livro \emph{Sobre o declínio da ``sinceridade'': Filosofia e
autobiografia de Jean"-Jacques Rousseau a Walter Benjamin} (Loyola, 2006) e organizadora de
coletâneas, entre as quais: \emph{Confluindo tradições estéticas} (2016), \emph{Estética em preto e branco} (2018) e \emph{Estéticas indígenas} (2019), como resultado de colóquios organizados pela linha de pesquisa em Estética e Filosofia da Arte do Programa de Pós"-graduação em Filosofia da \versal{UFG}.

\textbf{Pedro Hussak van Velthen Ramos} é professor de Estética na Universidade Federal Rural do
Rio de Janeiro (\versal{UFRRJ}), onde atua nos cursos de graduação e pós"-graduação em Filosofia.
Colabora também no Programa de Pós"-Graduação em Estudos Contemporâneos das Artes da
Universidade Federal Fluminense (\versal{UFF}). Entre outros títulos, publicou como organizador \emph{Educação Estética: de Schiller a Marcuse} (Nau Editora, 2011) e foi editor de dossiês temáticos sobre Jacques Rancière e arte contemporânea.

\textbf{Coleção Walter Benjamin} é um projeto acadêmico"-editorial que envolve pesquisa, tradução e publicação de obras e textos seletos desse importante filósofo, crítico literário e historiador da cultura judeu"-alemão, em volumes organizados por estudiosos versados em diferentes aspectos de sua obra, vida e pensamento. 

\textbf{Amon Pinho} é Professor Associado na Universidade Federal de Uberlândia (\versal{UFU}) e Pesquisador Associado no Centro de Filosofia da Universidade de Lisboa (\versal{CFUL}). Atua nas
áreas de História e Filosofia, com ênfase em Teoria da História e História da Filosofia
Contemporânea. Entre seus temas de eleição, dedica"-se à pesquisa e ao estudo da vida e obra de
Walter Benjamin.

\textbf{Francisco De Ambrosis Pinheiro Machado} é Professor Associado na Escola de Filosofia, Letras e Ciências Humanas da Universidade Federal de São Paulo (\versal{UNIFESP}). Pesquisa sobre filosofia da história, teoria crítica da cultura e estética na obra dos autores vinculados à Teoria Crítica, sobretudo, Walter Benjamin.


