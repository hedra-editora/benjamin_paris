%!TEX root=./LIVRO.tex 

\textbf{Walter Benjamin} (1892-1940) foi um filósofo, crítico literário, tradutor (de Baudelaire, Proust e Balzac, entre outros) e também um ficcionista alemão. Estudou filosofia num ambiente dominado pelo neokantismo, em Berlim, Freiburg, Munique e Berna, onde defendeu tese de doutorado sobre os primeiros românticos alemães. Durante o seu exílio em Paris, nos anos trinta, foi ligado ao Instituto de Pesquisa Social, embrião da chamada Escola de Frankfurt. Entre seus interlocutores e amigos, encontram"-se personalidades marcantes do século~\versal{XX} como Theodor W. Adorno, Hannah Arendt, Bertolt Brecht e Gershom Scholem.

\textbf{\titulo} \lipsum[1]

\textbf{Carla Milani Damião} \lipsum[2] 

\textbf{Pedro Hussak van Velthen Ramos} \lipsum[3]

\textbf{Coleção Walter Benjamin} é um projeto acadêmico"-editorial que envolve pesquisa, tradução e publicação de obras e textos seletos desse importante filósofo, crítico literário e historiador da cultura judeu"-alemão, em volumes organizados por estudiosos versados em diferentes aspectos de sua obra, vida e pensamento. 

\textbf{Amon Pinho} é Professor Associado na Universidade Federal de Uberlândia e Pesquisador Associado no Centro de Filosofia da Universidade de Lisboa. Tem experiência nas áreas de História e Filosofia, com ênfase em Teoria da História e História da Filosofia Contemporânea. Entre os seus temas de eleição, dedica"-se ao estudo da vida e obra de Walter Benjamin.

%\textbf{Francisco De Ambrosis Pinheiro Machado} é Professor Associado na Escola de Filosofia, Letras e Ciências Humanas da Universidade Federal de São Paulo (\versal{UNIFESP}). Pesquisa sobre filosofia da história, teoria crítica da cultura e estética na obra dos autores vinculados à Teoria Crítica, sobretudo, Walter Benjamin.


