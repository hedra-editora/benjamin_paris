\chapter{Três franceses\footnote[*]{``Drei Franzosen'',
  in Walter Benjamin, \emph{Gesammelte Schriften} {[}daqui em diante:
  \versal{GS}{]}, vol. \versal{III}: Kritiken und Rezensionen. Edição de Rolf Tiedemann e
  Hermann Schweppenhäuser. Frankfurt a. \versal{M}.: Suhrkamp, 1991, p. 79-81.
  Tradução de Carla M. Damião e Pedro Hussak van V. Ramos. Essa resenha
  foi publicada no Caderno de Literatura do Jornal de Frankfurt
  (\emph{Frankfurter Zeitung}), em 30 de outubro de 1927.
  [\versal{N. E.}]}}

Proust, Gide e Valéry, isto é, se quisermos, o triângulo equilátero da
nova literatura francesa, ao redor do qual Souday,\footnote{Walter
  Benjamin refere"-se à coletânea do jornalista e crítico literário Paul
  Souday, que reúne três volumes sob o título \emph{Les Documentaires},
  sendo o primeiro intitulado \emph{Marcel Proust}, o segundo,
  \emph{André Gide}, e o terceiro, \emph{Paul Valéry}, publicados em
  Paris, pela editora Simon Kra, em 1927. Souday é também autor do livro
  intitulado \emph{Les livres du Temps} e jornalista do jornal \emph{Les
  Temps}, referido por Benjamin nesta resenha apenas como \emph{Temps}.
  {[}\versal{N. E.}{]}} com sua pena crítica, traçou um círculo, tornando"-o assim
quase uma figura canônica. A isso corresponde o fato de que suas linhas
correm sob uma grande folha na qual está impresso o título \emph{Temps}.
Souday é cronista literário desse jornal. Isto garante, antes de mais
nada, o valor documental desta coletânea de apresentações. O
descontraído que vai e vem de suas reflexões, que se repõe a cada livro,
possui todas as possibilidades de tornar palpável aos leitores de hoje a
atmosfera especial que existia durante o surgimento de aproximadamente
40 volumes tratados na coletânea.

No caso de Proust, isso é o mais interessante. Souday foi, em 1913, um
dos poucos que reconheceu na primeira obra da grande série --- \emph{Du côté de chez Swam}\footnote{Em francês no original: \emph{O
caminho de Swann}.} --- algo mais do que um emaranhado desagradável de
notícias insignificantes e de meditações mórbidas. Nada mais difícil
para um resenhista do que essa obra, não digo para ler, compreender, mas
para apresentá"-la ao público. Antes que a guerra, num só golpe,
mostrasse a todos a própria existência em perspectiva extremamente
reduzida, na medida em que os colocava duramente diante do fim de sua
vida, a qual Proust tivera como um enfermo em seu destino; antes que a
guerra formasse para ele um público, esse crítico soube trazer à luz o
charme e a distinção do livro perturbador. Grande número de seus colegas
precisou de seis anos para segui"-lo em sua posição de vanguarda. Em
seguida, em 1919, o prêmio Goncourt é concedido ao escritor e, de lá
para cá, a crítica se transformou mais e mais na escrita da história de
sua fama. Mas como uma ``Gênese da fama'', apesar do excelente estudo de
Julian Hirsch,\footnote{Benjamin refere"-se à obra de Julian Hirsch,
  \emph{Die Genesis des Ruhmes. Ein Beitrag zur Methodenlehre der
  Geschichte} {[}\emph{A gênese da fama. Uma contribuição para uma
  doutrina do método da história}{]}. Leipzig: Johann Ambrosius Barth,
  1914. {[}\versal{N. E.}{]}} ainda esteja por ser escrita, aquilo que se mostra
de maneira muito diferente nos três escritores é tão cativante. Por
outro lado, pode"-se justamente lamentar que o ensaísta tenha tornado um
tanto apagada a origem jornalística de suas anotações. Sente"-se a
ausência em tais coletâneas do habitual prefácio e da data de publicação
de cada resenha. Seja como for: nas minúsculas nuvenzinhas do horizonte
intelectual do tempo, este olhar reconheceu a tempestade de poeira
formada por uma fama que se aproximava. Se esse olhar então mais tarde,
em todo caso, penetrou"-a e compreendeu precisamente o que estava por
detrás dela, é uma outra e mais complexa pergunta.

Aquilo que se pode ler aqui sobre Gide poderia tornar sua resposta
duvidosa. Assim que suas primeiras obras surgiram nos anos 1890, Souday,
também no que diz respeito a esse autor, inteirou"-se delas de modo
surpreendentemente rápido. Mas, com isso, para a sequência, nada estava
ainda assegurado. Proust pode permanecer inacessível a muitos leitores.
Mas, certamente, a quem ele se abre (cada frase pode tornar"-se a fresta
deste sésamo), sente"-se, de uma vez por todas, em casa, em seu círculo
mágico. Nada de semelhante ocorre com Gide. Aqui, feitiço e magia não
têm lugar. Pois ele pertence àquela terrível classe de escritores que
não enxerga no público a humanidade, deus ou a mulher, mas a besta. Gide
--- nisto, próximo a Oscar Wilde --- é um domador de palavras
(\emph{dompteur ès lettres}\footnote{Segundo o organizador Heinrich
  Kaulen (\emph{Walter Benjamin. Werke und Nachlass. Kritische
  Gesamtausgabe}, volume 13.2, p. 86-88), Benjamin repete a expressão de
  Gide sobre Baudelaire: ``\emph{magicien ès lettres}''. {[}\versal{N. E.}{]}}).
Um público adestrado na liberdade é seu sonho. E ouvia"-se o
rugido\footnote{Em alemão, \emph{Das Grollen} pode significar o
  ressentimento ou o rancor. Neste caso, o comentário de Heinrich Kaulen
  (\emph{ibidem}) é voltado para a reação da opinião pública diante da
  confissão feita por André Gide, nas obras citadas na sequência, de sua
  homossexualidade. Cauteloso, o organizador supõe ser isso, mas supõe
  também ser uma reação, de ordem política, à publicação da \emph{Voyage
  au Congo}, em 1927, embora esta obra não esteja citada na resenha em
  questão. {[}\versal{N. E.}{]}} por toda Paris, que acabou arruinando alguns
números da obra, nos quais se pensou mostrar seu domador. Dessas novas
insubordinações, Souday não está totalmente isento de culpa.

Mas ele não seria resenhista do \emph{Temps}, não seria o culto e
espirituoso representante de um centro burguês consolidado, se não
defendesse contra os \emph{Les Faux"-monnayeurs},\footnote{Em francês no original: \emph{Moedeiros falsos}. {[}\versal{N. T.}{]}} o \emph{Corydon} e a bela
autobiografia de Gide, publicada sob o título \emph{Si le grain ne meurt},\footnote{Em francês no original: \emph{Se o grão não
morre}. {[}\versal{N. T.}{]}} os direitos do instinto ``saudável'', mesmo com certa falta de
consideração. Pois, por mais que esse jornalista evidencie teimosamente
suas máximas e caprichos, no fundo, ele foi educado pelas melhores
tradições da burguesia francesa. Hugo é seu Deus; o clero, seu lenço
vermelho, e a democracia, sua profissão de fé. Um racionalismo
inteiramente humanista faz dele, então, por si mesmo, um dos intérpretes
de Valéry mais interessantes entre muitos nem sempre bem vindos.
Conhece"-se esse poeta e filósofo como o mais significativo entre os
oponentes da corrente surrealista, da psicologia profunda, da corrente
psicanalítica, dos cultos do inconsciente e da inspiração. Isso não pôde
evitar que, a partir do momento de sua fama, quando os contornos desta
surpreendente existência perderam a precisão, à medida que a atenção do
público aumentava, que então um abade\footnote{Benjamin refere"-se aqui
  ao abade Henri Bremond e ao debate sobre a ``poesia pura'', que durou
  algumas décadas na França, em parte motivada pelos três livros do
  abade: \emph{La Poésie pure} (1925-1926), \emph{Prière et poésie}
  (1926), \emph{Racine et Valéry} (1930), e pela proposição de que toda
  poesia advém do divino. {[}\versal{N. E.}{]}} um tanto espirituoso se
apoderasse de alguns de seus melhores pensamentos e uma pálida e
insignificante discussão sobre a afinidade entre poesia pura
(\emph{poésie pure}) e oração se espalhasse durante meses nas revistas.
Nessa discussão com os mesmos devaneios, aos quais Valéry presta"-se (não
em nome de sua honra), encontra"-se este homem com seu elemento mais íntimo: a
polêmica. E se assim se afasta dos tipos medianos da crítica francesa,
tornar"-se"-á então, justamente nesse aspecto, ainda mais acessível aos
leitores alemães. Para eles, esses três pequenos volumes compõem o mais
agradável esboço, que poderiam desejar, da mais nova luta literária
francesa.
