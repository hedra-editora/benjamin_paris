\chapter{Imagem de Proust\footnote[*]{``Zum Bilde Prousts'', in \versal{GS II}-1, p. 310-324. Tradução de Carla Milani Damião.}}

\section{I}

Os treze volumes de \emph{À la recherche du temps perdu} de Marcel
Proust são o resultado de uma síntese difícil de ser construída, na qual
a imersão do místico, a arte do prosador, a verve do satírico, o saber
do erudito e o ensimesmamento do monomaníaco, convergem em uma obra
autobiográfica. Diz"-se com razão que todas as grandes obras da
literatura fundam ou desfazem um gênero, em uma palavra, são casos
excepcionais. Entre elas, esta, porém, é uma das mais inclassificáveis.
A começar pela construção que apresenta poesia, memórias, comentário em
\emph{uma única obra}, até a sintaxe de frases infinitas, sem margens
(esse Nilo da linguagem que transborda aqui, fertilizando as planícies
da verdade), tudo está fora da norma. Que esse grande caso especial da
literatura represente, ao mesmo tempo, sua mais elevada realização nas
últimas décadas, é o primeiro reconhecimento esclarecedor que se abre ao
observador. Insalubres em mais alto grau, são as condições que a
sustentavam. Um sofrimento incomum, uma riqueza excepcional e uma
predisposição anormal. Nem tudo nessa vida é modelar, mas tudo é
exemplar. A obra determina, para a realização literária extraordinária
daquela época, seu espaço no coração da impossibilidade, no centro; e,
certamente, ao mesmo tempo, no ponto de indiferença de todos os riscos,
caracterizando essa grande realização da ``obra de uma vida'' como sendo
uma última por muito tempo. A imagem de Proust\footnote{Certamente
  essa penúltima frase do primeiro parágrafo do ensaio, conduziu à
  tradução desse como ``A imagem de Proust'' (\emph{Prousts Bild}). O
  sentido de imagem no ensaio, cujo título no original é ``Zum Bilde
  Prousts'', no entanto, não condiz apenas com a descrição fisionômica.
  Por isso, a opção de alguns tradutores por \emph{Retrato} de Proust
  parece arriscada, ao mesmo tempo, possível. Traduzir \emph{Zum} como
  \emph{para}, no sentido de em direção a, poderia indicar a procura por
  essa imagem que se constitui no ensaio de maneira ambígua e como um
  exercício imagético"-mimético do próprio Benjamin que, como Proust,
  também cria imagens de forma semelhante à construção imagética
  proustiana. Nossa opção foi por deixar a palavra ``imagem'' sem a
  determinação da preposição ou do artigo definido, de forma a ampliar
  seu sentido no ensaio. Vale indicar a análise do título por Ursula
  Link"-Heer (\emph{Benjamin Handbuch}, p. 514) ao comparar duas traduções
  para o francês do ensaio --- ``\emph{Pour le portrait de Proust}'' de
  Maurice de Gandillac e ``\emph{À propos de l'image chez Proust}'' de
  Rainer Rochlitz. Ambas as traduções, segundo a autora, estão corretas.
  A primeira como leitura fisionômica; a segunda, como Proust, ele
  próprio lidava com as imagens. O título, neste sentido, comportaria
  essa ambiguidade proposital, ao utilizar uma composição com o genitivo
  que se relaciona tanto com o sujeito (criação proustiana de imagens)
  quanto com o objeto (Proust retratado). Nossa opção por ``Imagem de
  Proust'', ao retirar a preposição inicial pode igualmente retirar a
  ambiguidade, por assim dizer, intraduzível, mas abre a possibilidade
  de perceber que a imagem de Proust é também composta por Benjamin e
  que o próprio título é composto, mimeticamente, como um arabesco
  proustiano. Vale ainda ressaltar, do comentário de Heer, o estrito
  parentesco do sentido de imagem no ensaio com o sentido de ``imagem
  dialética'' que surge em seus escritos posteriores. \versal{[N. T.]}} é a mais elevada
expressão fisionômica que a irresistível e crescente discrepância entre
poesia e vida pôde ganhar. Esta é a moral que justifica a tentativa de
trazê"-la à tona.

Sabemos que Proust não descreveu em sua obra uma vida como ela foi, mas
uma vida como aquele que a viveu, lembra\footnote{Traduzimos
  {[}\emph{die}{]} \emph{Erinnerung} por lembrança, {[}\emph{das}{]}
  \emph{Eingedenken} por rememoração, respeitando tanto o significado
  particular de cada substantivo quanto o jogo dialético que --- nesse
  ensaio --- ocorre entre os seguintes verbos substantivados: o ``lembrar'' (\emph{das Erinnern}) e o ``esquecer'' \emph{(das
  Vergessen}). Mais do que uma simples variação de termos --- por vezes
  traduzidos de maneira diferente na recepção de Walter Benjamin no
  Brasil ---, os verbos substantivados de ``lembrar'' e ``esquecer'' são
  de fundamental importância para a compreensão da dialética que se
  instaura na tecitura ou confecção de imagens literárias que se
  constituem no limite entre inconsciente e consciente. \versal{[N. T.]}} dessa vida.
Entretanto, isso ainda está expresso de maneira imprecisa e de longe de
forma muito grosseira. Pois aqui, para o autor que lembra, não
desempenha de forma alguma o papel principal aquilo que ele vivenciou,
mas sim o tecer de suas lembranças, o trabalho de Penélope da
rememoração. Ou não deveríamos falar de uma obra de Penélope do
esquecimento? Não está a rememoração involuntária,\footnote{No
  original em alemão: \emph{ungewollte Eingedenken}, aqui aproximada do
  conceito proustiano de memória involuntária (\emph{mémoire
  involontaire}). \versal{[N. T.]}} a \emph{mémoire involontaire} de Proust, muito mais
próxima do esquecimento daquilo o que na maioria das vezes se chama
lembrança? E essa obra de rememoração espontânea, na qual a lembrança é
a trama, e a urdidura o esquecer, não é muito mais o oposto à obra de
Penélope do que sua imagem semelhante?\footnote{No original em alemão:
  \emph{Ebenbild}. \versal{[N. T.]}} Pois aqui o dia desfaz o que a noite
tecia.\footnote{No original em alemão: \emph{wirken}, palavra que
  também significa tecer, especialmente tapeçaria. \versal{[N. T.]}} Toda manhã,
despertos, seguramos em nossas mãos, quase sempre de maneira fraca e
solta, apenas algumas franjas do tapete da existência vivida, como o
esquecer o teceu em nós. Mas cada dia desfaz o entrelaçamento, os
ornamentos do esquecer por meio da ação vinculada a um objetivo e, mais
ainda, por meio do lembrar preso a um objetivo. Por isso, Proust, ao
final, transformou seus dias em noite, para dedicar imperturbavelmente
todas suas horas à obra, sob luz artificial em seu quarto escurecido,
para não perder nenhum de seus enredados arabescos.

Quando os romanos denominam um texto tecitura, quase nenhum outro, assim
sendo, é mais intenso e mais denso do que o de Marcel Proust. Nada foi
para ele denso e duradouro o bastante. Seu editor Gallimard narrou como
os hábitos de Proust durante a revisão deixavam os gráficos
desesperados. As provas voltavam sempre cheias de anotações nas margens.
Mas nenhum erro ortográfico havia sido corrigido; todo espaço disponível
estava tomado por novos textos. Assim a legalidade do lembrar
efetivava"-se até na extensão da obra. Pois um acontecimento vivido é
finito, ao menos encerrado na esfera do vivenciado, um acontecimento
lembrado é ilimitado, porque ele é apenas chave\footnote{No
  original em alemão, com complementação entre chaves: \emph{weil} {[}\emph{es}{]} \emph{nur Schlussel} {[}\emph{ist}{]}. \versal{[N. T.]}} para tudo o que veio antes e tudo o que veio
depois dele. Ainda noutro sentido, a lembrança aqui dita a regra
rigorosa do tecer. A unidade do texto não é a pessoa do autor, muito
menos a ação, mas, exatamente, o \emph{actus purus} do próprio lembrar.
Pode"-se até mesmo dizer, suas intermitências são apenas o avesso
do \emph{continuum} do lembrar, o desenho invertido do tapete. Assim o
quis Proust, e assim ele deve ser compreendido, quando diz que
preferiria ver toda sua obra impressa em duas colunas, em um volume, e
sem nenhum parágrafo.

O que ele procurava tão freneticamente? Em que se fundamentavam esses
esforços infinitos? Podemos dizer que todas as vidas, obras, ações, que
importam, nunca foram outras do que o desdobramento inabalável das horas
mais banais e voláteis, mais sentimentais e mais frágeis na existência
daquele ao qual elas pertencem? E quando Proust, em uma passagem famosa,
descreveu este seu momento mais próprio, ele o fazia de tal maneira que
cada um o reencontrasse em sua própria existência. Por pouco poderíamos
denominá"-lo de corriqueiro. Ele vem com a noite, com um assobio
longínquo ou com o suspiro no peitoril da janela aberta. E não se pode
prever, quais encontros nos estariam predeterminados, se nós
estivéssemos menos dispostos a dormir. Proust não obedecia ao sono. E
mesmo assim, por isso mesmo e muito mais, pode Jean Cocteau falar num
belo ensaio sobre o timbre de sua voz, que ela foi obediente às leis da
noite e do mel. Sujeitando"-se ao seu domínio, ele venceu o luto
desesperançoso em seu interior (o que ele certa vez denominou
``l'essence même du présent une imperfection incurable''\footnote{Em francês no original: ``A imperfeição incurável na própria essência
  do presente''. Essa citação foi retirada da seguinte passagem de
  \emph{Les plaisirs et les jours} \emph{(Os prazeres e os dias)}, \versal{XXV},
  1§: ``Mais comme l'alchimiste, qui attribue chacun de ses insuccès à
  une cause accidentelle et chaque fois différente, loin de soupçonner
  dans \emph{l'essence même du présent une imperfection incurable}, nous
  accusons la malignité des circonstances particulières, les charges de
  telle situation enviée, le mauvais caractère de telle maîtresse
  désirée, les mauvaises dispositions de notre santé un jour qui aurait
  dû être un jour de plaisir, le mauvais temps ou les mauvaises
  hôtelleries pendant un voyage, d'avoir empoisonné notre bonheur.''
  {[}grifo nosso{]}. ``Mas como o alquimista que atribui a cada um dos
  seus insucessos uma causa acidental e a cada vez diferente, longe de
  supor \emph{uma imperfeição incurável na própria essência do
  presente}, acusamos a malignidade das circunstâncias particulares, os
  fardos de tal situação invejada, o caráter ruim da amante desejada, as
  más disposições de nossa saúde em um dia que deveria ser de prazer, o
  mau tempo ou más hospedarias durante uma viagem, de ter envenenado
  nossa felicidade.'' \versal{[N. T.]}}), e construiu dos favos da lembrança sua casa
para abrigar o zunir dos pensamentos. Cocteau viu o que deveria ocupar
todo leitor de Proust no mais elevado grau: ele viu o desejo de
felicidade cego, desatinado, possuído nessa pessoa. Reluzia de seus
olhares. Eles não eram felizes. Mas neles jazia a felicidade como
\emph{no} jogo ou \emph{no} amor. Não é difícil dizer, porque esse
desejo de felicidade hesitante, explosivo que penetra a poesia de
Proust é tão raramente entendido pelos seus leitores. Proust mesmo
facilitou"-lhes em considerar, em vários lugares, essa obra sob a
antiga perspectiva comprovada e confortável da renúncia, do heroísmo, da
ascese. Já que nada é mais óbvio aos alunos modelares da vida senão o
fato de que uma grande realização seja apenas fruto de esforços, lamento
e decepção. Pois que no belo também a felicidade possa participar, seria
um excesso do bem, além do qual o ressentimento deles jamais se
consolaria.

Existe, porém, uma vontade de felicidade dupla, uma dialética da
felicidade. Uma forma de felicidade hínica e elegíaca. Uma: o inaudito,
o que nunca existiu antes, o ápice da beatitude. Outra: o eterno mais
uma vez, a eterna restauração da felicidade originária, primeira. Essa
ideia de felicidade elegíaca, que se poderia também chamar de eleática,
é o que, para Proust, transforma a existência em uma floresta encantada
da lembrança. A ela, ele não só sacrificou na vida amigos e convívio
social, como também na obra, ação, unidade da personagem, fluxo
narrativo e jogo da fantasia. Max Unold, não o pior de seus leitores,
aquele que deu atenção ao ``tédio'' de tal modo condicionado de seus
escritos com as ``histórias de cobrador'', e que encontrou a fórmula:
``Ele conseguiu tornar interessante a história de cobrador''. Ele diz:
``reflita, senhor leitor, ontem mergulhei um biscoito no meu chá, então
ocorreu"-me que quando criança estive no campo --- e para isso ele gasta 80
páginas, e é tão fascinante, que nós não acreditamos ser mais o ouvinte,
mas o próprio sonhador desperto''. Nessas histórias de cobrador, Unold
encontrou uma ponte para o sonho --- ``todos os sonhos comuns, tão logo
são contados, tornam"-se histórias de cobrador''. Toda interpretação
sintética de Proust deve ser relacionada ao sonho. Não faltam portas
imperceptíveis que levem para dentro. Lá se encontra o estudo frenético
de Proust, seu culto apaixonado da semelhança. Não, onde ele a descobre
sempre perplexa, inesperadamente nas obras, fisionomias ou modo de
falar, que ela {[}a semelhança{]} deixa reconhecer os verdadeiros sinais
de seu domínio. A semelhança de um com o outro, com a qual contamos, que
nos ocupa no estado desperto, envolve apenas o mais profundo mundo do
sonho, no qual, o que ocorre, emerge, nunca de forma idêntica, mas de
forma semelhante, semelhante a si mesmo, de forma não transparente. As
crianças conhecem a marca desse mundo, a meia, que tem a estrutura do
mundo dos sonhos, quando ela, enrolada na cesta de roupa, é ao mesmo
tempo ``bolsa''\footnote{No original: ``Tasche'', entre aspas, uma espécie de bolsa que se forma ao enrolarmos um par de meias, virando uma das bordas para reunir cada meia em separado em uma peça única que, para Benjamin, assemelha"-se a uma bolsa. {[}\versal{N. T.}{]}} e ``o que está dentro''.\footnote{No original, o substantivo ``Mitgebrachtes'', também entre aspas, do verbo mitbringen que literalmente é ``trazer com'', pode ser o que ``veio junto''. Mitgebrachtes, em seguida, é substituído no jogo de palavras que formam a imagem lúdica do jogo das crianças, por ``was drin liegt'', literalmente, ``o que está dentro''. {[}\versal{N. T.}{]}} E assim como elas mesmas não
podem saciar"-se, de transformar ambas: bolsa e o que está dentro, com
\emph{um} gesto, num terceiro {[}elemento{]} --- em meia; também Proust
era insaciável em esvaziar, num único gesto, o falso modelo, o eu, para
sempre de novo entregar um terceiro: a imagem, que matava sua
curiosidade, ou mais ainda, sua nostalgia. Dilacerado pela nostalgia,
ele deitava na cama, com saudades do mundo distorcido no estado de
semelhança, no qual a verdadeira face surrealista da existência irrompe.
A ela pertence o que acontece em Proust, e quão cuidadosa e
elegantemente isso emerge. Isto é, nunca isolada, patética e
visionariamente, mas anunciada e multiplamente apoiada, carregando uma
frágil e preciosa realidade: a imagem. Ela se desprende da estrutura das
frases de Proust como em Balbec, sob as mãos de Françoise, o dia de
verão, velho, imemorial, mumificado, se desprende das cortinas de tule.

\section{II}

O mais importante que alguém tem a dizer, nem sempre ele proclama em voz
alta. E, mesmo em silêncio, ele nem sempre o confia para o mais
familiar, o mais próximo, nem sempre aquele que se prontificava mais
devotadamente a receber sua confissão. Se não somente as pessoas, mas
também as épocas, têm essa maneira casta, ou seja, formas tão astutas e
frívolas de comunicar o seu mais próprio a uma pessoa qualquer, e então,
para o século \versal{XIX}, não é Zola ou Anatole France, mas sim o jovem Proust,
o desprezível esnobe, o insincero frequentador de salões, que capturou
no ar as confidências mais surpreendentes do curso envelhecido do tempo
(como de um outro igualmente agonizante Swann). Só Proust tornou o
século dezenove memorável.\footnote{Note"-se a composição híbrida ---
  francês e alemão da palavra ``\emph{mémoirenfähig}'' que traduzimos
  por memorável, no sentido de tornar capaz de se lembrar. Quem se
  lembra? A época, o século \versal{XIX}. \versal{[N. T.]}} O que antes dele foi um período sem
tensão, tornou"-se um campo de força no qual as correntes mais diversas
foram despertadas por autores posteriores. Não é coincidência que o
trabalho mais interessante desse tipo venha de uma escritora, que era
pessoalmente próxima de Proust como admiradora e amiga. Já o título sob
o qual a duquesa Clermont"-Tonnerre apresenta o primeiro volume de suas
memórias --- \emph{Au temps des équipages}\footnote{Em francês no original: \emph{Nos tempos das tripulações}. \versal{[N.~T.]}} --- teria sido inimaginável antes de
Proust. Além disso, é o eco que ressoa suavemente o chamado ambíguo,
amoroso e desafiador do poeta de Faubourg Saint"-Germain. Acrescente"-se a
isso que essa apresentação (melódica) está cheia de relações diretas ou
indiretas a Proust, tanto em sua atitude como em suas personagens, entre
as quais estão ele próprio e alguns de seus objetos favoritos de estudo
do Ritz. Com isso, é claro que não podemos negar que estamos em um
ambiente muito feudal e com aparições como Robert de Montesquiou, que a
duquesa Clermont"-Tonnerre apresenta magistralmente, e, além disso, um
meio muito especial. Em Proust também estamos nesse meio, sabe"-se que
não falta um correspondente a Montesquiou. Isso tudo não compensaria a
discussão --- especialmente porque a questão dos modelos é secundária e
irrelevante para a Alemanha --- se a crítica alemã não gostasse tanto de
se acomodar. Antes de tudo, a crítica poderia deixar passar a
oportunidade de se acanalhar com a corja das bibliotecas. Nada ficou
mais perto de seus exponentes mais experientes, senão deduzir o autor do
ambiente esnobe da obra e rotular o trabalho de Proust, um caso interno
francês, um folhetim de entretenimento de Gotha.\footnote{Gotha é o
  título de catálogo genealógico da aristocracia europeia. \versal{[N. E.]}} Agora é
óbvio: os problemas das pessoas proustianas vêm de uma sociedade
saturada, mas não há um único que seja congruente com aqueles do autor.
Estes são subversivos. Se fosse para expressar esses problemas em uma
fórmula, então seu propósito seria o de construir toda a estrutura da
sociedade mais alta na forma de uma fisiologia da tagarelice. Não há no
tesouro de seus preconceitos e máximas nenhum que sua perigosa visão do
cômico não aniquile. Ter chamado atenção para essa não é o menor dos
significativos méritos que Léon Pierre"-Quint conquistou como o primeiro
intérprete de Proust. Quint escreve: ``Quando falamos de obras de humor,
geralmente pensamos em livros curtos e divertidos com capas ilustradas:
esquecemos de \emph{Dom Quixote}, \emph{Pantagruel} e \emph{Gil Blas},\footnote{\emph{L'Histoire de Gil Blas de Santillane} é o título completo da
  autobiografia ficcional (1715-1747) citada por Benjamin. \versal{[N. T.]}} volumes
pesados, desproporcionais e impressos em letras miúdas''. O lado
subversivo da obra de Proust aparece nesse contexto da maneira mais
concisa. E aqui é menos o humor do que o cômico, o centro verdadeiro de
sua força, ele não ergue o mundo em gargalhadas, mas o derruba nas
gargalhadas. Sob o risco de ele quebrar"-se em cacos, diante dos quais
ele mesmo desaba em prantos. E ele se quebra em pedaços: a unidade de
família e da personalidade, da moral sexual e da honra de classe. As
pretensões da burguesia são espatifadas nas gargalhadas. Sua fuga de
volta, sua reassimilação à aristocracia, é o tema sociológico da obra.

Proust não se cansou do treinamento que o contato com os círculos
feudais exigia. Perseverantemente e sem ter que se esforçar muito, ele
flexibilizou sua natureza a fim de torná"-la tão impermeável e engenhosa,
devota e difícil, como ele havia de se tornar pelo bem de sua tarefa.
Mais tarde, a mistificação, a cerimoniosidade, tornaram"-se tão
naturais que suas cartas são às vezes sistemas inteiros de parênteses ---
e não apenas gramaticais. Cartas que, apesar de sua composição
infinitamente inteligente e ágil, lembram, por vezes, o lendário
esquema: ``Honrada Senhora, acabo de perceber que esqueci ontem minha
bengala em sua casa e peço que a entregue ao portador desta mensagem.
\versal{P.S.}: Perdoe o incômodo, acabei de encontrá"-la''. Quão inventivo ele é
em apuros. Tarde da noite, ele aparece na casa da duquesa
Clermont"-Tonnerre, condicionando sua estadia a alguém que busque o
remédio em sua casa. E, então, ele envia o camareiro, dando"-lhe uma
longa descrição da região, da casa. Por último, diz: ``O senhor não tem
como errar. A única janela no Boulevard Haussmann onde ainda há luz''.
Tudo, menos o número. Tente conseguir o endereço de um bordel numa
cidade estrangeira e após obter as informações mais longas --- qualquer
coisa, menos a rua e o número da casa --- entenderá o que isso significa
aqui (e como isto está relacionado ao amor pelo cerimonial em Proust, a
sua veneração por Saint"-Simon e, não menos importante, ao seu
intransigente francesismo). Não é a quintessência da experiência:
experenciar como é difícil experenciar muita coisa que aparentemente
se deixaria dizer em poucas palavras? Só que tais palavras pertencem a
um linguajar de castas e estamentos\footnote{No original em alemão:
  \emph{Rotwelsch}. Trata"-se de uma espécie de dialeto que mistura
  alemão e romani, levada à Alemanha por migrantes. \versal{[N. T.]}} --- que não pode ser
entendido por estranhos. Não é nenhum milagre que a linguagem secreta
dos salões apaixonasse Proust. Mais tarde, quando ele empreendeu a
descrição impiedosa do \emph{petit clan}, dos Courvoisier, do
\emph{esprit d'Oriane},\footnote{O \emph{esprit d'Oriane} (in \versal{PROUST},
  M. \emph{Le côté de Guermantes}, p. 438) é descrito como o principal
  atrativo do Salão dos Guermantes. \versal{[N. E.]}} ele próprio havia familiarizado"-se
no contato com os Bibescos\footnote{Cf. Nota 2 de ``Documentos sobre Proust'', página \pageref{bibescos}. {[}\versal{N. E.}{]}} com a improvisação de uma linguagem cifrada,
à qual nós fomos apresentados nesse meio tempo.

Nos anos de sua vida de salão, Proust desenvolveu não só o vício da
bajulação num grau eminente --- pode"-se dizer: teológico ---, mas também o
da curiosidade. Em seus lábios havia um reflexo do sorriso que perpassa nos arcos de algumas das catedrais que tanto amava, passando pelos lábios das virgens tolas como fogo alastrado. É o sorriso da curiosidade.  Foi a curiosidade que fez dele, no fundo, um parodista tão grandioso? Assim, saberíamos, ao mesmo tempo, o que deveríamos pensar sobre a palavra ``parodista'' neste lugar. Não muito. Pois mesmo que se faça justiça à sua \emph{malice} abissal, isso ainda passa ao largo do que existe de amargo, selvagem e severo mordaz nesses relatos fantásticos que ele escreveu no estilo
de Balzac, Flaubert, Sainte"-Beuve, Henri de Régnier, dos Goncourt,
Michelet, Renan e, finalmente, de seu favorito Saint"-Simon que reuniu no
volume \emph{Pastiches et mélanges}. É o mimetismo do curioso, que foi o
artifício engenhoso dessa série, mas ao mesmo tempo um momento de todo
seu processo criativo, no qual a paixão pela vida vegetativa não pode
ser levada suficientemente a sério. Ortega y Gasset foi o primeiro a
chamar a atenção para a existência vegetativa das personagens de Proust,
que estão ligadas de forma tão sustentável ao seu lugar social,
determinadas pela posição do sol misericordioso feudal, movido pelo
vento que sopra de Guermantes ou Méséglise, entrelaçados um ao outro de
maneira impenetrável na densa mata de seus destinos. Desse círculo vital
brota, como método do poeta, o mimetismo. Seus conhecimentos mais exatos,
mais evidentes, estão pousados sob seus objetos, como insetos
sentam"-se nas folhas, flores e ramos, e não traem nada de sua
existência até que um salto, um bater de asas, um pulo\footnote{No
  original em alemão: \emph{Satz}, que além de pulo, significa frase, sentença. \versal{[N. T.]}}
mostram ao observador assustado que aqui uma vida própria, imprevisível,
esgueirou"-se discretamente num mundo estranho. ``A metáfora, por
mais inesperada que seja'', diz Pierre"-Quint, ``se apropria
estreitamente do pensamento''.

O verdadeiro leitor de Proust é continuamente sacudido por pequenos
sustos. Quanto ao resto, nas metáforas ele encontra a precipitação do
mesmo mimetismo que o surpreendia como uma luta pela existência desse
espírito na copa das árvores da sociedade. É preciso dizer uma palavra
de quão íntima e frutiferamente esses dois vícios, a curiosidade e a
bajulação, permearam"-se um ao outro. Uma passagem esclarecedora no livro
da duquesa Clermont"-Tonnerre diz: ``E ao final não podemos esconder:
Proust inebriou"-se com o estudo dos serviçais. Teria sido por que aqui um
elemento que ele não havia encontrado em nenhum outro lugar despertou
seu faro detetivesco? Ou ele os invejava por poderem observar melhor os
detalhes íntimos das coisas que lhe chamavam mais a atenção? Seja como
for, os serviçais em seus vários personagens e tipos, eram sua paixão''.
Nas sombras estranhas de um Jupien, um Monsieur Aimé, de uma Céleste
Albaret, a série deles desenha a forma de uma Françoise, a que com
traços rudes e angulosos de Santa Marta parece ter emergido de um livro
de orações,\footnote{No original em alemão: \emph{Stundenbuch}, que significa
  missal ou livro de orações. \versal{[N. T.]}} para aqueles \emph{grooms} e
\emph{chasseurs}\footnote{Termos utilizados no original, relativos
  à realização de tarefas de atendimento em hotéis, sendo a função de
  ambos, do \emph{groom} e do \emph{chasseur}, abrir portas aos hóspedes
  e realizar pequenas compras. \versal{[N.~T.]}} que são pagos não pelo trabalho, mas
pela ociosidade. E talvez a representação não reivindique o interesse
desse conhecedor das cerimônias de modo mais intenso que nestes mais
baixos escalões. Quem consegue medir quanta curiosidade de serviçal
impregnou a bajulação de Proust, quanta bajulação de serviçal sua
curiosidade, e onde essa cópia astuta do papel de serviçal teve seus
limites no topo da vida social? Ele a forneceu e não podia fazer de
outro modo. Pois, como ele mesmo uma vez revelou, ``\emph{voir}'' e
``\emph{désirer imiter}''\footnote{Em francês no original: ``ver'' e ``desejar imitar''. \versal{[N. T.]}}
eram uma e a mesma coisa para ele. Maurice Barrès expressou essa
atitude, por mais soberana e subalterna como foi, com uma das mais
distintas palavras já escritas sobre Proust, ``Un poète persan dans une
loge concierge''.\footnote{Em francês no original: ``Um poeta persa em um alojamento do
  porteiro''. \versal{[N. T.]}}

Havia na curiosidade de Proust um toque de detetive. Os dez mil da
elite, para ele, um clã de criminosos, uma gangue de conspiradores, com
a qual nenhum outro pode se comparar: a Camorra dos consumidores. Ela
exclui de seu mundo tudo o que tem participação na produção, exigindo ao
menos que essa parte se esconda graciosa e vergonhosamente atrás de um
gesto, como os profissionais perfeitos do consumo o exibem. A análise de
Proust do esnobismo, que é muito mais importante do que sua apoteose da
arte, representa o auge de sua crítica social. Pois a atitude do esnobe
não é outra coisa que a observação consequente, organizada e reforçada
da existência do ponto de vista quimicamente puro do consumidor. E, por
causa dessa encenação fantástica e satânica, tanto a lembrança mais
remota das forças produtivas da natureza, quanto a mais primitiva, deve
ser banida; por isso, para ele, mesmo no amor, a ligação invertida era
mais útil do que a normal. Mas o consumidor puro é o explorador puro.
Ele o é lógica e teoricamente; ele o é, em Proust, em toda a concretude
de sua atual existência histórica. Concreto, porque impenetrável e
difícil de aprender. Proust descreve uma classe comprometida em todas as
suas partes em camuflar sua base material e, por essa mesma razão,
formada em um feudalismo, que, sem significação econômica em si, é tanto
mais utilizável como máscara da grande burguesia. Desiludido e impiedoso
desmistificador do eu, do amor, da moral, como Proust gostava de ver a
si mesmo, torna toda a sua arte ilimitada em véu deste único e mais
importante mistério de sua classe: o econômico. Não que ele esteja, com
isso, a serviço dela; ele está apenas à frente dela. O que ela vive
começa a tornar"-se compreensível com ele. Mas muito da grandiosidade
desta obra permanecerá inexplorada ou irrevelado até que essa classe
tenha revelado suas características mais nítidas na batalha final.

\section{III}

No século passado, em Grenoble --- não sei se hoje ainda --- havia uma
taverna ``\emph{Au temps perdu}''. Também em Proust somos
frequentadores que sob a placa que balança entram pelo umbral atrás
do qual a eternidade e a embriaguez nos aguardam. Fernandez distinguiu
com razão um \emph{thème de l'éternité} em Proust do \emph{thème du
temps}.\footnote{Em francês no original: ``um tema da eternidade'' e ``um tema do tempo''. \versal{[N. T.]}}
Mas essa eternidade certamente não é platônica, nem utópica: é
inebriante. Se o ``tempo desvenda para todo aquele que se aprofunda em
seu curso, um novo tipo de eternidade, até o momento desconhecida'',
então o indivíduo não se aproxima dos ``campos superiores que um Platão
ou um Spinoza alcançaram com um bater de asas''. Não, pois há, em Proust,
rudimentos de um idealismo duradouro. Não são eles, porém, que
determinam o significado dessa obra. A eternidade na qual Proust abre
aspectos é o tempo entrecruzado, não o tempo ilimitado. Seu verdadeiro
interesse é direcionado ao curso do tempo em sua forma mais real, que é,
no entanto, sua forma entrecruzada, que em nenhum lugar reina de maneira
menos dissimulada do que no lembrar, dentro, e no envelhecimento, fora.
Seguir o jogo antagônico do envelhecimento e do lembrar significa entrar
no coração do mundo de Proust, o universo do entrecruzamento. O mundo
está em estado de semelhança e nele reinam as ``correspondências'', que
foram capturadas primeiro pelo romantismo, e de maneira mais íntima por
Baudelaire, mas que Proust (como o único) conseguiu trazer à luz em
nossa vida vivida. Esta é a obra da \emph{mémoire involontaire}, da
força rejuvenescente que se encontra à altura do envelhecimento
implacável. Onde o ocorrido reflete"-se no ``instante'' ainda fresco de
orvalho, um doloroso choque de rejuvenescimento o reúne mais uma vez tão
inexoravelmente quanto a direção de Guermantes entrecruzou"-se com a
direção de Swann para Proust, quando ele (no décimo terceiro volume) uma
última vez vagueia pela região de Combray e descobre o cruzamento dos
caminhos. E instantaneamente, a paisagem muda como um vento. ``Ah! Que
le monde est grand à la clarté des lampes! --- Aux yeux du
souvenir que le monde est petit!''\footnote{Em francês no original: ``Ah! Como o mundo é
  grande na claridade das lâmpadas! Como o mundo é pequeno aos olhos da
  lembrança''. \versal{[N. T.]}} Proust conseguiu algo monstruoso, em um instante, deixar
envelhecer o mundo inteiro em torno de uma vida humana inteira. Mas
justamente essa concentração, na qual, o que de outra forma só murcha e
obscurece, consome"-se como em um relâmpago, isso se chama
rejuvenescimento. \emph{À la recherche du temps perdu} é a tentativa
incessante de carregar uma vida com a mais alta presença de espírito.
Não é a reflexão --- mas o tornar presente, o método de Proust. Ele está
de fato convencido da verdade, que nós não temos tempo de viver os
verdadeiros dramas da existência que nos são destinados. Isso nos faz
envelhecer. Nada mais. As rugas e pregas no rosto, são as inscrições das
grandes paixões, dos vícios, dos aprendizados que vieram nos visitar ---
mas nós, os patrões, não estávamos em casa.

Dificilmente houve uma tentativa mais radical de autoimersão desde os
exercícios espirituais de Loyola na literatura ocidental. Essa também
tem, em seu meio, uma solidão que, com a força de Maelstrom,\footnote{\emph{Maelstrom} é o nome próprio de um tipo de corrente muito forte
  nos países escandinavos. \versal{[N. T.]}} suga o mundo em seu turbilhão. E a
tagarelice excessivamente alta e, para além de qualquer conceito, vazia,
que ruge dos romances de Proust em nossa direção, é o bramido com o qual
a sociedade tomba no abismo dessa solidão. As investidas de Proust
contra a amizade têm seu lugar aqui. O silêncio no fundo desse funil ---
seus olhos são os mais quietos e sugadores --- quis ser mantido. O que
aparece tão irritante e caprichoso em tantas anedotas é a relação de uma
intensidade sem precedentes de conversa com uma distância insuperável do
parceiro. Nunca houve alguém que pôde nos mostrar as coisas como ele. O
apontar de seu dedo é inigualável. Mas há outro gesto na companhia
amigável, na conversação: o toque. Este gesto não é mais estranho a
ninguém do que a Proust. Ele também não pode tocar seu leitor, não o
poderia por nada nesse mundo. Se alguém quisesse organizar a poesia em
torno desses polos --- a que aponta e a que toca --- o centro de uma seria
a obra de Proust, o da outra seria a de Péguy.\footnote{Trata"-se do
  escritor francês Charles Péguy. \versal{[N. E.]}} No fundo é isso que Fernandez
entendeu brilhantemente: ``A profundidade, ou melhor, a penetração, está
sempre ao lado dele, nunca ao lado do parceiro''. Com um toque de
cinismo e virtuosamente, isso vem à luz em sua crítica literária. Seu
documento mais importante é um ensaio escrito na elevada altura da fama e
na baixa altura do leito de morte: ``À propos de Baudelaire''.\footnote{Em francês no  original: ``A propósito de Baudelaire''. \versal{[N. T.]}} Jesuiticamente em acordo com seu próprio %À propos tem a crase mesmo, não?
sofrimento, excessivo na tagarelice daquele que repousa, assustador na
indiferença do moribundo, que quer falar aqui mais uma vez ainda e não
importa sobre o que. O que o inspirou aqui diante da morte também o
determina ao lidar com os contemporâneos: uma alternância tão dura e
golpeadora de sarcasmo e ternura, ternura e sarcasmo, que seu objeto,
esgotado sob isso, ameaça colapsar.

O provocativo, o instável do homem diz ainda respeito ao leitor das
obras. Basta para pensar na cadeia imprevisível de ``soit
que'',\footnote{Em francês no original: ``seja que''. \versal{[N. T.]}} mostrando uma ação de forma
exaustiva e deprimente, à luz da miríade de motivos incontáveis que
podem tê"-la sustentado. E mesmo assim, nessa fuga paratática, onde
fraqueza e genialidade em Proust são apenas uma coisa só, vem à luz: a
renúncia intelectual, o ceticismo comprovado que ele teve com as coisas.
Depois das interioridades presunçosas e românticas, ele veio, como
Jacques Rivière o expressa, decidido a não dar a menor fé às ``Sirènes
intérieures''.\footnote{Em francês no original: ``sereias interiores''. \versal{[N. T.]}} ``Proust aproxima
da vivência sem o menor interesse metafísico, sem a menor inclinação
construtivista, sem a menor tendência ao consolo.'' Nada é mais
verdadeiro. E assim é a figura básica desta obra, da qual Proust não se
cansou de reivindicar o que foi planejado, nada menos do que construído.
Planejada, no entanto, ela é como o curso de nossas linhas da mão ou o
arranjo dos estames no cálice. Proust, esta criança anciã, deixou"-se
cair, profundamente cansado, de volta no seio da natureza, não para
sugá"-lo, mas para sonhar com seu batimento cardíaco. Deve"-se vê"-lo tão
fraco e compreender com qual felicidade Jacques Rivière podia entendê"-lo
a partir de sua fraqueza e dizer: ``Marcel Proust morreu da mesma
inexperiência que lhe permitiu escrever sua obra. Ele morreu de
alheamento do mundo e porque ele não soube mudar as condições de vida,
que haviam se tornado devastadoras para ele. Ele morreu porque não sabia
como fazer fogo, como abrir uma janela''. E, claro, de sua asma nervosa.

Os médicos sentiam"-se impotentes para enfrentar esse sofrimento, bem
diferente do poeta que o colocou a seu serviço de modo muito planejado.
Ele foi --- para começar com o mais superficial --- um perfeito diretor de
sua doença. Por meses ele combina com ironia devastadora a imagem de um
devoto que lhe enviara flores, com um cheiro insuportável para ele. E
com os tempos e marés de seu sofrimento, ele alarmou amigos que temiam e
ansiavam pelo momento, quando o poeta aparecia repentinamente, muito
após a meia"-noite, no salão --- \emph{brisé de fatigue},\footnote{Em francês no original:
  morto de cansaço. \versal{[N. T.]}} e apenas por cinco minutos, como declarava, para
depois ficar até o amanhecer acinzentado, cansado demais para se
levantar, cansado demais até para interromper o discurso. Mesmo o
escritor de cartas não encontra fim para extrair desse sofrimento os
efeitos mais remotos. ``O chiado de minha respiração é mais alto do que
minha pena e do que um banho que alguém prepara no andar abaixo de
mim''. Mas não só isso. Nem que a doença o tenha arrancado da existência
mundana. Essa asma entrou em sua arte, senão foi sua arte que a criou.
Sua sintaxe reproduz ritmicamente passo a passo esse seu medo de
sufocamento. E sua reflexão irônica, filosófica e didática é sempre a
tomada de fôlego, com o qual o pesadelo de lembranças aliviam seu
coração. Em uma escala maior, é, porém, a morte, que ele constantemente
e principalmente, quando escrevia, tinha presente, a crise ameaçadora e
sufocante. Ela estava, pois, diante de Proust e muito antes de seu
sofrimento se tornar crítico. No entanto, não como um pensamento
esquisito hipocondríaco, mas como aquela ``\emph{réalité nouvelle}'',\footnote{Em francês no original: ``nova realidade''. \versal{[N. T.]}} a nova realidade,
da qual o reflexo sobre as coisas e sobre as pessoas são os traços do
envelhecimento. A estilística fisiológica levaria ao núcleo dessa obra.
Portanto, ninguém que conheça a tenacidade particular, com a qual
lembranças são guardadas no sentido do olfato (de forma alguma odores na
lembrança!) poderá declarar como um acaso a sensibilidade de Proust aos
odores. Certamente, a maioria das lembranças pelas quais procuramos
aparecem como imagens visuais à nossa frente. E também as formações da
\emph{mémoire} \emph{involontaire} que ascendem livremente ainda são
imagens visuais em boa parte isoladas, apenas enigmaticamente presentes.
Mas, por isso mesmo, para nos entregarmos conscientemente à mais íntima
vibração dessa poesia, é preciso colocar"-se em uma camada especial e
mais profunda dessa rememoração involuntária, na qual os momentos de
lembrança nos dão notícia de um todo, não mais individualmente como
imagens, mas sem imagem e sem forma, indefinido e pesado, como o peso da
rede dá ao pescador notícia de sua captura. O odor, esse é o sentido de
peso daquele que lança suas redes no mar do \emph{temps perdu}. E suas
frases são todo o jogo muscular do corpo inteligível, contendo todo o
esforço indescritível para içar essa captura.

Além do mais, quão íntima era a simbiose desse processo de criação
particular e desse sofrimento em particular, mostra"-se de maneira nítida
no fato de que em Proust nunca irrompe aquele heroico apesar de tudo,
com o qual, no mais, pessoas criativas levantam"-se contra o sofrimento.
E, portanto, por outro lado, pode"-se dizer: uma cumplicidade tão
profunda com o curso do mundo e a existência como foi aquela de Proust,
deveria inequivocamente ter levado a uma suficiência comum e inerte em
qualquer outra base a não ser esse sofrimento tão profundo e incessante.
Mas esse sofrimento estava destinado a deixar"-se mostrar, por um furor
sem desejo e sem arrependimento, seu lugar no grande processo da obra.
Pela segunda vez, ergueu"-se um andaime como o de Michelangelo, no qual o
artista, com a cabeça inclinada, pintou a criação no teto da Capela
Sistina: o leito de enfermo, no qual Marcel Proust dedicou as inúmeras
páginas, que cobriu no ar com sua letra manuscrita, à criação de seu
microcosmo.
