\movetooddpage
\addcontentsline{toc}{chapter}{\normalsize\textsc{andré gide}}
\part*{André Gide}

\chapter*{André Gide e a Alemanha\\
\emph{Conversação com o poeta}\footnote[*]{``André Gide und Deutschland. Gespräch mit dem Dichter'', in \versal{GS IV}-1, p.~497--502. Tradução de Carla Milani Damião. Entrevista publicada no jornal \emph{Deutschland Allgemeine Zeitung} em 29 de janeiro de 1928. [\versal{N.~O.}]}}

\addcontentsline{toc}{chapter}{André Gide e a Alemanha}
\hedramarkboth{Gide e a Alemanha}{Benjamin}{}

Há algumas semanas, conversando com um dos principais críticos da
França, perguntei: ``Quem entre os grandes franceses parece"-lhe, em sua
figura e em sua obra, o mais aparentado a nós?''; sua resposta:
``André Gide''. Não quero negar que essa resposta, se por ela não
aguardava, por ela ansiava. No entanto, evitemos um mal"-entendido
evidente. Se Gide, o homem, o pensador, possui, de certa forma, uma
afinidade inegável com o gênio alemão, isso não significa que, como
artista, ele viria ao encontro dos alemães, e que se tornaria fácil para
seus leitores alemães. Não o é para eles como não o é para os seus
compatriotas.

A Paris de onde ele vem não é a dos incontáveis ​​escritores de romances
e do mercado internacional de comédias. O talento e a família ligam"-no
mais do que a esta cidade ao norte, à Normandia e sobretudo ao
protestantismo. É preciso ler uma obra como a \emph{Porte étroite}\footnote{Em francês no original: \emph{A porta estreita}. {[}\versal{N.~T.}{]}} para
reconhecer com que amor Gide envolve essa paisagem e até que ponto a
paixão ascética de sua jovem heroína está envolvida com essa paisagem.

Um aspecto moralista e reformista era peculiar ao seu trabalho desde o
início. Não há escritor no qual a energia produtiva e a energia crítica
tenham estado mais estreitamente ligadas do que nele. E se foi há trinta
anos o protesto do jovem Gide contra o nacionalismo primitivo e estéril
de Barrès, ou se hoje seu último romance, os \emph{Faux"-monnayeurs},\footnote{Em francês no original: \emph{Os moedeiros falsos}. {[}\versal{N.~T.}{]}}
que propõe uma reforma criativa da forma corrente do romance no
espírito da filosofia da reflexão romântica --- uma coisa é certa, esse
espírito permaneceu absolutamente fiel: a necessidade de repelir os
dados, não importa que estejam do lado de fora ou encontrem"-se em si
mesmo.

Se é nisso que encontramos a essência deste importante autor tanto como
poeta quanto como moralista, então são dois grandiosos que lhe mostraram o
caminho: Oscar Wilde e Nietzsche. Talvez o espírito
europeu em sua figura ocidental, em contraste com sua face oriental
representada por Tolstói e Dostoiévski, nunca tenha sido expresso com
maior evidência do que nesta tríade. Se todavia mais tarde, quando o
poeta fala sobre o que deve à literatura alemã, o nome de
Nietzsche não surja, pode ser porque falar de Nietzsche
significaria para Gide lidar consigo mesmo num sentido excessivamente
intenso, excessivamente responsável. Pois, teria compreendido pouco de
Gide quem não soubesse que os pensamentos de Nietzsche eram mais para
ele do que o esboço de uma ``visão do mundo''. ``Nietzsche'', disse Gide
ocasionalmente em uma conversa, ``abriu uma estrada real onde eu
poderia ter feito apenas um caminho estreito. Ele não `me influenciou';
ele ajudou"-me.''

Isso é modéstia, mesmo se não ocorre uma outra palavra a esse respeito hoje.
Modéstia: essa virtude tem duas faces. Existe a pretensa, a baixa, a
encenada pelos pequenos, e a calorosa, serena e verdadeira dos grandes. Ela
irradia de forma convincente cada movimento desse homem. Sente"-se que ele está
acostumado a se mover na casa real das ideias. De lá, do contato com
rainhas, a entonação suave, o jogo hesitante, porém importante das
mãos, o olhar discreto, mas atento de seus olhos. E quando ele assegura"-me
ser geralmente um interlocutor desconfortável na conversação --- tímido e
selvagem ao mesmo tempo --- entendo: para ele, sair do círculo da existência habitual e
solitária, daquela casa real, significa, ao mesmo
tempo, perigo e sacrifício. Ele menciona o ditado de Chamfort: ``Se alguém
realizou uma obra"-prima, as pessoas não têm nada mais urgente a fazer do
que impossibilitar a próxima''. Como nenhum outro, Gide rejeitou
vigorosamente honras e títulos de glória. ``É verdade'', diz ele,
``Goethe diz que apenas os mendigos são modestos, no entanto'', continua
ele, ``não houve nenhum gênio mais modesto do que ele. Pois, o que
significa a paciência de, na velhice extrema, subordinar"-se a uma ordem
mais baixa e aprender o persa? Sim, mesmo ler no entardecer de um dia
enorme de trabalho já era, para esse homem, modéstia.''

Na França, houve rumores de que Gide queria traduzir \emph{As afinidades
eletivas}. E porque recentemente seu diário de viagem do Congo fala
sobre uma nova leitura do livro, pergunto a ele sobre isso. ``Não'',
Gide responde, ``traduzir agora é uma aventura remota para mim. É claro
que Goethe ainda me atraía.'' Uma ligeira hesitação segue, característica
nele. ``E certamente todo Goethe está em \emph{As afinidades
eletivas}, mas se eu tivesse que traduzir alguma coisa, preferiria
pensar em \emph{Prometeu}, algumas passagens de \emph{Pandora}, ou em
páginas de prosa menos frequentadas, como o escrito sobre Winckelmann.''

Penso, então, em uma tradução alemã que Gide publicou recentemente, um
capítulo de \emph{Henrique, o verde}\footnote{\emph{Der grüne
  Heinrich} (\emph{Henrique, o verde}, 1854--1855), romance de Gottfried Keller
  (1819--1890). \versal{[N.~O.]}} de Gottfried Keller. O que pode ter levado o poeta a
essa direção? Uma expressão de Jacques Rivière, o amigo desaparecido,
passa pela minha mente: aquele ``jardim encantado da hesitação'', no
qual Gide iria morar por toda a vida. Neste jardim também viveu Keller,
o poeta de profundas inibições e reservas apaixonadas, e a partir daqui
o encontro entre os dois grandes prosadores poderia ter nascido.

Mas nesse ponto não consegui retirar algo de Gide a esse
respeito, pois a conversa deu uma súbita virada: ``Gostaria de dizer"-lhe
algo mais sobre os propósitos da minha visita''. O objetivo era realizar
uma \emph{conférence}\footnote{Em francês no original: conferência. \versal{[N.~T.]}} em Berlim. E para sua preparação queria dedicar a
primeira semana da minha estada à paz e ao recolhimento. Mas as
coisas foram muito diferentes daquilo que havia previsto, pois a
amabilidade dos berlinenses, seu interesse por mim, provou ser tão
grande que a ociosidade, com a qual contava, não queria se apresentar.
Reuniões e palestras enchiam meu tempo. Por
outro lado, minha decisão de não me apresentar, exceto com um discurso
bem elaborado, permaneceu firme\emph{: Je voulais faire quelque chose de
très bien}.\footnote{Em francês no original: ``Eu queria fazer algo muito bem''. \versal{[N.~T.]}}
E ficaria contente se o senhor informasse isso e acrescentasse que
meu objetivo não foi abandonado, somente a execução foi adiada. Voltarei com minha \emph{conférence}.
Talvez então ela tenha um tema bastante diferente do que o que tinha em
mente desta vez. Posso dizer apenas isso: não pretendera, nem pretendo
no futuro, falar aqui da literatura francesa, como frequentemente
acontece. Em minhas conversas berlinenses pude verificar continuamente
o quão bem informados a este respeito estão todos os senhores, que se
interessam por isso.

``Estava pensando em falar sobre outra coisa. Gostaria de expor quais são
os elementos mais frutíferos e estimulantes de sua literatura para mim
como autor francês. Os senhores teriam me ouvido falar sobre qual papel
desempenhavam na França e, em particular, para mim, Goethe, Fichte,
Schopenhauer. Também aproveitaria a oportunidade de falar aos
senhores sobre o novo e intenso interesse que as coisas alemãs despertam em nós.
E se compararmos o estudioso francês atual com o da geração anterior,
posso dizer o seguinte: ele ficou mais ansioso para saber, sua visão
está prestes a se ampliar para além dos limites culturais e linguísticos
de sua terra natal. Compare isso com a afirmação de Barrès: `Aprendendo
idiomas? Por quê? Para dizer a mesma tolice de três ou quatro maneiras
diferentes?' O senhor percebe o significado dessa frase? Barrès,
sobretudo, pensa apenas em falar, ler em uma língua estrangeira,
adentrar em uma língua estrangeira, não vale nada para ele. Se em Barrès
isso era uma suficiência nacionalista, em Mallarmé na mesma época era
uma suficiência no mundo espiritual interior o que fazia cada olhar para
o exterior, o amor à viagem ou o conhecimento de línguas estrangeiras,
tornar"-se algo raro. Não teria, talvez, a filosofia do idealismo alemão
levado seus discípulos franceses a essa atitude?''

Gide conta então a curiosa anedota sobre como Villiers de L'Isle"-Adam
introduziu a doutrina hegeliana no círculo de Mallarmé. Parece que um
dia o jovem Villiers comprou um saco de batatas quentes numa esquina:
o papel do embrulho, porém, era uma folha de uma tradução da
\emph{Estética} de Hegel. Desta forma, e não no caminho oficial pela
Sorbonne e de Victor Cousin, o idealismo alemão teria chegado aos
simbolistas.

\emph{``Ne jamais profiter de l'élan acquis}'' --- nunca tire proveito do
ímpeto já alcançado: no \emph{Journal des faux"-monnyeurs},\footnote{Em francês no original: \emph{Diário dos moedeiros falsos}. {[}\versal{N.~T.}{]}} Gide define
assim uma das máximas de sua técnica literária. É, porém, muito mais do
que uma regra da escrita: é a expressão de uma atitude espiritual que
aborda cada problema como se fosse o primeiro, o único de um mundo que
acabou de sair do nada. E se o poeta, como a figura mais representativa
da cultura francesa, dirigir"-se aos ouvintes alemães, oxalá em um futuro
próximo, ele o fará, na direção de um recomeço, com um espírito
que não deverá em nada aos ânimos e aos humores da opinião pública aqui
e lá. Para aquele que escreveu essa frase há muitos anos:
``Reconhecemos como válida somente a obra que no seu elemento mais
profundo é a revelação do solo e da raça da qual emerge'', a comunidade
dos povos é uma coisa que se realiza apenas na mais elevada e precisa
expressão, e ao mesmo tempo também apenas na purificação espiritual mais
rigorosa dos caracteres nacionais. Obscuridade e imprecisão, onde quer
que estejam, são estranhas para ele: não é por acaso que Gide sempre se
confessou um fanático pelo desenho, pelo contorno nítido.

Nesse sentido, vamos esperá"-lo novamente na Alemanha, ansiosamente e com
alegria; ele, o grande francês que, com seu trabalho, paixão e
coragem, conseguiu dar à sua fisionomia o caráter europeu.
